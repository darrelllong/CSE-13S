\documentclass{article}
\usepackage{fullpage,fourier,amsmath,amssymb}
\usepackage{listings,color,url,hyperref}
\usepackage{mdframed}
\usepackage[x11names]{xcolor}

\usepackage{epigraph}
\title{Assignment 8 \\ Lempel-Ziv-Welch Compression}
\author{Prof. Darrell Long \\
CSE 13S -- Fall 2019}
\date{Due: December 6th at 11:59 pm}

\definecolor{codegreen}{rgb}{0,0.5,0}
\definecolor{codegray}{rgb}{0.5,0.5,0.5}
\definecolor{codepurple}{rgb}{0.58,0,0.82}

\lstloadlanguages{C,make,python,fortran}

\lstdefinestyle{c99}{
    morekeywords={bool, uint8_t, uint16_t, uint32_t, uint64_t, int8_t, int16_t, int32_t, int64_t},
    commentstyle=\color{codegreen},
    keywordstyle=\color{magenta},
    numberstyle=\tiny\color{codegray},
    identifierstyle=\color{blue},
    stringstyle=\color{codepurple},
    basicstyle=\ttfamily,
    breakatwhitespace=false,
    breaklines=true,
    captionpos=b,
    keepspaces=true,
    numbers=left,
    numbersep=5pt,
    showspaces=false,
    showstringspaces=false,
    showtabs=false,
    tabsize=4
}

\usepackage{fancyhdr}
\pagestyle{fancy}
\fancyhf{}

\fancypagestyle{plain}{%
  \fancyhf{}
  \renewcommand{\headrulewidth}{0pt}
  \renewcommand{\footrulewidth}{0pt}
  \lfoot{\textcopyright{} 2021 Darrell Long}
  \rfoot{\thepage}
}

\pagestyle{plain}


\begin{document}\maketitle

\lstset{language=C, style=c99}

\epigraphwidth=0.65\textwidth
\epigraph{\emph{Everyday, we create 2.5 quintillion bytes of data -- so much
that 90\% of the data in the world today has been created in the last two years
alone.}}{---IBM report on Big Data (2011)}

\section{Introduction}
Compressing data means reducing the number of bits needed to represent it.
Compressed data, due to its reduced size, is a lot faster to transfer and less expensive
to store, freeing up storage and increasing network capacity. Algorithms
that perform data compression are known as data compression algorithms. Data
compression algorithms are divided into two categories: lossy and lossless.
Lossy compression algorithms compress more than lossless compression algorithms,
but at the cost of losing some information, which can't be recovered. Lossy
compression algorithms are typically used for audio and video files, where a
loss in quality is tolerable and often not noticeable. Lossless compression
algorithms on the other hand don't compress as much, but do not lose any
data, meaning compressed information can be exactly reconstructed back into its
uncompressed form. Lossless compression algorithms are used for data that must
maintain integrity, such as binaries, text documents, and source code.

\section{Lempel-Ziv-Welch}
Lempel-Ziv-Welch (LZW) is a lossless compression algorithm commonly used for
GIF, PDF, and TIFF formatted files. The idea behind LZW is to identify patterns
in data and to assign each one a unique code. In LZW, codes are typically 12 or
16-bit integers, and sequences of characters found in data are commonly
referred to as words.

As an example of LZW, take some string $s =$ ``$abcabcabc$''. Assume that we are
using 16-bit codes and have a look-up table with $2^{16}$ indices. Each index of
the table stores a word. The table is initialized with each of the ASCII
characters as words, meaning indices 0 --- 255 are already taken, and the next
available code is 256.  We compress $s$ by parsing through it character by
character. The first character is an `$a$', which already exists in our table
with a code of 97, so we move onto the next character, keeping track that we
just read an `$a$'. We then read a `$b$', which we append onto our previously
read `$a$'. Now the current constructed word we have is ``$ab$'', which doesn't
yet exist in our table. We add ``$ab$'' to our table and assign it the next
available code (in this case, 256), reset the previously read word to just the
`$b$', output the code for `$b$' to a compressed file, which is 98, and move
onto processing the rest of the word in a similar fashion. In a file full of
repeated words, it's clear that the words added to the table only get longer and
longer, growing by at most one character each time, while our code numbers only
increase in bit-length each time the limit of the current bit-length is reached.

Of course, a compression algorithm is useless without a corresponding
decompression algorithm. Assume we're reading in the compressed file we
generated from the compression example. Again, assume we have the same look-up
table as we used for compression where each index is a word and is initialized
with the ASCII characters, and that the next available code is 256. To
decompress, we read one code at a time from the file. The first code that
compression output was 97, which decodes to an `$a$'. We keep track of 97 as the
previously read code. We move onto the next code, which is 98. This is the key
part of LZW: at this point, not only can we decode 98 into a `$b$', but we can
also append `$b$' onto `$a$' (which we get from decoding the previously read
code) to create a new word ``$ab$'', which we give the next available code of
256. This new word is now added to the table. As you can see, decompression
mirrors what compression does but one step behind. Although the table that
decompression and compression builds is the same, decompression needs to read
one extra code to match the table built by compression.

If the basic idea behind the compression and decompression algorithms do not
immediately make sense to you, or if you desire a more visual representation of
how the algorithms work, make sure to attend section; there will be a
presentation and walkthrough of the assignment.

\section{Your Task}

Your task is to implement a program that performs LZW compression and
decompression. Your program \texttt{must} be able to perform the following:
\begin{enumerate}
\item Compress any file, text or binary.
\item Decompress any file, text or binary, without losing any original information.
\item Use variable length codes up to 16 bits.
\item Operate on little and big endian systems. \emph{Interoperability is required}.
\item Perform read and writes in efficient blocks of 4KB.
\end{enumerate}

\section{Specifics}

You will need to implement some new ADTs for this assignment: an ADT for tries
and an ADT for words. In addition to these new ADTS, you will need to
be concerned about variable-length codes, I/O, and endianness.

\subsection{Tries}
The most costly part of compression is checking for existing prefixes. You could
utilize a hash table, or just an array of words to store prefixes, but that
wouldn't be optimal, as many of the prefixes you need to store are prefixes of
other prefixes. Instead you choose to utilize a \emph{trie}.

A trie\footnote{Edward Fredkin, ``{T}rie memory.'' \emph{Communications of the
ACM} 3, no. 9 (1960): 490--499.}
is an efficient information re-\emph{trie}-val data structure, commonly
known as a prefix tree. Each node in a trie represents a symbol, or a character,
and contains $n$ child nodes, where $n$
is the size of the alphabet you are using. In most cases, the alphabet used is
the set of ASCII characters, so $n = 256$. You will use a trie during
compression to store words.

As an example, suppose we are searching for the word ``$hello$'' in the trie. We
start at the root of the trie and read in the first character: `$h$'. We check
if the root node has a child node representing `$h$' exists. If it does, we step
down the trie to the child trie node representing `$h$', and read in the next
character, with the node we now step from the node we just stepped down to.  If
it doesn't, then the word doesn't exist in the trie. You repeat this process of
stepping down the trie to check if the word exists in the trie. You \emph{must}
follow the specification shown below when implementing your trie ADT.

\begin{lstlisting}[title=trie.h]
#ifndef __TRIE_H__
#define __TRIE_H__

#include <inttypes.h>
#include <stdbool.h>

// Typedef TrieNode to be just TrieNode.
typedef struct TrieNode TrieNode;

//
// Struct definition for a TrieNode.
//
// children:   An array of TrieNode pointers.
// code:       The TrieNode's unique code.
//
struct TrieNode {
  TrieNode *children[256];
  uint16_t code;
};

//
// Constructor for a TrieNode.
//
// code: The TrieNode's unique code.
//
TrieNode *trie_node_create(uint16_t code);

//
// Destructor for a TrieNode.
//
// n: The TrieNode to be freed.
//
void trie_node_delete(TrieNode *n);

//
// Creates and initializes a new trie.
// All ASCII characters are initially added as the root's children.
//
TrieNode *trie_create(void);

//
// Resets a trie back to its original state of only ASCII characters.
//
// root: The root of the trie to reset.
//
void trie_reset(TrieNode *root);

//
// Frees memory allocated for an entire trie.
//
// n: The root of the trie to free.
//
void trie_delete(TrieNode *n);

//
// Steps down to child the TrieNode that represents the symbol.
// Returns NULL if the symbol doesn't exist.
//
// n:   The TrieNode to step from.
// sym: The symbol to step to.
//
TrieNode *trie_step(TrieNode *n, uint8_t sym);

#endif
\end{lstlisting}

The \texttt{TrieNode struct} will have the three fields shown above. Each trie
node has an
array of 256 pointers to trie nodes as children, one for each ASCII character. It should be
easy to see how this simplifies searching for the next character in a word in
the trie. The \texttt{code} field stores the 16-bit code for the word that ends
with the trie node containing the code. This means that the code for
some word ``$abc$'' would be contained in the trie node for `$c$'. Note that
there isn't a field that indicates what character a trie node represents. This is
because the trie node's index in its parent's array of child nodes indicates what
character it represents. The \texttt{trie\_step()} function
will be repeatedly called to check if a word exists in the trie. A word only
exists if the trie node returned by the last step corresponding to the last
character in the word isn't \texttt{NULL}.

\subsection{Word Tables}

Although compression can be performed using a trie, decompression still needs to
use a look-up table for quick code to word translation. This look-up table will
be defined as a new \texttt{struct} called a \texttt{WordTable}. Your assignment
will only use up to 16-bit codes, so we can used a fixed word table size of
\texttt{UINT16\_MAX}, a macro defined in \texttt{inttypes.h} as the max value a
16-bit integer can have. Why not a hash table? Because there is a finite number
of codes. Each index of this word table will be a new \texttt{struct} called a
\texttt{Word}. You will store words in byte arrays, or arrays of
\texttt{uint8\_t}'s. This is because strings in \textbf{C} are null-terminated,
and problems with compression occur if a binary file is being compressed and
contains null characters that are placed into strings. Since we need to know how
long a word is, a \texttt{Word} will also have a field for storing the length of
the byte array, since we can't use \texttt{string.h} functions like
\texttt{strlen()} on byte arrays. You \emph {must} use the following
specification for the new \texttt{Word} ADT.

\begin{lstlisting}
#ifndef __WORD_H__
#define __WORD_H__

#include <inttypes.h>

//
// Struct definition of a Word.
// A word in LZW is just a byte array.
//
// word:    The byte array.
// length:  The length of the word.
//
typedef struct Word {
  uint8_t *word;
  uint64_t length;
} Word;

//
// Struct definition of a WordTable.
// Each index is a Word.
//
// entries:     An array of Word pointers.
//
typedef struct WordTable {
  Word *entries[UINT16_MAX];
} WordTable;

//
// Constructor for a Word.
//
// word:    The byte array.
// length:  The length of the word.
//
Word *word_create(uint8_t *word, uint64_t length);

//
// Destructor for a Word.
//
// w:   The Word.
//
void word_delete(Word *w);

//
// Constructor for a WordTable.
// Is initialized with all ASCII characters.
//
WordTable *wt_create(void);

//
// Resets a word table to its original state of just ASCII characters.
//
// wt: The word table to reset.
//
void wt_reset(WordTable *wt);

//
// Destructor for a WordTable.
//
// wt:  The WordTable.
//
void wt_delete(WordTable *wt);

#endif
\end{lstlisting}

\subsection{Codes}
Your implementation of LZW will use \emph{variable length codes}. This means
that the length in bits of codes that are written to a compressed file is the
\emph{minumum number of bits needed to represent the code}. What does this mean?
Assume we have the code of 256. The minimum number of bits needed to represent
256 is 9, so the bit length of the code for 256 is 9. You can calculate the
minimum number of bits to represent an integer $x$ using the formula $log_2(x) +
1$. That being said, codes can only increase in bit length. As you know,
compression writes codes to files and decompression reads codes from files. This
means that decompression \emph{must} know at all times how long in bits the next
code is. Assume that the code we are writing represents 5 and our current bit
length is 9. Thus, since 3 bits is all that is needed to represent 5, we need to
pad out the code with 6 zeroes, so ``101000000''.

\subsection{I/O}

You will need to be mindful about how I/O is done in this assignment. More
specifically, you will need to think about how to read and write codes to and
from files. Your initial thought might be to write each code to the output file
after constructing it. \emph{This will not work with variable length codes.} The
smallest unit of data you can write to a file is a \emph{byte}. If your codes
have extra bits of padding between them, your program won't work correctly.
Additionally, most, if not all, file systems write in blocks of 4KB, so you must
buffer your variable length codes and characters to write efficiently in blocks
of 4KB. The following is a module interface to help streamline all the I/O that
will be done for this assignment.

\begin{lstlisting}[title=io.h]
#ifndef __IO_H__
#define __IO_H__

#include "word.h"
#include <inttypes.h>

#define MAGIC           0xdeadcafe
#define BLOCK_SIZE      4096

//
// Struct definition for a compressed file's header.
//
// magic:      Magic number that verifies this compression program.
// file_size:  File size of original, uncompressed file.
// protection: Holds mask of original protection bits.
// padding:    Padding to align struct to 32 bits.
//
typedef struct FileHeader {
  uint32_t magic;
  uint64_t file_size;
  uint16_t protection;
  uint16_t padding;
} FileHeader;

//
// Reads a FileHeader struct from the input file.
// This should read sizeof(FileHeader) bytes into the header.
// Should verify the magic number in the header that is read in.
// Programs with an incorrect magic number should not be decompressed.
//
// infile:     The input file descriptor.
// header:     FileHeader to read into from the input file.
//
void read_header(int infile, FileHeader *header);

//
// Writes a FileHeader struct to the output file.
// This should write sizeof(FileHeader) bytes from the header.
//
// outfile:    The output file descriptor.
// header:     FileHeader to write from to the output file.
//
void write_header(int outfile, FileHeader *header);

//
// Returns the next character, or byte, from the input file.
// Should utilize a static 4KB uint8_t array to store bytes.
// 4KB (BLOCK_SIZE) are read into the buffer whenever the buffer is empty.
// A byte index is used to indicate the next character to return.
// This is called once for each byte in the input file.
//
// infile:     The input file descriptor.
//
uint8_t next_char(int infile);

//
// Adds the binary representation of a code to a buffer.
// Should utilize a static 4KB uint8_t array to store bits.
// The buffer is written to the output file whenever it is full.
// The binary code to add should be padded out to bit_len bits.
// Ex. buffer_code(outfile, 5, 6) means adding "101000".
// A bit index is used to indicate which bit to set in the
// buffer when adding a code.
//
// outfile:    The output file descriptor.
// code:       Code to buffer.
// bit_len:    The length in bits of the code to buffer.
//
void buffer_code(int outfile, uint16_t code, uint8_t bit_len);

//
// Flushes any remaining codes in the buffer used to store codes.
// Warning: this may not necessarily mean 4KB of codes.
//
// outfile:    The output file descriptor.
//
void flush_codes(int outfile);

//
// Returns the next code in the input file.
// Should utilize a static 4KB uint8_t array to store bits.
// 4KB (BLOCK_SIZE) are read into the buffer whenever it is empty.
// The binary of the next code is bit_len number of bits long.
// Converts the binary of the next code into a uint16_t and returns it.
// Called until main decompression loop decodes all characters.
//
// infile:     The input file descriptor.
// bit_len:    The length in bits of the code to read.
//
uint16_t next_code(int infile, uint8_t bit_len);

//
// Adds a Word into a buffer.
// Should utilize a static 4KB uint8_t array to store bytes.
// The buffer is written when it's full of characters.
// Should utilize a byte index to keep track of where each byte
// of the Word to add should be set in the static 4KB buffer.
// This will involve looping through the byte array in a Word.
// Why? Each byte in the array of a Word must be buffered.
//
// outfile:    The output file descriptor.
// w:          Word to buffer.
//
void buffer_word(int outfile, Word *w);

//
// Flushes the buffer used to store words to the output file.
// Warning: this many not necessarily mean 4KB of characters.
//
// outfile:    The output file descriptor.
//
void flush_words(int outfile);

#endif
\end{lstlisting}

Notice that there is a \texttt{struct} definition in the module interface. That
is the \texttt{struct} definition for the file header, which contains the magic
number for your program, the file size of the original, uncompressed file, the
protection bit mask for the original file, and some padding to align the
\texttt{struct} to 32 bits. The file header is the first thing that appears in a
compressed file. The magic number field, \texttt{magic}, serves as a unique
identifier for your program.  Your program should only be able to decompress
files which have the same magic number in the file header. The function
\texttt{read\_header()} reads in the header and verifies the magic number.  The
protection bit mask comes from the original file. The output file in which you
write to must have the same protection bits as the original file. The
\texttt{padding} field exists solely for alignment, and thus its value does not
matter. Before writing the file header to the compressed file using
\texttt{write\_header()}, you must swap the endianness of the fields if
necessary since \emph{interoperability is required}. If your program is run on a
system using big endian, the fields must be swapped to little endian, since
little endian is canonical. Here is another module specifically for handling
endianness:

\begin{lstlisting}[title=endian.h]
#ifndef __ENDIAN_H__
#define __ENDIAN_H__

#include <inttypes.h>
#include <stdbool.h>

//
// Checks if the order of bytes on the system is big endian.
//
static inline bool is_big(void) {
  union {
    uint8_t bytes[2];
    uint16_t word;
  } test;
  test.word = 0xFF00;
  return test.bytes[0];
}

//
// Checks if the order of bytes on the system is little endian.
//
static inline bool is_little(void) {
  return !is_big();
}

//
// Swaps the endianness of a uint16_t.
//
// x:  The uint16_t.
//
static inline uint16_t swap16(uint16_t x) {
  uint16_t result = 0;
  result |= (x & 0x00FF) << 8;
  result |= (x & 0xFF00) >> 8;
  return result;
}

//
// Swaps the endianness of a uint32_t.
//
// x:  The uint32_t.
//
static inline uint32_t swap32(uint32_t x) {
  uint32_t result = 0;
  result |= (x & 0x000000FF) << 24;
  result |= (x & 0x0000FF00) << 8;
  result |= (x & 0x00FF0000) >> 8;
  result |= (x & 0xFF000000) >> 24;
  return result;
}

//
// Swaps the endianness of a uint64_t.
//
// x:  The uint64_t.
//
static inline uint64_t swap64(uint64_t x) {
  uint64_t result = 0;
  result |= (x & 0x00000000000000FF) << 56;
  result |= (x & 0x000000000000FF00) << 40;
  result |= (x & 0x0000000000FF0000) << 24;
  result |= (x & 0x00000000FF000000) << 8;
  result |= (x & 0x000000FF00000000) >> 8;
  result |= (x & 0x0000FF0000000000) >> 24;
  result |= (x & 0x00FF000000000000) >> 40;
  result |= (x & 0xFF00000000000000) >> 56;
  return result;
}

#endif
\end{lstlisting}

All reads and writes in this program must be done using \texttt{read()} and
\texttt{write()}, which means that you must use the system calls \texttt{open()}
and \texttt{close()} to get your file descriptors.  The functions in
\texttt{io.h} that deal with \emph{reading} from files will require you to read
4KB at a time into a static buffer, which you iterate through for each
individual byte or bit. These functions are \texttt{next\_code()} and
\texttt{next\_char()}. The functions in \texttt{io.h} that deal with
\emph{writing} to files will similarly require you to store codes and words into
a static buffer to write 4KB at a time. These functions are
\texttt{buffer\_code()}, \texttt{buffer\_word()}, \texttt{flush\_codes()}, and
\texttt{flush\_words()}.  \emph{Hint: you will want to use a variable that is
designated to keep track of the byte or bit index of the static buffer you are
using. This goes for the static buffer for both reading from and writing to
files. This means two static 4KB uint8\_t arrays to serve as buffers: one to
store bits/codes and the other to store characters.}

So how does \texttt{buffer\_code()} work? The function takes in a file
descriptor, a code, and a bit length as its parameters. The file descriptor is
where to output codes to once your buffer is filled, the code is the code to
buffer, and the bit length is the length in bits to pad the code to before
buffering it. For example, if the code to buffer is 13 and the bit length is 9,
what is placed into the buffer is ``101100000''. Notice that the binary placed
into the buffer is reversed. This is because the function \texttt{next\_code()}
reads in a set number of bits for the next code given by the bit length
parameter and sums the bits into a \texttt{uint16\_t} code.

\section{Program Options}
You will use \texttt{getopt()} to handle the following options:
\begin{itemize}
    \item \texttt{-v} : verbose option
    \item \texttt{-c} : perform compression
    \item \texttt{-d} : perform decompression
    \item \texttt{-i <input file>} : sets the input file (default is \texttt{stdin})
    \item \texttt{-o <output file>} : sets the output file (default is \texttt{stdout})
\end{itemize}
Compression and decompression are mutually exclusive; only one may be chosen at
a time. You should also be able to perform compression or decompression if
\texttt{argv[0]} is ``$encode$'' or ``$decode$'' respectively. This is done
using \emph{linked files}. Read the man page on it for details (\texttt{man
ln}). If an input file or an output file is not specified, \texttt{stdin} and
\texttt{stdout} are used by default respectively. If any of these conditions
aren't held while processing command-line arguments, print an error message
about program usage and exit with a status code indicating an error occurred.
The verbose option enables a flag to print out informative statistics about the
compression or decompression that is performed. These statistics include the
original file size, the compressed file size, the compression ratio, and the
longest word length. The formula for the compression ratio is
$100\times(1-(\text{compressed size} / \text{original size}))$. The verbose
output of both compression and decompression must match the following:

\begin{lstlisting}
Original file size: X
Compressed file size: X
Compression ratio: XX.XXXX%
Longest word length: X
\end{lstlisting}

\section{Compression}
The following steps for compression will refer to the input file to compress as
\texttt{infile} and the compressed output file as \texttt{outfile}.

\begin{enumerate}
    \item Open \texttt{infile} with \texttt{open()}. If an error occurs, print a
        helpful message and exit with a status code indicating that an error
        occurred. If an input file name wasn't specified, set \texttt{infile} to
        be \texttt{stdin}.
    \item The first thing in \texttt{outfile} must be the file header, as
        defined in the file \texttt{io.h}. The magic number in the header must
        be \texttt{0xdeadcafe}. The file size and the protection bit mask you
        will obtain using \texttt{fstat()}. See the man page on it for details.
    \item Open \texttt{outfile} using \texttt{open()}. The permissions for
        \texttt{outfile} should match the protection bits as set in your file
        header. Any errors with opening \texttt{outfile} should be handled like
        with \texttt{infile}. If an output file name wasn't specified, set
        \texttt{outfile} to be \texttt{stdout}.
    \item Write the filled out file header to \texttt{outfile} using
        \texttt{write\_header()}. This means writing out the \texttt{struct}
        itself to the file, as described in the comment block of the function.
    \item Create a trie and add all 256 ASCII characters as children
        to the trie's root. Their respective codes are their respective
        ASCII values. You will need to make a copy of the root node and use the
        copy to step through the trie to check for existing prefixes. This root
        node copy will be referred to as \texttt{curr\_node}. The reason a copy
        is needed is that you will eventually need to reset whatever trie node
        you've stepped to back to the top of the trie, so using a copy lets you
        use the root node as a base to return to.
    \item You will need a monotonic counter to keep track of the next available
        code. This counter should start at 256, since code numbers 0 --- 255
        are reserved for each individual ASCII character that is added to the
        trie initially. This should be a \texttt{uint16\_t} since the largest
        code in your implementation is 16 bits. This will be referred to as
        \texttt{next\_avail\_code}.
    \item This is the main loop for compressing:
        \begin{enumerate}
            \item Get the next character from \texttt{infile} with
                \texttt{next\_char()}. Step down the trie based on the
                character you just read with \texttt{trie\_step()}.
                The returned node will be referred to as \texttt{next\_node}.
            \item If you haven't processed any characters yet, or
                \texttt{next\_node} isn't \texttt{NULL}, set \texttt{curr\_node}
                to be \texttt{next\_node} and move on to processing the next
                character.
            \item Else, \texttt{next\_node} must be \texttt{NULL}, meaning the
                word doesn't exist in the trie. Buffer the current code stored
                in \texttt{curr\_node} using \texttt{buffer\_code()}. The bit
                length for the code is the minimum number of bits needed to
                represent the next available code. You should be keeping
                track of this using your monotonic counter. Add a new child node
                to \texttt{curr\_node} whose code is the next available code.
                The index of this child node is the ASCII value of the character
                that was just read. Make sure to increment the next available
                code counter by 1 every time you assign a word a code.
                Reset \texttt{curr\_node} to be the child node of the trie's
                root for the character you just read, which should be one of the
                ASCII characters the trie was initialized with.
            \item Check if the next available code has hit the 16-bit limit,
                which is defined by the macro \texttt{UINT16\_MAX} in
                \texttt{inttypes.h}. If it has, reset the trie to just having
                the ASCII characters and reset the next available code as well
                back to 256. We have this limit since codes are of type
                \texttt{uint16\_t}, and we want to avoid integer overflow.
            \item If your 4KB code buffer has filled up, write the contents to
                \texttt{outfile}.
        \end{enumerate}
    \item Pseudocode for the compression loop:
\begin{lstlisting}
while (encoded_chars != header.file_size):
  curr_char = next_char()
  next_node = trie_step(curr_node, curr_char)

  if (encoded_chars == 0 or next_node != NULL):
    curr_node = next_node
  else:
    bit_len = log2(next_avail_code) + 1
    buffer_code(curr_code, bit_len)
    curr_node.children[curr_char] = trie_node_create(next_avail_code)
    curr_node = root.children[curr_char]
    next_avail_code += 1

  if (next_avail_code == UINT16_MAX):
    trie_reset(root)
    curr_node = root.children[curr_char]
    next_avail_code = 256
\end{lstlisting}

    \item After processing all the characters in \texttt{infile}, check if there
        are any characters left in the current word. If there are, this means
        that the word they comprise exists in the trie, so you can just buffer
        the code for that word.
    \item Make sure to use \texttt{flush\_codes()} to flush any codes left in
        your code buffer.
    \item Use \texttt{close()} to close \texttt{infile} and \texttt{outfile}.
\end{enumerate}

\section{Decompression}
The following steps for decompression will refer to the input file to decompress
as \texttt{infile} and the uncompressed output file as \texttt{outfile}.

\begin{enumerate}
    \item Open \texttt{infile} with \texttt{open()}. If an error occurs, print a
        helpful message and exit with a status code indicating that an error
        occurred. If an input file name wasn't specified, set \texttt{infile} to
        be \texttt{stdin}.
    \item Read in the file header with \texttt{read\_header()}, which also
        verifies the magic number. If the magic number is verified then
        decompression is good to go and you now have a header which contains the
        original file's size and protection bit mask. You will be decoding
        characters until the number of decoded characters matches the original
        file size.
    \item Open \texttt{outfile} using \texttt{open()}. The permissions for
        \texttt{outfile} should match the protection bits as set in your file
        header. Any errors with opening \texttt{outfile} should be handled like
        with \texttt{infile}. If an output file name wasn't specified, set
        \texttt{outfile} to be \texttt{stdout}.
    \item Create a new word table and make sure each of its entries are set to
        \texttt{NULL}. Initialize the table to have all ASCII characters.
    \item You will need to store the current and
        previously decoded words as byte arrays. You are recommended to use two
        buffers to store the words that are initially dynamically allocated to
        hold up to 4KB. If the buffers fill up, reallocate twice the amount of
        space for them. The worst case scenario is wasting half the buffer
        space.
    \item You will need two \texttt{uint16\_t}'s to keep track of the
        current and previously read codes.
    \item As with compression, you will need a monotonic counter for keeping
        track of the next available code. This counter will start at 256.
    \item During compression, the trie is reset when the next available code
        hits the 16-bit limit. Decompression mirrors this by resetting
        the word table. You will need to use a flag to keep track of whether or
        not the table was reset.
    \item The following is the main loop for decompression:
        \begin{enumerate}
            \item Read in the next code with \texttt{next\_code()}. The length
                of the next code in bits should be the minimum number of bits
                needed to represent the
                \emph{code after the next available code}.
\begin{lstlisting}
next_avail_code = 256
bit_len = log2(next_avail_code + 1) + 1
\end{lstlisting}
                The reason the length in bits of the next code is the minimum
                number of bits needed to represent the code after the next
                available code is because compression is one step ahead of
                decompression; decompression needs to read one extra code to add
                a word that compression would have already added. To account
                for this, decompression must shift code lengths one step in
                advance so that the codes that are read match the ones what were
                written.
            \item Check if the current code, the code you just read using
                \texttt{next\_code()}, already exists in the word table.
            \item If this is the first code read or the table was just reset,
                buffer the current word that the code decodes to (it should
                already be in your table). Then set the previous word to be the
                current word and move on to reading in the next code. Make sure
                to set the reset flag to false.
            \item If the current code already exists in the table, and
                isn't the first code read, then copy the current code's word
                into your current word buffer. You will create and add a new
                entry, a \texttt{Word}, to the table. The index in the table of
                the new entry is the next available code. The word for the new
                entry is the previous word appended with the first character of
                the current word. The previous word can be found in the table
                using the previous code as the index. After adding the new
                entry, buffer the current word and remember to increment the
                next available code counter.
            \item Else if it doesn't exist, you must construct and add the
                missing code entry to the table. Set the current word to be the
                previous word appended
                with the first character of the previous word. The index in the
                table of this missing entry is given by the next available code
                (again, remember to increment the counter). You can find the
                previous word in the table from the previous code. Add the
                current word to your buffer designated for writing with
                \texttt{buffer\_word()}. This is a rare edge-case for the
                LZW decompression algorithm and can only occur when the word
                the missing code entry corresponds to starts and ends with the
                same character.
            \item Set the previous code to be the current code.
            \item You will need to reset the table like you reset the trie in
                compression. The table is reset when the next available code
                reaches \texttt{UINT16\_MAX - 1}. Remember that decompression is
                one step behind compression, and thus to account for the reset
                performed by compression, decompression has to reset the table
                one step earlier. The table, once reset, should contain only the
                ASCII characters. Reset the next available code counter back to
                256, and set the reset flag to true.
            \item If the buffer for decoded words is filled at any
                point during the loop, write its contents out.
            \item Continue looping until \texttt{infile} is completely decoded.
        \end{enumerate}
    \item Pseudocode for the decompression loop:
\begin{lstlisting}
while (decoded_chars != header.file_size):
  bit_len = log2(next_avail_code + 1) + 1
  curr_code = next_code(bit_len)
  curr_entry = table[curr_code]

  if (decoded_chars == 0 or reset):
    buffer_word(curr_entry)
    prev_word = curr_char
    reset = false
  elif (curr_entry != NULL):
    curr_word = curr_entry.word
    prev_entry = table[prev_code]
    prev_word = prev_entry.word
    new_word = prev_word.append(curr_word[0])
    table[next_avail_code] = word_create(new_word)
    next_avail_code += 1
    buffer_word(curr_entry)
    decoded_chars += curr_entry.length
  else:
    prev_entry = table[prev_code]
    prev_word = prev_entry.word
    curr_word = prev_word.append(prev_word[0])
    missing_entry = word_create(curr_word)
    table[next_avail_code] = missing_entry
    next_avail_code += 1
    buffer_word(missing_entry)
    decoded_chars += missing_entry.length

  prev_code = curr_code

  if (next_avail_code == UINT16_MAX - 1):
    wt_reset()
    next_avail_code = 256
    reset = true
\end{lstlisting}
    \item Flush any remaining decoded characters in your buffer with
        \texttt{flush\_words()}.
    \item The number of decoded characters should exactly match the file size
        from the header if you decompressed \texttt{infile} correctly.
    \item Close \texttt{infile} and \texttt{outfile} with \texttt{close()}.
\end{enumerate}


\section{LZW Algorithm Pseudocode}

\subsection{Compression}
\begin{lstlisting}
write_header()
root = trie_create()
curr_node = root
next_avail_code = 256

while (encoded_chars != header.file_size):
  curr_char = next_char()
  next_node = trie_step(curr_node, curr_char)

  if (encoded_chars == 0 or next_node):
    curr_node = next_node
  else:
    bit_len = log2(next_avail_code) + 1
    buffer_code(curr_node.code, bit_len)
    curr_node.children[curr_char] = trie_node_create(next_avail_code)
    curr_node = root.children[curr_char]
    next_avail_code += 1

  encoded_chars += 1

  if (next_avail_code == UINT16_MAX):
    trie_reset()
    curr_node = root.children[curr_char]
    next_avail_code = 256

bit_len = log2(next_avail_code) + 1
buffer_code(curr_code, bit_len)
flush_codes()
\end{lstlisting}

\subsection{Decompression}
\begin{lstlisting}
read_header()
table = wt_create()
next_avail_code = 256
reset = false

while (decoded_chars != header.file_size):
  bit_len = log2(next_avail_code + 1) + 1
  curr_code = next_code(bit_len)
  curr_entry = table[curr_code]

  if (decoded_chars == 0 or reset):
    buffer_word(curr_entry)
    prev_word = curr_char
    reset = false
  elif (curr_entry != NULL):
    curr_word = curr_entry.word
    prev_entry = table[prev_code]
    prev_word = prev_entry.word
    new_word = prev_word.append(curr_word[0])
    table[next_avail_code] = word_create(new_word)
    next_avail_code += 1
    buffer_word(curr_entry)
    decoded_chars += curr_entry.length
  else:
    prev_entry = table[prev_code]
    prev_word = prev_entry.word
    curr_word = prev_word.append(prev_word[0])
    missing_entry = word_create(curr_word)
    table[next_avail_code] = missing_entry
    next_avail_code += 1
    buffer_word(missing_entry)
    decoded_chars += missing_entry.length

  prev_code = curr_code

  if (next_avail_code == UINT16_MAX - 1):
    wt_reset()
    next_avail_code = 256
    reset = true

flush_words()

\end{lstlisting}


\section{Deliverables}
You will need to turn in:

\begin{enumerate}
    \item \texttt{Makefile}:
    \begin{itemize}
        \item \texttt{CFLAGS=-Wall -Wextra -Werror -Wpedantic} must be included.
        \item \texttt{CC=clang} must be specified.
        \item \texttt{make clean} must remove all files that are compiler
            generated.
        \item \texttt{make infer} must run \texttt{infer} on your program.
            Complaints generated by \texttt{infer} must be either fixed or
            explained in your \texttt{README}.
        \item \texttt{make} should build your program, as should
            \texttt{make all}.
        \item Your program should have no memory leaks.
        \item The generated executable must be named \texttt{lzwcoder}.
    \end{itemize}
    \item Your program \emph{must} have the following source and header files:
    \begin{itemize}
        \item \texttt{trie.h}: the header file for the Trie ADT.
        \item \texttt{trie.c}: the source file for the Trie ADT.
        \item \texttt{word.h}: the header file for the Word ADT.
        \item \texttt{word.c}: the source file for the Word ADT.
        \item \texttt{io.h}: the header file for the I/O module.
        \item \texttt{io.c}: the source file for the I/O module.
        \item \texttt{endian.h}: the header file for the endianness module.
        \item \texttt{main.c}: the main source file for the program which should
            contain the implementations of the LZW compression and decompression
            algorithms.
    \end{itemize}

    The header files listed above will all be added to your GitLab repository.
    The \textbf{C}-files you will implement according to the functions defined
    in the header files. Remember, your program should be able to handle
    lossless compression/decompression for text \emph{and} binary files,
    as well as work on little endian and big endian systems.

    \item You may have other source and header files, but
        \emph{do not try to be overly clever}.
    \item \texttt{README.md}: This must be in markdown. This should
        describe how to use your program and Makefile. This also contains any
        explanations for complaints generated by \texttt{infer}.
    \item \texttt{DESIGN.pdf}: This must be a PDF. The design document should
        describe your design for your program with enough detail that a
        sufficiently knowledgeable programmer would be able to replicate your
        implementation. This does not mean copying your entire program in
        verbatim. You should instead describe how your program works with
        supporting pseudo-code.
    \item We will provide a compressed file called ``\emph{<CruzID>}.lzw'' which
        contains a unique, compressed checksum. Being able to decode this file
        is a good indication that your program works properly. That being said,
        it is \emph{highly recommended} that you test your program with other
        inputs, as the graders will test with other inputs as well. Your program
        isn't written to pass a test; it is written to \emph{work}.
\end{enumerate}


\section{Submission}

To submit your assignment, refer back to \texttt{assignment0} for the steps on
how to submit your assignment through \texttt{git}. Remember:
\emph{add, commit,} and \emph{push}!

\textcolor{red}{Your assignment is turned in \emph{only} after you have pushed.
If you forget to push, you have not turned in your assignment and you will get
a \emph{zero}. ``I forgot to push'' is not a valid excuse. It is \emph{highly}
recommended to commit and push your changes \emph{often}.}

\end{document}
