\documentclass[11pt]{article}
\usepackage{fullpage,fourier,amsmath,amssymb}
\usepackage{listings,color,url,hyperref}
\usepackage{epigraph}
\usepackage{graphicx}
\usepackage[x11names]{xcolor}
\usepackage[most]{tcolorbox}

\title{Assignment 0 \\ \texttt{git}'n Started}
\author{Prof. Darrell Long \\ CSE 13S -- Winter 2021}
\date{Due: January 10$^\text{th}$ at 11:59\,pm}

\usepackage{fancyhdr}
\pagestyle{fancy}
\fancyhf{}

\fancypagestyle{plain}{%
  \fancyhf{}
  \renewcommand{\headrulewidth}{0pt}
  \renewcommand{\footrulewidth}{0pt}
  \lfoot{\textcopyright{} 2021 Darrell Long}
  \rfoot{\thepage}
}

\pagestyle{plain}

\definecolor{codegreen}{rgb}{0,0.5,0}
\definecolor{codegray}{rgb}{0.5,0.5,0.5}
\definecolor{codepurple}{rgb}{0.58,0,0.82}

\lstloadlanguages{C,make,python,fortran,bash}

\lstdefinestyle{bashstyle}{
  language=bash,
  basicstyle=\small\ttfamily,
  numbers=none,
  frame=tb,
  columns=fullflexible,
  backgroundcolor=\color{yellow!10},
  linewidth=0.9\linewidth,
  xleftmargin=0.1\linewidth,
  aboveskip=1.25em,
  belowskip=1.25em,
  breakatwhitespace=false,
  breaklines=true,
  showstringspaces=false,
  keepspaces=true,
  showspaces=false,
  showtabs=false,
  tabsize=4
}

\lstdefinestyle{plainstyle}{
  keywords={},
  stringstyle=\color{black},
  basicstyle=\small\ttfamily,
  numbers=none,
  frame=tb,
  columns=fullflexible,
  backgroundcolor=\color{yellow!10},
  linewidth=0.9\linewidth,
  xleftmargin=0.1\linewidth,
  aboveskip=1.25em,
  belowskip=1.25em,
  breakatwhitespace=false,
  breaklines=true,
  showstringspaces=false,
  keepspaces=true,
  showspaces=false,
  showtabs=false,
  tabsize=4
}

\lstdefinestyle{c99}{
    language=C,
    morekeywords={
      bool, uint8_t, uint16_t, uint32_t, uint64_t,
      int8_t, int16_t, int32_t, int64_t
    },
    upquote=true,
    commentstyle=\color{codegreen},
    keywordstyle=\color{magenta},
    identifierstyle=\color{black},
    stringstyle=\color{codepurple},
    basicstyle=\ttfamily,
    breakatwhitespace=false,
    breaklines=true,
    captionpos=b,
    keepspaces=true,
    showspaces=false,
    showstringspaces=false,
    showtabs=false,
    numbers=left,
    numbersep=7pt,
    numberstyle=\ttfamily\color{black!25},
    xleftmargin=12pt,
    tabsize=4
}

\lstdefinestyle{python}{
    language=python,
    upquote=true,
    commentstyle=\color{codegreen},
    keywordstyle=\color{magenta},
    identifierstyle=\color{black},
    stringstyle=\color{codepurple},
    basicstyle=\ttfamily,
    breakatwhitespace=false,
    breaklines=true,
    captionpos=b,
    keepspaces=true,
    showspaces=false,
    showstringspaces=false,
    showtabs=false,
    numbers=left,
    numbersep=7pt,
    numberstyle=\ttfamily\color{black!25},
    xleftmargin=12pt,
    tabsize=4
}

\usepackage[most]{tcolorbox}

\newtcolorbox{prelab}[1]{
  colback=red!5!white, colframe=red!75!black, title=#1
}

\newtcblisting{codelisting}[2][]{
  title=#2, #1,
  fonttitle=\bfseries,
  colframe=black!80,
  listing only,
  listing options={basicstyle=\ttfamily,style=c99},
  top=-5pt,
  bottom=-5pt,
  enlarge top by=10pt
}

\newtcblisting{pythonlisting}[2][]{
  title=#2, #1,
  fonttitle=\bfseries,
  colframe=black!80,
  listing only,
  listing options={basicstyle=\ttfamily,style=python},
  top=-5pt,
  bottom=-5pt,
  enlarge top by=5pt
}

\newcommand\Warning{%
 \makebox[1.4em][c]{%
 \makebox[0pt][c]{\raisebox{.1em}{\small!}}%
 \makebox[0pt][c]{\color{red}\Large$\bigtriangleup$}}}%


\newcommand\asgn[0]{asgn0}

\begin{document}\maketitle

\section{Introduction}

\epigraphwidth=0.75\textwidth
\epigraph{\emph{It's nice if people can finally loosen up a little bit
and just go out and laugh at silliness. I mean, people take themselves
way too seriously sometimes.}  \texttt{git}\emph{'er done!}}{---Larry the
Cable Guy}

\noindent The aim of this first assignment will be for you to set up your
GitLab repositories and gain an understanding of how \texttt{git} works.
We will review several \texttt{git} commands that you will help you in the long
run. This document will be helpful for troubleshooting \texttt{git} issues in
the future and also includes the submission policy. You will find this quite
helpful in the future if you ever have any issues with \texttt{git} or
submitting. \emph{Ideally,} this assignment should be completed during your
discussion section.

\section{GitLab}

\epigraph{\emph{Dire Straits is a great band. Someone tells you they like
`Brothers in Arms' and immediately you know they're a stupid annoying}
\texttt{git}.}{---Alexei Sayle}

\noindent The deliverables for each of your assignments will be maintained
through your GitLab repository. GitLab is a service coupled with the
version-control capabilities of \texttt{git}. \texttt{git} allows you to
maintain multiple versions of your source files, also known as version control.
Version control is the practice of tracking and making changes to code, such
that in the event of some accident while coding, it is always possible to
restore your code to a previous state. \texttt{git} is used
through a set of commands within a repository, a version-controlled directory
that stores your files.

\subsection{Setting Up SSH Keys}

You will first need to \emph{clone} your GitLab repository. It is highly
recommended that you use \texttt{git} over SSH rather than HTTP.  SSH causes
GitLab to use secure communication between \texttt{git} and its servers using
SSH keys, and that using the SSH protocol allows for authentication
\emph{without} the need to enter your username or password each time.

SSH keys come in \emph{pairs}: a \emph{private} key and a \emph{public} key.
Data encrypted with some public key can only be decrypted with the corresponding
private key and vice versa; public key cryptography. To generate an SSH keypair:

\begin{lstlisting}[style=bashstyle]
  $ ssh-keygen
\end{lstlisting}

The SSH keypair generated by the default prompt answers will be sufficient for
your needs for use with GitLab. Make sure you add the \emph{public} key of the
generated key pair to GitLab and that it is an RSA key. To print your public key
so that you can copy it to your clipboard, enter the following:

\begin{lstlisting}[style=bashstyle]
  $ cat ~/.ssh/id_rsa.pub
\end{lstlisting}

After adding the key, you will be ready to clone your GitLab repository. For
more in-depth instructions on generating and adding SSH keys, as well as other
GitLab basics, please refer to this link:

\centerline{\url{https://git.ucsc.edu/help/gitlab-basics/README.md}}

\subsection{Cloning Your Repository}

To clone your repository, run the following command, substituting
\texttt{<CruzID>} with your CruzID:

\begin{lstlisting}[style=bashstyle]
  $ git clone git@git.ucsc.edu:cse13s/winter2021/<CruzID>.git cse13s
\end{lstlisting}

You will be prompted for permission to authenticate with the server.
When permitted, the command will clone your repository onto your machine into a
directory named \texttt{cse13s} in the current working directory. Use the
\texttt{cd} command to enter the \texttt{asgn0} directory in your cloned
\texttt{cse13s} repository to start your work for assignment 0.

\begin{lstlisting}[style=bashstyle]
  $ cd cse13s/asgn0
\end{lstlisting}

\section{Hello World!}

You will be creating a simple \textbf{C} program which will simply print
``\texttt{Hello World!}'' You can find also find a tutorial of this program in
Chapter 1 \S 1.1 in your textbook, \textit{The C Programming Language} by
Kernighan \& Ritchie.

\begin{enumerate}
  \item Make sure you are in the correct directory: \texttt{asgn0}. You can
    check your \emph{current working directory} using this command:

\begin{lstlisting}[style=bashstyle]
  $ pwd
\end{lstlisting}

  \item Create the program source \texttt{hello.c} with your text editor of
    choice. This means text editors such as \texttt{vi} and \texttt{emacs}.
    Notepad and Word are \emph{not} text editors. To open up \texttt{hello.c}
    for editing with \texttt{vi}:

\begin{lstlisting}[style=bashstyle]
  $ vi hello.c
\end{lstlisting}

  \item Include the header for the \texttt{<stdio.h>} library. This is needed by
    the \texttt{printf()} function that prints formatted strings to
    \texttt{stdout}, what you think of as the console.

\begin{codelisting}{\texttt{hello.c}}
#include <stdio.h>
\end{codelisting}

  \item Type your \texttt{main()} function. Every \textbf{C} program \emph{must}
    have a \texttt{main()} function which returns an \texttt{int}. A return of 0
    indicates program success, and a non-zero return indicates the occurrence of
    some error.

\begin{codelisting}{\texttt{hello.c}}
#include <stdio.h>

int main(void) {
    return 0;
}
\end{codelisting}

  \item In \texttt{main()} (between the curly braces) is where you will type the
    print statement. It is \emph{crucial} that your print statement matches the
    one given here. \emph{You will be docked points otherwise.}

\begin{codelisting}{\texttt{hello.c}}
#include <stdio.h>

int main(void) {
    printf("Hello World!\n");
    return 0;
}
\end{codelisting}

  \item Save your work and exit your text editor to return to the command line.
    With \texttt{vi} this means entering normal mode by hitting \texttt{esc} and
    entering the \texttt{vi} command ``\texttt{:wq}'' to save and quit.

  \item You should now be back on the command line. You should now compile and
    run your code to verify its correctness. To compile your code, run:

\begin{lstlisting}[style=bashstyle]
  $ clang -Wall -Wextra -Werror -Wpedantic -o hello hello.c
\end{lstlisting}

    This will compile your code with the compiler flags required by the class.
    \texttt{clang} is the \textbf{C} compiler that we will be using --- not
    \texttt{gcc}, not \texttt{cc}. You \emph{must} use \texttt{clang}. The
    \texttt{-Wall -Wextra -Werror -Wpedantic} arguments are the set of compiler
    flags you must use when compiling your code. This specific set of compiler
    flags is commonly referred to as the ``take no prisoners'' compiler flags.
    Simply put, together they catch pretty much everything that a compiler can
    catch (there are a few more esoteric warnings that can be enabled). Here are
    some links for you to investigate what each flag does:

    \centerline{\url{https://releases.llvm.org/10.0.0/tools/clang/docs/UsersManual.html}}
    \centerline{\url{https://releases.llvm.org/10.0.0/tools/clang/docs/DiagnosticsReference.html}}

  \item If you've done everything correctly up to this point, the compilation
    process should run silently and return no errors. However, if you do run
    into any errors, lab sections, and Piazza will be your best friends. Resist
    the urge to immediately use Google.

  \item After successfully compiling your program, there should now be an
    executable file named \texttt{hello} in the current working directory. To
    list out all the files in the current working directory use \texttt{ls}:

\begin{lstlisting}[style=bashstyle]
  $ ls
\end{lstlisting}

    To run the \texttt{hello} program, enter:

\begin{lstlisting}[style=bashstyle]
  $ ./hello
\end{lstlisting}

  \item The \texttt{.} (usually called ``dot'') refers to the \emph{current
    working directory.} Your shell has a \texttt{PATH} environment variable, a
    colon-delimited list of directories that it looks through when you enter a
    command. Since your current working directory is most likely not in your
    \texttt{PATH}, you must specify the directory that your program can be found
    in order to run it. If the output of running your program is correct, you
    should then submit your working \emph{source code} to \texttt{git}. You
    should submit source code \emph{only}: no executables.

\begin{lstlisting}[style=bashstyle]
  $ git add hello.c
  $ git commit -m "Adding finished hello.c"
  $ git push
\end{lstlisting}

    The above three commands will add, commit, and push \texttt{hello.c} to
    \texttt{git}. In-depth description of each of these commands will be
    provided in the following section. To verify that \texttt{hello.c} was
    added, check your repository:

    \centerline{\url{https://git.ucsc.edu/gitlab/cse13s/winter2021/<CruzID>/asgn0}}

    Only in this case do you perform one commit at the end. In general, you
    should commit after every significant change.

  \item The only other file to be submitted for assignment 0 is your signed
    \texttt{CHEATING.pdf}. This file can be found on Canvas and/or under Piazza
    resources. \textcolor{red}{Note:} \textcolor{violet}{You do not create a
    PDF file by simply appending \texttt{.pdf} to its name.} You will submit
    \texttt{CHEATING.pdf} the same way you did \texttt{hello.c}: adding,
    committing, then pushing.
\end{enumerate}

\section{Git}

The following commands are used through \texttt{git} for version control. For
this assignment, you will have used the \texttt{clone}, \texttt{add}, and
\texttt{push} commands. This section will serve as a brief description and use
of frequently used \texttt{git} commands that you will most likely use
throughout the quarter, if not your entire career as a computing professional.

\subsection{\texttt{git clone}}

This command clones a repository from a server onto your local machine. This
downloads a copy of the repository which is stored on a server for local
editing. Meaning, any changes that need to be sent back to the server will need
to be \emph{added}, \emph{committed} and \emph{pushed}. Here is an example of
cloning over \texttt{ssh}:

\begin{lstlisting}[style=bashstyle]
  $ git clone user@somemachine:path/to/repo
\end{lstlisting}

\subsection{\texttt{git add}}

This command allows you to add files into your repository and stages them to
the \texttt{git} source tree. Any file that has been changed since the time it was last
added needs to be added again.

\begin{lstlisting}[style=bashstyle]
  $ git add file1 file2
\end{lstlisting}

Keep in mind, adding files with this command does \emph{not} commit them. You
still need to commit the changes with the \texttt{git commit} command.

\subsection{\texttt{git commit}}

This command creates a checkpoint for each file which was added using the
previous command, \texttt{git add}. You can think of it like capturing a
snapshot of the current staged changes. These snapshots are then safely
committed. Each commit has an unique commit ID along with a message about the
commit.

\begin{lstlisting}[style=bashstyle]
  $ git commit -m "A short informative message about any changes"
\end{lstlisting}

To commit all the changed files, you can use the command \texttt{git commit -a}
which can also be combined with the \texttt{-m} option. This will only commit
files that have been added and committed at least once before. Without the
\texttt{-m} flag, you will be prompted into an editor to enter your commit
message. A forewarning: don't commit rude comments -- the TA's will see them.

You should commit working versions of your code frequently so in the case you
mess something up, like accidentally deleting your code, you can use \texttt{git
checkout HEAD} to revert to the most recent commit.

\subsection{\texttt{git checkout}}

This command allows you to navigate between branches created by \texttt{git}
branch. It can help you undo changes in the case you mess up and come to the
rescue. Checking out a branch is similar to checking out old commits. The files
in the current working directory is updated to match the selected branch or
commit ID. It also tells \texttt{git} to store all new commits on that branch.
You can display the history of commits you can check out to with \texttt{git
log}.

\begin{lstlisting}[style=bashstyle]
  $ git checkout <branch>
\end{lstlisting}


\subsection{\texttt{git log}}

This command provides a list of the commits that have been made on the
repository. It provides access to look up commit times, messages, and IDs.

\begin{lstlisting}[style=bashstyle]
  $ git log
\end{lstlisting}


\subsection{\texttt{git push}}

This command pushes all of your local commits to the upstream repository. It
pushes all of your changes to the directory which is stored on-line. You
\emph{must} do this to turn in your work for this class. If you do not run this
command after committing, \emph{none} of your work will be turned in.

\begin{lstlisting}[style=bashstyle]
  $ git push
\end{lstlisting}

\subsection{\texttt{git pull}}

This command fetches and downloads content from a remote repository. Your local
repository is immediately updated to match the fetched content. \texttt{git
pull} is actually a combination of \texttt{git fetch} followed by \texttt{git
merge}. The first half of \texttt{git pull} will execute \texttt{git fetch} on
the local branch that HEAD is pointed at. After the contents are fetched, the
second half of \texttt{git pull} will merge the work-flow creating a new merge
commit ID and HEAD is updated to point to the new commit.

\begin{lstlisting}[style=bashstyle]
  $ git pull
\end{lstlisting}

\subsection{\texttt{git ls-files}}

This command lists all files in the current directory that have been checked
into the repository. This will be useful for making sure you have submitted all
required deliverables for each assignment.

\begin{lstlisting}[style=bashstyle]
  $ git ls-files
\end{lstlisting}

\subsection{\texttt{git status}}

This command provides a status of which files have been added and staged for the
next commit, as well as unpushed changes.

\begin{lstlisting}[style=bashstyle]
  $ git status
\end{lstlisting}


\section{Honesty}

Academic honesty is very important in computer science, and life in general. The
goal of this course is for you to learn the material, not simply for you to get
a mark on your transcript saying you passed the class. All students in the class
must sign and turn in an acknowledgment that they understand the cheating policy
for the class. We will not accept or grade any assignments from a student unless
they have turned in the \texttt{CHEATING.pdf}. We encourage you to ask for
clarifications in the academic policy if you have any questions.

\section{Deliverables}

\epigraph{\emph{If there was no Black Sabbath, I could still possibly be a
morning newspaper delivery boy. No fun.}}{---Lars Ulrich}

\noindent For this class, you will be turning in all of your work through
\texttt{git}. All the files you need to turn in for an assignment will be found
and listed in the Deliverables section of every assignment PDF. Files will need
to be added to the corresponding assignment directory, committed, and pushed.

You will need to turn in:
\begin{enumerate}
	\item \texttt{\asgn/CHEATING.pdf}
	\item \texttt{\asgn/hello.c}
\end{enumerate}

\section{Submission}

\epigraph{\emph{The cost of freedom is always high, but Americans have always
paid it. And one path we shall never choose, and that is the path of
surrender, or submission.}}{---John F.\ Kennedy}\noindent

\noindent Now that you have learned about some useful \texttt{git} commands,
it's time to put them to use. The steps to submitting assignments will not
change throughout the course. If you ever forget the steps, refer back to this
PDF. Remember: \emph{add, commit,} and \emph{push}! In the case you do mess
something up, \emph{don't panic.} Take a step back and think things throughly.
The Internet, TAs and tutors are here as resources.

\begin{enumerate}
  \item Add it!

\begin{lstlisting}[style=bashstyle]
  $ git add CHEATING.pdf hello.c
\end{lstlisting}

    As mentioned before, you will need to first add the files to
    your repository using the \texttt{git add <filenames>} command. You
    will be submitting these files into the \texttt{\asgn}
    directory.

  \item Commit it!

\begin{lstlisting}[style=bashstyle]
  $ git commit -m "Your commit message here"
\end{lstlisting}

  Changes to these files will be committed to the repository with \texttt{git
  commit}. The command should also include a commit message describing what
  changes are included in the commit. For your final and last commit for
  submission, your commit message should be ``final submission''.

  \item Push it!

\begin{lstlisting}[style=bashstyle]
  $ git push
\end{lstlisting}

  The committed changes are then sync'd up with the remote server
  using the \texttt{git push} command. You must be sure to push your
  changes to the remote server or else they will not be received by
  the graders.

  \textcolor{red}{Your assignment is turned in \emph{only} after you have
  pushed. If you forget to push, you have not turned in your assignment and you
  will get a \emph{zero}. ``I forgot to push'' is not a valid excuse. It is
  \emph{highly} recommended to commit and push your changes \emph{often}.}
\end{enumerate}

\section{Supplemental Readings}

\epigraph{\emph{The more that you read, the more things you will know. The
more that you learn, the more places you'll go.}}{---Dr.\ Seuss}\noindent

\begin{itemize}

  \item \textcolor{red}{\textit{Version Control with Git} by Loeliger \&
    McCullough} $\leftarrow$ Read this! Now!
    \begin{itemize}
      \item Chapter 3 -- Getting Started (pg. 22--25)
    \end{itemize}

  \item \textit{The C Programming Language} by Kernighan \& Ritchie $\leftarrow$
    It is a \emph{huge} mistake to not read this!
    \begin{itemize}
      \item Chapter 1 \S 1.1
    \end{itemize}

	\item \textit{vi and Vim Editors} by Robbins \& Lamb
    \begin{itemize}
      \item Chapter 1 \S 1.4 \& \S 1.5
    \end{itemize}

\end{itemize}

\end{document}



