\documentclass[11pt]{article}

\usepackage{fullpage,fourier,amsmath,amssymb}
\usepackage{listings,color,url,hyperref}
\usepackage{mdframed}
\usepackage{epigraph}
\usepackage[x11names]{xcolor}
\usepackage{gensymb}
\usepackage{forest}
\usepackage{tikz}
\usepackage{xeCJK}
\usetikzlibrary{positioning,matrix}
\setmainfont{heuristica}[Scale=0.9]

\title{Assignment 3 \\ Sorting: Putting your affairs in order}
\author{Prof. Darrell Long \\ CSE 13S -- Spring 2021}
\date{Due: April 25$^\text{th}$ at 11:59\,pm}

\usepackage{fancyhdr}
\pagestyle{fancy}
\fancyhf{}

\fancypagestyle{plain}{%
  \fancyhf{}
  \renewcommand{\headrulewidth}{0pt}
  \renewcommand{\footrulewidth}{0pt}
  \lfoot{\textcopyright{} 2021 Darrell Long}
  \rfoot{\thepage}
}

\pagestyle{plain}

\definecolor{codegreen}{rgb}{0,0.5,0}
\definecolor{codegray}{rgb}{0.5,0.5,0.5}
\definecolor{codepurple}{rgb}{0.58,0,0.82}

\lstloadlanguages{C,make,python,fortran,bash}

\lstdefinestyle{bashstyle}{
  language=bash,
  basicstyle=\small\ttfamily,
  numbers=none,
  frame=tb,
  columns=fullflexible,
  backgroundcolor=\color{yellow!10},
  linewidth=0.9\linewidth,
  xleftmargin=0.1\linewidth,
  aboveskip=1.25em,
  belowskip=1.25em,
  breakatwhitespace=false,
  breaklines=true,
  showstringspaces=false,
  keepspaces=true,
  showspaces=false,
  showtabs=false,
  tabsize=4
}

\lstdefinestyle{plainstyle}{
  keywords={},
  stringstyle=\color{black},
  basicstyle=\small\ttfamily,
  numbers=none,
  frame=tb,
  columns=fullflexible,
  backgroundcolor=\color{yellow!10},
  linewidth=0.9\linewidth,
  xleftmargin=0.1\linewidth,
  aboveskip=1.25em,
  belowskip=1.25em,
  breakatwhitespace=false,
  breaklines=true,
  showstringspaces=false,
  keepspaces=true,
  showspaces=false,
  showtabs=false,
  tabsize=4
}

\lstdefinestyle{c99}{
    language=C,
    morekeywords={
      bool, uint8_t, uint16_t, uint32_t, uint64_t,
      int8_t, int16_t, int32_t, int64_t
    },
    upquote=true,
    commentstyle=\color{codegreen},
    keywordstyle=\color{magenta},
    identifierstyle=\color{black},
    stringstyle=\color{codepurple},
    basicstyle=\ttfamily,
    breakatwhitespace=false,
    breaklines=true,
    captionpos=b,
    keepspaces=true,
    showspaces=false,
    showstringspaces=false,
    showtabs=false,
    numbers=left,
    numbersep=7pt,
    numberstyle=\ttfamily\color{black!25},
    xleftmargin=12pt,
    tabsize=4
}

\lstdefinestyle{python}{
    language=python,
    upquote=true,
    commentstyle=\color{codegreen},
    keywordstyle=\color{magenta},
    identifierstyle=\color{black},
    stringstyle=\color{codepurple},
    basicstyle=\ttfamily,
    breakatwhitespace=false,
    breaklines=true,
    captionpos=b,
    keepspaces=true,
    showspaces=false,
    showstringspaces=false,
    showtabs=false,
    numbers=left,
    numbersep=7pt,
    numberstyle=\ttfamily\color{black!25},
    xleftmargin=12pt,
    tabsize=4
}

\usepackage[most]{tcolorbox}

\newtcolorbox{prelab}[1]{
  colback=red!5!white, colframe=red!75!black, title=#1
}

\newtcblisting{codelisting}[2][]{
  title=#2, #1,
  fonttitle=\bfseries,
  colframe=black!80,
  listing only,
  listing options={basicstyle=\ttfamily,style=c99},
  top=-5pt,
  bottom=-5pt,
  enlarge top by=10pt
}

\newtcblisting{pythonlisting}[2][]{
  title=#2, #1,
  fonttitle=\bfseries,
  colframe=black!80,
  listing only,
  listing options={basicstyle=\ttfamily,style=python},
  top=-5pt,
  bottom=-5pt,
  enlarge top by=5pt
}

\newcommand\Warning{%
 \makebox[1.4em][c]{%
 \makebox[0pt][c]{\raisebox{.1em}{\small!}}%
 \makebox[0pt][c]{\color{red}\Large$\bigtriangleup$}}}%


\begin{document}\maketitle

\epigraphwidth=0.65\textwidth
\epigraph{\emph{Any inaccuracies in this index may be explained by the
fact that it has been sorted with the help of a computer.}}{---Donald
Knuth, Vol.  III, \emph{Sorting and Searching}}


\section{Introduction}

Putting items into a sorted order is one of the most common tasks in
Computer Science. As a result, there are a myriad of library routines
that will do this task for you, but that does not absolve you of the
obligation of understanding how it is done. In fact it behooves you to
understand the various algorithms in order to make wise choices.

The best execution time that can be accomplished, also referred to as
the \emph{lower bound}, for sorting using \emph{comparisons} is
$\Omega(n \log n)$, where $n$ is the number is elements to be sorted. If
the universe of elements to be sorted is small, then we can do better
using a \emph{Count Sort} or a \emph{Radix Sort} both of which have a
time complexity of $O(n)$. The idea of \emph{Count Sort} is to count the
number of occurrences of each element in an array. For \emph{Radix
Sort}, a digit by digit sort is done by starting from the least
significant digit to the most significant digit. It may also use
\emph{Count Sort} as a subroutine.

What is this $O$ and $\Omega$ stuff? It's how we talk about the
execution time (or space used) by a program. We will discuss it in
lecture and in sections, and you will see it again in your Data
Structures and Algorithms class, now named CSE 101.

The sorting algorithms that you are expected to implement are Bubble
Sort, Shell Sort, and two Quicksorts. The purpose of this assignment is
to get you fully familiarized with each sorting algorithm.
\textcolor{red}{They are well-known sorts (save for the latter
  Quicksort). You can use the Python pseudocode provided to you as
  guidelines. Do not get the code for the sorts from the Internet or you
  will be referred to for cheating. We will be running plagiarism
checkers.}


\section{Bubble Sort}

\epigraph{\emph{\textbf{C} is peculiar in a lot of ways, but it, like
many other successful things, has a certain unity of approach that stems
from development in a small group.}}{---Dennis Ritchie}

\noindent Bubble Sort works by examining adjacent pairs of items, call
them $i$ and $j$. If item $j$ is smaller than item $i$, swap them. As a
result, the largest element \emph{bubbles} up to the top of the array in
a single pass. Since it is in fact the largest, we do not need to
consider it again. So in the next pass, we only need to consider $n-1$
pairs of items. The first pass requires $n$ pairs to be examined; the
second pass, $n-1$ pairs; the third pass $n-2$ pairs, and so forth. If
you can pass over the entire array and no pairs are out of order, then
the array is sorted.

\medskip
\begin{prelab}{Pre-lab Part 1}
  \begin{enumerate}
    \item How many rounds of swapping will need to sort the numbers ${8,
      22, 7, 9, 31, 5, 13}$ in ascending order using Bubble Sort?
    \item How many comparisons can we expect to see in the worse case
      scenario for Bubble Sort? Hint: make a list of numbers and attempt
      to sort them using Bubble Sort.
  \end{enumerate}
\end{prelab}

In 1784, when Carl Friedrich Gau{\ss} was only 7 years old, he was
reported to have amazed his elementary school teacher by how quickly he
summed up the integers from $1$ to $100$. The precocious little
Gau{\ss} produced the correct answer immediately after he quickly
observed that the sum was actually 50 pairs of numbers, with each pair
summing to 101 totaling to 5,050. We can then see that:

\[
  n+(n-1)+(n-2) + \ldots + 1 = \frac{n(n+1)}{2}
\]

\noindent so the \emph{worst case} time complexity is $O(n^2)$. However, it could
be much better if the list is already sorted. If you haven't seen the
inductive proof for this yet, you will in the discrete
mathematics class, CSE 16.

\begin{pythonlisting}{Bubble Sort in Python}
def bubble_sort(arr):
    n = len(arr)
    swapped = True
    while swapped:
        swapped = False
        for i in range(1, n):
            if arr[i] < arr[i - 1]:
                arr[i], arr[i - 1] = arr[i - 1], arr[i]
                swapped = True
        n -= 1
\end{pythonlisting}


\section{Shell Sort}

\epigraph{\emph{There are two ways of constructing a software design.
    One way is to make it so simple that there are obviously no
    deficiencies, and the other way is to make it so complicated that
    there are no obvious deficiencies. The first method is far more
difficult.}}{---C.A.R. Hoare}

\noindent
Donald L. Shell (March 1, 1924--November 2, 2015) was an American
computer scientist who designed the Shell sort sorting algorithm.
He earned his Ph.D. in Mathematics from the University of Cincinnati
in 1959, and published the Shell Sort algorithm in the Communications
of the ACM in July that same year.

Shell Sort is a variation of insertion sort, which sorts pairs of
elements which are far apart from each other. The \emph{gap} between the
compared items being sorted is continuously reduced. Shell Sort starts
with distant elements and moves out-of-place elements into position
faster than a simple nearest neighbor exchange. What is the expected
time complexity of Shell Sort? It depends entirely upon the gap
sequence.

The following is the pseudocode for Shell Sort. The gap sequence is
represented by the array \texttt{gaps}. You will be given a gap
sequence, the Pratt sequence ($2^p 3^q$ also called $3$-smooth), in the
header file \texttt{gaps.h}.  For each \texttt{gap} in the gap sequence,
the function compares all the pairs in \texttt{arr} that are
\texttt{gap} indices away from each other. Pairs are swapped if they are
in the wrong order.

\begin{pythonlisting}{Shell Sort in Python}
def shell_sort(arr):
    for gap in gaps:
        for i in range(gap, len(arr)):
            j = i
            temp = arr[i]
            while j >= gap and temp < arr[j - gap]:
                arr[j] = arr[j - gap]
                j -= gap
            arr[j] = temp
\end{pythonlisting}

\medskip
\begin{prelab}{Pre-lab Part 2}
  \begin{enumerate}
    \item The worst time complexity for Shell Sort depends on the
      sequence of gaps. Investigate why this is the case. How can you
      improve the time complexity of this sort by changing the gap size?
      Cite any sources you used.
  \end{enumerate}
\end{prelab}


\section{Quicksort}

\epigraph{\emph{If debugging is the process of removing software bugs,
then programming must be the process of putting them in.}}{---Edsger
Dijkstra}

\noindent
Quicksort (sometimes called partition-exchange sort) was developed by
British computer scientist C.A.R. ``Tony'' Hoare in 1959 and published
in 1961. It is perhaps the most commonly used algorithm for sorting (by
competent programmers).  When implemented well, it is the fastest known
algorithm that sorts using \emph{comparisons}. It is usually two or
three times faster than its main competitors, Merge Sort and Heapsort.
It does, though, have a worst case performance of $O(n^2)$ while its
competitors are strictly $O(n \log n)$ in their worst case.

Quicksort is a divide-and-conquer algorithm. It partitions
arrays into two sub-arrays by selecting an element from the array and
designating it as a pivot. Elements in the array that are less than the
pivot go to the left sub-array, and elements in the array that are
greater than or equal to the pivot go to the right sub-array.

Note that Quicksort is an \emph{in-place} algorithm, meaning it doesn't
allocate additional memory for sub-arrays to hold partitioned elements.
Instead, Quicksort utilizes a subroutine called \texttt{partition()}
that places elements less than the pivot into the left side of the array
and elements greater than or equal to the pivot into the right side and
returns the index that indicates the division between the partitioned
parts of the array. In a recursive implementation, Quicksort is then
applied recursively on the partitioned parts of the array, thereby
sorting each array partition containing at least one element.

Instead of a recursive Quicksort, you will be writing an
two \emph{iterative} Quicksorts: one that utilizes a \emph{stack} and
one that utilizes a \emph{queue}. In the implementation using a stack,
two indices are pushed into the stack at a time. The first and second
indices respectively represent the leftmost and rightmost indices of the
array partition to sort.  Similarly, in the implementation using a
queue, two indices are enqueued at a time, where the first and second
indices respectively represent the leftmost and rightmost indices of the
array partition to sort. \textcolor{red}{Hint: the return type of
\texttt{partition()} should be \texttt{int64\_t}.}

\begin{pythonlisting}{Partition in Python}
def partition(arr, lo, hi):
    pivot = arr[lo + ((hi - lo) // 2)]; # Prevent overflow.
    i = lo - 1
    j = hi + 1
    while i < j:
        i += 1 # You may want to use a do-while loop.
        while arr[i] < pivot:
            i += 1
        j -= 1
        while arr[j] > pivot:
            j -= 1
        if i < j:
            arr[i], arr[j] = arr[j], arr[i]
    return j
\end{pythonlisting}

\begin{pythonlisting}{Quicksort in Python with a stack}
def quick_sort_stack(arr):
    lo = 0
    hi = len(arr) - 1
    stack = []
    stack.append(lo) # Pushes to the top.
    stack.append(hi)
    while len(stack) != 0:
        hi = stack.pop() # Pops off the top.
        lo = stack.pop()
        p = partition(arr, lo, hi)
        if lo < p:
            stack.append(lo)
            stack.append(p)
        if hi > p + 1:
            stack.append(p + 1)
            stack.append(hi)
\end{pythonlisting}

\begin{pythonlisting}{Quicksort in Python with a queue}
def quick_sort_queue(arr):
    lo = 0
    hi = len(arr) - 1
    queue = []
    queue.append(lo) # Enqueues at the head.
    queue.append(hi)
    while len(queue) != 0:
        lo = queue.pop(0) # Dequeues from the tail.
        hi = queue.pop(0)
        p = partition(arr, lo, hi)
        if lo < p:
            queue.append(lo)
            queue.append(p)
        if hi > p + 1:
            queue.append(p + 1)
            queue.append(hi)
\end{pythonlisting}

\medskip
\begin{prelab}{Pre-lab Part 3}
  \begin{enumerate}
    \item Quicksort, with a worse case time complexity of
      $\operatorname{O}(n^2)$, doesn't seem to live up to its name.
      Investigate and explain why Quicksort isn't doomed by its worst
      case scenario. Make sure to cite any sources you use.
  \end{enumerate}
\end{prelab}

\subsection{Stacks}

You will need to implement the \emph{stack} ADT for the first of your
two iterative Quicksorts. An \textbf{ADT} is an \emph{abstract data
type}. With any ADT comes an interface comprised of \emph{constructor},
\emph{destructor}, \emph{accessor}, and \emph{manipulator} functions.
The following subsections define the interface for a stack. A stack is
an ADT that implements a \emph{last-in, first-out}, or \textbf{LIFO},
policy. Consider a stack of pancakes. A pancake can only be placed
(\emph{pushed}) on top of the stack and can only be removed
(\emph{popped}) from the top of the stack. The header file containing
the interface will be given to you as \texttt{stack.h}.
\textcolor{red}{You may not modify this file.} If you borrow code from
any place --- including Prof. Long --- you must cite it. It is \emph{far
better} if you write it yourself.

\subsubsection{\texttt{Stack}}

The stack datatype will be abstracted as a \texttt{struct} called
\texttt{Stack}. We will use a \texttt{typedef} to construct a new type,
which you should treat as opaque --- which means that you cannot
manipulate it directly. We will \emph{declare} the \texttt{stack} type
in \texttt{stack.h} and you will define its concrete implementation in
\texttt{stack.c}. \textcolor{red}{This means the following \texttt{struct} definition
\emph{must} be in \texttt{stack.c}}.

\begin{codelisting}{}
struct Stack {
  uint32_t top;       // Index of the next empty slot.
  uint32_t capacity;  // Number of items that can be pushed.
  int64_t *items;     // Array of items, each with type int64_t.
};
\end{codelisting}

\subsubsection{\texttt{Stack *stack\_create(uint32\_t capacity)}}

The constructor function for a \texttt{Stack}. The \texttt{top} of a
stack should be initialized to 0. The capacity of a stack is set to the
specified capacity. The specified capacity also indicates the number of
items to allocate memory for, the items in which are held in the
dynamically allocated array \texttt{items}. A working constructor for a
stack is provided below. Note that it uses \texttt{malloc()} and
\texttt{calloc()} for \emph{dynamic memory allocation}. Both these
functions are included from \texttt{<stdlib.h>}.

\begin{codelisting}{}
Stack *stack_create(uint32_t capacity) {
  Stack *s = (Stack *) malloc(sizeof(Stack));
  if (s) {
    s->top = 0;
    s->capacity = capacity;
    s->items = (int64_t *) calloc(capacity, sizeof(int64_t));
    if (!s->items) {
      free(s);
      s = NULL;
    }
  }
  return s;
}
\end{codelisting}

\subsubsection{\texttt{void stack\_delete(Stack **s)}}

The destructor function for a stack. A working destructor for a stack is
provided below. Any memory that is allocated using one of
\texttt{malloc()}, \texttt{realloc()}, or \texttt{calloc()} \emph{must}
be freed using the \texttt{free()} function. The job of the destructor
is to free all the memory allocated by the constructor.
\textcolor{red}{Your programs are expected to be free of memory leaks.}
A pointer to a pointer is used as the parameter because we want to avoid
\emph{use-after-free} errors. A use-after-free error occurs when a
program uses a pointer that points to freed memory. To avoid this, we
pass the \emph{address of a pointer} to the destructor function. By
\emph{dereferencing} this double pointer, we can make sure that the
pointer that pointed to allocated memory is updated to be \texttt{NULL}.

\begin{codelisting}{}
void stack_delete(Stack **s) {
  if (*s && (*s)->items) {
    free((*s)->items);
    free(*s);
    *s = NULL;
  }
  return;
}

int main(void) {
  Stack *s = stack_create();
  stack_delete(&s);
  assert(s == NULL);
}
\end{codelisting}


\subsubsection{\texttt{bool stack\_empty(Stack *s)}}

Since we will be using \texttt{typedef} to create \emph{opaque} data
types, we need functions to access fields of a data type. These
functions are called \emph{accessor} functions. An opaque data type
means that users do not need to know its implementation outside of the
implementation itself. This also means that it is incorrect to write
\texttt{s->top} outside of \texttt{stack.c} since it violates opacity.
This accessor function returns \texttt{true} if the stack is empty and
\texttt{false} otherwise.

\subsubsection{\texttt{bool stack\_full(Stack *s)}}

Returns \texttt{true} if the stack is full and \texttt{false} otherwise.

\subsubsection{\texttt{uint32\_t stack\_size(Stack *s)}}

Returns the number of items in the stack.

\subsubsection{\texttt{bool stack\_push(Stack *s, int64\_t x)}}

The need for \emph{manipulator} functions follows the rationale behind
the need for accessor functions: there needs to be some way to alter
fields of a data type. \texttt{stack\_push()} is a manipulator function
that pushes an item \texttt{x} to the top of a stack.

This function returns a \texttt{bool} in order to signify either success
or failure when pushing onto a stack. When can pushing onto a stack result
in failure? \emph{When the stack is full.} If the stack is full prior to
pushing the item \texttt{x}, return \texttt{false} to indicate failure.
Otherwise, push the item and return \texttt{true} to indicate success.

\subsubsection{\texttt{bool stack\_pop(Stack *s, int64\_t *x)}}

This function pops an item off the specified stack, passing the value
of the popped item back through the pointer \texttt{x}. Like with
\texttt{stack\_push()}, this function returns a \texttt{bool} to
indicate either success or failure. When can popping a stack result in
failure? \emph{When the stack is empty.} If the stack is empty prior to
popping it, return \texttt{false} to indicate failure. Otherwise, pop
the item, set the value in the memory \texttt{x} is pointing to as the
popped item, and return \texttt{true} to indicate success.

\begin{codelisting}{}
// Dereferencing x to change the value it points to.
*x = s->items[s->top];
\end{codelisting}

\subsubsection{\texttt{void stack\_print(Stack *s)}}

This is a debug function that you should write early on. It will help
greatly in determining whether or not your stack implementation is
working correctly.


\subsection{Queues}

You will need to implement the \emph{queue} ADT for the second of your
two iterative Quicksorts. The following subsections define the interface
for a queue. The header file containing the interface will be given to
you as \texttt{queue.h}. \textcolor{red}{You may not modify this file.}
A queue is an ADT that implements a \emph{first-in, first-out}, or
\textbf{FIFO}, policy. Consider a checkout line. Customers are
\emph{dequeued} from the front (the \emph{head}) of the line and
\emph{enqueued} at the end (the \emph{tail}) of the line.

\subsubsection{\texttt{Queue}}

\textcolor{red}{The queue \texttt{struct} must be defined as follows in
\texttt{queue.c}:}

\begin{codelisting}{}
struct Queue {
  uint32_t head;      // Index of the head of the queue.
  uint32_t tail;      // Index of the tail of the queue.
  uint32_t size;      // The number of elements in the queue.
  uint32_t capacity;  // Capacity of the queue.
  int64_t *items;     // Holds the items.
}
\end{codelisting}

\subsubsection{\texttt{Queue *queue\_create(uint32\_t capacity)}}

The constructor for a \texttt{Queue}. The structure of this function is
very similar to the stack constructor function given in \S 4.1.2. The
\texttt{head} and \texttt{tail} of a queue should be initialized to 0.
The capacity of a queue is set to the specified capacity. The capacity
also indicates the number of items to allocate memory for, the items in
which are contained in the dynamically allocated array \texttt{items}.
The \texttt{size} of a queue tracks the number of enqueued items.
Return a pointer to the dynamically allocated \texttt{Queue}. If
\texttt{malloc()} or \texttt{calloc()} fail, return a \texttt{NULL}
pointer to indicate failure.

\subsubsection{\texttt{void queue\_delete(Queue **q)}}

The destructor for a \texttt{Queue}.

\subsubsection{\texttt{bool queue\_empty(Queue *q)}}

Returns \texttt{true} if the queue is empty and \texttt{false}
otherwise.

\subsubsection{\texttt{bool queue\_full(Queue *q)}}

Returns \texttt{true} if the queue is full and \texttt{false} otherwise.

\subsubsection{\texttt{uint32\_t queue\_size(Queue *q)}}

Returns the number of items in the queue.

\subsubsection{\texttt{bool enqueue(Queue *q, int64\_t x)}}

Enqueues an item \texttt{x} at the tail of a queue. If the queue is full prior to
enqueuing \texttt{x}, return \texttt{false} to indicate failure.
Otherwise, enqueue \texttt{x} and return \texttt{true} to indicate
success.

\subsubsection{\texttt{bool dequeue(Queue *q, int64\_t *x)}}

Dequeues an item from the head of a queue, passing the value of the
dequeued item back through the pointer \texttt{x}. If the queue is
empty prior to dequeuing the item, return \texttt{false} to indicate
failure. Otherwise, dequeue the item, set the value in the memory
\texttt{x} is pointing to as the dequeued item, and return \texttt{true}
to indicate success.

\subsubsection{\texttt{void queue\_print(Queue *q)}}

This is a debug function that you should write early on. It will help
greatly in determining whether or not your queue implementation is
working correctly.


\section{Your Task}

\epigraph{\emph{Die Narrheit hat ein gro\ss{}es Zelt; Es lagert bei ihr alle
Welt, Zumal wer Macht hat und viel Geld.}}{---Sebastian Brant, \emph{Das
Narrenschiff}}

\noindent For this assignment you have 3 tasks:

\begin{enumerate}
  \item Implement Bubble Sort, Shell Sort, and both iterative Quicksorts
    based on the provided Python pseudocode in \textbf{C}. The interface
    for these sorts will be given as the header files \texttt{bubble.h},
    \texttt{shell.h}, and \texttt{quick.h}. \textcolor{red}{You are not
    allowed to modify these files for any reason other than gathering
  statistics.}
  \item Implement a test harness for your implemented sorting
    algorithms. In your test harness, you will creating an array of
    random elements and testing each of the sorts. Your test harness
    \emph{must} be in the file \texttt{sorting.c}.
  \item Gather statistics about each sort and its performance. The
    statistics you will gather are the \emph{size} of the array, the
    number of \emph{moves} required, and the number of
    \emph{comparisons} required. \textcolor{red}{Note: a comparison is
    counted only when two array elements are compared.} For the
    iterative Quicksorts, you will additionally need to gather
    statistics on the maximum stack and queue sizes. You will want to
    create some graphs.
\end{enumerate}


\section{Specifics}

\epigraph{\emph{Vielleicht sind alle Drachen unseres Lebens
  Prinzessinnen, die nur darauf warten uns einmal sch\"on und mutig zu
sehen. Vielleicht ist alles Schreckliche im Grunde das Hilflose, das von
uns Hilfe will.}}{---Rainer Maria Rilke}

\noindent The following subsections cover the requirements of your test harness.
It is crucial that you follow the instructions.

\subsection{Command-line Options}

Your test harness must support any combination of the following
command-line options:

\begin{itemize}
  \item \texttt{-a}\ : Enables \emph{all} sorting algorithms.
  \item \texttt{-b}\ : Enables Bubble Sort.
  \item \texttt{-s}\ : Enables Shell Sort.
  \item \texttt{-q}\ : Enables the Quicksort that utilizes a stack.
  \item \texttt{-Q}\ : Enables the Quicksort that utilizes a queue.
  \item \texttt{-r seed}\ : Set the random seed to \texttt{seed}.
    The \emph{default} seed should be 13371453.
  \item \texttt{-n size}\ : Set the array size to \texttt{size}. The
    \emph{default} size should be 100.
  \item \texttt{-p elements}\ : Print out \texttt{elements} number of
    elements from the array. The \emph{default} number of elements to
    print out should be 100. \textcolor{red}{If the size of the array is
      less than the specified number of elements to print, print out the
    entire array and nothing more.}
\end{itemize}

It is important to read this \emph{carefully}. None of these options are
\emph{exclusive} of any other (you may specify any number of them,
including \emph{zero}). The most natural data structure for this
problem is a \emph{set}.

% \subsection{Sets}

% For this assignment, you are required to implement simple
% \emph{sets} using bits to track which command-line options are specified
% when your program is run. The number of bits will be small, but in
% subsequent assignments you will be implementing sets for an
% \emph{arbitrary} number of bits. Your set will be initialized using an
% unsigned int of a size equivalent to the number of bits as shown below.

% \begin{codelisting}{}
% typedef uint32_t Set;
% \end{codelisting}

% For manipulating the bits in a set, we use bit-wise operators. These
% operators as the name suggests will perform an operation on every bit in
% a number. The following are the six bit-wise operators specified in
% \textbf{C}:

% \begin{center}
%   \begin{tabular}{|c|l|l|}
%     \hline
%     \verb|&| & bit-wise AND & Performs the AND operation on every bit
%     of two numbers. \\
%     \hline
%     \verb||| & bit-wise OR & Performs the OR operation on every bit of
%     two numbers. \\
%     \hline
%     \verb|~| & bit-wise NOT & Inverts all bits in the given number. \\
%     \hline
%     \verb|^| & bit-wise XOR & Performs the exclusive-OR operation on
%     every bit of two numbers. \\
%     \hline
%     \verb|<<| & left shift & Shifts bits in a number to the left by a
%     specified number of bits. \\
%     \hline
%     \verb|>>| & right shift & Shifts bits in a number to the right by a
%     specified number of bits. \\
%     \hline
%   \end{tabular}
% \end{center}

% \noindent Recall that the basic set operations are: \emph{membership},
% \emph{union}, \emph{intersection} and \emph{negation}. For this
% assignment you will be implementing functions for these operations and a
% few more helper functions. Using these functions, you will set (make the
% bit 1) or clear (make the bit 0) bits in the \texttt{Set} depending on
% the command-line options read by \texttt{getopt()}. You can then check
% the states of all the bits (the members) of the \texttt{Set} using a
% single \texttt{for} loop and execute the corresponding sort. Note: you
% will not use all of the functions, but we will check their presence in
% \texttt{set.c}. Again, you \emph{must} use sets to track which
% command-line options are specified when running your program.

% \subsubsection{\texttt{Set set\_empty(void)}}

% This function is used to return an empty set. In this context, an empty
% set would be a set in which all bits are equal to 0.

% \subsubsection{\texttt{bool set\_member(Set s, uint8\_t x)}}

% \begin{align*}
%   x \in s \iff x\, \text{is a member of set}\, s
% \end{align*}

% % return (1 << x) & A
% \noindent This function returns a \texttt{bool} indicating the presence
% of the given value \texttt{x} in the set \texttt{s}. You will use the
% bit-wise AND operator to determine set membership. The first operand for
% the AND operation is the set \texttt{s}. The second operand is the value
% obtained by left shifting 1 \texttt{x} number of times. If the result of
% the AND operation is a non-zero value, then \texttt{x} is a member of
% \texttt{s} and \texttt{true} is returned to indicate this. Return
% \texttt{false} if the result of the AND operation is 0.

% \subsubsection{\texttt{Set set\_insert(Set s, uint8\_t x)}}

% This function returns a set with the bit corresponding to \texttt{x}
% equal to 1. Here, the bit is set using the bit-wise OR operator. The
% first operand for the OR operation is the set \texttt{s}. The second
% operand is value obtained by left shifting 1 by \texttt{x} number of
% bits.

% \subsubsection{\texttt{Set set\_remove(Set s, uint8\_t x)}}

% Similar to insertion, removing a member from the set means to clear the
% bit corresponding to \texttt{x}. Here, the bit is cleared using the
% bit-wise AND operator. The first operand for the AND operation is the
% set \texttt{s}. The second operand is a negation of the number 1 left
% shifted to the same position that \texttt{x} would occupy in the set.
% This means that the second operand would be all 1s except for the bit at
% \texttt{x}'s position. The function returns set \texttt{s} after
% removing \texttt{x}.

% \subsubsection{\texttt{Set set\_intersect(Set s, Set t)}}

% \begin{align*}
%   s \cap t = \{x | x \in s \land x \in  t\}
% \end{align*}

% % return A & B
% An intersection of two sets is a collection of elements that are common
% to both sets. Here, to calculate the intersection of the two sets,
% \texttt{s} and \texttt{t} we need to use the AND operator. Only the bits
% corresponding to members in both \texttt{s} and \texttt{t} need to be
% equal to 1 and the new set is returned by the function.

% \subsubsection{\texttt{Set set\_union(Set s, Set t)}}

% \begin{align*}
%   s \cup t = \{x | x \in s \lor x \in  t\}
% \end{align*}

% % return A | B
% A union of two sets is a collection of all elements in both sets. Here,
% to calculate the union of the two sets, \texttt{s} and \texttt{t} we
% need to use the OR operator. The bits corresponding to members in
% \texttt{s} or \texttt{t} equal to 1 therefore are in the new set is
% returned by the function.

% \subsubsection{\texttt{Set set\_complement(Set s)}}

% This function is used to return the complement of a given set. By
% complement we mean that all bits in the set are flipped using the NOT
% operator.

% \subsubsection{\texttt{Set set\_difference(Set s, Set t)}}

% The difference of two sets refers to the elements of set \texttt{s}
% which are not in set \texttt{t}. In other words, it refers to the
% members of set \texttt{s} that are unique to set \texttt{s}. The
% difference is calculated using the AND operator where the two operands
% are set \texttt{s} and the negation of set \texttt{t}.

% This function can be used to find the complement of a given set as well,
% in which case the first operand would be the universal set $\mathbb{U}$
% and the second operand would be the set you want to complement as shown
% below.

% \begin{align*}
%   \overline{s} = \{ x | x \notin s\} = \mathbb{U} -s
% \end{align*}

% \subsection{Testing}

% \begin{itemize}
%   \item You will test each of the sorts specified by command-line option
%     by sorting an array of pseudorandom numbers generated with
%     \texttt{random()}. Each of your sorts should sort the \emph{same}
%     pseudorandom array. \textcolor{red}{Hint: make use of
%     \texttt{srandom()}.}
%   \item The pseudorandom numbers generated by \texttt{random()} should
%     be \emph{bit-masked} to fit in \emph{30} bits. \textcolor{red}{Hint: use
%     bit-wise AND.}
%   \item Your test harness \emph{must} be able to test your sorts with
%     array sizes \emph{up to the memory limit of the computer}. That
%     means that you will need to dynamically allocate the array.
%   \item Your program should have no \emph{memory leaks}. Make sure you
%     \texttt{free()} before exiting. \texttt{valgrind} should pass
%     cleanly with any combination of the specified command-line options.
%   \item Your algorithms \emph{must} correctly sort. Any algorithm that
%     does not sort correctly will receive a \emph{zero}.
% \end{itemize}

% A large part of this assignment is understanding and comparing the
% performance of various sorting algorithms. You essentially conducting an
% experiment. As stated in \S 6, you \emph{must} collect the following
% statistics on each algorithm:

% \begin{itemize}
%   \item The \emph{size} of the array,
%   \item The number of \emph{moves} required (each time you transfer an
%     element in the array, that counts), and
%   \item The number of \emph{comparisons} required (comparisons
%     \emph{only} count for \emph{elements}, not for logic).
% \end{itemize}

\subsection{Output}

The output your test harness produces \emph{must} be formatted like in
the following examples:

\begin{lstlisting}[style=bashstyle]
  $ ./sorting -bq -n 1000 -p 0
  Bubble Sort
  1000 elements, 758313 moves, 498465 compares
  Quick Sort (Stack)
  1000 elements, 7257 moves, 13643 compares
  Max stack size: 22
  $ ./sorting -bqQ -n 15 -r 2021
  Bubble Sort
  15 elements, 228 moves, 105 compares
       45003895     46620555    199644728    203770850    218081181
      230022357    273593510    314322227    377988577    458735007
      553822818    570456718    653251166    802708864    890975627
  Quick Sort (Stack)
  15 elements, 60 moves, 98 compares
  Max stack size: 6
       45003895     46620555    199644728    203770850    218081181
      230022357    273593510    314322227    377988577    458735007
      553822818    570456718    653251166    802708864    890975627
  Quick Sort (Queue)
  15 elements, 60 moves, 98 compares
  Max queue size: 8
       45003895     46620555    199644728    203770850    218081181
      230022357    273593510    314322227    377988577    458735007
      553822818    570456718    653251166    802708864    890975627
\end{lstlisting}

For each sort that was specified, print its name, the statistics for the
run, then the specified number of array elements to print. The array
elements should be printed out in a table with 5 columns. Each array
element should be printed with a width of 13. You should make use of the
following \texttt{printf()} statement:

\begin{codelisting}{}
printf("%13" PRIu32); // Include <inttypes.h> for PRIu32.
\end{codelisting}

\medskip
\begin{prelab}{Pre-lab Part 4}
    \begin{enumerate}
        \item Explain how you plan on keeping track of the number of
          moves and comparisons since each sort will reside within its
          own file.
    \end{enumerate}
\end{prelab}


\section{Deliverables}

\epigraph{\emph{Dr.\ Long, put down the Rilke and step away from the
computer.}}{---Michael Olson}

You will need to turn in:

\begin{enumerate}
  \item Your program \emph{must} have the following source and header files:
  \begin{itemize}
    \item Each sorting method will have its own pair of header file and source file.
    \begin{itemize}
      \item \texttt{bubble.h} specifies the interface to \texttt{bubble.c}.
      \item \texttt{bubble.c} implements Bubble Sort.
      \item \texttt{gaps.h} contains the Pratt gap sequence for Shell
        Sort.
      \item \texttt{shell.h} specifies the interface to \texttt{shell.c}.
      \item \texttt{shell.c} implements Shell Sort.
      \item \texttt{quick.h} specifies the interface to \texttt{quick.c}.
      \item \texttt{quick.c} implements \emph{both} iterative Quicksorts.
      \item \texttt{stack.h} specifies the interface to the stack ADT.
      \item \texttt{stack.c} implements the stack ADT.
      \item \texttt{queue.h} specifies the interface to the queue ADT.
      \item \texttt{queue.c} implements the queue ADT.
      % \item \texttt{set.h} specifies the interface to the set ADT.
      % \item \texttt{set.c} implements the set ADT.
    \end{itemize}
    \item \texttt{sorting.c} contains \texttt{main()} and \emph{may}
      contain any other functions necessary to complete the assignment.
  \end{itemize}

You can have other source and header files, but \emph{do not try to be
overly clever}. The header files for each of the sorts are provided to
you. Each sort function takes the array of \texttt{uint32\_t}s to sort
as the first parameter and the length of the array as the second
parameter.

  \item \texttt{Makefile}: This is a file that will allow the grader to
    type \texttt{make} to compile your program. Typing \texttt{make}
    must build your program and \texttt{./sorting} alone as well as
    flags must run your program.

    \begin{itemize}
      \item \texttt{CFLAGS=-Wall -Wextra -Werror -Wpedantic}
        must be included.
      \item \texttt{CC=clang} must be specified.
      \item \texttt{make clean} must remove all files that are compiler
        generated.
      \item \texttt{make} should build your program, as should
        \texttt{make all}.
      \item \texttt{make format} should format all your source code,
        including the header files.
      \item Your program executable \emph{must} be named
        \texttt{sorting}.
    \end{itemize}

  \item Your code must pass \texttt{scan-build} \emph{cleanly}.

  \item \texttt{README.md}: This \emph{must} be in \emph{Markdown}.
    This must describe how to use your program and \texttt{Makefile}.

  \item \texttt{DESIGN.pdf}: This \emph{must} be a PDF. The design
    document should contain answers to the pre-lab questions at the
    beginning and describe your design for your program with enough
    detail that a sufficiently knowledgeable programmer would be able to
    replicate your implementation. This does not mean copying your
    entire program in verbatim. You should instead describe how your
    program works with supporting pseudo-code.

  \item \texttt{WRITEUP.pdf}: This document \emph{must} be a PDF. The
    writeup must include the following:
    \begin{itemize}
      \item Identify the respective time complexity for each sort and
        include what you have to say about the constant.
      \item What you learned from the different sorting algorithms.
      \item How you experimented with the sorts.
      \item How big do the stack and queue get? Does the size relate to
        the input? How?
      \item Graphs explaining the performance of the sorts on a variety
        of inputs, such as arrays in reverse order, arrays with a small
        number of elements, and arrays with a large number of elements.
      \item Analysis of the graphs you produce.
    \end{itemize}
\end{enumerate}


\section{Submission}

\epigraph{\emph{Da\ss{} Gott ohn Arbeit Lohn verspricht, Verla\ss{} dich
darauf und backe nicht Und wart, bis dir 'ne Taube gebraten Vom Himmel
k\"onnt in den Mund geraten!}}{---Sebastian Brant, \emph{Das
Narrenschiff}}

To submit your assignment, refer back to \texttt{asgn0} for the steps on
how to submit your assignment through \texttt{git}.  Remember:
\emph{add, commit,} and \emph{push}!

\textcolor{red}{Your assignment is turned in \emph{only} after you have
  pushed \emph{and} submitted the commit ID on Canvas. If you forget to
  push, you have not turned in your assignment and you will get a
  \emph{zero}. ``I forgot to push'' is not a valid excuse. It is
\emph{highly} recommended to commit and push your changes \emph{often}.}


\section{Supplemental Readings}

\epigraph{\emph{The more you read, the more things you will know. The more that
you learn, the more places you'll go.}}{---Dr.\ Seuss}\noindent

\begin{itemize}
    \item \textit{The C Programming Language} by Kernighan \& Ritchie
    \begin{itemize}
	\item Chapter 1 \S 1.10
	\item Chapter 3 \S 3.5
        \item Chapter 4 \S 4.10--4.11
        \item Chapter 5 \S 5.1--5.3 \& 5.10
    \end{itemize}
     \item  \emph{C in a Nutshell} by T.\ Crawford \& P.\ Prinz.
     \begin{itemize}
      \item Chapter 6 -- Example 6.5		% Not a section
      \item Chapter 7 -- Recursive Functions
      \item Chapter 8 -- Arrays as Arguments of Functions
      \item Chapter 9 -- Pointers to Arrays
      \end{itemize}
\end{itemize}

\begin{center}
  \includegraphics[width=0.5\textwidth]{../../images/monkey-chainsaw.jpg} \\
  \vspace{10pt}
  \textbf{用C編程就像給猴子電鋸一樣 。}
\end{center}

\end{document}
