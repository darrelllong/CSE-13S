\section{Mathematics of RSA}\label{proof}

\epigraphwidth=0.7\textwidth
\epigraph{\emph{If $\,{\cal P} = \NNP$, then all of modern cryptography collapses. On this happy thought\ldots}}{Michael O.\xspace Rabin, November 1998}

\noindent
The mathematics of RSA are based on arithmetic in a group of integers
modulo $n$, denoted $\mathbb{Z}/n$. This is the set $\{ 0, \ldots, n - 1
\}$ and all sets that are isomorphic to it. For example, $\{ n, \ldots,
2n -1 \} \pmod{n} = \{ 0, \ldots, n - 1 \} \pmod{n}$, and there are an
infinite number of such sets. Since they are \emph{all the same} we will
only concern ourselves with the one with the smallest numbers.  What do
we mean when we say that are the same? We mean that if $x \equiv k
\pmod{n}$ then $a n + x \equiv k \pmod{n}, \forall a \ge 0, a \in
\mathbb{Z}$. In other words, additional integer products of $n$ do not
matter.

The Euler-Fermat theorem says that if $a\in\mathbb{N}$ and
$n\in\mathbb{N}$ are \emph{coprime}, that is $\gcd(a, n) = 1$, then
$$a^{\varphi(n)} \equiv 1 \pmod{n}.$$ This will allow us, for example,
to take a message $M$ and have $M^{\varphi(n)} \equiv 1 \pmod{n}$.

What is $\varphi(n)$? It is the Euler totient function, and gives the
number of positive integers than that or equal to $n$ that are
relatively prime to $n$. For any prime number $p$,
$$
\varphi(p) = p - 1.
$$

For the RSA algorithm, we choose two large primes $p$ and $q$, and we
make $n = p q$. What does this mean for us with respect to $\varphi(n)$?
\begin{align*}
        \varphi(n) & = \varphi(p) \times \varphi(q) \\
        & = (p - 1) (q - 1) \\
        & = n - (p + q) + 1 .
\end{align*}

We choose an encryption key $e$ such that is it relatively prime to
$\varphi(n)$, that is, $\gcd(e, \varphi(n)) = 1$. We can then deduce our
decryption key $d$ such that $e \times d \equiv 1 \pmod{\varphi(n)}$.
Our encryption algorithm is simply $E(M) = M^e \pmod{n} = C$, and our
decryption algorithm is $D(C) = c^d \pmod{n} = M$.

We have the additional property of being mutual inverses:
$E(D(x)) = D(E(x)) = x$,
and that will give us the ability to provide \emph{digital signatures}.
Observe that,
\begin{align*}
        D(E(M)) &\equiv E(M)^d \equiv M^{e d} \pmod{n} \\
        E(D(M)) &\equiv D(M)^e \equiv M^{d e} \pmod{n}.
\end{align*}
Continuing in this manner, observe that,
\[
  M^{e d} \equiv M^{k \varphi(n) + 1} \pmod{n}
\]
for some multiplier $k \ge 1$. This is true because, prior to the
modular reduction, $ed$ must be one greater than some multiple of
$\varphi(n)$; applying the modulus is what makes
$e d \equiv 1 \pmod{\varphi(n)}$. We can rewrite this as
\[
  M^{e d} \equiv (M^{\varphi(n)})^k \times M \pmod{n}.
\]
Here we apply Euler's theorem, which states that if $a$ and $n$ are
coprime integers, then $a^{\varphi(n)} \equiv 1 \pmod{n}$. So, assuming
that $M$ is coprime with $n$, we can simplify the above equation to
\begin{align*}
  M^{e d} &\equiv (1)^k \times M  &&\pmod{n} \\
          &\equiv (1) \times M  &&\pmod{n} \\
          &\equiv M  &&\pmod{n}
\end{align*}
which shows that the encryption and decryption functions are
mutual inverses.
