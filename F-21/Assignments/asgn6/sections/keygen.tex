\section{Key Generator}
\epigraph{\emph{This method, seemingly very clever, actually played into our hands! And so it often happens that an apparently ingenious idea is in fact a weakness which the scientific cryptographer seizes on for his solution.}}{---Herbert Yardley, \emph{The American Black Chamber}}

\noindent
Your key generator program should accept the following command-line
options:
\begin{itemize}
  \item \texttt{-b}\,: specifies the minimum bits needed for the public
    modulus $n$.
  \item \texttt{-i}\,: specifies the number of Miller-Rabin iterations
    for testing primes (default: 50).
  \item \texttt{-n pbfile}\,: specifies the public key file (default:
    rsa.pub).
  \item \texttt{-d pvfile}\,: specifies the private key file (default:
    rsa.priv).
  \item \texttt{-s}\,: specifies the random seed for the random state
    initialization (default: the seconds since the \textsc{Unix} epoch,
    given by \texttt{time(NULL)}).
  \item \texttt{-v}\,: enables verbose output.
  \item \texttt{-h}\,: displays program synopsis and usage.
\end{itemize}

The program should follow these steps:
\begin{enumerate}
  \item Parse command-line options using \texttt{getopt()} and handle
    them accordingly.
  \item Open the public and private key files using \texttt{fopen()}.
    Print a helpful error and exit the program in the event of failure.
  \item Using \texttt{fchmod()} and \texttt{fileno()}, make sure that
    the private key file permissions are set to \texttt{0600},
    indicating read and write permissions for the user, and no
    permissions for anyone else.
  \item Initialize the random state using \texttt{randstate\_init()},
    using the set seed.
  \item Make the public and private keys using \texttt{rsa\_make\_pub()}
    and \texttt{rsa\_make\_priv()}, respectively.
  \item Get the current user's name as a string. You will want to use
    \texttt{getenv()}.
  \item Convert the username into an \texttt{mpz\_t} with
    \texttt{mpz\_set\_str()}, specifying the base as 62. Then, use
    \texttt{rsa\_sign()} to compute the signature of the username.
  \item Write the computed public and private key to their respective
    files.
  \item If verbose output is enabled print the following, each with a
    trailing newline, in order:
    \begin{enumerate}
      \item username
      \item the signature \texttt{s}
      \item the first large prime \texttt{p}
      \item the second large prime \texttt{q}
      \item the public modulus \texttt{n}
      \item the public exponent \texttt{e}
      \item the private key \texttt{d}
    \end{enumerate}
    All of the \texttt{mpz\_t} values should be printed with information
    about the number of bits that constitute them, along with their
    respective values in \emph{decimal}. See the reference key generator
    program for an example.
  \item Close the public and private key files, clear the random state
    with \texttt{randstate\_clear()}, and clear any \texttt{mpz\_t}
    variables you may have used.
\end{enumerate}
