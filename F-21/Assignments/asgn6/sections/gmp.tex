\section{GNU Multiple Precision Arithmetic}
\epigraph{\emph{One reason you should not use web applications to do your computing is
that you lose control. It's just as bad as using a proprietary program. Do your own
computing on your own computer with your copy of a freedom-respecting program.
If you use a proprietary program or somebody else's web server, you're defenceless.}}{%
---Richard Stallman}

\noindent
As you should know by now, \textbf{C}, unlike languages like
\textbf{Python}, does not natively support arbitrary precision integers.
The security of RSA, however, relies on large integers. So, we elect to
use the GNU multiple precision arithmetic library, usually referred to
as GMP\@. You can find the manual and documentation for the library
here:
\centerline{\url{https://gmplib.org/manual}.}
Take some time to look through the manual, taking note of which
functions may be useful.

You will need to install both \texttt{gmp} and \texttt{pkg-config}. The
latter is a utility used to assist in finding and linking libraries,
instead of having the program hard-code where to find specific headers
and libraries during program compilation. To install these packages on
Ubuntu 20.04, run the following:

\begin{shlisting}{}
$ sudo apt install pkg-config libgmp-dev
\end{shlisting}

Get started on this \emph{as soon as possible}. Make sure to attend
section for assistance on using \texttt{pkg-config} in a
\texttt{Makefile} to direct the compilation process for your programs.

You may notice that GMP already provides number theoretic functions,
several of which \emph{could} be used in RSA. \textcolor{red}{You may
not use any GMP-implemented number theoretic functions. You \emph{must}
implement those functions yourself.}

The following two sections (\S6 and \S7) will present the functions that
you have to implement, but they both will require the use of random,
arbitrary-precision integers.

GMP requires us to explicitly initialize a random state variable and
pass it to any of the random integer functions in GMP\@. Not only that,
we also need to call a dedicated function to clean up any memory used by
the initialized random state. To remedy this, you will implement a small
random state module, which contains a single \texttt{extern} declaration
to a global random state variable called \texttt{state}, and two
functions: one to initialize the state, and one to clear it. The
interface for the module will be given in \texttt{randstate.h} and the
implementation must go in \texttt{randstate.c}.

\begin{funcdoc}{void randstate\_init(uint64\_t seed)}
  Initializes the global random state named \texttt{state} with a
  Mersenne Twister algorithm, using \texttt{seed} as the random seed.
  This entails a call to \texttt{gmp\_randinit\_mt()} and a call to
  \texttt{gmp\_randseed\_ui()}.
\end{funcdoc}

\begin{funcdoc}{void randstate\_clear(void)}
  Clears and frees all memory used by the initialized global random
  state named \texttt{state}. This should just be a single call to
  \texttt{gmp\_randclear()}.
\end{funcdoc}
