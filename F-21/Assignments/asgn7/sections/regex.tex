\section{Lexical Analysis with Regular Expressions}

\epigraphwidth=0.5\textwidth
\epigraph{\emph{Ideas are more powerful than guns. We would not let our
enemies have guns, why should we let them have ideas.}}{---Joseph
Stalin}

\noindent
Back to regulating your citizens of the GPRSC. You will need a function
to parse out the words that they speak, which will be passed to you in
the form of an input stream. The words that they will use are valid
words, which can include \emph{contractions} and \emph{hyphenations}.
A valid word is any sequence of one or more characters that are part of
your regular expression word character set. Your word character set
should contain characters from a--z, A--Z, 0--9, and the underscore
character. Since you also accept contractions like ``don't'' and
``y'all've'' and hyphenations like ``pseudo-code'' and
``move-to-front'', your word character set should include apostrophes
and hyphens as well.

You will need to write your own \emph{regular expression} for a word,
utilizing the \texttt{regex.h} library to lexically analyze the input
stream for words. You will be given a parsing module that lexically
analyzes the input stream using your regular expression. You are not
required to use the module itself, but it is \emph{mandatory} that you
parse through an input stream for words using at least one regular
expression. The interface for the parsing module will be in
\texttt{parser.h} and its implementation will be in \texttt{parser.c}.

The function \texttt{next\_word()} requires two inputs, the input stream
\texttt{infile}, and a pointer to a compiled regular expression,
\texttt{word\_regex}. Notice the word \emph{compiled}: you must first compile
your regular expression using \texttt{regcomp()} before passing it to the
function. Make sure you remember to call the function \texttt{clear\_words()} to
free any memory used by the module when you're done reading in words.
Here is a small program that prints out words input to \texttt{stdin}
using the parsing module. In the program, the regular expression for a
word matches one or more lowercase and uppercase letters. The regular
expression you will have to write for your assignment will be more
complex than the one displayed here, as it is just an example.

\begin{clisting}{Example program using the parsing module.}
#include "parser.h"
#include <regex.h>
#include <stdio.h>

#define WORD "[a-zA-Z]+"

int main(void) {
    regex_t re;
    if (regcomp(&re, WORD, REG_EXTENDED)) {
        fprintf(stderr, "Failed to compile regex.\n");
        return 1;
    }

    char *word = NULL;
    while ((word = next_word(stdin, &re)) != NULL) {
        printf("Word: %s\n", word);
    }

    clear_words();
    regfree(&re);
    return 0;
}
\end{clisting}
