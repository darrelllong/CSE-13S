\section{Your Task}

\epigraphwidth=0.5\textwidth
\epigraph{\emph{The people will believe what the media tells them they
believe.}}{---George Orwell}

\noindent
\begin{itemize}
  \item Initialize your Bloom filter and hash table.
  \item Read in a list of \emph{badspeak} words with \texttt{fscanf()}.
    Again, badspeak is simply oldspeak without a newspeak translation.
    Badspeak is strictly forbidden. Each badspeak word should be added
    to the Bloom filter and the hash table. The list of proscribed words
    will be in \texttt{badspeak.txt}, which can be found in the
    \texttt{resources} repository.
  \item Read in a list of \emph{oldspeak} and \emph{newspeak} pairs with
    \texttt{fscanf()}. Only the oldspeak should be added to the Bloom
    filter. The oldspeak \emph{and} newspeak are added to the hash
    table. The list of oldspeak and newspeak pairs will be in
    \texttt{newspeak.txt}, which can also be found in the
    \texttt{resources} repository.
  \item Now that the lexicon of badspeak and oldspeak/newspeak
    translations has been populated, you can start to filter out words.
    Read words in from \texttt{stdin} using the supplied parsing module.
  \item For each word that is read in, check to see if it has been added
    to the Bloom filter. If it has not been added to the Bloom filter,
    then no action is needed since the word isn't a proscribed word.
  \item If the word has most likely been added to the Bloom filter,
    meaning \texttt{bf\_probe()} returned \texttt{true}, then further
    action needs to be taken.
    \begin{enumerate}
      \item If the hash table contains the word and the word \emph{does
        not} have a newspeak translation, then the citizen who used this
        word is guilty of \texttt{thoughtcrime}. Insert this badspeak
        word into a list of badspeak words that the citizen used in
        order to notify them of their errors later. What data structure
        could be used to store these words?
      \item If the hash table contains the word, and the word \emph{does}
        have a newspeak translation, then the citizen requires
        counseling on proper \emph{Rightspeak}. Insert this oldspeak
        word into a list of oldspeak words with newspeak translations in
        order to notify the citizen of the revisions needed to be made
        in order to practice Rightspeak.
      \item If the hash table does not contain the word, then all is
        good since the Bloom filter issued a false positive. No
        disciplinary action needs to be taken.
    \end{enumerate}
  \item If the citizen is accused of thoughtcrime \emph{and} requires
    counseling on proper \emph{Rightspeak}, then they are given a
    reprimanding \emph{mixspeak message} notifying them of their
    transgressions and promptly sent off to \emph{joycamp}. The message
    should contain the list of badspeak words that were used followed by
    the list of oldspeak words that were used with their proper newspeak
    translations.

  \begin{shlisting}{}
Dear beloved citizen of the GPRSC,

We have some good news, and we have some bad news.
The good news is that there is bad news. The bad news is that you will
be sent to joycamp and subjected to a week-long destitute existence.
This is the penalty for using degenerate words, as well as using
oldspeak in place of newspeak. We hope you can correct your behavior.

Your transgressions, followed by the words you must think on:

kalamazoo
antidisestablishmentarianism
write -> papertalk
sad -> happy
read -> papertalk
music -> noise
liberty -> badfree\end{shlisting}

  \item If the citizen is accused solely of thoughtcrime, then they are
    issued a thoughtcrime message and also sent off to \emph{joycamp}.
    The \emph{badspeak message} should contain the list of badspeak
    words that were used.

  \begin{shlisting}{}
Dear beloved citizen of the GPRSC,

You have been caught using degenerate words that may cause
distress among the moral and upstanding citizens of the GPSRC.
As such, you will be sent to joycamp. It is there where you will
sit and reflect on the consequences of your choice in language.

Your transgressions:

kalamazoo
antidisestablishmentarianism\end{shlisting}

  \item If the citizen only requires counseling, then they are issued an
    encouraging \emph{goodspeak message}. They will read it, correct
    their \emph{wrongthink}, and enjoy the rest of their stay in the
    GPRSC. The message should contain the list of oldspeak words that
    were used with their proper newspeak translations.

    \begin{shlisting}{}
Dear beloved citizen of the GPRSC,

We recognize your efforts in conforming to the language standards
of the GPSRC. Alas, you have been caught uttering questionable words
and thinking unpleasant thoughts. You must correct your wrongspeak
and badthink at once. Failure to do so will result in your deliverance
to joycamp.

Words that you must think on:

write -> papertalk
sad -> happy
read -> papertalk
music -> noise
liberty -> badfree\end{shlisting}

  \item Each of the messages are defined for you in \texttt{messages.h}.
    \textcolor{red}{You may not modify this file}.

  \item The list of the command-line options your program must support
    is listed below. \emph{Any} combination of the command-line options
    must be supported.
    \begin{itemize}
      \item \texttt{-h} prints out the program usage. Refer to the
      reference program in the resources repository for what to print.
      \item \texttt{-t size} specifies that the hash table
        will have \texttt{size} entries (the default will be $2^{16}$).
      \item \texttt{-f size} specifies that the Bloom filter
        will have \texttt{size} entries (the default will be $2^{20}$).
      \item \texttt{-s} will enable the printing of statistics to
        \texttt{stdout}. The statistics to calculate are:
        \begin{itemize}
          \item Average binary search tree size
          \item Average binary search tree height
          \item Average branches traversed
          \item Hash table load
          \item Bloom filter load
        \end{itemize}
        The latter three statistics are computed as follows:
        \begin{align*}
          \text{Average branches traversed} &=
          \frac{\texttt{branches}}{\texttt{lookups}} \\
          \text{Hash table load} &= 100 \times \frac{\texttt{ht\_count()}}{\texttt{ht\_size()}} \\
          \text{Bloom filter load} &= 100 \times \frac{\texttt{bf\_count()}}{\texttt{bf\_size()}}
        \end{align*}
        The hash table load and Bloom filter load should be printed with
        up to 6 digits of precision. \textcolor{red}{Enabling the
        printing of statistics should \emph{suppress all messages} the
        program may otherwise print.} The number of lookups is defined
        as the number of times \texttt{ht\_lookup()} and
        \texttt{ht\_insert()} is called. The number of branches is
        defined as the count of links traversed during calls to
        \texttt{bst\_find()} and \texttt{bst\_insert()}. The global
        variable \texttt{lookups} should be defined in \texttt{ht.c} and
        the global variable \texttt{branches} should be defined in
        \texttt{bst.c}.
    \end{itemize}
\end{itemize}
