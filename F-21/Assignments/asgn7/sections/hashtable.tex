\section{Hash Tables}\label{hashtable}

\epigraph{\emph{He tried his best to educate the children not to be
traitors.}} {---Mea Son (widow of Pol Pot)}

\noindent Armed with a Bloom filter, you now exercise the power to catch
and punish those who practice wrongthink and continue to use oldspeak.
It comes to mind however, that a Bloom filter is probabilistic and that
it is better to exercise mercy and counsel the oldspeakers so that they
may atone and use \emph{newspeak}. To remedy this, another solution pops
into your brilliant and pure mind---a \emph{hash table}.

A hash table is a data structure that maps keys to values and provides
fast, $\operatorname{O}(1)$, look-up times. It does so typically by
taking a key $k$, hashing it with some hash function $h(x)$, and placing
the key's corresponding value in an underlying array at index $h(k)$.
This is the perfect way not only to store translations from oldspeak to
newspeak, but also as a way to store all prohibited oldspeak words
without newspeak translations. We will refer to oldspeak without
newspeak translations as \emph{badspeak}. So what happens when two
\emph{oldspeak} words have the same hash value?  This is called a
\emph{hash collision}, and must be resolved. Rather than doing
\emph{open addressing} (as will be discussed in lecture), we will be
using \emph{binary search trees} to resolve \emph{oldspeak} hash
collisions.

\begin{wrapfigure}{r}{0.16\textwidth}
\centering
  \includegraphics[width=0.155\textwidth]{images/facebook-big-brother-is-watching.png}
\end{wrapfigure}

A hash function maps data of arbitrary size to fixed-size values. Its
values are called hash values, hash codes, digests, or simply hashes,
which index a fixed-size table called a hash table.  Hash functions and
their associated hash tables are used in data storage and retrieval
applications to access data (on average) at a nearly constant time per
retrieval. They require a storage space that is only fractionally
greater than the total space needed for the data. Hashing avoids the
non-linear access time of ordered and unordered lists and some trees and
the often exponential storage requirements of direct access of large
state spaces.

Hash functions rely on statistical properties of key and function
interaction: worst-case behavior is an exhaustive search but with a
vanishingly small probability, and average-case behavior can be nearly
constant (with minimal collisions).

Below is the \texttt{struct} definition for a hash table. Similar to a
Bloom filter, a hash table contains a salt which is passed to
\texttt{hash()} whenever a new oldspeak entry is being inserted. As
mentioned in \S\ref{hashtable}, we will be using binary search trees to
resolve oldspeak hash collisions, which is why a hash table contains an
array of trees.

\begin{clisting}{}
struct HashTable {
  uint64_t salt[2];
  uint32_t size;
  Node **trees;
};
\end{clisting}

This \texttt{struct} definition \emph{must} go in \texttt{ht.c}.

\begin{funcdoc}{\texttt{HashTable *ht\_create(uint32\_t size)}}
  The constructor for a hash table. The \texttt{size} parameter denotes
  the number of indices, or binary search trees, that the hash table can
  index up to. The salt for the hash table is provided in
  \texttt{salts.h}.
\end{funcdoc}

\begin{funcdoc}{\texttt{void ht\_delete(HashTable **ht)}}
  The destructor for a hash table. Each of the binary search trees
  \texttt{trees}, the underlying array of binary search tree root nodes,
  is freed. The pointer that was passed in should be set to
  \texttt{NULL}.
\end{funcdoc}

\begin{funcdoc}{\texttt{uint32\_t ht\_size(HashTable *ht)}}
  Returns the hash table's size.
\end{funcdoc}

\begin{funcdoc}{\texttt{Node *ht\_lookup(HashTable *ht, char *oldspeak)}}
  Searches for an entry, a node, in the hash table that contains
  \texttt{oldspeak}. A node stores oldspeak and its newspeak
  translation. The index of the binary search tree to perform a look-up
  on is calculated by hashing the \texttt{oldspeak}. If the node is
  found, the pointer to the node is returned. Else, a \texttt{NULL}
  pointer is returned.
\end{funcdoc}

\begin{funcdoc}{\texttt{void ht\_insert(HashTable *ht, char *oldspeak, char *newspeak)}}
  Inserts the specified \texttt{oldspeak} and its corresponding
  \texttt{newspeak} translation into the hash table. The index of the
  binary search tree to insert into is calculated by hashing the
  \texttt{oldspeak}.
\end{funcdoc}

\begin{funcdoc}{\texttt{uint32\_t ht\_count(HashTable *ht)}}
Returns the number of non-\texttt{NULL} binary search trees in the hash
table.
\end{funcdoc}

\begin{funcdoc}{double ht\_avg\_bst\_size(HashTable *ht)}
  Returns the average binary search tree size. This is computed as the
  sum of the sizes over all the binary search trees divided by the
  number of non-\texttt{NULL} binary search trees in the hash table. You
  will need to use \texttt{bst\_size()}, presented in \S\ref{bstsize},
  to compute this.
\end{funcdoc}

\begin{funcdoc}{double ht\_avg\_bst\_height(HashTable *ht)}
  Returns the average binary search tree size. This is computed as the
  sum of the heights over all the binary search trees divided by the
  number of non-\texttt{NULL} binary search trees in the hash table. You
  will need to use \texttt{bst\_height()}, presented in
  \S\ref{bstheight}, to compute this.
\end{funcdoc}

\begin{funcdoc}{\texttt{void ht\_print(HashTable *ht)}}
  A debug function to print out the contents of a hash table.
  \textcolor{red}{Write this immediately after the constructor}.
\end{funcdoc}
