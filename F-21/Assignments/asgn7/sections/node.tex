\section{Nodes}
\epigraph{\emph{Tyrants have always some slight shade of virtue; they support the laws before destroying them.}}{---Voltaire}

\noindent
As mentioned in \S\ref{hashtable}, binary search trees will be used to
resolve hash collisions. As discussed in lecture, binary search trees,
and trees in general, are made up of nodes. For this assignment, each
node contains \emph{oldspeak} and its \emph{newspeak} translation if it
exists. The \emph{key} to search with in a binary search tree is
\emph{oldspeak}. Each node, in typical binary search tree fashion, will
contain pointers to its left and right children. The node
\texttt{struct} is defined for you in \texttt{node.h} as follows:

\begin{clisting}{}
struct Node {
    char *oldspeak;
    char *newspeak;
    Node *left;
    Node *right;
};
\end{clisting}

If the \texttt{newspeak} field is \texttt{NULL}, then the oldspeak
contained in this node is badspeak, since there is no newspeak
translation. The rest of the interface for the node ADT is provided in
\texttt{node.h}. The node ADT is made transparent in order to simplify
the implementation of the binary search tree ADT.

\begin{funcdoc}{\texttt{Node *node\_create(char *oldspeak, char *newspeak)}}
  The constructor for a node. You will want to make a \emph{copy} of the
  \texttt{oldspeak} and its \texttt{newspeak} translation that are passed
  in. What this means is \emph{allocating memory} and copying over the
  characters for both \texttt{oldspeak} and \texttt{newspeak}. You may
  find \texttt{strdup()} useful.
\end{funcdoc}

\begin{funcdoc}{\texttt{void node\_delete(Node **n)}}
  The destructor for a node. Only the node \texttt{n} is freed. The
  previous and next nodes that \texttt{n} points to \emph{are not}
  deleted. Since you have allocated memory for \texttt{oldspeak} and
  \texttt{newspeak}, remember to free the memory allocated to both of
  those as well. The pointer to the node should be set to \texttt{NULL}.
\end{funcdoc}

\begin{funcdoc}{\texttt{void node\_print(Node *n)}}
  While helpful as debug function, you will use this function to produce
  correct program output. Thus, it is imperative that you print out the
  contents of a node in the following manner:

  \begin{itemize}
    \item If the node \texttt{n} contains oldspeak \emph{and} newspeak,
      print out the node with this print statement:
      \begin{clisting}{}
printf("%s -> %s\n", n->oldspeak, n->newspeak);
      \end{clisting}
    \item If the node \texttt{n} contains \emph{only} oldspeak, meaning
      that newspeak is null, then print out the node with this print
      statement:
      \begin{clisting}{}
printf("%s\n", n->oldspeak);
      \end{clisting}
  \end{itemize}
\end{funcdoc}
