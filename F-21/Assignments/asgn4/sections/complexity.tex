\section{Computational Complexity}

\epigraphwidth=0.65\textwidth \epigraph{\emph{Many a trip continues long
    after movement in time and space have ceased. I remember a man in
    Salinas who in his middle years traveled to Honolulu and back, and
    that journey continued for the rest of his life. We could watch him
    in his rocking chair on his front porch, his eyes squinted,
    half-closed, endlessly traveling to Honolulu.}}{---John Steinbeck,
    \emph{Travels with Charley: In Search of Amerca}}

\noindent
How long will this take?
The answer is, it will take a very long time if there are a large number
of vertices. The running time of the simplest algorithm is
$\operatorname{O}(n!)$ and there is no known algorithm that runs in less
than $\operatorname{O}(2^n)$ time.  In fact, the TSP has been shown to
be NP-hard, which means that it is as difficult as any problem in the
class NP (you will learn more about this in CSE\,104: Computability and
Computational Complexity).  Basically, it means that it can be solved in
polynomial time if you have a magical computer that at each if-statement
is takes both branches every time (creating a copy of the computer for
each such branch).
