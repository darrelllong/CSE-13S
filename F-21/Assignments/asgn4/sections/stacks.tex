\section{Stacks}

You will need to implement the \emph{stack} ADT for tracking the path
Denver has traveled. This section defines the interface for a stack. A
stack is an ADT that implements a \emph{last-in, first-out}, or
\textbf{LIFO}, policy. Consider a stack of pancakes. A pancake can only
be placed (\emph{pushed}) on top of the stack and can only be removed
(\emph{popped}) from the top of the stack. We do not eat pancakes from
the middle or bottom of a stack. The header file containing the
interface will be given to you as \texttt{stack.h}. \textcolor{red}{You
may not modify this file.} If you borrow code from any place ---
including Prof. Long --- you must cite it. It is \emph{far better} if
you write it yourself.

The stack datatype will be abstracted as a \texttt{struct} called
\texttt{Stack}. We will use a \texttt{typedef} to construct a new type,
which you should treat as opaque --- which means that you cannot
manipulate it directly, like the graph ADT. We will \emph{declare} the
\texttt{stack} type in \texttt{stack.h} and you will define its concrete
implementation in \texttt{stack.c}. \textcolor{red}{This means the
following \texttt{struct} definition \emph{must} be in
\texttt{stack.c}}.

\begin{clisting}{}
struct Stack {
    uint32_t top;       // Index of the next empty slot.
    uint32_t capacity;  // Number of items that can be pushed.
    uint32_t *items;    // Array of items, each with type uint32_t.
};
\end{clisting}

\begin{funcdoc}{Stack *stack\_create(uint32\_t capacity)}
  The constructor function for a \texttt{Stack}. The \texttt{top} of a
  stack should be initialized to 0. The capacity of a stack is set to the
  specified capacity. The specified capacity also indicates the number of
  items to allocate memory for, the items in which are held in the
  dynamically allocated array \texttt{items}. A working constructor for a
  stack is provided below. Note that it uses \texttt{malloc()}, as well
  as \texttt{calloc()}, for \emph{dynamic memory allocation}. Both these
  functions are included from \texttt{<stdlib.h>}.

  \begin{clisting}{}
Stack *stack_create(uint32_t capacity) {
    Stack *s = (Stack *) malloc(sizeof(Stack));
    if (s) {
        s->top = 0;
        s->capacity = capacity;
        s->items = (uint32_t *) calloc(capacity, sizeof(uint32_t));
        if (!s->items) {
            free(s);
            s = NULL;
        }
    }
    return s;
}
  \end{clisting}
\end{funcdoc}

\begin{funcdoc}{void stack\_delete(Stack **s)}
  The destructor function for a stack. A working destructor is provided
  below. Pay attention to the things that are freed and what is set to
  \texttt{NULL}.

  \begin{clisting}{}
void stack_delete(Stack **s) {
    if (*s && (*s)->items) {
        free((*s)->items);
        free(*s);
        *s = NULL;
    }
    return;
}
  \end{clisting}
\end{funcdoc}


\begin{funcdoc}{bool stack\_empty(Stack *s)}
  Returns \texttt{true} if the stack is empty and \texttt{false}
  otherwise.
\end{funcdoc}

\begin{funcdoc}{bool stack\_full(Stack *s)}
  Returns \texttt{true} if the stack is full and \texttt{false} otherwise.
\end{funcdoc}

\begin{funcdoc}{uint32\_t stack\_size(Stack *s)}
  Returns the number of items in the stack.
\end{funcdoc}

\begin{funcdoc}{bool stack\_push(Stack *s, uint32\_t x)}
  The need for \emph{manipulator} functions follows the rationale behind
  the need for accessor functions: there needs to be some way to alter
  fields of a data type. \texttt{stack\_push()} is a manipulator function
  that pushes an item \texttt{x} to the top of a stack.

  This function returns a \texttt{bool} in order to signify either
  success or failure when pushing onto a stack. When can pushing onto a
  stack result in failure? \emph{When the stack is full.} If the stack
  is full prior to pushing the item \texttt{x}, return \texttt{false} to
  indicate failure. Otherwise, push the item and return \texttt{true} to
  indicate success.
\end{funcdoc}

\begin{funcdoc}{bool stack\_pop(Stack *s, uint32\_t *x)}
  This function pops an item off the specified stack, passing the value
  of the popped item back through the pointer \texttt{x}. Like with
  \texttt{stack\_push()}, this function returns a \texttt{bool} to
  indicate either success or failure. When can popping a stack result in
  failure? \emph{When the stack is empty.} If the stack is empty prior to
  popping it, return \texttt{false} to indicate failure. Otherwise, pop
  the item, set the value in the memory \texttt{x} is pointing to as the
  popped item, and return \texttt{true} to indicate success.

  \begin{clisting}{}
// Dereferencing x to change the value it points to.
*x = s->items[s->top];
  \end{clisting}
\end{funcdoc}

\begin{funcdoc}{bool stack\_peek(Stack *s, uint32\_t *x)}
  Peeking into a stack is synonymous with querying a stack about the
  element at the top of the stack. If the stack is empty prior to
  peeking into it, return \texttt{false} to indicate failure. This
  functions like \texttt{stack\_pop()}, but doesn't modify the contents
  of the stack.
\end{funcdoc}

\begin{funcdoc}{void stack\_copy(Stack *dst, Stack *src)}
  Assuming that the destination stack \texttt{dst} is properly
  initialized, and that \texttt{dst} has the same capacity as
  \texttt{src}, make \texttt{dst} a copy of the source stack
  \texttt{src}. This means making the \emph{contents} of
  \texttt{dst->items} the same as \texttt{src->items}. The top of
  \texttt{dst} should also match the top of \texttt{src}.
\end{funcdoc}

\begin{funcdoc}{void stack\_print(Stack *s, FILE *outfile, char *cities[])}
  Prints out the contents of the stack to \texttt{outfile} using
  \texttt{fprintf()}. Working through each vertex in the stack starting
  from the \emph{bottom}, print out the name of the city each vertex
  corresponds to. This function will be given to aid you.

  \begin{clisting}{}
void stack_print(Stack *s, FILE *outfile, char *cities[]) {
    for (uint32_t i = 0; i < s->top; i += 1) {
        fprintf(outfile, "%s", cities[s->items[i]]);
        if (i + 1 != s->top) {
            fprintf(outfile, " -> ");
        }
    }
    fprintf(outfile, "\n");
}
  \end{clisting}
\end{funcdoc}
