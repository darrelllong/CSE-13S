\section{Depth-first Search}

\epigraphwidth=0.65\textwidth \epigraph{\emph{Again it might have been
the American tendency in travel. One goes, not so much to see but to
tell afterward.}}{---John Steinbeck, \emph{Travels with Charley: In
Search of Amerca}}

\noindent
We need a methodical procedure for searching through the graph. Once we
have examined a vertex, we do not want to do so again---we don't want
Denver going through cities where he has already been (he has been known
to wear out his welcome: charming women and fighting men).

Depth-first search (DFS) first marks the vertex $v$ as having been visited,
then it iterates through all of the edges $\langle v, w \rangle$,
recursively calling itself starting at $w$ if $w$ has not already been
visited.

\begin{clisting}{}
procedure DFS(G,v):
    label v as visited
    for all edges from v to w in G.adjacentEdges(v) do
       if vertex w is not labeled as visited then
          recursively call DFS(G,w)
    label v as unvisited
\end{clisting}

Finding a Hamiltonian path then reduces to:
\begin{enumerate}
\item
Using DFS to find paths that pass through all vertices, and
\item
There is an edge from the last vertex to the first. The solutions to the
Traveling Salesman Problem are then the shortest found Hamiltonian
paths.
\end{enumerate}
