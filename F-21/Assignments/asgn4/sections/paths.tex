\section{Paths}

\epigraphwidth=0.65\textwidth \epigraph{\emph{We find after years of
struggle that we do not take a trip; a trip takes us.} }{---John
Steinbeck, \emph{Travels with Charley: In Search of Amerca}}

\noindent
Given that vertices are added to and removed from the traveled path in a
stack-like manner, we decide to abstract a path as follows:

\begin{clisting}{}
struct Path {
    Stack *vertices; // The vertices comprising the path.
    uint32_t length; // The total length of the path.
};
\end{clisting}

The following functions define the interface for the path ADT.

\begin{funcdoc}{Path *path\_create(void)}
  The constructor for a path. Set \texttt{vertices} as a freshly created
  stack that can hold up to \texttt{VERTICES} number of vertices.
  Initialize \texttt{length} to be 0. The \texttt{length} field will
  track the length of the path. In other words, it holds the sum of the
  edge weights between consecutive vertices in the \texttt{vertices}
  stack.
\end{funcdoc}

\begin{funcdoc}{void path\_delete(Path **p)}
  The destructor for a path. Remember to set the pointer \texttt{p} to
  \texttt{NULL}.
\end{funcdoc}

\begin{funcdoc}{bool path\_push\_vertex(Path *p, uint32\_t v, Graph
*G)}
Pushes vertex \texttt{v} onto path \texttt{p}. The \texttt{length} of the
path is \emph{increased} by the edge weight connecting the vertex at the top of
the stack and \texttt{v}. Return \texttt{true} if the vertex was
successfully pushed and \texttt{false} otherwise.
\end{funcdoc}

\begin{funcdoc}{bool path\_pop\_vertex(Path *p, uint32\_t *v, Graph
*G)}
  Pops the \texttt{vertices} stack, passing the popped vertex back through
  the pointer \texttt{v}. The \texttt{length} of the path is
  \emph{decreased} by the edge weight connecting the vertex at the top of
  the stack and the popped vertex. Return \texttt{true} if the vertex was
  successfully popped and \texttt{false} otherwise.
\end{funcdoc}

\begin{funcdoc}{uint32\_t path\_vertices(Path *p)}
  Returns the number of vertices in the path.
\end{funcdoc}

\begin{funcdoc}{uint32\_t path\_length(Path *p)}
  Returns the length of the path.
\end{funcdoc}

\begin{funcdoc}{void path\_copy(Path *dst, Path *src)}
  Assuming that the destination path \texttt{dst} is properly initialized,
  makes \texttt{dst} a copy of the source path \texttt{src}. This will
  require making a copy of the \texttt{vertices} stack as well as the
  \texttt{length} of the source path.
\end{funcdoc}

\begin{funcdoc}{void path\_print(Path *p, FILE *outfile, char
*cities[])}
  Prints out a path to \texttt{outfile} using \texttt{fprintf()}. Requires
  a call to \texttt{stack\_print()}, as defined in \S 7.10, in order to
  print out the contents of the \texttt{vertices} stack.
\end{funcdoc}

% \section{Stacks, Revisited}

% \epigraphwidth=0.55\textwidth \epigraph{\emph{I believe that what we become depends on
% what our fathers teach us at odd moments, when they aren't trying to teach us.
% We are formed by little scraps of wisdom.}}{---Umberto Eco, \emph{Foucault's Pendulum}}

% \noindent
% You will use the stack that you implemented for assignment 3 with slight
% modifications. If there were any problems with your stack for that
% assignment, make sure to fix them for this assignment. Here is the
% modified stack interface for this assignment.

% \begin{clisting}{}
% struct Stack {
%     uint32_t top;
%     uint32_t capacity;
%     uint32_t *items;
% };
% \end{clisting}

% \subsection{\texttt{Stack *stack\_create(uint32\_t capacity)}}

% The constructor function for a \texttt{Stack}. The \texttt{top} of a
% stack should be initialized to 0. The capacity of a stack is set to the
% specified capacity. The specified capacity also indicates the number of
% items to allocate memory for, the items in which are held in the
% dynamically allocated array \texttt{items}.

% \subsection{\texttt{void stack\_delete(Stack **s)}}

% The destructor function for a stack. Remember to set the pointer
% \texttt{s} to \texttt{NULL}.

% \subsection{\texttt{bool stack\_empty(Stack *s)}}

% Return \texttt{true} if the stack is empty and \texttt{false} otherwise.

% \subsection{\texttt{bool stack\_full(Stack *s)}}

% Return \texttt{true} if the stack is full and \texttt{false} otherwise.

% \subsection{\texttt{uint32\_t stack\_size(Stack *s)}}

% Return the number of items in the stack.

% \subsection{\texttt{bool stack\_push(Stack *s, uint32\_t x)}}

% If the stack is full prior to pushing the item \texttt{x}, return
% \texttt{false} to indicate failure.  Otherwise, push the item and return
% \texttt{true} to indicate success.

% \subsection{\texttt{bool stack\_peek(Stack *s, uint32\_t *x)}}

% Peeking into a stack is synonymous with querying a stack about the
% element at the top of the stack.  If the stack is empty prior to peeking
% into it, return \texttt{false} to indicate failure.

% \subsection{\texttt{bool stack\_pop(Stack *s, uint32\_t *x)}}

% If the stack is empty prior to popping it, return \texttt{false} to
% indicate failure. Otherwise, pop the item, set the value in the memory
% \texttt{x} is pointing to as the popped item, and return \texttt{true}
% to indicate success.

% \subsection{\texttt{void stack\_copy(Stack *dst, Stack *src)}}

% Assuming that the destination stack \texttt{dst} is properly initialized,
% make \texttt{dst} a copy of the source stack \texttt{src}. This means
% making the \emph{contents} of \texttt{dst->items} the same as
% \texttt{src->items}. The top of \texttt{dst} should also match the top
% of \texttt{src}.

% \subsection{\texttt{void stack\_print(Stack *s, FILE *outfile, char
% *cities[])}}

% Prints out the contents of the stack to \texttt{outfile} using
% \texttt{fprintf()}. Working through each vertex in the stack starting
% from the \emph{bottom}, print out the name of the city each vertex
% corresponds to. This function will be given to aid you.

% \begin{clisting}{}
% void stack_print(Stack *s, FILE *outfile, char *cities[]) {
%     for (uint32_t i = 0; i < s->top; i += 1) {
%         fprintf(outfile, "%s", cities[s->items[i]]);
%         if (i + 1 != s->top) {
%             fprintf(outfile, " -> ");
%         }
%     }
%     fprintf(outfile, "\n");
% }
% \end{clisting}
