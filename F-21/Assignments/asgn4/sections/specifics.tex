\section{Specifics}

\epigraphwidth=0.5\textwidth
\epigraph{\emph{The gladdest moment in
human life, methinks, is a departure into unknown lands.}}{---Sir
Richard Burton}

\noindent
Here are the specifics for your program implementation.

\begin{enumerate}
  \item Parse command-line options with looped calls to
    \texttt{getopt()}. This should be familiar from assignments 2 and 3.
  \item Scan in the first line from the input. This will be the
    number of vertices, or cities, that will be in the graph. Print an
    error if the number specified is greater than \texttt{VERTICES}, the
    macro defined in \texttt{vertices.h}.
  \item Assuming the number of specified vertices is $n$, read the next
    $n$ lines from the input using \texttt{fgets()}. Each line is the
    name of a city. Save the name of each city to an array. You will
    want to either make use of \texttt{strdup()} from
    \texttt{<string.h>} or implement your own \texttt{strdup()}
    function. If the line is malformed, print an error
    and exit the program. Note: using \texttt{fgets()} will leave in the
    newline character at the end, so you will manually have to change it
    to the null character to remove it.
  \item Create a new graph $G$, making it undirected if specified.
  \item Scan the input line by line using \texttt{fscanf()} until the
    end-of-file is reached.  Add each edge to $G$. If the line is
    malformed, print an error and exit the program.
  \item Create two paths. One will be for tracking the current traveled path and
    the other for tracking the shortest found path.
  \item Starting from the origin vertex, defined by the macro
    \texttt{START\_VERTEX} in \texttt{vertices.h}, perform a depth-first
    search on $G$ to find the shortest Hamiltonian path. Here is an
    example function prototype that you may use as the recursive
    depth-first function:

\begin{clisting}{}
void dfs(Graph *G, uint32_t v, Path *curr, Path *shortest, char *cities[], FILE *outfile);
\end{clisting}

    The parameter \texttt{v} is the vertex that you are currently on.
    The currently traversed path is maintained with \texttt{curr}. The
    shortest found path is tracked with \texttt{shortest}. The array of
    city names is \texttt{cities}. Finally, \texttt{outfile} is the
    output to print to.

  \item After the search, print out the length of the shortest path
    found, the path itself (remember to return back to the origin), and
    the number of calls to \texttt{dfs()}.

\begin{shlisting}{}
$ ./tsp < mythical.graph
Path length: 21
Path: Asgard -> Shangri-La -> Olympus -> Elysium -> Asgard
Total recursive calls: 4
\end{shlisting}

    If the verbose command-line option was enabled, print out \emph{all}
    the Hamiltonian paths that were found as well. It is recommended
    that you print out the paths as you find them.

\begin{shlisting}{}
$ ./tsp -v < ucsc.graph
Path length: 7
Path: Cowell -> Stevenson -> Merrill -> Cowell
Path length: 6
Path: Cowell -> Merrill -> Stevenson -> Cowell
Path length: 6
Path: Cowell -> Merrill -> Stevenson -> Cowell
Total recursive calls: 5
\end{shlisting}
\end{enumerate}

\textcolor{red}{The output of your program must match the output of the
reference program when given a properly formatted graph exactly to
receive full credit.}
