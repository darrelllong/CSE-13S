\section{Deliverables}

You will need to turn in the following source code and header files:

\begin{enumerate}
  \item \texttt{bbp.c}: This contains the implementation of the
    Bailey-Borwein-Plouffe formula to approximate $\pi$ and the function
    to return the number of computed terms.
  \item \texttt{e.c}: This contains the implementation of the Taylor
    series to approximate Euler's number $e$ and the function to return
    the number of computed terms.
  \item \texttt{euler.c}: This contains the implementation of Euler's
    solution used to approximate $\pi$ and the function to return the
    number of computed terms.
  \item \texttt{madhava.c}: This contains the implementation of the
    Madhava series to approximate $\pi$ and the function to return the
    number of computed terms.
  \item \texttt{mathlib-test.c}: This contains the \texttt{main()}
    function which tests each of your math library functions.
  \item \texttt{mathlib.h}: This contains the interface for your math
    library.
  \item \texttt{newton.c}: This contains the implementation of the
    square root approximation using Newton's method and the function to
    return the number of computed iterations.
  \item \texttt{viete.c}: This contains the implementation of
    Vi\`{e}te's formula to approximate $\pi$ and the function to return
    the number of computed factors.
\end{enumerate}

You may have other source and header files, but \emph{do not make things
over complicated}. \textcolor{red}{Any additional source code and header
files that you may use must not use global variables.}
You will also need to turn in the following:

\begin{enumerate}
  \item \texttt{Makefile}:
    \begin{itemize}
      \item \texttt{CC = clang} must be specified.
      \item \texttt{CFLAGS = -Wall -Wextra -Werror -Wpedantic} must be specified.
      \item \texttt{make} must build the \texttt{mathlib-test}
        executable, as should \texttt{make all} and \texttt{make
        mathlib-test}.
      \item \texttt{make clean} must remove all files that are compiler
        generated.
      \item \texttt{make format} should format all your source code,
        including the header files.
    \end{itemize}
  \item \texttt{README.md}: This must use proper Markdown syntax. It
    must describe how to use your program and \texttt{Makefile}. It
    should also list and explain any command-line options that your
    program accepts. Any false positives reported by \texttt{scan-build}
    should be documented and explained here as well. Note down any known
    bugs or errors in this file as well for the graders.
  \item \texttt{DESIGN.pdf}: This document \emph{must} be a proper
    PDF\@. This design document must describe your design and design
    process for your program with enough detail such that a sufficiently
    knowledgeable programmer would be able to replicate your
    implementation. \textcolor{red}{This does not mean copying your
    entire program in verbatim}. You should instead describe how your
    program works with supporting pseudocode.
  \item \texttt{WRITEUP.pdf}: This document \emph{must} be a proper
    PDF\@. This writeup must include, at least, the following:
      \begin{itemize}
        \item Graphs displaying the difference between the values
          reported by your implemented functions and that of the math
          library's. Use a \textsc{Unix} tool --- not some website ---
          to produce these graphs. \texttt{gnuplot} is recommended.
          Attend section for examples of using \texttt{gnuplot} and
          other \textsc{Unix} tools. An example script for using
          \texttt{gnuplot} to help plot your graphs will be supplied in
          the resources repository.
        \item Analysis and explanations for any discrepancies and
          findings that you glean from your testing.
      \end{itemize}
\end{enumerate}
