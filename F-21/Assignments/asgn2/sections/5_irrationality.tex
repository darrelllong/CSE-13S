\section{The Problem of Irrationality}
\epigraph{\emph{What good your beautiful proof on the transcendence of $\pi$: Why investigate such problems, given that irrational numbers do not even exist?}}{---Leopold Kronecker}


\noindent
Hippasus of Metapontum (c.\xspace 530 -- c.\xspace 450 BC) was a
Pythagorean philosopher, usually credited with the discovery of the
\emph{irrational numbers}. According to Prof.\xspace Dimotakis, the
existence of irrational numbers was known to the Pythagoreans but
it was a closely guarded secret. Hippasus was so enthralled by the
knowledge that he went to the center of town and expounded the
discovery.  This so angered his fellow Pythagoreans, that they
drowned him in the sea.  Pappus of Alexandria merely says that the
knowledge of irrational numbers originated in the Pythagorean school,
and that the member who first divulged the secret perished by
drowning.

\begin{quote}
\emph{It is related to Hippasus that he was a Pythagorean, and that,
owing to his being the first to publish and describe the sphere
from the twelve pentagons, he perished at sea for his impiety, but
he received credit for the discovery, though really it all belonged
to \emph{him} (for in this way they refer to Pythagoras, and they
do not call him by his name).}

        \hfill{---Iamblichus the Syrian}
\end{quote}

Among irrational numbers are $\pi$ most commonly known as the ratio the
circumference of a circle to its diameter, Euler's number $e$, the
golden ratio $\varphi=\tfrac{1+\sqrt{5}}{2}$, and, as we know, the
square root of two. In fact, one can show that all square roots of
natural numbers, other than of perfect squares, are irrational.

The irrational numbers are all the real numbers (denoted by
$\mathbb{R}$) which are not rational numbers (denoted by $\mathbb{Q}$)
and those are \emph{equinumerous} with the integers (denoted by
$\mathbb{Z}$). That is, irrational numbers cannot be expressed as the
ratio of two integers. You may also be surprised to learn that
$|\mathbb{Q}| = |\mathbb{Z}|$.

When the ratio of lengths of two line segments is an irrational number,
the line segments are also described as being \emph{incommensurable},
meaning that they share no \emph{measure} in common, that is, there is
no length, no matter how short, that could be used to express the
lengths of both of the two segments as integers. Consider the triangle
$ABC$ where sides $a=1$ and $b=1$ then by the Pythagorean theorem $c^2 =
a^2 + b^2$, thus, $c = \sqrt{a^2+b^2} = \sqrt{1^2+1^2} = \sqrt{2}$. And
for this, Hippasus lost his life.

The first number to be proved irrational was the $\sqrt{2}$, probably by
our old friend Hippasus. Proofs by ancient Greeks were usually geometric
in nature, and for many, difficult to follow. Instead, let us use this
classical proof.  Suppose that $\sqrt{2}$ is rational, which means that
there exist $p$ and $q$ such that $$\sqrt{2} = \frac{p}{q}.$$ We assume
that $p$ and $q$ have no common factors, in other words, $\gcd(p,q) =1
$. If there are common factors, then we simply cancel them using the
usual method for reducing fractions. We then square both sides of the
equation resulting in $$2 = \frac{p^2}{q^2},$$ which means that $p^2 = 2
q^2$, and so $p^2$ is \emph{even}. That means that $p$ is even, and so
$p^2$ is a multiple of $4$. We can divide out $2$ and the quotient is
still even, including $q^2$, and so is $q$. If both $p$ and $q$ are
even, then they share $2$ as a common factor, contradicting our
assumption.

Like all real numbers, irrational numbers can be expressed (approximately) in positional
notation, notably as a decimal number. In the case of irrational
numbers, the decimal expansion does not terminate, nor end with a
repeating sequence. Conversely, a decimal expansion that terminates or
repeats must be a rational number. These are provable properties of
rational numbers and positional number systems, and are not used as
definitions in mathematics.

As a consequence of Georg Cantor's (1845--1918) proof that the real
numbers are uncountable and the rationals countable, it follows that
almost all real numbers are irrational. In other words, there are more
irrational numbers than there are rational numbers (in fact,
$|\mathbb{R}-\mathbb{Q}| = |\mathbb{R}|$).

In practice, this means that we can never exactly represent an
irrational number, even if we had infinite time and infinite space in
which to do so. We must instead be content with approximations, but
these must be of sufficient accuracy for us to engage in science,
engineering, and technology (and in the case of $\varphi$, art and
architecture).

Since, aside from perfect squares, all square roots are irrational, we
must have a method to approximately compute them. In our case, to
compute $\sqrt{x}$, you will use Newton's method, also called the
\emph{Newton-Raphson method}, by computing the inverse of $x^2$. It is
an iterative algorithm to approximate roots of real-valued functions,
\emph{i.e.}\xspace, solving $f(x) = 0$. Starting with an initial guess,
each iteration of Newton's method produces successively better
approximations. A \emph{Newton iterate} is defined as:
$$
x_{k+1} = x_k - \frac{f(x_k)}{f'(x_k)} .
$$
Each guess $x_{k+1}$ gives a successive improvement over the previous
guess $x_k$. In essence, we are using the \emph{slope of the line} at
the evaluation point to guide the next guess.  The function begins with
an initial guess $x_0 = 1.0$ that it uses to compute better
approximations. \texttt{sqrt()} is sufficiently calculated once the
value converges, \emph{i.e.}\xspace when the difference between
consecutive approximations is sufficiently small. In this case, $f(x) =
x^2 - y$, so you can see that $f(x)=0$ when $x = \sqrt{y}$.

\begin{pylisting}{}
def sqrt(x):
    z = 0.0
    y = 1.0
    while abs(y - z) > epsilon:
        z = y
        y = 0.5 * (z + x / z)
    return y
\end{pylisting}

% $$
% \frac{1}{32} \sqrt{\frac{1}{2} \left(2+\sqrt{2}\right) \left(2+\sqrt{2+\sqrt{2}}\right)
%    \left(2+\sqrt{2+\sqrt{2+\sqrt{2}}}\right)
%    \left(2+\sqrt{2+\sqrt{2+\sqrt{2+\sqrt{2}}}}\right)
%    \left(2+\sqrt{2+\sqrt{2+\sqrt{2+\sqrt{2+\sqrt{2}}}}}\right)}
% $$
