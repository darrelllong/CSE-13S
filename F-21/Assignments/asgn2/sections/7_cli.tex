\section{Command-line Options}
\epigraphwidth=0.6\textwidth
\epigraph{\emph{A few dud universes can really clutter up your basement.}}%
{---Neal Stephenson, \emph{In the Beginning\ldots Was the Command Line}}

\noindent
Your test harness will determine which of your implemented functions to
run through the use of \emph{command-line option}. In most \textbf{C}
programs, the \texttt{main()} function has two parameters: \texttt{int
argc} and \texttt{char **argv}. A command, such as \texttt{./exec arg1
arg2}, is split into an array of strings referred as arguments. The
parameter \texttt{argv} is this array of strings. The parameter
\texttt{argc} serves as the argument counter, which is the number of
arguments that were supplied. Try the following code, and make sure that
you understand it.

\begin{clisting}{Printing out supplied command-line arguments.}
#include <stdio.h>

int main(int argc, char **argv) {
    for (int i = 0; i < argc; i += 1) {
        printf("argv[%d] = %s\n", i, argv[i]);
    }
    return 0;
}
\end{clisting}

A command-line option is an argument, usually prefixed with a hyphen,
that modifies the behavior of a command or program. They are typically
parsed using the \texttt{getopt()} function. \textcolor{red}{Do not
attempt to parse the command-line arguments yourself.} Instead, use the
\texttt{getopt()} function. Command-line options must be defined in
order for \texttt{getopt()} to parse them. These options are defined in
a string, where each character in the string corresponds to an option
character that can be specified on the command-line. Upon running the
executable, \texttt{getopt()} should be used to scan through the
command-line arguments, checking for option characters in a loop.

\begin{clisting}{Parsing options using \texttt{getopt()}.}
#include <stdio.h>
#include <unistd.h>

#define OPTIONS "pi:"

int main(int argc, char **argv) {
    int opt = 0;
    while ((opt = getopt(argc, argv, OPTIONS)) != -1) {
        switch (opt) {
        case 'p':
            printf("-p option.\n");
            break;
        case 'i':
            printf("-i option: %s is parameter.\n", optarg);
            break;
        }
    }
    return 0;
}
\end{clisting}

This example program supports two command-line options, \texttt{-p} and
\texttt{-i}. Note that the option character \texttt{`i'} in the defined
option string \texttt{OPTIONS} has a trailing colon. The colon
signifies, when the \texttt{-i} option is enabled on the command-line,
that \texttt{getopt()} should search for a option argument following it.
An error is thrown by \texttt{getopt()} if an argument for a option, or
\emph{flag}, requiring one is not supplied.

