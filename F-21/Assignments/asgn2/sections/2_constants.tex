\section{Fundamental Constants}

\epigraph{\emph{Transcendental numbers, they transcend the power of algebraic methods.}}{---Leonhard Euler}

\noindent
We live in a world (especially if you are Platonist) that is described
by and perhaps governed by mathematics and mathematical objects. Our
world is populated by numbers---fundamental constants---that we know to
exist, but which we cannot write exactly using our decimal (or any
positional) number system. We are thus presented with a pleasant
conundrum: How do we calculate these numbers that need in order to
pursue science?

There are many little things that remind us of the wonders of the
physical world in which we live. One of the most beautiful things in
mathematics, which Richard Feynman called ``our jewel,'' is
\emph{Euler's identity}:
$$
e^{i \pi} + 1 = 0
$$
which unites the most fundamental numbers in a single formula. A simple
informal proof will show you why this is so.

The tool we will use for our proof is the \emph{Taylor series} (Brook
Taylor, 1685--1731). Taylor showed that you can write most functions as
an infinite sum (there are some restrictions). This allows us to
evaluate a function $f(x)$ near a point $a$ by writing it like this:
$$
f(x) = \sum_{k=0}^\infty \frac{f^{(k)}(a)}{k!}(x-a)^k = f(a) +
\frac{f'(a)}{1!}(x-a) + \frac{f''(a)}{2!}(x-a)^2 +
\frac{f^{(3)}(a)}{3!}(x-a)^3 + \cdots
$$
where $f'$ is the \emph{first derivative}, $f''$ is the second,
$f^{(3)}$ is the third, and so forth. For example,
$$
\sin x = x - \frac{x^3}{3!} + \frac{x^5}{5!} - \frac{x^7}{7!} +
\frac{x^9}{9!} + \cdots
$$
and similarly,
$$
\cos x = 1 - \frac{x^2}{2!} + \frac{x^4}{4!} - \frac{x^6}{6!} +
\frac{x^8}{8!} + \cdots
$$
Now, notice that \emph{sine} has the odd powers of $x$, while
\emph{cosine} has the even powers of $x$. The reason for this is simple,
it is that we are using $a = 0$, and since the $\sin ' x = \cos x$ and
$\cos ' x = - \sin x$ and since $\sin 0 = 0$ and $\cos 0 = 1$ all the
trigonometric functions drop out and just leave us with powers of $x$.
We also need to look at the exponential function,
$$
e^x = \sum_{k=0}^\infty \frac{x^k}{k!} = 1 + x + \frac{x^2}{2!} +
\frac{x^3}{3!} + \frac{x^4}{4!} + \cdots
$$
If you look carefully at the Taylor series expansion of the exponential
function, you will see that it has both the even and odd powers of $x$,
but the signs are all positive, while for \emph{sine} and \emph{cosine}
half of the terms are negative.

The imaginary number $i = \sqrt{-1}$ solves the problem for us,
$$
e^{i x} = 1+i x-\frac{x^2}{2!}-\frac{i x^3}{3!}+\frac{x^4}{4!}+\frac{i
x^5}{5!}-\frac{x^6}{6!}-\frac{i x^7}{7!}+\frac{x^8}{8!}+\frac{i
x^9}{9!}-\frac{x^{10}}{10!}+\cdots
$$
and look closely at the even numbered terms. You will see that they are
the same as $\cos x$, and if you look at the odd numbered terms, you
will see that they are the same as $i \sin x$ (you simply have to factor
out the $i$). Consequently, we see that
$$
e^{i x} = \cos x + i \sin x.
$$
If you let $x = \pi$, then
$$
e^{i \pi} = \cos \pi + i \sin \pi = -1 + i \times 0 = -1.
$$
Thus, if $e^{i \pi} = -1$ then $e^{i \pi} + 1 = 0$. If you are like most
scientists, you are left with a feeling of awe.

Can we make use of this jewel to calculate a value of $\pi$? We will
start with
        $e^{i \pi} + 1  = 0$ and subtract $1$ from both sides, and we get
        $e^{i \pi}  = -1$.
        We take the square root of both sides:
        $$\sqrt{e^{i \pi}}  = \sqrt{-1}$$ and we get
        $e^\frac{i \pi}{2}  = i$.
        We take the $i^\text{th}$ root of both sides:
        $$\sqrt[i]{e^\frac{i \pi}{2}} = \sqrt[i]{i}$$ which simplifies to
        $e^\frac{\pi}{2} = \sqrt[i]{i}$.
        We are almost finished once we take the natural logarithm of both sides:
        $\log(e^\frac{\pi}{2}) = \log(\sqrt[i]{i})$ yielding
        $\frac{\pi}{2} = \log(\sqrt[i]{i})$ which simplifies to
        $\pi = 2 \log(\sqrt[i]{i})= 2\log(i^{-i})$.
So ultimately we find that $\pi$ is the logarithm of the $i^{\text{th}}$
root of an imaginary number. We may want to contemplate the words of
Benjamin Peirce (1809--1880), a professor of mathematics at Harvard, who
said ``Gentlemen, that is surely true, it is absolutely paradoxical; we
cannot understand it, and we don’t know what it means. But we have
proved it, and therefore we know it is the truth.''
