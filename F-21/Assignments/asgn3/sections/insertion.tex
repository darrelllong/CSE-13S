\section{Insertion Sort}

Insertion Sort is a sorting algorithm that considers elements one at a
time, placing them in their correct, ordered position. It is so simple
and so ancient that we do not know who invented it. Assume an array of
size $n$. For each $k$ in increasing value from $1 \le k \le n$ (using
1-based indexing), Insertion Sort compares the $k$-th element with each
of the preceding elements in descending order until its position is
found. Assume we're sorting an array $A$ in increasing order. We start from
and check if $A[k]$ is in the correct order by comparing it
the element $A[k - 1]$. There are two possibilities at this point:

\begin{enumerate}
  \item \textbf{$A[k]$ is in the right place.} This means that $A[k]$ is
    greater or equal to $A[k - 1]$, and thus we can move onto sorting
    the next element.
  \item \textbf{$A[k]$ is in the wrong place.} Thus, $A[k - 1]$ is
    shifted up to $A[k]$, and the original value of $A[k]$ is further
    compared to $A[k - 2]$. Intuitively, Insertion Sort simply shifts
    elements back until the element to place in sorted order is slotted
    in.
\end{enumerate}

\begin{pylisting}{Insertion Sort in Python}
def insertion_sort(A: list):
    for i in range(1, len(A)):
        j = i
        temp = A[i]
        while j > 0 and temp < A[j - 1]:
            A[j] = A[j - 1]
            j -= 1
        A[j] = temp
\end{pylisting}
