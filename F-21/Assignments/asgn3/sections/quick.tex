\section{Quicksort}

\epigraph{\emph{If debugging is the process of removing software bugs,
then programming must be the process of putting them in.}}{---Edsger
Dijkstra}

\noindent
Quicksort (sometimes called partition-exchange sort) was developed by
British computer scientist C.A.R. ``Tony'' Hoare in 1959 and published
in 1961. It is perhaps the most commonly used algorithm for sorting (by
competent programmers).  When implemented well, it is the fastest known
algorithm that sorts using \emph{comparisons}. It is usually two or
three times faster than its main competitors, Merge Sort and Heapsort.
It does, though, have a worst case performance of $O(n^2)$ while its
competitors are strictly $O(n \log n)$ in their worst case.

Quicksort is a divide-and-conquer algorithm. It partitions
arrays into two sub-arrays by selecting an element from the array and
designating it as a pivot. Elements in the array that are less than the
pivot go to the left sub-array, and elements in the array that are
greater than or equal to the pivot go to the right sub-array.

Note that Quicksort is an \emph{in-place} algorithm, meaning it doesn't
allocate additional memory for sub-arrays to hold partitioned elements.
Instead, Quicksort utilizes a subroutine called \texttt{partition()}
that places elements less than the pivot into the left side of the array
and elements greater than or equal to the pivot into the right side and
returns the index that indicates the division between the partitioned
parts of the array. Quicksort is then applied recursively on the
partitioned parts of the array, thereby sorting each array partition
containing at least one element. Like with the Heapsort algorithm, the
provided Quicksort pseudocode operates on 1-based indexing, subtracting
one to account for 0-based indexing whenever array elements are
accessed.

\begin{pylisting}{Partition in Python}
def partition(A: list, lo: int, hi: int):
    i = lo - 1
    for j in range(lo, hi):
        if A[j - 1] < A[hi - 1]:
            i += 1
            A[i - 1], A[j - 1] = A[j - 1], A[i - 1]
    A[i], A[hi - 1] = A[hi - 1], A[i]
    return i + 1
\end{pylisting}

\begin{pylisting}{Recursive Quicksort in Python}
# A recursive helper function for Quicksort.
def quick_sorter(A: list, lo: int, hi: int):
    if lo < hi:
        p = partition(A, lo, hi)
        quick_sorter(A, lo, p - 1)
        quick_sorter(A, p + 1, hi)

def quick_sort(A: list):
    quick_sorter(A, 1, len(A))
\end{pylisting}
