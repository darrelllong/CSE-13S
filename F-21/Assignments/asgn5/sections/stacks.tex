\section{Stacks}
\epigraphwidth=0.25\textwidth
\epigraph{\emph{The future is unwritten.}}{---Joe Strummer}

You will need to use a \emph{stack} in your decoder to reconstruct a
Huffman tree. The interface of the stack should be familiar from
assignment 4. The difference is that the stack this time around will
store \emph{nodes}. The interface for the stack is defined in
\texttt{stack.h}.

\begin{clisting}{}
struct Stack {
    uint32_t top;
    uint32_t capacity;
    Node **items;
};
\end{clisting}

\begin{funcdoc}{\texttt{Stack *stack\_create(uint32\_t capacity)}}
  The constructor for a stack. The maximum number of nodes the stack can
  hold is specified by \texttt{capacity}.
\end{funcdoc}

\begin{funcdoc}{\texttt{void stack\_delete(Stack **s)}}
  The destructor for a stack. Remember to set the pointer to \texttt{NULL}
  after you free the memory allocated by the stack.
\end{funcdoc}

\begin{funcdoc}{\texttt{bool stack\_empty(Stack *s)}}
  Returns \texttt{true} if the stack is empty and \texttt{false}
  otherwise.
\end{funcdoc}

\begin{funcdoc}{\texttt{bool stack\_full(Stack *s)}}
  Returns \texttt{true} if the stack is full and \texttt{false} otherwise.
\end{funcdoc}

\begin{funcdoc}{\texttt{uint32\_t stack\_size(Stack *s)}}
  Returns the number of nodes in the stack.
\end{funcdoc}

\begin{funcdoc}{\texttt{bool stack\_push(Stack *s, Node *n)}}
  Pushes a node onto the stack. Returns \texttt{false} if the stack is
  full prior to pushing the node and \texttt{true} otherwise to indicate
  the successful pushing of a node.
\end{funcdoc}

\begin{funcdoc}{\texttt{bool stack\_pop(Stack *s, Node **n)}}
  Pops a node off the stack, passing it back through the double pointer
  \texttt{n}. Returns \texttt{false} if the stack is empty prior to
  popping a node and \texttt{true} otherwise to indicate the
  succesuccessful popping of a node.
\end{funcdoc}

\begin{funcdoc}{\texttt{void stack\_print(Stack *s)}}
  A debug function to print the contents of a stack.
\end{funcdoc}
