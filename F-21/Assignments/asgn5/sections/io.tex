\section{I/O}
\epigraphwidth=0.5\textwidth
\epigraph{\emph{
When I was a child I truly loved:
unthinking love as calm and deep
as the North Sea. But I have lived,
and now I do not sleep.}}{---John Gardner, \emph{Grendel}}

\noindent Now that we have covered all the essential ADTs necessary for
the encoder, we will discuss I/O. Instead of the buffered I/O functions
from \texttt{<stdio.h>} that you have become acquainted with in previous
assignments, we will use low-level system calls (\emph{syscalls}) such
as \texttt{read()}, \texttt{write()}, \texttt{open()} and
\texttt{close()}. The former two functions can be included with
\texttt{<unistd.h>} and the latter two can be included with
\texttt{<fcntl.h>}. Functions defined by the following I/O module will
be used by both the encoder and decoder. The interface for the I/O
module will be supplied in \texttt{io.h}. You will notice two
\texttt{extern} variables defined in \texttt{io.h}: \texttt{bytes\_read}
and \texttt{bytes\_written}. These are here for the purposes of
collecting statistics. \textcolor{red}{These two variables \emph{must}
be defined in \texttt{io.c}.}

\begin{funcdoc}{\texttt{int read\_bytes(int infile, uint8\_t *buf, int
nbytes)}}
  This will be a useful wrapper function to perform reads. As you may
  know, the \texttt{read()} syscall \emph{does not} always guarantee
  that it will read all the bytes specified (as is the case with
  \emph{pipes}). For example, a call could be issued to read a a block
  of bytes, but it might only read part of a block. So, we write a
  wrapper function to \emph{loop calls} to \texttt{read()} until we have
  either read all the bytes that were specified (\texttt{nbytes}) into
  the byte buffer \texttt{buf}, or there are no more bytes to read. The
  number of bytes that were read from the input file descriptor,
  \texttt{infile}, is returned. \textcolor{red}{You should use this
  function whenever you need to perform a read.}
\end{funcdoc}

\begin{funcdoc}{\texttt{int write\_bytes(int outfile, uint8\_t *buf, int
nbytes)}}
  This functions is very much the same as \texttt{read\_bytes()}, except
  that it is for looping calls to \texttt{write()}. As you may imagine,
  \texttt{write()} is not guaranteed to write out all the specified
  bytes (\texttt{nbytes}), and so we must loop until we have either
  written out all the bytes specified from the byte buffer \texttt{buf},
  or no bytes were written. The number of bytes written out to the
  output file descriptor, \texttt{outfile}, is returned.
  \textcolor{red}{You should use this function whenever you need to
  perform a write.}
\end{funcdoc}

\begin{funcdoc}{\texttt{bool read\_bit(int infile, uint8\_t *bit)}}
  You should all know by now that it is \emph{not} possible to read a
  single bit from a file. What you \emph{can} do, however, is read in a
  block of bytes into a buffer and dole out bits one at a time. Whenever
  all the bits in the buffer have been doled out, you can simply fill
  the buffer back up again with bytes from \texttt{infile}. This is
  exactly what you will do in this function. You will maintain a
  \texttt{static} buffer of bytes and an index into the buffer that
  tracks which bit to return through the pointer \texttt{bit}. The
  buffer will store \texttt{BLOCK} number of bytes, where \texttt{BLOCK}
  is yet another macro defined in \texttt{defines.h}. This function
  returns \texttt{false} if there are no more bits that can be read and
  \texttt{true} if there are still bits to read. It may help to treat
  the buffer as a \emph{bit vector}.
\end{funcdoc}

\begin{funcdoc}{\texttt{void write\_code(int outfile, Code *c)}}
  The same bit-buffering logic used in \texttt{read\_bit()} will be used
  in here as well. This function will also make use of a \texttt{static}
  buffer (we recommend this buffer to be static to the file, not just
  this function) and an index. Each bit in the code \texttt{c} will be
  buffered into the buffer. The bits will be buffered starting from the
  $0^\text{th}$ bit in \texttt{c}. When the buffer of \texttt{BLOCK}
  bytes is filled with bits, write the contents of the buffer to
  \texttt{outfile}.
\end{funcdoc}

\begin{funcdoc}{\texttt{void flush\_codes(int outfile)}}
  It is not always guaranteed that the buffered codes will align nicely
  with a block, which means that it is possible to have bits leftover in
  the buffer used by \texttt{write\_code()} after the input file has
  been completely encoded. The sole purpose of this function is to write
  out any leftover, buffered bits. Make sure that any extra bits in the
  last byte are zeroed before flushing the codes.
\end{funcdoc}
