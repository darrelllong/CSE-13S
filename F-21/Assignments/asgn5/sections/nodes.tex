\section{Nodes}

The first ADT that we will cover is a \emph{node}. Huffman trees
are composed of nodes, with each node containing a pointer to its left
child, a pointer to its right child, a symbol, and the frequency of that
symbol. The node's frequency is only needed for the encoder.

\begin{clisting}{}
typedef struct Node Node;

struct Node {
    Node *left;         // Pointer to left child.
    Node *right;        // Pointer to right child.
    uint8_t symbol;     // Node's symbol.
    uint64_t frequency; // Frequency of symbol.
};
\end{clisting}

Immediately, we notice that a symbol is a \texttt{uint8\_t}, and not a
\texttt{char}. This is because we want to interpret the input file as
\emph{raw bytes}, not as a string. The following subsections define the
interface for a \texttt{Node} and will be supplied in \texttt{node.h}.
The definition of a \texttt{Node} will be made transparent in order to
simplify things.

\begin{funcdoc}{Node *node\_create(uint8\_t symbol, uint64\_t frequency)}
  The constructor for a node. Sets the node's symbol as \texttt{symbol}
  and its frequency as \texttt{frequency}.
\end{funcdoc}

\begin{funcdoc}{\texttt{void node\_delete(Node **n)}}
  The destructor for a node. Make sure to set the pointer to
  \texttt{NULL} after freeing the memory for a node.
\end{funcdoc}

\begin{funcdoc}{\texttt{Node *node\_join(Node *left, Node *right)}}
  Joins a left child node and right child node, returning a pointer to a
  created parent node. The parent node's left child will be
  \texttt{left} and its right child will be \texttt{right}. The parent
  node's symbol will be \texttt{`\$'} and its frequency the \emph{sum}
  of its \emph{left} child's frequency and its \emph{right} child's
  frequency.
\end{funcdoc}

\begin{funcdoc}{\texttt{void node\_print(Node *n)}}
A debug function to verify that your nodes are created and joined
correctly.
\end{funcdoc}
