\section{Codes}

After constructing a Huffman tree, you will need to maintain a stack of
bits while traversing the tree in order to create a code for each
symbol. We will create a new ADT, a \texttt{Code}, that represents a
stack of bits. The interface for a \texttt{Code} is very much like the
interface for an ADT you will implement later on in the quarter: the
\emph{bit vector}. The logic for setting a bit, clearing a bit, and
getting a bit implemented here will be used later on for your bit
vectors, so make sure to nail down the implementation here to save
yourself trouble in the future.

\begin{clisting}{}
typedef struct {
    uint32_t top;
    uint8_t bits[MAX_CODE_SIZE];
} Code;
\end{clisting}

The \texttt{struct} definition of a \texttt{Code} will be made
transparent. \textcolor{red}{This is done for the sole purpose of being
able to pass a \texttt{struct} by value.} The macro
\texttt{MAX\_CODE\_SIZE} reflects the maximum number of bytes needed to
store any valid code. The definition of this macro --- and other macros
--- will be given in \texttt{defines.h}. You will need to combine your
knowledge of bit vectors and stacks in order to implement this ADT. The
interface, given in \texttt{code.h}, is defined in the the following
subsections.

\begin{clisting}{Macros defined in \texttt{defines.h}}
// 4KB blocks.
#define BLOCK 4096

// ASCII + Extended ASCII.
#define ALPHABET 256

// 32-bit magic number.
#define MAGIC 0xBEEFD00D

// Bytes for a maximum, 256-bit code.
#define MAX_CODE_SIZE (ALPHABET / 8)

// Maximum Huffman tree dump size.
#define MAX_TREE_SIZE (3 * ALPHABET - 1)
\end{clisting}

\begin{funcdoc}{\texttt{Code code\_init(void)}}
  You will immediately notice that this ``constructor'' function is
  unlike any of the other constructor functions you have implemented in
  the past. You may also have noticed, if you glanced slightly ahead,
  that there is no corresponding destructor function. This is an
  engineering decision that was made when considering the constraints of
  the Huffman coding algorithm.

  This function \emph{will not} require any dynamic memory allocation.
  You will simply create a new \texttt{Code} on the stack, setting
  \texttt{top} to 0, and zeroing out the array of bits, \texttt{bits}.
  The initialized \texttt{Code} is then returned.
\end{funcdoc}

\begin{funcdoc}{\texttt{uint32\_t code\_size(Code *c)}}
  Returns the size of the \texttt{Code}, which is exactly the number of
  bits pushed onto the \texttt{Code}.
\end{funcdoc}

\begin{funcdoc}{\texttt{bool code\_empty(Code *c)}}
  Returns \texttt{true} if the \texttt{Code} is empty and \texttt{false}
  otherwise.
\end{funcdoc}

\begin{funcdoc}{\texttt{bool code\_full(Code *c)}}
  Returns \texttt{true} if the \texttt{Code} is empty and \texttt{false}
  otherwise. The maximum length of a code in bits is 256, which we have
  defined using the macro \texttt{ALPHABET}. Why 256? Because there are
  exactly 256 ASCII characters (including the extended ASCII).
\end{funcdoc}

\begin{funcdoc}{\texttt{bool code\_set\_bit(Code *c, uint32\_t i)}}
  Sets the bit at index \texttt{i} in the \texttt{Code}, setting it to
  1. If \texttt{i} is out of range, return \texttt{false}. Otherwise,
  return \texttt{true} to indicate success.
\end{funcdoc}

\begin{funcdoc}{\texttt{bool code\_clr\_bit(Code *c, uint32\_t i)}}
  Clears the bit at index \texttt{i} in the \texttt{Code}, clearing it
  to 0. If \texttt{i} is out of range, return \texttt{false}. Otherwise,
  return \texttt{true} to indicate success.
\end{funcdoc}

\begin{funcdoc}{\texttt{bool code\_get\_bit(Code *c, uint32\_t i)}}
  Gets the bit at index \texttt{i} in the \texttt{Code}. If \texttt{i}
  is out of range, or if bit \texttt{i} is equal to 0, return
  \texttt{false}. Return \texttt{true} if and only if bit \texttt{i} is
  equal to 1.
\end{funcdoc}

\begin{funcdoc}{\texttt{bool code\_push\_bit(Code *c, uint8\_t bit)}}
  Pushes a bit onto the \texttt{Code}. The value of the bit to push is
  given by \texttt{bit}. Returns \texttt{false} if the \texttt{Code} is
  full prior to pushing a bit and \texttt{true} otherwise to indicate
  the successful pushing of a bit.
\end{funcdoc}

\begin{funcdoc}{\texttt{bool code\_pop\_bit(Code *c, uint8\_t *bit)}}
  Pops a bit off the \texttt{Code}. The value of the popped bit is
  passed back with the pointer \texttt{bit}. Returns \texttt{false} if
  the \texttt{Code} is empty prior to popping a bit and \texttt{true}
  otherwise to indicate the successful popping of a bit.
\end{funcdoc}

\begin{funcdoc}{\texttt{void code\_print(Code *c)}}
  A debug function to help you verify whether or not bits are pushed
  onto and popped off a \texttt{Code} correctly.
\end{funcdoc}
