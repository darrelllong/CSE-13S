\section{The Encoder}
\epigraphwidth=0.5\textwidth
\epigraph{\emph{No one with access to a convertible, an empty highway and a good radio station should ever need a psychiatrist.}}{---Terry Allen}

\noindent
Your first task for this assignment is to implement a Huffman encoder.
This encoder will read in an input file, find the Huffman encoding of
its contents, and use the encoding to compress the file. Your encoder
program, named \texttt{encode}, must support any combination of the
following command-line options:

\begin{itemize}
  \item \textbf{\texttt{-h}}\,: Prints out a help message describing the purpose
    of the program and the command-line options it accepts, exiting the
    program afterwards. Refer to the reference program in the resources
    repo for an idea of what to print.

  \item \textbf{\texttt{-i infile}}\,: Specifies the input file to encode using
    Huffman coding. The default input should be set as \texttt{stdin}.

  \item \textbf{\texttt{-o outfile}}\,: Specifies the output file to
    write the compressed input to. The default output should be set as
    \texttt{stdout}.

  \item \textbf{\texttt{-v}}\,: Prints compression statistics to
    \texttt{stderr}. These statistics include the uncompressed file
    size, the compressed file size, and \emph{space saving}. The formula
    for calculating space saving is:
    \[
        100 \times (1 - (\text{compressed size} / \text{uncompressed
        size})).
    \]
    Refer to the reference program in the resources repository for the
    exact output.
\end{itemize}

\noindent The algorithm to encode a file, or to compress it, is as follows:

\begin{enumerate}
  \item Compute a histogram of the file. In other words, count the
    number of occurrences of each unique symbol in the file.

  \item Construct the Huffman tree using the computed histogram. This
    will require a \emph{priority queue}.

  \item Construct a code table. Each index of the table represents a
    symbol and the value at that index the symbol's code. You will need
    to use a \emph{stack of bits} and perform a traversal of the Huffman
    tree.

  \item Emit an encoding of the Huffman tree to a file. This will be
    done through a \emph{post-order traversal} of the Huffman tree. The
    encoding of the Huffman tree will be referred to as a \emph{tree
    dump}.

  \item Step through each symbol of the input file again. For each
    symbol, emit its code to the output file.
\end{enumerate}
