\section{An Small Example}
\epigraphwidth=0.3\textwidth
\epigraph{\emph{I don't like half the folks I love.}}{---Paul Thorn}

We will now go through a simple example of Huffman encoding and
decoding. Assume we are encoding the input \texttt{banana}. We will need
to create a histogram of the input.

\begin{center}
  \begin{tabular}{ |c|c| }
   \hline
     Symbol & Frequency \\
   \hline
     \texttt{a} & 3 \\
   \hline
     \texttt{b} & 1 \\
   \hline
     \texttt{n} & 2 \\
   \hline
  \end{tabular}
\end{center}

Nodes corresponding to each symbol are then placed into a priority
queue. The lower the frequency of the symbol, the higher the priority of
its corresponding node. The priority queue will be visually represented
using a \emph{min heap}. The priority queue is used to construct the
Huffman tree, which will be shown alongside the priority queue. The
Huffman tree is initially empty.

\begin{figure}[H]
  \centering
  \begin{subfigure}[b]{0.4\linewidth}
    \centering
    \begin{forest} for tree={circle,draw,inner sep=5pt,l=10pt,l sep=6pt,s sep=14pt}
      [\texttt{b:1}
        [\texttt{a:3}]
        [\texttt{n:2}]
      ]
    \end{forest}
    \caption{Priority queue}
  \end{subfigure}
  \begin{subfigure}[b]{0.4\linewidth}
    \caption{Huffman tree}
  \end{subfigure}
\end{figure}

First, we dequeue the node containing \texttt{b}. After fixing the heap,
we dequeue the node containing \texttt{s}. We have now dequeued the two
nodes of highest priority. We \emph{join} the dequeued nodes together
into a new node, giving it the symbol \texttt{\$}. The frequency of the
new node is the sum of the frequency of \texttt{b} and the frequency of
\texttt{s}. The new node is then enqueued.

\begin{figure}[H]
  \centering
  \begin{subfigure}[b]{0.4\linewidth}
    \centering
    \begin{forest} for tree={circle,draw,inner sep=5pt,l=10pt,l sep=6pt,s sep=14pt}
      [\texttt{a:3}
        [\texttt{\$:3}]
        [,phantom]
      ]
    \end{forest}
    \caption{Priority queue}
  \end{subfigure}
  \begin{subfigure}[b]{0.4\linewidth}
    \centering
      \begin{forest} for tree={circle,draw,inner sep=5pt,l=10pt,l sep=6pt,s sep=14pt}
        [\texttt{\$:3}
          [\texttt{b:1}]
          [\texttt{n:2}]
        ]
      \end{forest}
    \caption{Huffman tree}
  \end{subfigure}
\end{figure}

We repeat the same step of dequeuing two nodes and joining them
together.
We will give the new node the symbol \texttt{!}, just for
distinction. The new node is then enqueued.

\begin{figure}[H]
  \centering
  \begin{subfigure}[b]{0.4\linewidth}
    \centering
    \begin{forest} for tree={circle,draw,inner sep=5pt,l=10pt,l sep=6pt,s sep=14pt}
      [\texttt{!:6}]
    \end{forest}
    \caption{Priority queue}
  \end{subfigure}
  \begin{subfigure}[b]{0.4\linewidth}
    \centering
      \begin{forest} for tree={circle,draw,inner sep=5pt,l=10pt,l sep=6pt,s sep=14pt}
        [\texttt{!:6}
          [\texttt{a:3}]
          [\texttt{\$:3}
            [\texttt{b:1}]
            [\texttt{n:2}]
          ]
        ]
      \end{forest}
    \caption{Huffman tree}
  \end{subfigure}
\end{figure}

We stop when there is exactly one node left in the priority queue. The
remaining node is the root of the constructed Huffman tree. Here is some
Python pseudocode to help with the construction of the Huffman tree.

\begin{pylisting}{}
def construct(q):
    while len(q) > 1:
        left = dequeue(q)
        right = dequeue(q)
        parent = join(left, right)
        enqueue(q, parent)
    root = dequeue(q)
    return root
\end{pylisting}

We will now proceed to assign unique codes to each symbol by traversing
the tree. Here is some more Python pseudocode to help with the building
of the codes.

\begin{pylisting}{}
Code c = code_init()

def build(node, table):
    if node is not None:
        if not node.left and not node.right:
            table[node.symbol] = c
        else:
            push_bit(c, 0)
            build(node.left, table)
            pop_bit(c)

            push_bit(c, 1)
            build(node.right, table)
            pop_bit(c)
\end{pylisting}

The assigned codes after building the tree is shown in the following
code table:

\begin{figure}[H]
  \centering
  \begin{subfigure}[b]{0.4\linewidth}
    \centering
    \begin{tabular}{ |c|c| }
     \hline
       Symbol & Code \\
     \hline
       \texttt{a} & 0 \\
     \hline
       \texttt{b} & 10 \\
     \hline
       \texttt{n} & 11 \\
     \hline
    \end{tabular}
    \caption{Code table}
  \end{subfigure}
  \begin{subfigure}[b]{0.4\linewidth}
    \centering
      \begin{forest} for tree={circle,draw,inner sep=5pt,l=10pt,l sep=6pt,s sep=14pt}
        [\texttt{!:6}
          [\texttt{a:3}, edge label={node[midway,above left,font=\scriptsize]{0}}]
          [\texttt{\$:3}, edge label={node[midway,above right,font=\scriptsize]{1}}
            [\texttt{b:1}, edge label={node[midway,above left,font=\scriptsize]{0}}]
            [\texttt{n:2}, edge label={node[midway,above right,font=\scriptsize]{1}}]
          ]
        ]
      \end{forest}
    \caption{Huffman tree}
  \end{subfigure}
\end{figure}

Next, we dump the constructed tree and output the encoding of the input.
The tree dump produced through the post-order traversal of the tree
following the specification of \texttt{dump\_tree()}: \texttt{LaLbLnII}.
Here is some Python pseudocode to help with the dumping of the tree.

\begin{pylisting}{}
def dump(outfile, root):
    if root:
        dump(outfile, root.left)
        dump(outfile, root.right)

        if not root.left and not root.right:
            # Leaf node.
            write('L')
            write(node.symbol)
        else:
            # Interior node.
            write('I')
\end{pylisting}

All that remains is to iterate over each symbol in the input,
\texttt{banana}, and emit each symbol's corresponding code. The binary
emitted, written from the \textbf{least significant bit (LSB)} to the
\textbf{most significant bit (MSB)}, is: \texttt{100110110}.

To decode this, we need to first reconstruct the Huffman tree from its
tree dump. We iterate over the tree dump (\texttt{LaLbLnII}), following
the algorithm described in step (\ref{rebuild}) in
\S\ref{decoder_specifics}. The first symbol we push onto the stack is
\texttt{a}. The frequency of the node is not of particular importance,
so we leave it as 0.

\begin{figure}[H]
  \centering
  \begin{subfigure}[b]{0.4\linewidth}
    \centering
    \begin{tikzpicture}[stack/.style={rectangle split, rectangle split parts=#1,draw, anchor=center}]
      \node[stack=5]  {
        \nodepart{five}\texttt{a:0}
      };
    \end{tikzpicture}
    \caption{Stack}
  \end{subfigure}
  \begin{subfigure}[b]{0.4\linewidth}
    \caption{Huffman tree}
  \end{subfigure}
\end{figure}

The next symbol we push onto the stack is \texttt{b}.

\begin{figure}[H]
  \centering
  \begin{subfigure}[b]{0.4\linewidth}
    \centering
    \begin{tikzpicture}[stack/.style={rectangle split, rectangle split parts=#1,draw, anchor=center}]
      \node[stack=5]  {
        \nodepart{four}\texttt{b:0}
        \nodepart{five}\texttt{a:0}
      };
    \end{tikzpicture}
    \caption{Stack}
  \end{subfigure}
  \begin{subfigure}[b]{0.4\linewidth}
    \caption{Huffman tree}
  \end{subfigure}
\end{figure}

The last symbol we push onto the stack is \texttt{n}.

\begin{figure}[H]
  \centering
  \begin{subfigure}[b]{0.4\linewidth}
    \centering
    \begin{tikzpicture}[stack/.style={rectangle split, rectangle split parts=#1,draw, anchor=center}]
      \node[stack=5]  {
        \nodepart{three}\texttt{n:0}
        \nodepart{four}\texttt{b:0}
        \nodepart{five}\texttt{a:0}
      };
    \end{tikzpicture}
    \caption{Stack}
  \end{subfigure}
  \begin{subfigure}[b]{0.4\linewidth}
    \centering
    \caption{Huffman tree}
  \end{subfigure}
\end{figure}

Continuing the iteration, we encounter our first \texttt{`I'}. This
means we pop the stack twice to obtain two nodes. The first node popped
is the right child and the second node is the left child. The nodes are
joined together and the newly created node is pushed onto the stack.

\begin{figure}[H]
  \centering
  \begin{subfigure}[b]{0.4\linewidth}
    \centering
    \begin{tikzpicture}[stack/.style={rectangle split, rectangle split parts=#1,draw, anchor=center}]
      \node[stack=5]  {
        \nodepart{four}\texttt{\$:0}
        \nodepart{five}\texttt{a:0}
      };
    \end{tikzpicture}
    \caption{Stack}
  \end{subfigure}
  \begin{subfigure}[b]{0.4\linewidth}
      \centering
      \begin{forest} for tree={circle,draw,inner sep=5pt,l=10pt,l sep=6pt,s sep=14pt}
        [\texttt{\$:0}
          [\texttt{b:0}]
          [\texttt{n:0}]
        ]
      \end{forest}
    \caption{Huffman tree}
  \end{subfigure}
\end{figure}

We encounter another \texttt{`I'}. We repeat what was done before,
popping two nodes, joining them, and pushing the joined node onto the
stack. Like with the encoding example previously, the joined node's
symbol is set to \texttt{!} for distinction.

\begin{figure}[H]
  \centering
  \begin{subfigure}[b]{0.4\linewidth}
    \centering
    \begin{tikzpicture}[stack/.style={rectangle split, rectangle split parts=#1,draw, anchor=center}]
      \node[stack=5]  {
        \nodepart{five}\texttt{!:0}
      };
    \end{tikzpicture}
    \caption{Stack}
  \end{subfigure}
  \begin{subfigure}[b]{0.4\linewidth}
      \centering
      \begin{forest} for tree={circle,draw,inner sep=5pt,l=10pt,l sep=6pt,s sep=14pt}
        [\texttt{!:0}
          [\texttt{a:0}, edge label={node[midway,above left,font=\scriptsize]{0}}]
          [\texttt{\$:0}, edge label={node[midway,above right,font=\scriptsize]{1}}
            [\texttt{b:0}, edge label={node[midway,above left,font=\scriptsize]{0}}]
            [\texttt{n:0}, edge label={node[midway,above right,font=\scriptsize]{1}}]
          ]
        ]
      \end{forest}
    \caption{Huffman tree}
  \end{subfigure}
\end{figure}

The last node in the stack is the root of the reconstructed Huffman
tree. To decode the encoded input, we iterate over the emitted binary:
\texttt{100110110}. Starting from the root of the tree, we walk down to
the left child on a 0-bit and walk down to the right child on a 1-bit.
Whenever a leaf is reached, we output its symbol and reset back to the
root of the tree. Following this procedure produces the decoded output:
\texttt{banana}.
