\documentclass{article}
\usepackage{fullpage,fourier,amsmath,amssymb}
\usepackage{listings,color,url,hyperref}
\usepackage{epigraph, graphicx}
\usepackage[x11names]{xcolor}

\title{Assignment 1 \\ Left, Right and Center}
\author{Prof. Darrell Long \\
CSE 13S -- Winter 2020}
\date{Due: January 19$^\text{th}$ at 11:59\,pm}

\usepackage{fancyhdr}
\pagestyle{fancy}
\fancyhf{}

\fancypagestyle{plain}{%
  \fancyhf{}
  \renewcommand{\headrulewidth}{0pt}
  \renewcommand{\footrulewidth}{0pt}
  \lfoot{\textcopyright{} 2021 Darrell Long}
  \rfoot{\thepage}
}

\pagestyle{plain}

\definecolor{codegreen}{rgb}{0,0.5,0}
\definecolor{codegray}{rgb}{0.5,0.5,0.5}
\definecolor{codepurple}{rgb}{0.58,0,0.82}

\lstloadlanguages{C,make,python,fortran,bash}

\lstdefinestyle{bashstyle}{
  language=bash,
  basicstyle=\small\ttfamily,
  numbers=none,
  frame=tb,
  columns=fullflexible,
  backgroundcolor=\color{yellow!10},
  linewidth=0.9\linewidth,
  xleftmargin=0.1\linewidth,
  aboveskip=1.25em,
  belowskip=1.25em,
  breakatwhitespace=false,
  breaklines=true,
  showstringspaces=false,
  keepspaces=true,
  showspaces=false,
  showtabs=false,
  tabsize=4
}

\lstdefinestyle{plainstyle}{
  keywords={},
  stringstyle=\color{black},
  basicstyle=\small\ttfamily,
  numbers=none,
  frame=tb,
  columns=fullflexible,
  backgroundcolor=\color{yellow!10},
  linewidth=0.9\linewidth,
  xleftmargin=0.1\linewidth,
  aboveskip=1.25em,
  belowskip=1.25em,
  breakatwhitespace=false,
  breaklines=true,
  showstringspaces=false,
  keepspaces=true,
  showspaces=false,
  showtabs=false,
  tabsize=4
}

\lstdefinestyle{c99}{
    language=C,
    morekeywords={
      bool, uint8_t, uint16_t, uint32_t, uint64_t,
      int8_t, int16_t, int32_t, int64_t
    },
    upquote=true,
    commentstyle=\color{codegreen},
    keywordstyle=\color{magenta},
    identifierstyle=\color{black},
    stringstyle=\color{codepurple},
    basicstyle=\ttfamily,
    breakatwhitespace=false,
    breaklines=true,
    captionpos=b,
    keepspaces=true,
    showspaces=false,
    showstringspaces=false,
    showtabs=false,
    numbers=left,
    numbersep=7pt,
    numberstyle=\ttfamily\color{black!25},
    xleftmargin=12pt,
    tabsize=4
}

\lstdefinestyle{python}{
    language=python,
    upquote=true,
    commentstyle=\color{codegreen},
    keywordstyle=\color{magenta},
    identifierstyle=\color{black},
    stringstyle=\color{codepurple},
    basicstyle=\ttfamily,
    breakatwhitespace=false,
    breaklines=true,
    captionpos=b,
    keepspaces=true,
    showspaces=false,
    showstringspaces=false,
    showtabs=false,
    numbers=left,
    numbersep=7pt,
    numberstyle=\ttfamily\color{black!25},
    xleftmargin=12pt,
    tabsize=4
}

\usepackage[most]{tcolorbox}

\newtcolorbox{prelab}[1]{
  colback=red!5!white, colframe=red!75!black, title=#1
}

\newtcblisting{codelisting}[2][]{
  title=#2, #1,
  fonttitle=\bfseries,
  colframe=black!80,
  listing only,
  listing options={basicstyle=\ttfamily,style=c99},
  top=-5pt,
  bottom=-5pt,
  enlarge top by=10pt
}

\newtcblisting{pythonlisting}[2][]{
  title=#2, #1,
  fonttitle=\bfseries,
  colframe=black!80,
  listing only,
  listing options={basicstyle=\ttfamily,style=python},
  top=-5pt,
  bottom=-5pt,
  enlarge top by=5pt
}

\newcommand\Warning{%
 \makebox[1.4em][c]{%
 \makebox[0pt][c]{\raisebox{.1em}{\small!}}%
 \makebox[0pt][c]{\color{red}\Large$\bigtriangleup$}}}%

\lstset{language=C, style=c99}

\newcommand\asgn[1]{asgn#1}

\begin{document}\maketitle

%%%%%%%%%%%%%%%%%%%%%%%%%%%%%%%%%%%%%%%%%%%%%%%%%%%%%%%%

\section{Introduction}
\epigraphwidth=0.5\textwidth
\epigraph{\emph{The gambling known as business looks with austere disfavor
upon the business known as gambling.}}{---Ambrose Bierce}

\noindent We are going to implement a simple game called \emph{Left, Right, and Center}.
It requires no skill, no real decision-making, and a player that is out of the
game can suddenly come back in and often win.

Some of you may have religious or moral prohibitions against gambling. This
program is not gambling, since (i) you neither win nor lose, and (ii) only
fictitious people lose or win any money. But, if you do have any qualms, let
us know and we will give you an alternative assignment.

\section{Playing the Game}
\epigraphwidth=0.75\textwidth
\epigraph{\emph{No sympathy for the devil; keep that in mind. Buy
the ticket, take the ride\ldots and if it occasionally gets a little
heavier than what you had in mind, well\ldots maybe chalk it off
to forced conscious expansion: Tune in, freak out, get beaten.}}{---Hunter
S. Thompson, \emph{Fear \& Loathing in Las Vegas}}

Some number of $k$ players, $1 < k \le 10$, sit around a table.
Each player has in her hand \$3. There are three dice, and each die
has $6$ faces and is labeled: $3 \times \textbf{\textbullet}$, $1
\times \textbf{L}$, $1 \times \textbf{R}$ or $1 \times \textbf{C}$.
As a result, we know that there is a $50$\% chance of rolling
\textbf{\textbullet}, and $16.6\overline{6}$\% chance of rolling
each of \textbf{L}, \textbf{R}, or \textbf{C}.

\begin{enumerate}
\item
Beginning with player $1$, roll the dice:
\begin{enumerate}
\item
If the player has \$3 or more then she rolls three dice;
if she has \$2 then she rolls two dice;
if she has only \$1 then she rolls one die;
if she has no money then she must pass.

\item For each die:
\begin{enumerate}
\item If the player rolls \textbf{L} then she gives \$1 to the player on her \emph{left}.

\item If the player rolls \textbf{R} then she gives \$1 to the player on her \emph{right}.

\item If the player rolls \textbf{C} then she puts \$1 in the pot in the
    \emph{center}.

\item If the player rolls \textbf{\textbullet} then she ignores it.
\end{enumerate}

\end{enumerate}
\item Move to the next player in sequence: to the
right. The players are numbered. There is \texttt{you}. Then there
is the left player which is (\texttt{you} $-$ 1) \texttt{mod} the number
of players and there is the right player which is (\texttt{you} $+$
1) \texttt{mod} the number of players.
Be careful: What does \texttt{-2 \% 10} mean? Consequently, you may find this code useful:
\begin{lstlisting}
//
// Returns the position of the player to the left.
//
// pos:     The position of the current player.
// players: The number of players in the game.
//
uint32_t left(uint32_t pos, uint32_t players) {
  return ((pos + players - 1) % players);
}

//
// Returns the position of the player to the right.
//
// pos:     The position of the current player.
// players: The number of players in the game.
//
uint32_t right(uint32_t pos, uint32_t players) {
  return ((pos + 1) % players);
}
\end{lstlisting}

\item Repeat until only one player has any money remaining (who then wins the pot).
\end{enumerate}

%%%%%%%%%%%%%%%%%%%%%%%%%%%%%%%%%%%%%%%%%%%%%%%%%%%%%%%%

\section{Your Task}
\epigraphwidth=0.6\textwidth
\epigraph{
\emph{Any one who considers arithmetical methods of producing random digits
is, of course, in a state of sin.}}{---John von Neumann, 1951}

\begin{itemize}

\item You must have one source file: \texttt{lrc.c}. Do not name
your source file anything else. \textcolor{red}{You will lose points}.


\item For grading purposes the numbering of the faces matters, and
so they must be defined as follows (do not change it):

\begin{lstlisting}
typedef enum faciem {LEFT, RIGHT, CENTER, PASS} faces;
faces die[] = {LEFT, RIGHT, CENTER, PASS, PASS, PASS};
\end{lstlisting}

\textcolor{red}{This means you may not use \texttt{int} as a substitute.}

\item You must give your players names, and for grading purposes the names
must correspond to these (do not change them):

{\small
\begin{lstlisting}
const char *names[] = {"Dude",   "Walter", "Maude", "Big Lebowski", 
                       "Brandt", "Bunny",  "Karl",  "Kieffer", 
                       "Franz",  "Smokey"};
\end{lstlisting}
}

\item Typing \texttt{make} must build your program and
\texttt{./lrc} must run your program. Since you have not learned
about \texttt{Makefiles} yet, here is one that you can use for now.

\begin{lstlisting}[title=\texttt{Makefile}]
CFLAGS=-Wall -Wextra -Werror -pedantic
CC=clang $(CFLAGS)

lrc     :       lrc.o
        $(CC) -o lrc lrc.o
lrc.o   :       lrc.c
        $(CC) -c lrc.c
clean   :
        rm -f lrc lrc.o
infer   :
        make clean; infer-capture -- make; infer-analyze -- make
\end{lstlisting}

\item You will need to use a random number generator to simulate
rolling a dice. You can do this by calling the function \texttt{rand()}
(read the \emph{man page}) to get a random value. You can then use \texttt{mod}
to limit this value to the range of 0--5 inclusive (in Computer Science, we start from 0).
This value is used to determine what was rolled.

\item In order that your program be \emph{reproducible}, you must start from a
known place. This is accomplished by \emph{setting the random seed}
using the function \texttt{srand()} (again read the \emph{man page}).
Your program will ask for two numbers: the random seed and the
number of players. You should assign these inputs to variables to use
in your program. The random seed completely determines the outcome of your program. If you give it
the same random seed and the number of players you \emph{must} get
the same answer. Here is an example of using \texttt{srand()} and
\texttt{rand()}:
\begin{lstlisting}
srand(1); // This sets your random seed to 1.
int a = rand(); // Declares & initializes variable to random number.
\end{lstlisting}

\item \emph{Comment your code.} Include block comments to tell the grader what a certain block of code does. Use a line comment to explain something that is not as obvious. 
\textcolor{red}{Refer to the coding standards}.

\item Your program must use \texttt{clang-format}.

\item Your program \emph{must} run on the time share. If it does not, your program will receive a 0. To avoid this, test your program on the time share.

\item Your program must pass \texttt{infer} with no errors. If there are any, fix them; for those
that you cannot fix, make sure to document them in your \texttt{README}.

\end{itemize}

In lecture we talked briefly about \emph{random} and \emph{pseudo-random}
numbers. As we know, computers produce pseudo-random numbers,
and in this case it is to your benefit since \emph{reproducibility}
is essential. That means that in reality you program though it
appears to be random is actually \emph{deterministic}. This is why
starting with the same seed produces the same sequence.

%%%%%%%%%%%%%%%%%%%%%%%%%%%%%%%%%%%%%%%%%%%%%%%%%%%%%%%%
\section{Hints}
\epigraph{\emph{It is no use trying to sum people up. One must follow hints,
not exactly what is said, nor yet entirely what is done.}}{---Virginia Woolf}

\begin{enumerate}

\item \emph{\textcolor{red}{Start Now!}} You can always finish early.

\item You may find it helpful to draw out and run through a game to get a feel
    for the rules. Doing so may make it easier to visualize the game and design
    your program.
\item The game itself should be an infinite loop, where the condition to break
    out of this loop is when there is one active player remaining.
\item You should think carefully about what quantities that you
must track in order for your program to function. At a \emph{minimum}
you must keep track of the bank balance of each player, the amount
of money in the pot, and the number of players that are \emph{in}.
Be careful: players that were \emph{out} may be brought back in if
money is passed to the \emph{left} or \emph{right}.

\end{enumerate}

%%%%%%%%%%%%%%%%%%%%%%%%%%%%%%%%%%%%%%%%%%%%%%%%%%%%%%%%

\section{Deliverables}
\epigraph{\emph{The design process is about designing and prototyping and
making. When you separate those, I think the final result
suffers.}}{---Jonathan Ive}

You will need to turn in:

\begin{enumerate}
\item \texttt{lrc.c}: The source file which is your program.

\item \texttt{Makefile}: This is a file that will allow the grader to type
    \texttt{make} to compile your program. Typing \texttt{make} with no
    arguments must build your program.
\begin{itemize}
\item \texttt{CFLAGS=-Wall -Wextra -Werror -Wpedantic} must be included.
\item \texttt{CC=clang} must be specified.
\item \texttt{make clean} must remove all files that are compiler generated.
\item \texttt{make} should build your program, as should \texttt{make all}.
\item Your program executable must be named \texttt{lrc}.
\end{itemize}

\item \texttt{README.md}: This must be in markdown.
This must describe how to use your program and \texttt{Makefile}.

\item \texttt{DESIGN.md}: This must be in markdown. The design document
should describe your design for your program with enough detail
that a sufficiently knowledgeable programmer would be able to
replicate your implementation. This does not mean copying your
entire program in verbatim. You should instead describe how your
program works with supporting pseudo-code. For this program, pay
extra attention to how you describe your logic flow/algorithm in
\textbf{C}.

\end{enumerate}

All of these files must be in the directory \texttt{\asgn{1}}.

%%%%%%%%%%%%%%%%%%%%%%%%%%%%%%%%%%%%%%%%%%%%%%%%%%%%%%%%

\section{Submission}
\epigraphwidth=0.5\textwidth
\epigraph{\emph{Better three hours too soon than a minute too
late.}}{---William Shakespeare}

To submit your assignment, refer back to \texttt{\asgn{0}} for the steps on how to submit your assignment through \texttt{git}. Remember: \emph{add, commit,} and \emph{push}!

\textcolor{red}{Your assignment is turned in \emph{only} after you have pushed.
If you forget to push, you have not turned in your assignment and you will get
a \emph{zero}. ``I forgot to push'' is not a valid excuse. It is \emph{highly} recommended to commit and push your changes \emph{often}.}
%\end{enumerate}

%%%%%%%%%%%%%%%%%%%%%%%%%%%%%%%%%%%%%%%%%%%%%%%%%%%%%%%%
\section{Supplemental Readings}
\epigraph{\emph{The more that you read, the more things you will know. The
more that you learn, the more places you'll go.}}{---Dr.\ Seuss}\noindent

 \begin{itemize}

 	\item \textit{The C Programming Language} by Kernighan \& Ritchie
	\begin{itemize}
		\item Chapter 1 \S 1.6, 1.7
		\item Chapter 2 \S 2.3
		\item Chapter 3 \S 3.1-3.7
	\end{itemize}
 \end{itemize}

%%%%%%%%%%%%%%%%%%%%%%%%%%%%%%%%%%%%%%%%%%%%%%%%%%%%%%%%%%%

\newpage
\section*{Example}
\begin{verbatim}
darrell@hilbert code % ./lrc
Random seed: 2020
How many players? 7
Dude rolls... gets a pass gets a pass gives $1 to Karl 
Walter rolls... gives $1 to Dude puts $1 in the pot gets a pass 
Maude rolls... gives $1 to Big Lebowski gets a pass gets a pass 
Big Lebowski rolls... gives $1 to Maude gives $1 to Maude gets a pass 
Brandt rolls... puts $1 in the pot gets a pass gives $1 to Bunny 
Bunny rolls... gives $1 to Karl gets a pass gives $1 to Karl 
Karl rolls... gets a pass gives $1 to Dude gets a pass 
Dude rolls... gets a pass gets a pass gives $1 to Walter 
Walter rolls... gets a pass gets a pass 
Maude rolls... gives $1 to Big Lebowski gives $1 to Walter puts $1 in the pot 
Big Lebowski rolls... gets a pass puts $1 in the pot gets a pass 
Brandt rolls... gives $1 to Big Lebowski 
Bunny rolls... gets a pass gets a pass 
Karl rolls... gets a pass gets a pass gives $1 to Dude 
Dude rolls... gets a pass gives $1 to Walter gets a pass 
Walter rolls... gets a pass gets a pass gets a pass 
Maude rolls... gives $1 to Walter 
Big Lebowski rolls... gives $1 to Maude gets a pass gets a pass 
Bunny rolls... puts $1 in the pot gets a pass 
Karl rolls... puts $1 in the pot gets a pass gets a pass 
Dude rolls... gives $1 to Walter gets a pass puts $1 in the pot 
Walter rolls... puts $1 in the pot gives $1 to Maude gives $1 to Dude 
Maude rolls... puts $1 in the pot gives $1 to Big Lebowski 
Big Lebowski rolls... gets a pass puts $1 in the pot gives $1 to Maude 
Bunny rolls... gives $1 to Karl 
Karl rolls... gets a pass gets a pass gets a pass 
Dude rolls... gives $1 to Karl gives $1 to Karl 
Walter rolls... gives $1 to Maude gets a pass gets a pass 
Maude rolls... gives $1 to Big Lebowski gets a pass 
Big Lebowski rolls... puts $1 in the pot gets a pass 
Karl rolls... gives $1 to Dude gives $1 to Bunny gives $1 to Dude 
Dude rolls... puts $1 in the pot gets a pass 
Walter rolls... gets a pass gives $1 to Dude 
Maude rolls... gives $1 to Big Lebowski 
Big Lebowski rolls... gets a pass gives $1 to Brandt 
Brandt rolls... gives $1 to Bunny 
Bunny rolls... gets a pass gives $1 to Brandt 
Karl rolls... gets a pass puts $1 in the pot gives $1 to Dude 
Dude rolls... gives $1 to Karl gives $1 to Karl gets a pass 
Walter rolls... gives $1 to Dude 
Big Lebowski rolls... gets a pass 
Brandt rolls... gets a pass 
Bunny rolls... gives $1 to Karl 
Karl rolls... gives $1 to Dude gets a pass gets a pass 
Dude rolls... puts $1 in the pot gets a pass gives $1 to Karl 
Big Lebowski rolls... gets a pass 
Brandt rolls... gives $1 to Big Lebowski 
Karl rolls... gives $1 to Dude puts $1 in the pot gets a pass 
Dude rolls... gets a pass gets a pass 
Big Lebowski rolls... gives $1 to Brandt gets a pass 
Brandt rolls... gives $1 to Big Lebowski 
Karl rolls... gives $1 to Bunny gives $1 to Dude 
Dude rolls... gets a pass gets a pass gets a pass 
Big Lebowski rolls... puts $1 in the pot gives $1 to Maude 
Bunny rolls... puts $1 in the pot 
Dude rolls... gets a pass gives $1 to Walter gets a pass 
Walter rolls... gives $1 to Dude 
Maude rolls... gets a pass 
Dude rolls... gives $1 to Walter gives $1 to Walter gives $1 to Walter 
Walter rolls... gives $1 to Maude puts $1 in the pot gets a pass 
Maude rolls... gets a pass gets a pass 
Walter rolls... gives $1 to Dude 
Maude rolls... puts $1 in the pot gets a pass 
Dude rolls... gets a pass 
Maude rolls... gets a pass 
Dude rolls... gives $1 to Walter 
Walter rolls... gives $1 to Maude 
Maude wins the $19 pot with $2 left in the bank!
\end{verbatim}

\begin{figure}[ht]
  \centering
    \includegraphics[width=0.5\textwidth]{obey.jpg}
\end{figure}

\end{document}
