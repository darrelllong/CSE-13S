\documentclass[11pt,twocolumn]{article}
\usepackage{fullpage,fourier,amsmath,amssymb}
\usepackage{listings,color,url,hyperref}
\usepackage{mdframed}
\usepackage{epigraph}
\usepackage[x11names]{xcolor}
\usepackage{tcolorbox}
\usepackage{gensymb}

\title{Assignment 4 \\ Optimus Prime (fun with bit vectors)}
\author{Prof. Darrell Long \\
CSE 13S -- Winter 2020}
\date{Due: February 9$^\text{th}$ at 11:59\,pm}

\usepackage{fancyhdr}
\pagestyle{fancy}
\fancyhf{}

\fancypagestyle{plain}{%
  \fancyhf{}
  \renewcommand{\headrulewidth}{0pt}
  \renewcommand{\footrulewidth}{0pt}
  \lfoot{\textcopyright{} 2021 Darrell Long}
  \rfoot{\thepage}
}

\pagestyle{plain}

\definecolor{codegreen}{rgb}{0,0.5,0}
\definecolor{codegray}{rgb}{0.5,0.5,0.5}
\definecolor{codepurple}{rgb}{0.58,0,0.82}

\lstloadlanguages{C,make,python,fortran}

\lstdefinestyle{c99}{
    morekeywords={bool, uint8_t, uint16_t, uint32_t, uint64_t, int8_t, int16_t, int32_t, int64_t},
    commentstyle=\color{codegreen},
    keywordstyle=\color{magenta},
    numberstyle=\tiny\color{codegray},
    identifierstyle=\color{blue},
    stringstyle=\color{codepurple},
    basicstyle=\ttfamily,
    breakatwhitespace=false,
    breaklines=true,
    captionpos=b,
    keepspaces=true,
    numbers=left,
    numbersep=5pt,
    showspaces=false,
    showstringspaces=false,
    showtabs=false,
    tabsize=4
}


\newtcolorbox{prelab}[1]{colback=red!5!white, colframe=red!75!black, title=#1}
\lstset{language=C, style=c99}
\lstdefinestyle{stylePython}{
language = Python,
}

\onecolumn
\begin{document}\maketitle

%%%%%%%%%%%%%%%%%%%%%%%%%%%%%%%%%%%%%%%%%%%%%%%%%%%%%%%%
\section{Introduction}

We are going to answer some simple but fundamental questions about
the sequence of natural numbers ${\mathbb N} = 1, 2, 3, \ldots$ We
are going to look at primality and some other curious types of primes.
Primes numbers are fundamental in mathematics and Computer Science. You should also be concerned with the execution time of your programs.

\subsection{Prime Numbers}
\epigraphwidth=0.65\textwidth
\epigraph{\emph{God may not play dice with the universe, but something strange is going on with the prime numbers.}}{---Paul Erd\"os}

\noindent Prime numbers are among the most interesting of the natural numbers.
A number $p \in \mathbb{N}$ is \emph{prime} if it is evenly divisible
by only $1$ and $p$.  That means that $(\forall m \in \mathbb{N},
1 < m < p) m \nmid p$ or alternatively, $(\forall m \in \mathbb{N},
1 < m < p)p \bmod m \ne 0$.  The first prime number is $2$, all
primes except $2$ must be odd, since all even numbers are divisible
by $2$.  There are an infinite number of primes, which was proved by Euclid
about 300 B.C.  There is no formula for finding the primes, but the
\emph{prime number theorem} tells us that the probability of a given
$m\in \mathbb{N}, 1<m\le N$ being prime is very close to $\tfrac{1}{\ln N}$, since
the number of primes less than $N$, denoted $\pi(N)$, is
$$
\frac{N}{\ln N - (1 - \epsilon)} < \pi(N) < \frac{N}{\ln N - (1 +
\epsilon)}.
$$

Determining whether a number $n$ is prime and if it evenly divides by $n$ is conceptually simple, all that must be done is to try every $k \in \{ 2, \ldots, n - 1\}$.
The execution time for this method is $O(n)$, which for large $n$ is prohibitive. You are invited to try this for a relatively small number $2^{64} - 1$.
A little thought shows that we do not need to check so many, and that evaluating
$k \mid n$ for $k\in \{2, \ldots, \lceil \sqrt{n} \,\rceil \}$ is sufficient.
The resulting execution time of $O(\sqrt{n})$ is a great improvement.
Is that the best that we can hope for? No, but a lower bound is \emph{unknown}.

All natural numbers that are not prime are called \emph{composite}.
The \emph{fundamental theorem of arithmetic,}
also called the \emph{unique factorization theorem},
states that every integer $m>1$ is either prime or a unique
product of primes ($p_0, \ldots, p_k$).
That is,
$$
m = p_0^{\alpha_0} \times p_1^{\alpha_1} \times \ldots \times p_k^{\alpha_k}
= \prod_{i=0}^{k} p_i^{\alpha_k}.
$$
For example, $8\,3736=2^3\times 3^2\times 1163$.
In 1801, the fundamental theorem of arithmetic was proved by Carl Friedrich Gauss in his book
\emph{Disquisitiones Arithmetic\ae{}}.

\subsection{Positional Number Systems}

A positional number system is one that has its roots in the Babylonian
numeral system (base 60) dating back to around 2000 B.C. --  which is
why we have 60 seconds in a minute, 60 minutes in
an hour, 360 degrees in a circle ($1\degree = 60'$).
The modern number system which is most familiar to us is
decimal (base 10), also known as the Hindu-Arabic system. This is
the most commonly used base for counting, and stems from the fact
that most humans have ten fingers. In the positional number
system, a number is represented asc an ordered set of digits, where
the contribution of each digit depends on its position. Each digit
is multiplied by a value which determines its contribution to the
number. A any natural number
$$
N = a_{k} b^{k}+a_{k-1} b^{k-1}+\ldots+a_{1} b^{1}+a_{0} b^{0}
$$
has $b$ as its base for each digit (such as 2 (\emph{binary}) or
16 (\emph{hexadecimal})), and values on the range of $a_{i}\in
\{0,1,\ldots,b-1\}$. $k$ represents the highest-ordered digit
position in the number $N$. For example, the number 2020 can be
expressed in base 10 as,
$$
2020 = 2 \times 10^{3} + 0 \times 10^{2} + 10 \times 10^{1} + 0 \times 10^{0}
$$
or alternatively, in base 2 and base 16,
$$2020 = 11111100100_{2} = 7E4_{16}.$$

\subsubsection{Base Change}

A single number found in a positional number system can be represented
in any number of different bases. A \emph{base change} occurs when
a number is converted from one numeral base to another, such as
going from base 10 to base 16. For example, the decimal number
$7562_{10}$ can be converted to hexadecimal, noting that each decimal
remainder is converted to its hexadecimal value.

\begin{center}
\begin{tabular}{rcccl}
\multicolumn{1}{l}{\textbf{base 16}} & \multicolumn{1}{l}{\textbf{quotient}} & \multicolumn{1}{l}{\textbf{remainder (decimal)}} & \multicolumn{1}{l}{\textbf{remainder (hexadecimal)}} &  \\ \hline
7562/16                              & 472                                   & 10                                               & A                                            &  \\
472/16                               & 29                                    & 8                                                & 8                                            &  \\
29/16                                & 1                                     & 13                                               & D                                            &  \\
1/16                                 & 0                                     & 1                                                & 1                                            &
\end{tabular}
\end{center}

The result which is written as 1D8A$_{16}$. Additionally, the decimal
number $100_{10}$ can be converted to base 3.

\begin{center}
\begin{tabular}{rcc}
\multicolumn{1}{l}{\textbf{base 3}} & \multicolumn{1}{l}{\textbf{quotient}} & \multicolumn{1}{l}{\textbf{remainder (decimal)}} \\ \hline
100/3                               & 33                                    & 1                                                \\
33/3                                & 11                                    & 0                                                \\
11/3                                & 3                                     & 2                                                \\
3/3                                 & 1                                     & 0                                                \\
1/3                                 & 0                                     & 1
\end{tabular}
\end{center}

The result which is written as 10201$_{3}$.


\subsection{Interesting Primes}
\epigraphwidth=0.75\textwidth \epigraph{\emph{The problem of distinguishing
prime numbers from composite numbers and of resolving the latter into their
prime factors is known to be one of the most important and useful in
arithmetic.}}{---Carl Friedrich Gauss}

\noindent A Fibonacci number is defined by a linear recurrence equation:
$$F_n = F_{n-1} + F_{n-2}$$
It is conventional to define $F_0 = 0$. The above equation defines
the next Fibonacci number as the sum of the two numbers at the
$(n-1)^{\text{st}}$ and the $(n-2)^{\text{nd}}$ position. The first
$10$ Fibonacci numbers are $0$, $1$, $1$, $2$, $3$, $5$, $8$, $13$,
$21$ and $34$.

A \textit{\textbf{Fibonacci prime}} is a Fibonacci number that is
prime. For example out of the above list of Fibonacci numbers, the
numbers $2$, $3$, $5$ and $13$ are Fibonacci prime numbers. One of
your tasks for this assignment will be to find out if a given prime
number is a Fibonacci prime.

Francois Edouard Anatole Lucas was a French mathematician who devised
methods like the Lucas Lehmer primality test for testing primality
of numbers. He is also the person who gave the Fibonacci numbers their name, after Leonardo Pisano
detto il Fibonacci.
A Lucas number, named after the same mathematician, is
defined by a linear recurrence equation: $$L_n = L_{n-1} + L_{n-2}$$
It is conventional to define $L_0 = 2$. Similar to Fibonacci numbers,
the above equation defines the next Lucas number as the sum
of the previous two numbers at the $(n-1)^{\text{st}}$ and the
$(n-2)^{\text{nd}}$ position. The first $10$ Lucas numbers are $2$,
$1$, $3$, $4$, $7$, $11$, $18$, $29$, $47$ and $76$.

A \textit{\textbf{Lucas prime}} is a Lucas number that is prime.
For example out of the above list of Lucas numbers, the numbers
$2$, $3$, $7$, $11$, $29$ and $47$ are Lucas prime numbers. One of
your tasks for this assignment will be to find out if a given prime
number is a Lucas prime.

A \textit{\textbf{Mersenne Prime}}, named after Marin Mersenne, a
$17^\text{th}$ century French polymath.
He also developed Mersenne's laws, which describe the harmonics of a vibrating string (such as may
be found on guitars and pianos), and his seminal work on music theory, \emph{Harmonie
universelle}.
He may be best known for a list of primes with
exponents up to $257$, is a prime number that can be represented
as one less than an $n^{\text{th}}$ power of two. For an integer
$n$, a Mersenne prime is represented in the form: $$M_{n} = 2^{n}
- 1.$$

The first four Mersenne primes are $M_{2} = 3$, $M_{3} = 7$, $M_{5}
= 31$ and $M_{7} = 127$, and can be represented in the above format
as:

\begin{align*}
M_{2} & = 2^{2} - 1= 3 \\
M_{3} & = 2^{3} - 1= 7 \\
M_{5} & = 2^{5} - 1= 31 \\
M_{7} & = 2^{7} - 1= 127.
\end{align*}

\subsection{Palindromic Primes}
A palindrome is a word, verse, or sentence or a number that reads
the same backward or forward. Examples of palindromes are: ``Able
was I ere I saw Elba'' and the year 1881. The pseudocode to check if a
string is Palindrome is as follows:

\begin{lstlisting}[style = stylePython]
def isPalindrome(s):
   f = True
   for i in range(len(s) / 2):
       if s[i] != s[-(i + 1)]:
           f = False
   return f

w = raw_input("word = ")
if isPalindrome(w):
    print w, "is a palindrome"
else:
    print w, "is not a palindrome"
\end{lstlisting}

Whether a number is a palindrome or not depends on the \emph{radix}
or \emph{base} of the number system being used. Let us look at the
two examples.

\textbf{Example 1}: For the decimal number $3855$, the binary number is $111100001111$, and the hexadecimal number is $\text{F0F}$.

\textbf{Example 2}: For the decimal number $195$, the binary number is $11000011$, and the hexadecimal number is $\text{C}3$.

In Example 1, the decimal number $3855$ is not a palindrome but the hexadecimal and binary counterparts of the number are palindromes, while in example 2, only the binary number is a palindrome.

A \textbf{\textit{palindromic prime}} is a prime number, whose
corresponding string at a certain base is a palindrome. Few examples
of palindromic primes are $(373)_{10}$, the binary version of the
number $7$ which is $111$, the binary version of the number $5$
which is $101$.

\section{Your Task}
\epigraph{\emph{Mathematics reveals its secrets only to those who approach it
with pure love, for its own beauty.}}{---Archimedes}

\noindent For this assignment you have two tasks:

\begin{itemize}

\item \textbf{Task 1:} Use a sieve to generate a list of prime numbers up to a
    value $n$, inclusive. Go through each prime number and determine whether it
    is a \emph{Fibonacci Prime} (\texttt{F}), \emph{Lucas Prime} (\texttt{L}) or
    a \emph{Mersenne Prime} (\texttt{M}).
\item \textbf{Task 2:} For each prime number check if its a palindromic prime
    for the following bases:
	\begin{itemize}
		\item Base 2
		\item Base 10
		\item Base 14
        \item Base of the first letter of your last name + 10. For example,
            Professor Long would compute his base as $12 + 10 = 22$, since the
            letter L is the $12^{\text{th}}$ letter of the alphabet.
	\end{itemize}
    For each base only print the primes that are palindromic primes.

\end{itemize}

We will test your program by comparing its output with the output of a known
correct program. Refer to the example output given at the end of the assignment.
Your program should produce an output in the exact same format as shown in the
example output. The values in the output will change based on your $n$ value and
the bases for palindromic prime.

%%%%%%%%%%%%%%%%%%%%%%%%%%%%%%%%%%%%%%%%%%%%%%%%%%%%%%%%
\medskip
\begin{prelab}{Pre-lab Part 1}
	\begin{enumerate}
    \item Assuming you have a list of primes to consult, write pseudo-code to
        determine if a number is a Fibonacci prime, a Lucas prime, and/or a
        Mersenne prime.
	\item Assuming you have a list of primes to consult, write pseudo-code to determine if a
	       number in base 10 is a Fibonacci prime, Lucas prime, or Mersenne prime.
	\end{enumerate}
\end{prelab}

%%%%%%%%%%%%%%%%%%%%%%%%%%%%%%%%%%%%%%%%%%%%%%%%%%%%%%%%

\section{Specifics}
\epigraph{\emph{Don't try to cover your mistakes with false words. Rather,
correct your mistakes with examination.}}{---Pythagoras}

Writing the program using the na\"ive approach of finding primes
up to the value of a given number into order to find its prime
factorization is, frankly, silly. You can and will do much better,
and you will do so by using a \emph{sieve}. Your sieve will take as its sole
argument a \emph{BitVector}.

\begin{lstlisting}[title=sieve.h]
#ifndef __SIEVE_H__
#define __SIEVE_H__

#include "bv.h"

//
// The Sieve of Eratosthenes.
// Set bits in a BitVector represent prime numbers.
// Composite numbers are sieved out by clearing bits.
//
// v:  The BitVector to sieve.
//
void sieve(BitVector *v);

#endif
\end{lstlisting}

%%%%%%%%%%%%%%%%%%%%%%%%%%%%%%%%%%%%%%%%%%%%%%%%%%%%%%%%

\section{Bit Vectors}
\epigraph{\emph{Symmetrical equations are good in their place, but `vector'
is a useless survival, or offshoot from quaternions, and has never been of the
slightest use to any creature.}}{---Lord Kelvin}

\textit{\textbf{Bit Vectors}} are a rarely taught but essential tool in the kit
of all computer scientists and engineers. A Bit Vector is an ADT that represents
an array of bits, the bits in which are used to denote if something is true or
false (1 or 0). This is an efficient ADT since, in order to represent the truth
or falsity of an array of $n$ items, we can use $n / 8 + 1$ \texttt{uint8\_t}'s
instead of $n$, and being able to access $8$ indices with a single integer
access is extremely cost efficient. You are expected to implement the following
ADT interface for Bit Vectors.

\begin{lstlisting}[title=bv.h]
// bv.h - Contains the function declarations for the BitVector ADT.

#ifndef __BV_H__
#define __BV_H__

#include <inttypes.h>
#include <stdbool.h>

//
// Struct definition for a BitVector.
//
// vector:     The vector of bytes to hold bits.
// length:     The length in bits of the BitVector.
//
typedef struct BitVector {
  uint8_t *vector;
  uint32_t length;
} BitVector;

//
// Creates a new BitVector of specified length.
//
// bit_len:    The length in bits of the BitVector.
//
BitVector *bv_create(uint32_t bit_len);

//
// Frees memory allocated for a BitVector.
//
// v:  The BitVector.
//
void bv_delete(BitVector *v);

//
// Returns the length in bits of the BitVector.
//
// v:  The BitVector.
//
uint32_t bv_get_len(BitVector *v);

//
// Sets the bit at index in the BitVector.
//
// v:  The BitVector.
// i:  Index of the bit to set.
//
void bv_set_bit(BitVector *v, uint32_t i);

//
// Clears the bit at index in the BitVector.
//
// v:  The BitVector.
// i:  Index of the bit to clear.
//
void bv_clr_bit(BitVector *v, uint32_t i);

//
// Gets a bit from a BitVector.
//
// v:  The BitVector.
// i:  Index of the bit to get.
//
uint8_t bv_get_bit(BitVector *v, uint32_t i);

//
// Sets all bits in a BitVector.
//
// v:  The BitVector.
//
void bv_set_all_bits(BitVector *v);

#endif
\end{lstlisting}

The header file \texttt{bv.h} defines the BitVector ADT and its associated
operations. Even though \textbf{C} will not prevent you from directly accessing
\texttt{struct} fields, you must avoid that temptation and only use the
functions defined in \texttt{bv.h} --- no exceptions!

You must implement each of the functions specified in the header file. Most of
them are just a line of two of \textbf{C} code, but their implementation can be
subtle. You are warned \emph{again} against using code that you may find on the
Internet.


\begin{lstlisting}
//
// The Sieve of Eratosthenes.
// Set bits in a BitVector represent prime numbers.
// Composite numbers are sieved out by clearing bits.
//
// v:  The BitVector to sieve.
//
void sieve(BitVector *v) {
  bv_set_all_bits(v);   // Assume all are prime.
  bv_clr_bit(v, 0);     // 0 is, well, zero.
  bv_clr_bit(v, 1);     // 1 is unity.
  bv_set_bit(v, 2);     // 2 is prime.
  for (uint32_t i = 2; i <= sqrtl(bv_get_len(v)); i += 1) {
    // Prime means bit is set.
    if (bv_get_bit(v, i)) {
      for (uint32_t k = 0; (k + i) * i <= bv_get_len(v); k += 1) {
        bv_clr_bit(v, (k + i) * i); // Its multiples are composite.
      }
    }
  }
  return;
}
\end{lstlisting}

Since you have been given the code for the Sieve of Eratosthenes,
you \emph{must} cite it and give proper credit if you use it. If,
for example, you were to implement the Sieve of Sundaram, or the
more modern Sieve of Atkin, you would not need to cite beyond the
source of the algorithm and any pseudocode that you followed.

\textcolor{red}{Hint: You are not penalized for using this code. Choose another
\emph{only} if you are highly motivated.}

\medskip
\begin{prelab}{Pre-lab Part 2}
    \begin{enumerate}
        \item Implement each BitVector ADT function.
        \item  Explain how you avoid memory leaks when you free allocated memory
            for your BitVector ADT.
        \item While the algorithm in sieve() is correct, it has room for
            improvement. What change would you make to the code in sieve() to
            improve the runtime?
    \end{enumerate}
\end{prelab}

%%%%%%%%%%%%%%%%%%%%%%%%%%%%%%%%%%%%%%%%%%%%%%%%%%%%%%%%
\section{Deliverables}

You will need to turn the following files into the \texttt{assignment4}
directory:

\begin{enumerate}
    \item Your program \emph{must} have the source and header files:
    \begin{itemize}
        \item \texttt{bv.h} to specify the bit vector operations and abstract
            data type \texttt{bitV}.
        \item \texttt{bv.c} to implement the functionality.
        \item \texttt{sieve.h} specifies the interface to the sieve.
        \item \texttt{sieve.c} to implement the sieve algorithm of your choice.
        \item \texttt{sequence.c} contains \texttt{main()} and \emph{may}
            contain the other functions necessary to complete the assignment.
    \end{itemize}

    You may have other source and header files, but \emph{do not try to be
    overly clever}.

    \item \texttt{Makefile}: This is a file that will allow the grader to type
        \texttt{make} to compile your program. At this point you will have
        learned about \texttt{make} and can create your own \texttt{Makefile}.
    \begin{itemize}
        \item \texttt{CFLAGS=-Wall -Wextra -Werror -Wpedantic} must be included.
        \item \texttt{CC=clang} must be specified.
        \item \texttt{make clean} must remove all files that are compiler
            generated.
        \item \texttt{make} should build your program, as should \texttt{make
            all}.
        \item Your program executable must be named \texttt{sequence}.
    \end{itemize}

    \item \texttt{README.md}: This \emph{must} be in \emph{markdown}. This must
        describe how to use your program and Makefile.

    \item \texttt{DESIGN.pdf}: This \emph{must} be a PDF. The design document
        should contain answers to the pre-lab questions at the beginning and
        describe your design for your program with enough detail that a
        sufficiently knowledgeable programmer would be able to replicate your
        implementation. This does not mean copying your entire program in
        verbatim. You should instead describe how your program works with
        supporting pseudocode.

 You \emph{must} push the \texttt{DESIGN.pdf} before you
          push \textcolor{red}{\emph{any}} code.
\end{enumerate}

\noindent Your program must also:
\begin{itemize}
    \item Support the following \texttt{getopt()} flags:
        \begin{enumerate}
            \item \texttt{-s} : Print out all primes and identify whether or not
                they are interesting (Lucas, Mersenne, Fibonacci).
            \item \texttt{-p} : Print out palindromic primes in bases 2, 10,
                14, and first letter of your last name + 10.
            \item \texttt{-n <value>} : Specifies the largest value to consider,
                inclusively, for your prime sieve. By default your program runs
                up through $1000$.
        \end{enumerate}


    \item Dynamically allocate memory.
    \item Not have \emph{any} memory leaks.
    \item Cleanly pass infer --- fix any errors or explain any complaints that
        infer has in your \texttt{README}.
    \item \emph{Not} have any compiler warnings.
\end{itemize}

%%%%%%%%%%%%%%%%%%%%%%%%%%%%%%%%%%%%%%%%%%%%%%%%%%%%%%%%

\section{Submission}

To submit your assignment, refer back to \texttt{assignment0} for the steps on
how to submit your assignment through \texttt{git}. Remember: \emph{add,
commit,} and \emph{push}!

\textcolor{red}{Your assignment is turned in \emph{only} after you have pushed
\emph{and} submitted the commit ID on Canvas.  If you forget to push, you have
not turned in your assignment and you will get a \emph{zero}. ``I forgot to
push'' is not a valid excuse. It is \emph{highly} recommended to commit and push
your changes \emph{often}.}

\section{Supplemental Readings}
\epigraph{\emph{The more you read, the more things you will know. The
more that you learn, the more places you'll go.}}{---Dr.\ Seuss}\noindent

\begin{itemize}
    \item \textit{The C Programming Language} by Kernighan \& Ritchie
    \begin{itemize}
	\item Chapter 2 \S 2.9
        \item Chapter 4 \S 4.5--4.11
	\item Chapter 5 \S 5.1--5.5
	\item Chapter 6 \S 6.1--6.2, 6.7
    \end{itemize}
\end{itemize}

\section{Examples}
Consider the two examples shown below. This shows the \emph{required} format of
the output when running the two commands \texttt{./sequence -s -n 100} and
\texttt{./sequence -p -n 50}, respectively. You will notice that the output for
the second command does not match what is specified for your program; it is only
used to showcase program output format using other bases.

\begin{lstlisting}[title=./sequence -s -n 100]
2: prime, fibonacci
3: prime, mersenne, lucas, fibonacci
5: prime, fibonacci
7: prime, mersenne, lucas
11: prime, lucas
13: prime, fibonacci
17: prime
19: prime
23: prime
29: prime, lucas
31: prime, mersenne
37: prime
41: prime
43: prime
47: prime, lucas
53: prime
59: prime
61: prime
67: prime
71: prime
73: prime
79: prime
83: prime
89: prime, fibonacci
97: prime
\end{lstlisting}

\begin{lstlisting}[title=./sequence -p -n 50]
Base  3
---- --
2 = 2
13 = 111
23 = 212

Base  4
---- --
2 = 2
3 = 3
5 = 11
17 = 101
29 = 131

Base  5
---- --
2 = 2
3 = 3
31 = 111
41 = 131

Base  6
---- --
2 = 2
3 = 3
5 = 5
7 = 11
37 = 101
43 = 111

Base  7
---- --
2 = 2
3 = 3
5 = 5

Base  8
---- --
2 = 2
3 = 3
5 = 5
7 = 7

Base  9
---- --
2 = 2
3 = 3
5 = 5
7 = 7

Base 11
---- --
2 = 2
3 = 3
5 = 5
7 = 7

Base 12
---- --
2 = 2
3 = 3
5 = 5
7 = 7
11 = b
13 = 11

Base 13
---- --
2 = 2
3 = 3
5 = 5
7 = 7
11 = b

Base 15
---- --
2 = 2
3 = 3
5 = 5
7 = 7
11 = b
13 = d
\end{lstlisting}

\end{document}
