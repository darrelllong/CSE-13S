\documentclass{article}
\usepackage{fullpage,fourier,amsmath,amssymb}
\usepackage{listings,color,url,hyperref}
\usepackage{graphicx}
\usepackage{epigraph,wrapfig}
\usepackage[x11names]{xcolor}
\usepackage{tcolorbox}
\usepackage{gensymb}

\title{Assignment 6\\The Great Firewall of Santa Cruz:\\Bloom Filters,
Linked Lists and Hash Tables}
\author{Prof. Darrell Long \\
CSE 13S -- Winter 2020}
\date{Due: March $1^\text{st}$ at 11:59 pm}

\usepackage{fancyhdr}
\pagestyle{fancy}
\fancyhf{}

\fancypagestyle{plain}{%
  \fancyhf{}
  \renewcommand{\headrulewidth}{0pt}
  \renewcommand{\footrulewidth}{0pt}
  \lfoot{\textcopyright{} 2021 Darrell Long}
  \rfoot{\thepage}
}

\pagestyle{plain}

\definecolor{codegreen}{rgb}{0,0.5,0}
\definecolor{codegray}{rgb}{0.5,0.5,0.5}
\definecolor{codepurple}{rgb}{0.58,0,0.82}

\lstloadlanguages{C,make,python,fortran,bash}

\lstdefinestyle{bashstyle}{
  language=bash,
  basicstyle=\small\ttfamily,
  numbers=none,
  frame=tb,
  columns=fullflexible,
  backgroundcolor=\color{yellow!10},
  linewidth=0.9\linewidth,
  xleftmargin=0.1\linewidth,
  aboveskip=1.25em,
  belowskip=1.25em,
  breakatwhitespace=false,
  breaklines=true,
  showstringspaces=false,
  keepspaces=true,
  showspaces=false,
  showtabs=false,
  tabsize=4
}

\lstdefinestyle{plainstyle}{
  keywords={},
  stringstyle=\color{black},
  basicstyle=\small\ttfamily,
  numbers=none,
  frame=tb,
  columns=fullflexible,
  backgroundcolor=\color{yellow!10},
  linewidth=0.9\linewidth,
  xleftmargin=0.1\linewidth,
  aboveskip=1.25em,
  belowskip=1.25em,
  breakatwhitespace=false,
  breaklines=true,
  showstringspaces=false,
  keepspaces=true,
  showspaces=false,
  showtabs=false,
  tabsize=4
}

\lstdefinestyle{c99}{
    language=C,
    morekeywords={
      bool, uint8_t, uint16_t, uint32_t, uint64_t,
      int8_t, int16_t, int32_t, int64_t
    },
    upquote=true,
    commentstyle=\color{codegreen},
    keywordstyle=\color{magenta},
    identifierstyle=\color{black},
    stringstyle=\color{codepurple},
    basicstyle=\ttfamily,
    breakatwhitespace=false,
    breaklines=true,
    captionpos=b,
    keepspaces=true,
    showspaces=false,
    showstringspaces=false,
    showtabs=false,
    numbers=left,
    numbersep=7pt,
    numberstyle=\ttfamily\color{black!25},
    xleftmargin=12pt,
    tabsize=4
}

\lstdefinestyle{python}{
    language=python,
    upquote=true,
    commentstyle=\color{codegreen},
    keywordstyle=\color{magenta},
    identifierstyle=\color{black},
    stringstyle=\color{codepurple},
    basicstyle=\ttfamily,
    breakatwhitespace=false,
    breaklines=true,
    captionpos=b,
    keepspaces=true,
    showspaces=false,
    showstringspaces=false,
    showtabs=false,
    numbers=left,
    numbersep=7pt,
    numberstyle=\ttfamily\color{black!25},
    xleftmargin=12pt,
    tabsize=4
}

\usepackage[most]{tcolorbox}

\newtcolorbox{prelab}[1]{
  colback=red!5!white, colframe=red!75!black, title=#1
}

\newtcblisting{codelisting}[2][]{
  title=#2, #1,
  fonttitle=\bfseries,
  colframe=black!80,
  listing only,
  listing options={basicstyle=\ttfamily,style=c99},
  top=-5pt,
  bottom=-5pt,
  enlarge top by=10pt
}

\newtcblisting{pythonlisting}[2][]{
  title=#2, #1,
  fonttitle=\bfseries,
  colframe=black!80,
  listing only,
  listing options={basicstyle=\ttfamily,style=python},
  top=-5pt,
  bottom=-5pt,
  enlarge top by=5pt
}

\newcommand\Warning{%
 \makebox[1.4em][c]{%
 \makebox[0pt][c]{\raisebox{.1em}{\small!}}%
 \makebox[0pt][c]{\color{red}\Large$\bigtriangleup$}}}%


\newtcolorbox{prelab}[1]{colback=red!5!white, colframe=red!75!black, title=#1}
\lstset{language=C, style=c99}
\lstdefinestyle{stylePython}{
language = Python,
}

\begin{document}\maketitle

\section{Introduction}
\epigraphwidth=0.5\textwidth \epigraph{\emph{War is peace. Freedom
is slavery. Ignorance is strength.}}{ ---George Orwell, 1984}

\noindent
You have been selected through thoroughly democratic processes (and the
machinations of your friend Ernst Blofeld) to be the Dear and Beloved Leader
of the Glorious People's Republic of Santa Cruz following the failure of the
short-lived anarcho-syndicalist commune, where each person in turn acted as
a form of executive officer for the week. In order to promote virtue and
prevent vice, and to preserve social cohesion and discourage unrest, you have
decided that the Internet content must be filtered so that your beloved
children are not corrupted through the use of unfortunate, hurtful, offensive,
and far too descriptive language.

\section{Background}
\epigraphwidth=0.75\textwidth
\epigraph{\emph{The Ministry of Peace concerns itself with war, the Ministry
of Truth with lies, the Ministry of Love with torture and the Ministry of
Plenty with starvation. These contradictions are not accidental, nor do they
result from from ordinary hypocrisy: they are deliberate exercises in
\textbf{doublethink}.}}{---George Orwell, 1984}
\subsection{Bloom Filters}
\noindent
The Internet is very large, very fast and full of \emph{badthink}. The masses
spend their days sending each other cat videos. Your hero, Ernst Blofeld,
decided that in his country of Pacifica that a more neutral \emph{newspeak}
was required to keep the people content, pure, and from thinking
too much. But how do you process so many words as they flow into
your little country at $10\,$Gbits/second? The solution that comes to
your brilliant and pure mind --- a \emph{Bloom filter}.
\begin{wrapfigure}{r}{0.133\textwidth}
\centering
\includegraphics[width=0.130\textwidth]{1984-Big-Brother.jpg}
\end{wrapfigure}


A Bloom filter is similar to a hash table, where the entries in the hash table are simply
single bits.  Consider \texttt{h(}\emph{text}\texttt{) = k}, then if $B_k = 0$
then the entry is definitely missing; if $B_k = 1$ then the entry may be
present. The latter is called a false positive, and the false positive rate
depends on the size of the Bloom filter and the number of texts that hash to the
same position (hash collisions).

You assign your programming minions to take every \emph{potentially offensive}
word that they can find in an English dictionary and hash them into the Bloom
filter. That is, for each word the corresponding bit in the filter is set to
$1$.

You will make a Bloom filter with \emph{three} salts for \emph{three} different
hash functions. Why? To reduce the chance of a \emph{false positive}.

\lstset{language=C, style=c99}
\begin{lstlisting}[title=bf.h]
#ifndef NIL
#define NIL (void *)0
#endif

#ifndef __BF_H__
#define __BF_H__

#include "bv.h"
#include <inttypes.h>
#include <stdbool.h>

//
// Struct definition for a BloomFilter.
//
// primary:     Primary hash function salt.
// secondary:   Secondary hash function salt.
// tertiary:    Tertiary hash function salt.
// filter:      BitVector that determines membership of a key.
//
typedef struct BloomFilter {
  uint64_t primary  [2]; // Provide for three different hash functions
  uint64_t secondary[2];
  uint64_t tertiary [2];
  BitVector *filter;
} BloomFilter;

//
// Constructor for a BloomFilter.
//
// size:  The number of entries in the BloomFilter.
//
BloomFilter *bf_create(uint32_t size);

//
// Destructor for a BloomFilter.
//
// bf:  The BloomFilter.
//
void bf_delete(BloomFilter *bf);

//
// Inserts a new key into the BloomFilter.
// Indices to set bits are given by the  hash functions.
//
// bf:    The BloomFilter.
// key:   The key to insert into the BloomFilter.
//
void bf_insert(BloomFilter *bf, char *key);

//
// Probes a BloomFilter to check if a key has been inserted.
//
// bf:    The BloomFilter.
// key:   The key in which to check membership.
//
bool bf_probe(BloomFilter *bf, char *key);

#endif
\end{lstlisting}

\begin{lstlisting}[title=bf.c]
#include "bf.h"
#include <stdlib.h>

//
// Constructor for a BloomFilter.
//
// size:  The number of entries in the BloomFilter.
//
BloomFilter *bf_create(uint32_t size) {
  BloomFilter *bf = (BloomFilter *)malloc(sizeof(BloomFilter));
  if (bf) {
    bf->primary  [0] = 0xfc28ca6885711cf7; // U.S. Constitution
    bf->primary  [1] = 0x2841af568222f773;
    bf->secondary[0] = 0x85ae998311115ae3; // Il nome della rosa
    bf->secondary[1] = 0xb6fac2ae33a40089;
    bf->tertiary [0] = 0xd37b01df0ae8f8d0; // The Cremation of Sam McGee
    bf->tertiary [1] = 0x911d454886ca7cf7;
    bf->filter = bv_create(size);
    return bf;
  }
  return (BloomFilter *)NIL;
}

// The rest of the functions you must implement yourselves.
\end{lstlisting}

A stream of words is passed to the Bloom filter. If the Bloom filter rejects all
words then the person responsible is \emph{innocent} of a \emph{thoughtcrime}.
But if any word has a corresponding bit set in the Bloom filter, it is likely
that they are \emph{guilty} of a \emph{thoughtcrime}, and that is very ungood
indeed. To make certain, we must consult the \emph{hash table} and if the word
is there as a \emph{proscribed} word then they will be placed into the care of
\emph{Miniluv} and is sent off to \emph{joycamp}.  If the word passed both Bloom
filters and is not a forbidden word, the hash table will provide a translation
that will replace offensive, insensitive, and otherwise dangerous words with new
approved words. The advantage is that your government can augment this list at
any time via the \emph{Minitrue}. Minitrue can also add words that you, in your wisdom, have determined are not wholesome for your
people to use. Simply put, there are three cases to consider:

\begin{enumerate}
\item Words that are approved will \emph{not} appear in the Bloom filter.
\item Words that should be replaced, which will have a mapping from the old
    word, \emph{oldspeak}, to the new approved word, \emph{newspeak} (also
    referred to as a \emph{translation}).
\item Words without translations to new approved words means that it's
    off to \emph{joycamp}.
\end{enumerate}

You will use the \emph{BitVector} data structure that you developed
for \emph{Assignment 4} to implement your Bloom filter.

%%%%%%%%%%%%%%%%%%%%%%%%%%%%%%%%%%%%%%%%%%%%%%%%%%%%%%%%
\medskip
\begin{prelab}{Pre-lab Part 1}
  \begin{enumerate}
    \item Write down the pseudocode for inserting and deleting elements from a
        Bloom filter.
    \item Assuming that you are creating a bloom filter with $m$ bits and $k$ hash
        functions, discuss its time and space complexity.
  \end{enumerate}
\end{prelab}
%%%%%%%%%%%%%%%%%%%%%%%%%%%%%%%%%%%%%%%%%%%%%%%%%%%%%%%%

\subsection{Hash Tables}
\epigraph{\emph{To send men to the firing squad, judicial proof is unnecessary
\ldots These procedures are an archaic bourgeois detail.}}{Ernesto Ch\'e
Guevara}
\noindent
In general, rather than punish them, we would rather counsel them how to \emph{Rightspeak}. You
decide that the easiest way to make word replacements is through a translation
table, with entries of the form:

\lstset{language=C, style=c99}
\begin{lstlisting}[title=\texttt{GoodSpeak struct}]
typedef struct GoodSpeak {
  char *oldspeak;
  char *newspeak;
} GoodSpeak;
\end{lstlisting}

\begin{wrapfigure}{r}{0.133\textwidth}
\centering
\includegraphics[width=0.130\textwidth]{Blofeld.png}
\end{wrapfigure}
If a word is in your Bloom filter, then it is either a proscribed word or a word that needs to be translated by the hash table. You will use the hash
table to locate that word. If it is found, the system
helpfully provides the appropriate new improved word in its place.

A hash table is one that is indexed by a function, \texttt{hash()}, applied to
the \emph{key}. The key in this case will be the word in \emph{oldspeak}.  As we
have said, this provides a mapping from \emph{oldspeak} to \emph{newspeak}.
Words for which there is no translation result in \emph{joycamp}.  All other
words must have a replacement \emph{newspeak} word.

What happens when two \emph{oldspeak} words have the same hash value? This is
called a \emph{hash collision}, and must be resolved. Rather than doing
\emph{open addressing} (as discussed in class), we will be using \emph{linked
lists} to resolve \emph{oldspeak} hash collisions.

\subsection{Hashing with the SPECK Cipher}
\epigraph{\emph{Every record has been destroyed or falsified, every book has
been rewritten, every picture has been repainted, every statue and street
building has been renamed, every date has been altered. And that process is
continuing day be day and minute by minute. History has stopped. Nothing
exists except the endless present in which the party is always
right.}}{---George Orwell}

\noindent
First of all, you need a good hash function. We have discussed hash functions in
class, and rather than risk having a poor one implemented, we will simply
provide you one. The
SPECK\footnote{
Ray Beaulieu, Stefan Treatman-Clark, Douglas Shors, Bryan Weeks, Jason Smith,
and Louis Wingers, ``The {SIMON} and {SPECK} lightweight block ciphers.'' In
Proceedings of the 52nd ACM/EDAC/IEEE Design Automation Conference (DAC), pp.
1--6. IEEE, 2015.}
block cipher is provided for use as a hash function.

SPECK is a family of lightweight block ciphers publicly released by the National
Security Agency (NSA) in June 2013.  SPECK has been optimized for performance in
software implementations, while its sister algorithm, SIMON, has been optimized
for hardware implementations. SPECK is an add-rotate-xor (ARX) cipher. The
reason a cipher is used for this is because encryption generates random output
given some input; exactly what we want for a hash.

Encryption is the process of taking some file you wish to protect, usually
called plaintext, and transforming its data such that only authorized parties
can access it. This transformed data is referred to as ciphertext. Decryption
is the inverse operation of encryption, taking the ciphertext and transforming
the encrypted data back to its original state as found in the original
plaintext.

Encryption algorithms that utilize the same key for both encryption and
decryption, like SPECK, are symmetric-key algorithms, and algorithms that don't,
such as RSA, are asymmetric-key algorithms.

\lstset{language=C, style=c99}
\begin{lstlisting}[title=speck.h]
#ifndef __SPECK_H__
#define __SPECK_H__

#include <inttypes.h>

uint32_t hash(uint64_t salt[], char *key);

#endif
\end{lstlisting}
\begin{lstlisting}[title=speck.c]
#include "speck.h"
#include <inttypes.h>
#include <stddef.h>
#include <string.h>

#define LCS(X, K)                                                              \
  (X << K) | (X >> (sizeof(uint64_t) * 8 - K)) // left circular shift
#define RCS(X, K)                                                              \
  (X >> K) | (X << (sizeof(uint64_t) * 8 - K)) // right circular shift

// Core SPECK operation
#define R(x, y, k) (x = RCS(x, 8), x += y, x ^= k, y = LCS(y, 3), y ^= x)


void speck_expand_key_and_encrypt(uint64_t pt[], uint64_t ct[], uint64_t K[]) {
  uint64_t B = K[1], A = K[0];
  ct[0] = pt[0];
  ct[1] = pt[1];

  for (size_t i = 0; i < 32; i += 1) {
    R(ct[1], ct[0], A);
    R(B, A, i);
  }
}

uint64_t keyed_hash(const char *s, uint32_t length, uint64_t key[]) {
  uint64_t accum = 0;

  union {
    char b[sizeof(uint64_t)]; // 16 bytes fit into the same space as
    uint64_t ll[2]; // 2 64 bit numbers.
  } in;

  uint64_t out[2]; // SPECK results in 128 bits of ciphertext
  uint32_t count;

  count = 0; // Reset buffer counter
  in.ll[0] = 0x0;
  in.ll[1] = 0x0; // Reset the input buffer (zero fill)

  for (size_t i = 0; i < length; i += 1) {
    in.b[count++] = s[i]; // Load the bytes

    if (count % (2 * sizeof(uint64_t)) == 0) {
      speck_expand_key_and_encrypt(in.ll, out, key); // Encrypt 16 bytes
      accum ^= out[0] ^ out[1]; // Add (XOR) them in for a 64 bit result
      count = 0; // Reset buffer counter
      in.ll[0] = 0x0;
      in.ll[1] = 0x0; // Reset the input buffer
    }
  }

  // There may be some bytes left over, we should use them.
  if (length % (2 * sizeof(uint64_t)) != 0) {
    speck_expand_key_and_encrypt(in.ll, out, key);
    accum ^= out[0] ^ out[1];
  }

  return accum;
}

//
// Wrapper function to get a 32-bit hash value by using SPECK's key hash.
// SPECK's key hash requires a key and a salt.
//
// ht:     The HashTable.
// key:    The key to hash.
//
uint32_t hash(uint64_t salt[], char *key) {
  union {
    uint64_t full;
    uint32_t half[2];
  } value;

  value.full = keyed_hash(key, strlen(key), salt);

  return value.half[0] ^ value.half[1];
}
\end{lstlisting}

\lstset{language=C, style=c99}
\begin{lstlisting}[title=hash.h]

#ifndef __HASH_H__
#define __HASH_H__

#ifndef NIL
#define NIL (void *)0
#endif

#include <inttypes.h>
#include "ll.h"

//
// Struct definition for a HashTable.
//
// salt:       The salt of the HashTable (used for hashing).
// length:     The maximum number of entries in the HashTable.
// heads:      An array of linked list heads.
//
typedef struct HashTable {
  uint64_t salt[2];
  uint32_t length;
  ListNode **heads;
} HashTable;

//
// Constructor for a HashTable.
//
// length:  Length of the HashTable.
// salt:    Salt for the HashTable.
//
HashTable *ht_create(uint32_t length);

//
// Destructor for a HashTable.
//
// ht:  The HashTable.
//
void ht_delete(HashTable *ht);

//
// Searches a HashTable for a key.
// Returns the ListNode if found and returns NULL otherwise.
// This should call the ll_lookup() function.
//
// ht:      The HashTable.
// key:     The key to search for.
//
ListNode *ht_lookup(HashTable *ht, char *key);

//
// First creates a new ListNode from GoodSpeak.
// The created ListNode is then inserted into a HashTable.
// This should call the ll_insert() function.
//
// ht:          The HashTable.
// gs:          The GoodSpeak to add to the HashTable.
//
void ht_insert(HashTable *ht, GoodSpeak *gs);

#endif
\end{lstlisting}

\lstset{language=C, style=c99}
\begin{lstlisting}[title=hash.c]
#include "hash.h"
#include "ll.h"
#include <inttypes.h>
#include <stdlib.h>

//
// Constructor for a HashTable.
//
// length:  Length of the HashTable.
// salt:    Salt for the HashTable.
//
HashTable *ht_create(uint32_t length) {
  HashTable *ht = (HashTable *)malloc(sizeof(HashTable));
  if (ht) {
    ht->salt[0] = 0x85ae998311115ae3; // Il nome della rosa
    ht->salt[1] = 0xb6fac2ae33a40089;
    ht->length = length;
    ht->heads = (ListNode **)calloc(length, sizeof(ListNode *));
    return ht;
  }

  return (HashTable *)NIL;
}

// The rest of the functions you must implement yourselves.
\end{lstlisting}


\subsection{Linked Lists}
\epigraph{\emph{Education is a weapon, whose effect depends on who holds it in
his hands and at whom it is aimed.}}{---Joseph Stalin}

\noindent
A \emph{linked list} will be used to resolve hash collisions. Each node of
the linked list contains a \texttt{GoodSpeak struct} which contains
\emph{oldspeak} and its \emph{newspeak} translation if it exists. The \emph{key}
to search with in the linked list is \emph{oldspeak}.

\lstset{language=C, style=c99}
\begin{lstlisting}[title=ll.h]
#ifndef __LL_H__
#define __LL_H__

#ifndef NIL
#define NIL (void *)0
#endif

#include <stdbool.h>

// If flag is set, ListNodes that are queried are moved to the front.
extern bool move_to_front;

typedef struct ListNode ListNode;

//
// Struct definition of a ListNode.
//
// gs:  GoodSpeak struct containing oldspeak and its newspeak translation.
//
struct ListNode {
  GoodSpeak *gs;
  ListNode *next;
};

//
// Constructor for a ListNode.
//
// gs:  GoodSpeak struct containing oldspeak and its newspeak translation.
//
ListNode *ll_node_create(GoodSpeak *gs);

//
// Destructor for a ListNode.
//
// n:  The ListNode to free.
//
void ll_node_delete(ListNode *n);

//
// Destructor for a linked list of ListNodes.
//
// head:   The head of the linked list.
//
void ll_delete(ListNode *head);

//
// Creates and inserts a ListNode into a linked list.
//
// head:  The head of the linked list to insert in.
// gs:    GoodSpeak struct.
//
ListNode *ll_insert(ListNode **head, GoodSpeak *gs);

//
// Searches for a specific key in a linked list.
// Returns the ListNode if found and NULL otherwise.
//
// head:   The head of the linked list to search in.
// key:    The key to search for.
ListNode *ll_lookup(ListNode **head, char *key);

#endif
\end{lstlisting}


You will be implementing this in two forms, and comparing the performance of
the methods:
\begin{itemize}
\item Inserting each new word at the front of the list and
\item Inserting each new word at the front of the list, but \emph{each} time
it is searched for it is \emph{moved to the front} of the list.
\end{itemize}

The move-to-front technique moves a node that was just searched for in the
linked list to the front. This decreases look-up times for
frequently-searched-for nodes. You will learn more about optimality in your
future classes. You will keep track of the \emph{average number of links
followed} in order to find each word in a linked list.

%%%%%%%%%%%%%%%%%%%%%%%%%%%%%%%%%%%%%%%%%%%%%%%%%%%%%%%%
\medskip
\begin{prelab}{Pre-lab Part 2}
  \begin{enumerate}
    \item Draw the pictures to show the how elements are being inserted in different ways in the Linked list.
    \item Write down the pseudocode for the above functions in the Linked List
        data type.
  \end{enumerate}
\end{prelab}

%%%%%%%%%%%%%%%%%%%%%%%%%%%%%%%%%%%%%%%%%%%%%%%%%%%%%%%%

\subsection{Lexical Analysis with Regular Expressions}
\epigraph{\emph{Ideas are more powerful than guns. We would not let our enemies
have guns, why should we let them have ideas.}}{---Joseph Stalin}

\noindent
You will need a function to pick words from an input stream. The words should be
just valid words, and these can include \emph{contractions}. A valid word is any
sequence of one or more characters, that are considered part of the word
character set in regex. A word character set in regex contains characters from
$a-z$, $A-Z$, $0-9$, including the underscore character. Since your program also
accepts contractions like ``don't'' where you will have to account for apostrophes, and words like
``pseudo-code'' where you will have to account for hyphens.

You will need to write your own \emph{regular expression} for a word, utilizing
the \texttt{regex.h} library to lexically analyze the input stream for words. To
aid you with this, here is a simple module that lexically analyzes the input
stream using your regular expression. You are not required to use the module
itself, but it is \emph{mandatory} that you parse through an input stream for
words using at least one regular expression.

\begin{lstlisting}[title=parser.h]
#ifndef __PARSER_H__
#define __PARSER_H__

#include <regex.h>
#include <stdio.h>

//
// Returns the next word that matches the specified regular expression.
// Words are buffered and returned as they are read from the input file.
//
// infile:      The input file to read from.
// word_regex:  Pointer to a compiled regular expression for a word.
// returns:     The next word if it exists, a null pointer otherwise.
//
char *next_word(FILE *infile, regex_t *word_regex);

//
// Clears out the static word buffer.
//
void clear_words(void);

#endif
\end{lstlisting}

\begin{lstlisting}[title=parser.c]
#include "parser.h"
#include <regex.h>
#include <inttypes.h>
#include <stdio.h>
#include <stdlib.h>
#include <string.h>

#define BLOCK   4096

static char *words[BLOCK] = { NULL };  // Stores a block of words maximum.

//
// Returns the next word that matches the specified regular expression.
// Words are buffered and returned as they are read from the input file.
//
// infile:      The input file to read from.
// word_regex:  Pointer to a compiled regular expression for a word.
// returns:     The next word if it exists, a null pointer otherwise.
//
char *next_word(FILE *infile, regex_t *word_regex) {
  static uint32_t index = 0;  // Track the word to return.
  static uint32_t count = 0;  // How many words have we stored?

  if (!index) {
    clear_words();

    regmatch_t match;
    uint32_t matches = 0;
    char buffer[BLOCK] = { 0 };

    while (!matches) {
      if (!fgets(buffer, BLOCK, infile)) {
        return NULL;
      }

      char *cursor = buffer;

      for (uint16_t i = 0; i < BLOCK; i += 1) {
        if (regexec(word_regex, cursor, 1, &match, 0)) {
          break;  // Couldn't find a match.
        }

        if (match.rm_so < 0) {
          break;  // No more matches.
        }

        uint32_t start  = (uint32_t)match.rm_so;
        uint32_t end    = (uint32_t)match.rm_eo;
        uint32_t length = end - start;

        words[i] = (char *)calloc(length + 1, sizeof(char));
        if (!words[i]) {
          perror("calloc");
          exit(1);
        }

        memcpy(words[i], cursor + start, length);
        cursor += end;
        matches += 1;
      }

      count = matches;  // Words stored is number of matches.
    }
  }

  char *word = words[index];
  index = (index + 1) % count;
  return word;
}

//
// Clears out the static word buffer.
//
void clear_words(void) {
  for (uint16_t i = 0; i < BLOCK; i += 1) {
    if (words[i]) {
      free(words[i]);
      words[i] = NULL;
    }
  }

  return;
}
\end{lstlisting}

The function \texttt{next\_word()} requires two inputs, the input stream
\texttt{infile}, and a pointer to a compiled regular expression,
\texttt{word\_regex}. Notice the word \emph{compiled}: you must first compile
your regular expression using \texttt{regcomp()} before passing it to the
function. Make sure you remember to call the function \texttt{clear\_words()} to
free any memory used by the module when you're done reading in words. Note that
you will need to transform all of your words from \emph{mixed case} to
\emph{lowercase} before adding them to your Bloom filter and hash table.

\section{Your Task}
\epigraph{\emph{The people will believe what the media tells them they
believe.}}{---George Orwell}

\begin{itemize}
\item Read in a list of \emph{forbidden} words, setting the corresponding bit
    for each word in the Bloom filter. This list is the badspeak.txt file we mentioned earlier in the assignment.
\item Create a \texttt{GoodSpeak struct} for each forbidden word. The word
    should be stored in \texttt{oldspeak}, and \texttt{newspeak} should be
    set to \texttt{NULL}; forbidden words do not have translations.
\item Read in a space-separated list of \emph{oldspeak}, \emph{newspeak} pairs.
    This list is the newspeak.txt file we mentioned earlier in the assignment.
\item Create a \texttt{GoodSpeak struct} for each \emph{oldspeak},
    \emph{newspeak} pair, placing them in \texttt{oldspeak} and
    \texttt{newspeak} respectively.
\item The hash index for each \emph{forbidden} word is determined by using
    \texttt{oldspeak} as the key.
\item Read text from \emph{standard input} (I/O redirection must be supported).
\item Words that pass through the Bloom filter but have no translation are forbidden,
    which constitutes a \emph{thoughtcrime}.
\item The use of forbidden words constitutes a \emph{thoughtcrime}. If only
    forbidden words were used, you will send them a \emph{thoughtcrime message}.
\begin{lstlisting}[title=Example thoughtcrime message.]
Dear Comrade,

You have chosen to use degenerate words that may cause hurt
feelings or cause your comrades to think unpleasant thoughts.
This is doubleplus bad. To correct your wrongthink and
preserve community consensus we will be sending you to joycamp
administered by Miniluv.

Your errors:

kalamazoo
antidisestablishmentarianism
\end{lstlisting}
\item The use of only \emph{oldspeak} words that have \emph{newspeak}
    translations elicits a \emph{goodspeak message} to provide encouragement.
\begin{lstlisting}[title=Example goodspeak message.]
Dear Comrade,

Submitting your text helps to preserve feelings and prevent
badthink. Some of the words that you used are not goodspeak.
The list shows how to turn the oldspeak words into newspeak.

sad -> happy
liberty -> badfree
music -> noise
read -> papertalk
write -> papertalk
\end{lstlisting}
\item The use of forbidden words and words that have \emph{newspeak}
    translations warrants a combination of a \emph{thoughtcrime message} and a
    \emph{goodspeak} encouragement message.
\begin{lstlisting}[title=Example thoughtcrime/goodspeak message.]
Dear Comrade,

You have chosen to use degenerate words that may cause hurt
feelings or cause your comrades to think unpleasant thoughts.
This is doubleplus bad. To correct your wrongthink and
preserve community consensus we will be sending you to joycamp
administered by Miniluv.

Your errors:

kalamazoo
antidisestablishmentarianism

Think of these words on your vacation!

sad -> happy
liberty -> badfree
music -> noise
read -> papertalk
write -> papertalk
\end{lstlisting}
\item The list of proscribed words is available as
    \verb+~euchou/Newspeak/badspeak.txt+
    on \texttt{unix.ucsc.edu}.
\item The list of \emph{oldspeak} words and their respective \emph{newspeak}
    translations is available as \verb+~euchou/Newspeak/newspeak.txt+ on
    \texttt{unix.ucsc.edu}.
\item Your executable, \texttt{./newspeak}, will provide the following options:
        \begin{itemize}
        \item \texttt{./newspeak -s} will suppress the letter from the censor,
            and instead print the statistics that were computed.
        \item \texttt{./newspeak -h size} specifies that the hash table will
            have \texttt{size} entries (the default will be $10000$).
        \item \texttt{./newspeak -f size} specifies that the Bloom filter will
            have \texttt{size} entries (the default will be $2^{20}$).
        \item \texttt{./newspeak -m} will use the \emph{move-to-front rule}.
        \item \texttt{./newspeak -b} will not use the \emph{move-to-front rule}.
        \item ANY combination of these flags except for \texttt{-m -b}
            \emph{must be supported}.
        \end{itemize}
\end{itemize}

%%%%%%%%%%%%%%%%%%%%%%%%%%%%%%%%%%%%%%%%%%%%%%%%%%%%%%%%

\section{Deliverables}
\epigraph{\emph{I would rather have questions that can't be answered than
answers that can't be questioned.}}{---Richard P. Feynman}

You will need to turn in:

\begin{enumerate}
\item \texttt{newspeak.c}: This contains \texttt{main()} and \emph{may} contain
    the other functions necessary to complete the assignment.
\item \texttt{speck.c} and \texttt{speck.h}: These two files have already been pushed to your git repository in the asgn6 folder.
\item \texttt{hash.c} and \texttt{hash.h}: These contain your implementation of
    the hash table ADT.
\item \texttt{ll.c} and \texttt{ll.h}: These contain your implementation of the
    linked list ADT.
\item You may have other source and header files, but \emph{do not make things overly complicated}. For example the \texttt{parser.c} and
    \texttt{parser.h} have already been pushed to your git respository under the asgn6 folder.
\item \texttt{Makefile}: This is a file that will allow the grader to type
    \texttt{make} to compile your program. Typing \texttt{make} must build your
    program.
\begin{itemize}
    \item \texttt{CFLAGS=-Wall -Wextra -Werror -Wpedantic -std=c99} must be
        included.
    \item \texttt{CC=clang} must be specified.
    \item \texttt{make clean} must remove all files that are compiler generated.
    \item \texttt{make infer} runs \texttt{infer} on your program. Complaints
        generated by \texttt{infer} must either be fixed or explained in your
        \texttt{README}.
    \item \texttt{make} should build your program, as should \texttt{make all}.
    \item Your program executable \emph{must} be named \texttt{newspeak}.
\end{itemize}

\item \texttt{README.md}: This must be in markdown. This must describe how to
    use your program and \texttt{Makefile}. This should also contain
    explanations for any complaints from \texttt{infer}.

\item \texttt{DESIGN.pdf}: This \emph{must} be a PDF. The design document
    should contain answers to the pre-lab questions at the beginning and
    describe your design for your program with enough detail that a sufficiently
    knowledgeable programmer would be able to replicate your implementation.
    This does not mean copying your entire program in verbatim. You should
    instead describe how your program works with supporting pseudo-code. For
    this program, pay extra attention to how you build each necessary component.

\item \texttt{WRITEUP.pdf}: Your writeup should contain a discussion on the following topics:
\begin{itemize}
\item {What happens when you vary the size of a hash table?}
\item {What happens when you vary the Bloom filter size?}
\item {Do you really need the \textit{move to front} rule?}
\end{itemize}
    
    \textcolor{red}{You \emph{must} push the \texttt{DESIGN.pdf} before you push
    \emph{any} code.}
\end{enumerate}

All of these files must be in the directory \texttt{asgn6}.

%%%%%%%%%%%%%%%%%%%%%%%%%%%%%%%%%%%%%%%%%%%%%%%%%%%%%%%%
\section{Submission}
\epigraph{\emph{We can and must write in a language which sows among the
masses hate, revulsion, and scorn toward those who disagree with
us.}}{---Vladimir Lenin}

\noindent
To submit your assignment, refer back to \texttt{assignment0} for
the steps on how to submit your assignment through \texttt{git}.
Remember: \emph{add, commit,} and \emph{push}!

\textcolor{red}{Your assignment is turned in \emph{only} after you
have pushed and turned in the commit ID to Canvas. If you forget to push,
you have not turned in your assignment and you will get a \emph{zero}.
``I forgot to push'' is not a valid excuse. It is \emph{highly} recommended
to commit and push your changes \emph{often}.}
\end{document}

%%%%%%%%%%%%%%%%%%%%%%%%%%%%%%%%%%%%%%%%%%%%%%%%%%%%%%%%
\section{Supplemental Readings}
\epigraph{\emph{The more you read, the more things you will know. The
more that you learn, the more places you'll go.}}{---Dr.\ Seuss}\noindent

\begin{itemize}
    \item \textit{The C Programming Language} by Kernighan \& Ritchie
    \begin{itemize}
	\item Chapter 1 \S 1.5
        \item Chapter 5 \S 5.6--5.12
	\item Chapter 8 \S 8.1--8.3
    \end{itemize}
\end{itemize}
