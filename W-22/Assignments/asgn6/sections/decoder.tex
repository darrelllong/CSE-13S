\section{The Decoder}
\epigraphwidth=0.5\textwidth
\epigraph{\emph{The way I see it, there are just two ways to go, livin' fast or dyin' slow! Which way you gonna go?}}{---Robert Earl Keen}

\noindent
The second task for this assignment is to implement a Huffman decoder.
This decoder will read in a compressed input file and decompress it,
expanding it back to its original, uncompressed size. Your decoder
program, named \texttt{decode}, must support any combination of the
following command-line options.

\begin{itemize}
  \item \textbf{\texttt{-h}}\,: Prints out a help message describing the purpose
    of the program and the command-line options it accepts, exiting the
    program afterwards. Refer to the reference program in the resources
    repo for an idea of what to print.

  \item \textbf{\texttt{-i infile}}\,: Specifies the input file to
    decode using Huffman coding. The default input should be set as
    \texttt{stdin}.

  \item \textbf{\texttt{-o outfile}}\,: Specifies the output file to
    write the decompressed input to. The default output should be set as
    \texttt{stdout}.

  \item \textbf{\texttt{-v}}\,: Prints decompression statistics to
    \texttt{stderr}. These statistics include the compressed file size,
    the decompressed file size, and \emph{space saving}. The formula for
    calculating space saving is:
    \[
      100 \times (1 - (\text{compressed size} / \text{decompressed
      size})).
    \]

    Refer to the reference program in the resources repository for the
    exact output.
\end{itemize}

\noindent The algorithm to decode a file, or to decompress it, is as follows:

\begin{enumerate}
  \item Read the emitted (\emph{dumped}) tree from the input file. A
    \emph{stack of nodes} is needed in order to reconstruct the Huffman
    tree.

  \item Read in the rest of the input file bit-by-bit, traversing down
    the Huffman tree one link at a time. Reading a 0 means walking down
    the left link, and reading a 1 means walking down the right link.
    Whenever a leaf node is reached, its symbol is emitted and you start
    traversing again from the root.
\end{enumerate}
