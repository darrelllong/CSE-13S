\section{A Huffman Coding Module}
\epigraphwidth=0.66\textwidth
\epigraph{\emph{Fiction does not spring into the world fully grown, like Athena. It is the process of writing and rewriting that makes a fiction original, if not profound.}}
{---John Gardner, \emph{The Art of Fiction: Notes on Craft for Young Writers}}

An interface for a Huffman coding module that you will need to implement
will be given in \texttt{huffman.h}. Do not worry if you do not
initially understand the exact purpose of each function, as they will be
clarified in \S 10 and \S 11. The interface is just given now as a
reference for which functions are used in the aforementioned sections.

\begin{funcdoc}{\texttt{Node *build\_tree(uint64\_t hist[static ALPHABET])}}
  Constructs a Huffman tree given a computed histogram. The histogram
  will have \texttt{ALPHABET} indices, one index for each possible
  symbol. Returns the root node of the constructed tree. The use of
  \texttt{static} array indices in parameter declarations is a C99
  addition. In this case, it informs the compiler that the histogram
  \texttt{hist} should have \emph{at least} \texttt{ALPHABET} number of
  indices.
\end{funcdoc}

\begin{funcdoc}{\texttt{void build\_codes(Node *root, Code table[static ALPHABET])}}
  Populates a code table, building the code for each symbols in the
  Huffman tree. The constructed codes are copied to the code table,
  \texttt{table}, which has \texttt{ALPHABET} indices, one index for
  each possible symbol.
\end{funcdoc}

\begin{funcdoc}{\texttt{void dump\_tree(int outfile, Node *root)}}
  Conducts a \emph{post-order traversal} of the Huffman tree rooted at
  \texttt{root}, writing it to \texttt{outfile}. This should write an
  \texttt{`L'} followed by the byte of the symbol for each leaf, and an
  \texttt{`I'} for interior nodes. You \emph{should not} write a symbol
  for an interior node.
\end{funcdoc}

\begin{funcdoc}{\texttt{Node *rebuild\_tree(uint16\_t nbytes, uint8\_t tree\_dump[static nbytes])}}
  Reconstructs a Huffman tree given its post-order tree dump stored in
  the array \texttt{tree\_dump}. The length in bytes of
  \texttt{tree\_dump} is given by \texttt{nbytes}. Returns the root node
  of the reconstructed tree.
\end{funcdoc}

\begin{funcdoc}{\texttt{void delete\_tree(Node **root)}}
  The destructor for a Huffman tree. This will require a post-order
  traversal of the tree to free all the nodes. Remember to set the
  pointer to \texttt{NULL} after you are finished freeing all the
  allocated memory.
\end{funcdoc}
