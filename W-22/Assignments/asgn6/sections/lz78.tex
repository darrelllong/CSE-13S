\section{Lempel-Ziv Compression}\label{sec:lz78}

Abraham Lempel and Jacob Ziv published papers for two lossless compression
algorithms, LZ77 and LZ78, published in 1977 and 1978, respectively. The core
idea behind both of these compression algorithms is to represent repeated
patterns in data with using pairs which are each comprised of a code and a
symbol. A code is typically an unsigned 16-bit value and a symbol is a 8-bit
ASCII character.

\subsection{A Compression Example}

We will explain this algorithm by working through an example. Assume we want to
compress, or \emph{encode}, some sequence of symbols ``$abab$''. To do this, we
will need a dictionary. Each dictionary entry is a key-value pair, where the key
is some \emph{prefix} string, or a \emph{word}, and the value is a code. If we
establish that codes can be any 16-bit unsigned value, then we have must have
$2^{16}$ different codes.

Two of these $2^{16}$ codes are \emph{reserved} for very specific purposes: the
\emph{stop code} and the \emph{empty code}. The stop code is used to indicate
the end of encoded data, and the empty code is used to indicate the \emph{empty
word}, which is just a word consisting of zero symbols. The encoding dictionary
is initialized with the empty word.

\begin{clisting}{defines.h}
#pragma once

#include <stdint.h>

typedef uint8_t Symbol;
typedef uint32_t Code;

#define STOP_CODE  0          // Signals end of encoding/decoding.
#define EMPTY_CODE 1          // Code denoting the empty word.
#define START_CODE 2          // Starting code of new words.
#define MAX_CODE   (1 << 16)  // First unusable code.
#define ALPHABET   256        // 256 symbols (ASCII + extended ASCII).
#define MAGIC      0x8badbeef // 32-bit magic number.
#define BLOCK      4096       // 4KB block size.
\end{clisting}

You may be wondering why a \texttt{Code} is \texttt{typedef}'d as a
\texttt{uint32\_t} instead of a \texttt{uint16\_t}. After all, isn't a code just
a unsigned 16-bit integer? The reason for this will be made clear at the end of
this compression example.

We start off the example by considering the first symbol: `$a$'. For the
example, all symbols will be surrounded with single quotes, and all words will
be surrounded with double quotes. We will need to track the word consisting of
all previously seen symbols up to this point in time. We'll call this the
\emph{current word}. Since we are currently on the first symbol, the current
word must not contain anything, and thus must be the empty word.

We now append the current symbol to the current word, which produces a new word:
``$a$''. We check the dictionary to see if we've encountered this word before.
You can refer to the state of the dictionary during each stage of this example
in Figure~\ref{fig:compression-example}. Since this word isn't in the
dictionary, we will need to add it, assigning it the next available code of 2,
or \texttt{START\_CODE}, since this is the first non-reserved code that's been
assigned. We then reset our current word back to being the empty word and output
the pair (\texttt{EMPTY\_CODE},\;`$a$').

\begin{figure}[H]
    \begin{tabular}{ccc}
        \begin{minipage}{.24\textwidth}
            \caption*{\textbf{Initial state}}
            \centering
            \begin{tabular}{c|c}
                \textbf{Word} & \textbf{Code} \\
                \hline
                ``'' & \texttt{EMPTY\_CODE} \\
                \hline
                     & \texttt{START\_CODE} \\
                \hline
                & 3 \\
                \hline
                & 4 \\
                \hline
                & $\ldots$ \\
           \end{tabular}
        \end{minipage}
        \begin{minipage}{.24\textwidth}
            \caption*{\textbf{Adding ``$a$''}}
            \centering
            \begin{tabular}{c|c}
                \textbf{Word} & \textbf{Code} \\
                \hline
                ``'' & \texttt{EMPTY\_CODE} \\
                \hline
                ``$a$'' & \texttt{START\_CODE} \\
                \hline
                & 3 \\
                \hline
                & 4 \\
                \hline
                & $\ldots$ \\
           \end{tabular}
        \end{minipage}
        \begin{minipage}{.24\textwidth}
            \caption*{\textbf{Adding ``$b$''}}
            \centering
            \begin{tabular}{c|c}
                \textbf{Word} & \textbf{Code} \\
                \hline
                ``'' & \texttt{EMPTY\_CODE} \\
                \hline
                ``$a$'' & \texttt{START\_CODE} \\
                \hline
                ``$b$'' & 3 \\
                \hline
                & 4 \\
                \hline
                & $\ldots$ \\
           \end{tabular}
        \end{minipage}
        \begin{minipage}{.24\textwidth}
            \caption*{\textbf{Adding ``$ab$''}}
            \centering
            \begin{tabular}{c|c}
                \textbf{Word} & \textbf{Code} \\
                \hline
                ``'' & \texttt{EMPTY\_CODE} \\
                \hline
                ``$a$'' & \texttt{START\_CODE} \\
                \hline
                ``$b$'' & 3 \\
                \hline
                ``$ab$'' & 4 \\
                \hline
                & $\ldots$ \\
           \end{tabular}
        \end{minipage}
    \end{tabular}
    \caption{The state of the encoding dictionary at each stage of the LZ78
    compression example.}
    \label{fig:compression-example}
\end{figure}

This last step is important. Whenever appending a symbol to the current word
yields a word that hasn't yet been encountered, it is assigned the next
available code and added to the dictionary before emitting its corresponding
pair. The pair itself is constructed using the code that corresponds to the
current word and the symbol that was just processed.

We continue on, now considering the second symbol: `$b$'. We append this
symbol to our current word, which yields ``$b$'' as a new word. We check the
dictionary to see if we've encountered this word before. Since we haven't, we
add it to the dictionary and assign it the next available code of 3. We set the
current word to the empty word and output the pair
(\texttt{EMPTY\_CODE},\;`$b$').

The next symbol we see is `$a$'. We append this symbol to our current word,
which yields ``$a$'' as a new word. We check the dictionary to see if we've
encountered this word before. Indeed we have. We update the current word to
``$a$'' and proceed to the next symbol.

We read in the last symbol, `$b$'. We append this character to our current word,
which yields ``$ab$''. We check if this new word is in the dictionary. Since it
isn't, we add it in, assigning it the next available code of 4. We then output
the pair (2,\;`$b$'). Why is 2 used in the code? Because that is the code that
corresponds to the current word, ``$a$''.

To finish off the encoding, we output the final pair (\texttt{STOP\_CODE},\;0).
As you can imagine, this pair signals the end of compressed data. The symbol in
this final pair isn't used and the value is of no significance.

Back to the earlier question of why a \texttt{Code} is \texttt{typedef}'d as a
\texttt{uint32\_t} instead of a \texttt{uint16\_t}. It should be clear that
codes are assigned in \emph{monotonically increasing order}. At one point, we
will run out of codes to assign. How do we know when we've run out of codes?
When the next available code to assign is equal to \emph{one more than the
largest value that a 16-bit unsigned integer can hold.} This is $2^{16}$, or
\texttt{MAX\_CODE}. Since a \texttt{uint16\_t} can't hold this value, a
\texttt{uint32\_t} is used instead.

\subsection{A Decompression Example}

Of course, a compression algorithm is useless without a corresponding
decompression algorithm. For this example, we will decode the output produced by
the compression example from before. The output contains the following pairs in
order: (\texttt{EMPTY\_CODE},\;`$a$'), (\texttt{EMPTY\_CODE},\;`$b$'),
(2,\;`$b$'), (\texttt{STOP\_CODE},\,0). Similar to the compression algorithm,
the decompression algorithm initializes a dictionary containing only the empty
word. The catch is that the key and value for the decompressing dictionary is
swapped; each key is instead a code and each value a word. As you might imagine,
the decompression algorithm is the inverse of the compression algorithm, and
thus from the output pairs, the decompression algorithm will recreate the same
dictionary used during compression to output the decompressed data.

We start with the first pair: (\texttt{EMPTY\_CODE},\;`$a$'). We append the
symbol `$a$' to the word corresponding to \texttt{EMPTY\_CODE}, which is the
empty word. Thus, the current word is ``$a$''. We add this word to the
dictionary and assign it the next available code of 2, then output the current
word ``$a$''. You can refer to the state of the dictionary at each stage of this
example in Figure~\ref{fig:decompression-example}.

We consider the next pair: (\texttt{EMPTY\_CODE},\;`$b$'). We append the symbol
`$b$' to the word corresponding to \texttt{EMPTY\_CODE}, which is the empty
word. Thus, the current word is ``$b$''. We add this word to the dictionary and
assign it the next available code of 3, then output the current word ``$b$''.

We consider the next pair: (2,\;`$b$'). We append the symbol `$b$' to the word
corresponding to code 2, which we previously added to our dictionary. The word
denoted by this code is ``$a$'', whence we obtain our current word of ``$ab$''.
We add this word to the dictionary and assign it the next available code of 4,
then output the current word ``$ab$''. Finally, we read in the last pair:
(\texttt{STOP\_CODE},\;0). Since the code is \texttt{STOP\_CODE}, we know that
we have finished decompression.

\begin{figure}[htb]
    \begin{tabular}{ccc}
        \begin{minipage}{.24\textwidth}
            \caption*{\textbf{Initial state}}
            \centering
            \begin{tabular}{c|c}
                \textbf{Code} & \textbf{Word} \\
                \hline
                \texttt{EMPTY\_CODE} & ``'' \\
                \hline
                \texttt{START\_CODE} & \\
                \hline
                3 & \\
                \hline
                4 & \\
                \hline
                $\ldots$ & \\
           \end{tabular}
        \end{minipage}
        \begin{minipage}{.24\textwidth}
            \caption*{\textbf{Adding ``$a$''}}
            \centering
            \begin{tabular}{c|c}
                \textbf{Code} & \textbf{Word} \\
                \hline
                \texttt{EMPTY\_CODE} & ``'' \\
                \hline
                \texttt{START\_CODE} & ``$a$'' \\
                \hline
                3 & \\
                \hline
                4 & \\
                \hline
                $\ldots$ & \\
           \end{tabular}
        \end{minipage}
        \begin{minipage}{.24\textwidth}
            \caption*{\textbf{Adding ``$b$''}}
            \centering
            \begin{tabular}{c|c}
                \textbf{Code} & \textbf{Word} \\
                \hline
                \texttt{EMPTY\_CODE} & ``'' \\
                \hline
                \texttt{START\_CODE} & ``$a$'' \\
                \hline
                3 & ``$b$'' \\
                \hline
                4 & \\
                \hline
                $\ldots$ & \\
           \end{tabular}
           \label{ahaha}
        \end{minipage}
        \begin{minipage}{.24\textwidth}
            \caption*{\textbf{Adding ``$ab$''}}
            \centering
            \begin{tabular}{c|c}
                \textbf{Code} & \textbf{Word} \\
                \hline
                \texttt{EMPTY\_CODE} & ``'' \\
                \hline
                \texttt{START\_CODE} & ``$a$'' \\
                \hline
                3 & ``$b$'' \\
                \hline
                4 & ``$ab$'' \\
                \hline
                $\ldots$ & \\
           \end{tabular}
        \end{minipage}
    \end{tabular}
    \caption{The state of the decoding dictionary at each stage of the LZ78
    decompression example.}
    \label{fig:decompression-example}
\end{figure}

If the basic idea behind the compression and decompression algorithms do not
immediately make sense to you, or if you desire a more visual representation of
how the algorithms work, make sure to attend section and get help early!
\textcolor{red}{Things will not get easier as time goes on.}
