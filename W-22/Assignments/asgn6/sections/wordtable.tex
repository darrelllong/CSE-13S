\section{Word Tables}\label{sec:wordtable}

Although compression can be performed using a trie, decompression still needs to
use a look-up table for quick code-to-word translation. This look-up table will
be encapsulated in a \texttt{struct} called a \texttt{WordTable}, which contains
an array of words. Since we only have $2^{16}$ codes, we can used a fixed array
of size \texttt{MAX\_CODE}, as defined in \S\ref{sec:lz78}.

\begin{clisting}{}
struct WordTable {
    Word *words[MAX_CODE];
};

\end{clisting}

Each index of the word array is a pointer to a \texttt{Word}, a \texttt{struct}
created to abstractly represent a word. A \texttt{Word} contains an array of
symbols, each symbol in which is a byte. The length of this byte array is
contained in the \texttt{Word} as well.

\begin{clisting}{}
struct Word {
    uint8_t *syms;   // Array of bytes.
    uint32_t length; // Length of array.
};
\end{clisting}

The reason we aren't using strings for words is because we want to be able to
encode and decode any file. Binary files typically have zeroes strewn
throughout, which causes problems with strings since they are
\texttt{NULL}-terminated in \textbf{C}. To avoid this, we use byte arrays to
represent words. This approach, however, does require us to track the length of
each byte array. You \emph{must} use the following specification for the
\texttt{Word} and \texttt{WordTable} ADTs.

\begin{funcdoc}{Word *word\_create(uint8\_t *syms, uint32\_t length)}
  Constructor for a word where \texttt{syms} is the array of symbols a
  \texttt{Word} represents. The length of the array of symbols is given by
  \texttt{length}. To create the empty word, a \texttt{NULL} pointer is passed
  for \texttt{syms} and a 0 is used for the \texttt{length}. How exactly the
  empty word is created is up to you. Do keep in mind, however, that allocating
  0 bytes using either \texttt{malloc()} or \texttt{calloc()} is undefined.
\end{funcdoc}

\begin{funcdoc}{Word *word\_append\_sym(Word *w, Symbol s)}
  Returns a new \texttt{Word} containing the symbols in \texttt{w} appended with
  the symbol \texttt{s}. Keep in mind that the word specified to append to may
  be the empty word.
\end{funcdoc}

\begin{funcdoc}{void word\_delete(Word **w)}
  Destructor for a \texttt{Word}. Like with \texttt{trie\_node\_create()}, a
  double pointer is used here. Make sure to set the pointer to the word to
  \texttt{NULL} after freeing all allocated memory.
\end{funcdoc}

\begin{funcdoc}{WordTable *wt\_create(void)}
  Creates a new \texttt{WordTable}, initialized with the empty word. Make sure
  the other indices in the table are initialized to \texttt{NULL}.
\end{funcdoc}

\begin{funcdoc}{void wt\_reset(WordTable *wt)}
  Resets a \texttt{WordTable} to its initial state of only containing the empty
  word. Make sure all the other words in the table have been deleted and
  properly set to \texttt{NULL}.
\end{funcdoc}

\begin{funcdoc}{void wt\_delete(WordTable **wt)}
  Destructor for a \texttt{WordTable}. Remember to set the passed pointer to
  \texttt{NULL} after freeing all allocated memory.
\end{funcdoc}
