\section{Decompression}\label{sec:decompression}

The following steps for decompression will refer to the input file to decompress
as \texttt{infile} and the uncompressed output file as \texttt{outfile}.

\begin{enumerate}
    \item Open \texttt{infile} with \texttt{open()}. If an error occurs, print a
      helpful message and exit with a status code indicating that an error
      occurred. \texttt{infile} should be \texttt{stdin} if an input file wasn't
      specified.
    \item Read in the file header with \texttt{read\_header()}, then verify the
      magic number. If it's verified, then decompression is good to go and you
      now have a header which contains the original protection bit mask.
      Otherwise, print an error message reporting the invalid magic number and
      stop the decoder program.
    \item Open \texttt{outfile} using \texttt{open()}. The permissions for
      \texttt{outfile} should match the protection bits as set in your file
      header that you just read. Any errors with opening \texttt{outfile} should
      be handled like with \texttt{infile}. \texttt{outfile} should be
      \texttt{stdout} if an output file wasn't specified.
    \item Create a new word table with \texttt{wt\_create()} and make sure each
      of its entries are set to \texttt{NULL}. Initialize the table to have just
      the empty word, a word of length 0, at the index \texttt{EMPTY\_CODE}. We
      will refer to this table as \texttt{table}.
    \item You will need two \texttt{Code}s to keep track of the current code and
      next code. These will be referred to as \texttt{curr\_code} and
      \texttt{next\_code}, respectively. \texttt{next\_code} should be
      initialized as \texttt{START\_CODE} and functions exactly the same as the
      monotonic counter used during compression, which was also called
      \texttt{next\_code}.
    \item Use \texttt{read\_pair()} in a loop to read all the pairs from
      \texttt{infile}. We will refer to the code and symbol from each read pair
      as \texttt{curr\_code} and \texttt{curr\_sym}, respectively. The
      bit-length of the code to read is the bit-length of \texttt{next\_code}.
      The loop breaks when the code read is \texttt{STOP\_CODE}. For each read
      pair, perform the following:
        \begin{enumerate}
            \item As seen in the decompression example, we will need to append
                the read symbol with the word denoted by the read code and add
                the result to \texttt{table} at the index \texttt{next\_code}.
                The word denoted by the read code is stored in
                \texttt{table->words[curr\_code]}. We will append
                \texttt{table->words[curr\_code]} and \texttt{curr\_sym} using
                \texttt{word\_append\_sym()}.
            \item Write the word that we just constructed and added to the
                table with \texttt{write\_word()}. This word should have been
            \item Increment \texttt{next\_code} and check if it equals
                \texttt{MAX\_CODE}. If it has, reset the table using
                \texttt{wt\_reset()} and set \texttt{next\_code} to be
                \texttt{START\_CODE}. This mimics the resetting of the trie
                during compression.
        \end{enumerate}
    \item Flush any buffered words using \texttt{flush\_words()}. Like
      with \texttt{write\_pair()}, \texttt{write\_word()} buffers words
      under the hood, so we have to remember to flush the contents of
      our buffer.
    \item Close \texttt{infile} and \texttt{outfile} with \texttt{close()}.
\end{enumerate}
