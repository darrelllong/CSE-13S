\section{Command-line Options}

Your \texttt{encode} program must support the following \texttt{getopt()}
options:

\begin{itemize}
    \item \texttt{-h} : Display program help and usage.
    \item \texttt{-v} : Print compression statistics to \texttt{stderr}.
    \item \texttt{-i <input>} : Specify input to compress (\texttt{stdin} by
        default)
    \item \texttt{-o <output>} : Specify output of compressed input
        (\texttt{stdout} by default)
\end{itemize}

Your \texttt{decode} program must support the following \texttt{getopt()}
options:

\begin{itemize}
    \item \texttt{-h} : Display program help and usage.
    \item \texttt{-v} : Print decompression statistics to
      \texttt{stderr}.
    \item \texttt{-i <input>} : Specify input to decompress (\texttt{stdin} by
        default)
    \item \texttt{-o <output>} : Specify output of decompressed input
        (\texttt{stdout} by default)
\end{itemize}

The verbose option enables a flag to print out informative statistics about the
compression or decompression that is performed. These statistics include the
compressed file size, the uncompressed file size, and \emph{space saving}. The
formula for calculating space saving is:
\[
  100\times\left ( 1-\frac{\text{compressed size}} {\text{uncompressed size}} \right).
\]
The verbose output of both \texttt{encode} and \texttt{decode} must match the
following:

\begin{shlisting}{}
Compressed file size: X bytes
Uncompressed file size: X bytes
Space saving: XX.XX%
\end{shlisting}
