\section{Specifics}

The following subsections will cover the specifics for the encoder and
the decoder, covering each step of the encoding and decoding algorithm.

\subsection{Specifics for the Encoder}\label{encoder_specifics}

For this section, the input file to compress will be referred to as
\texttt{infile} and the compressed output file as \texttt{outfile}. Your
encoder, after parsing command-line options with \texttt{getopt()}, must
perform the following steps exactly to produce the correct Huffman
encoding:

\begin{enumerate}
  \item Read through \texttt{infile} to construct a histogram. Your
    histogram should be a simple array of 256 (\texttt{ALPHABET})
    \texttt{uint64\_t}s.

  \item Increment the count of element 0 and element 255 by one in the
    histogram. This is so that at the very minimum, the histogram will
    have two elements present. Do this regardless of what you read in.
    While doing this may result in a slightly sub-optimal Huffman tree
    later on, it is a quick and clean solution to handling the case when
    a file has no bytes or contains only one unique symbol.

  \item Construct the Huffman tree using a priority queue. This will
    be done using \texttt{build\_tree()}.

    \begin{enumerate}
      \item Create a priority queue. For each symbol histogram where its
        frequency is greater than $0$ (there should be at minimum two
        elements because of step 2), create a corresponding
        \texttt{Node} and insert it into the priority queue.

      \item While there are two or more nodes in the priority queue,
        dequeue two nodes. The first dequeued node will be the left
        child node. The second dequeued node will be the right child
        node. Join these nodes together using \texttt{node\_join()} and
        enqueue the joined parent node. The frequency of the parent node
        is the sum of its left child's frequency and its right child's
        frequency.

      \item Eventually, there will only be one node left in the priority
        queue. This node is the \emph{root} of the constructed Huffman
        tree.
    \end{enumerate}

  \item Construct a code table by traversing the Huffman tree. This will
    be done using \texttt{build\_codes()}. The code table is a simple
    array of 256 (\texttt{ALPHABET}) \texttt{Code}s.

    \begin{enumerate}
    \item Create a new \texttt{Code} \texttt{c} using \texttt{code\_init()}.
        Starting at the root of the Huffman tree, perform a
        \emph{post-order} traversal.
      \item If the current node is a leaf, the current code \texttt{c}
        represents the path to the node, and thus is the code for the
        node's symbol. Save this code into code table.
      \item Else, the current node must be an interior node. Push a 0 to
        \texttt{c} and recurse down the left link.
      \item After you return from the left link, pop a bit from
        \texttt{c}, push a 1 to \texttt{c} and recurse down the right
        link. Remember to pop a bit from \texttt{c} when you return from
        the right link.
      \end{enumerate}

  \item Construct a \emph{header}. A header is defined by the following
    \texttt{struct} definition, which will be supplied to you in
    \texttt{header.h}:
    \begin{clisting}{}
typedef struct {
    uint32_t magic;       // 32-bit magic number.
    uint16_t permissions; // Input file permissions.
    uint16_t tree_size;   // Emitted tree size in bytes.
    uint64_t file_size;   // Input file size.
} Header;
    \end{clisting}

    The header's magic number field, \texttt{magic}, should be set to
    the macro \texttt{MAGIC}, as defined in \texttt{defines.h}. This
    magic number identifies a file as one which has been compressed
    using your encoder. It is crucial that you use this magic number and
    nothing else.

    The header's \texttt{permissions} field stores the original
    permission bits of \texttt{infile}. You can get these permissions by
    using \texttt{fstat()}. You should also set the permissions of
    \texttt{outfile} to match the permissions of \texttt{infile} using
    \texttt{fchmod()}.

    The header's \texttt{tree\_size} field represents the number of
    bytes that make up the Huffman tree dump. This size is calculated as
    $(3 \times \text{unique symbols}) - 1$.

    Finally, the header's \texttt{file\_size} field is the size in bytes
    of the file to compress, \texttt{infile}. You obtain this size
    through \texttt{fstat()} as well.

  \item Write the constructed header to \texttt{outfile}.

  \item Write the constructed Huffman tree to \texttt{outfile} using
    \texttt{dump\_tree()}.

  \item Starting at the beginning of \texttt{infile}, write the
    corresponding code for each symbol to \texttt{outfile} with
    \texttt{write\_code()}. When finished with all the symbols, make
    sure to flush any remaining buffered codes with
    \texttt{flush\_codes()}.

  \item Close \texttt{infile} and \texttt{outfile}.
\end{enumerate}

\subsection{Specifics for the Decoder}\label{decoder_specifics}

For this section, the input file to decompress will be referred to as
\texttt{infile} and the compressed output file as \texttt{outfile}. Your
decoder, after parsing command-line options with \texttt{getopt()}, must
perform the following steps exactly to decode the file:

\begin{enumerate}
  \item Read in the header from \texttt{infile} and verify the magic
    number. If the magic number does not match \texttt{0xBEEFBBAD}
    (defined as \texttt{MAGIC} in \texttt{defines.h}), then an invalid
    file was passed to your program. Display a helpful error message and
    quit.

  \item The \texttt{permissions} field in the header indicates the
    permissions that \texttt{outfile} should be set to. Set the
    permissions using \texttt{fchmod()}.

  \item The size of the dumped tree is given by the \texttt{tree\_size}
    field in the header. Read the dumped tree from \texttt{infile} into
    an array that is \texttt{tree\_size} bytes long. Then, reconstruct
    the Huffman tree using \texttt{rebuild\_tree()}.\label{rebuild}

    \begin{enumerate}
      \item The array containing the dumped tree will be referred to as
        \texttt{tree\_dump}. The length of this array will be
        \texttt{nbytes}. A stack of nodes will be needed to reconstruct
        the tree.

      \item Iterate over the contents \texttt{tree\_dump} from $0$ to
        \texttt{nbytes}.

      \item If the element of the array is an \texttt{`L'}, then the
        next element will be the symbol for the leaf node. Use that
        symbol to create a new node with \texttt{node\_create()}. Push
        the created node onto the stack.

      \item If the element of the array is an \texttt{`I'}, then you
        have encountered an interior node. Pop the stack once to get the
        \emph{right} child of the interior node, then pop again to get
        the \emph{left} child of the interior node. Note: the pop order
        \emph{is important}. Join the left and right nodes with
        \texttt{node\_join()} and push the joined parent node on the
        stack.

      \item There will be one node left in the stack after you finish
        iterating over the contents \texttt{tree\_dump}. This node is
        the root of the Huffman tree.
    \end{enumerate}

  \item Read \texttt{infile} one bit at a time using
    \texttt{read\_bit()}. You will be traversing down the tree one link
    at a time for each bit that is read.

    \begin{enumerate}
      \item Begin at the root of the Huffman tree. If a bit of value 0
        is read, then walk down to the left child of the current node.
        Else, if a bit of value 1 is read, then walk down to the right
        child of the current node.

      \item If you find yourself at a leaf node, then write the leaf
        node's symbol to \texttt{outfile}. Note: you may alternatively
        buffer these symbols and write out the buffer whenever it is
        filled (this will be more efficient). After writing the symbol,
        reset the current node back to the root of the tree.

      \item Repeat until the number of decoded symbols matches the
        original file size, which is given by the \texttt{file\_size}
        field in the header that was read from \texttt{infile}.
    \end{enumerate}

  \item Close \texttt{infile} and \texttt{outfile}.
\end{enumerate}
