\section{Tries}\label{sec:tries}

The most costly part of compression is checking for existing prefixes, or words.
You could utilize a hash table, or just an array to store words, but that
wouldn't be optimal, as many of the words you need to store are prefixes of
other words. Instead you choose to utilize a \emph{trie}.

A trie\footnote{Edward Fredkin, ``{T}rie memory.'' \emph{Communications of the
ACM} 3, no. 9 (1960): 490--499.} is an efficient information re-\emph{trie}-val
data structure, commonly known as a prefix tree. Each node in a trie represents
a symbol, or a character, and contains $n$ child nodes, where $n$ is the size of
the alphabet you are using. In most cases, the alphabet used is the set of ASCII
characters, so $n = 256$. You will use a trie during compression to store words.

\begin{center}
    \tikzset{
        sibling distance=1.2cm, level distance=0.8cm,
        split/.style={
            draw, rectangle split, rectangle split parts=2,draw,inner
            sep=0pt,rectangle split horizontal,minimum size=0ex,text
            width=1ex,align=center,rectangle split part align=base
        },
        boxed/.style={draw,minimum size=3ex,inner sep=0pt,align=center},
        edge from parent/.style={draw, edge from parent path={[->,thick]
        (\tikzparentnode)  -- ($(\tikzchildnode.north) + 2*(0pt,1pt)$) }}
    }

    \begin{tikzpicture}
        \Tree
        [.\node[boxed] (S1) {\nodepart{one}\texttt{ ROOT }};
            [.\node[boxed] (L1) {\nodepart{one}b};
                [.\node[boxed] (L2) {y}; ]
            ]
            [.\node[boxed] (M1) {\nodepart{one}s};
                [.\node[boxed] (L3) {e};
                    [.\node[boxed] (L4) {a};]
                    [.\node[boxed] (R4) {l};
                        [.\node[boxed] (M2) {l};
                            [.\node[boxed] (M3) {s}; ]
                        ]
                    ]
                ]
                [.\node[boxed] (R3) {h};
                    [.\node[boxed] (L5) {e};
                        [.\node[boxed] (M4) {l};
                            [.\node[boxed] (M5) {l};
                                [.\node[boxed] (M6) {s}; ]
                            ]
                        ]
                    ]
                    [.\node[boxed] (R5) {o};
                        [.\node[boxed] (M7) {r};
                            [.\node[boxed] (M8) {e};]
                        ]
                    ]
                ]
            ]
            [.\node[boxed] (R1) {\nodepart{one}t};
                [.\node[boxed] (R2) {h};
                    [.\node[boxed] (M9) {e}; ]
                ]
            ]
        ]
    \end{tikzpicture}
\end{center}

Above is an example of a trie containing the following words: ``$She$'',
``$sells$'', ``$sea$'', ``$shells$'', ``$by$'', ``$the$'', ``$sea$'',
``$shore$''. Searching for a word in the trie means stepping down for each
symbol in the word, starting from the root. Stepping down the trie is simply
checking if the current node we have traversed to has a child node representing
the symbol we are looking for, and setting the current node to be the child node
if it does exist. Thus, to find ``$sea$'', we would start from the trie's root
and step down to `$s$', then `$e$', then `$a$'.  If any symbol is missing, or
the end of the trie is reached without fully matching a word, while stepping
through the trie, then the word is not in the trie.  You \emph{must} follow the
specification shown below when implementing your trie ADT.

\begin{clisting}{}
struct TrieNode {
    TrieNode *children[ALPHABET];
    Code code;
};
\end{clisting}

The \texttt{TrieNode struct} will have the two fields shown above. Each trie
node has an array of 256 pointers to trie nodes as children, one for each ASCII
character. It should be easy to see how this simplifies searching for the next
character in a word in the trie. The \texttt{code} field stores the 16-bit code
for the word that ends with the trie node containing the code. This means that
the code for some word ``$abc$'' would be contained in the trie node for `$c$'.
Note that there isn't a field that indicates what character a trie node
represents. This is because the trie node's index in its parent's array of child
nodes indicates what character it represents. The \texttt{trie\_step()} function
will be repeatedly called to check if a word exists in the trie. A word only
exists if the trie node returned by the last step corresponding to the last
character in the word isn't \texttt{NULL}.

\begin{funcdoc}{TrieNode *trie\_node\_create(Code c)}
  Constructor for a \texttt{TrieNode}. The node's code is set to \texttt{c}.
  Make sure each of the children node pointers are \texttt{NULL}.
\end{funcdoc}

\begin{funcdoc}{void trie\_node\_delete(TrieNode **n)}
  Destructor for a \texttt{TrieNode}. Note that a \emph{double} pointer is
  passed here. Unlike the destructors you may have written in the past, a double
  pointer is used here to set the passed pointer to \texttt{NULL}. Making sure
  to \texttt{NULL} out the pointer in the destructor reduces the chance of
  accidentally using a pointer that points to deallocated memory
  (use-after-free).
\end{funcdoc}

\begin{funcdoc}{TrieNode *trie\_create(void)}
  Initializes a trie: a root \texttt{TrieNode} with the code
  \texttt{EMPTY\_CODE}. Returns the root, a \texttt{TrieNode *}, if
  successful, \texttt{NULL} otherwise.
\end{funcdoc}

\begin{funcdoc}{void trie\_reset(TrieNode *root)}
  Resets a trie to just the root \texttt{TrieNode}. Since we are working with a
  finite number of codes, we eventually will use all of them. At that point, we
  must reset the trie by deleting its children so that we can continue
  compressing/decompressing the file. Make sure that each of the root's children
  nodes are \texttt{NULL}.
\end{funcdoc}

\begin{funcdoc}{void trie\_delete(TrieNode **n)}
  Deletes a sub-trie starting from the trie rooted at node \texttt{n}. This will
  require recursive calls on each of \texttt{n}'s children. Make sure each
  the pointer to each deleted node is set to \texttt{NULL}. This should already
  happen if the trie node destructor is written properly.
\end{funcdoc}

\begin{funcdoc}{TrieNode *trie\_step(TrieNode *n, Symbol s)}
  Returns a pointer to the child node reprsenting the symbol \texttt{s}. If the
  symbol doesn't exist, \texttt{NULL} is returned.
\end{funcdoc}
