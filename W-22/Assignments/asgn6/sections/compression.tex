\section{Compression}\label{sec:compression}

The following steps for compression will refer to the input file descriptor to
compress as \texttt{infile} and the compressed output file descriptor as
\texttt{outfile}.

\begin{enumerate}
    \item Open \texttt{infile} with \texttt{open()}. If an error occurs,
      print a helpful message and exit with a status code indicating
      that an error occurred. \texttt{infile} should be \texttt{stdin}
      if an input file wasn't specified.
    \item The first thing in \texttt{outfile} must be the file header,
      as defined in the file \texttt{io.h}. The magic number in the
      header must be \texttt{0x8badbeef}. The file size and the
      protection bit mask you will obtain using \texttt{fstat()}. See
      the \texttt{man} page on it for details.
    \item Open \texttt{outfile} using \texttt{open()}. The permissions
      for \texttt{outfile} should match the protection bits as set in
      your file header. Any errors with opening \texttt{outfile} should
      be handled like with \texttt{infile}. \texttt{outfile} should be
      \texttt{stdout} if an output file wasn't specified.
    \item Write the filled out file header to \texttt{outfile} using
        \texttt{write\_header()}. This means writing out the \texttt{struct}
        itself to the file, as described in the function specification.
    \item Create a trie. The trie initially has no children and consists
      solely of the root. The code stored by this root trie node should
      be \texttt{EMPTY\_CODE} to denote the empty word. You will need to
      make a copy of the root node and use the copy to step through the
      trie to check for existing prefixes. This root node copy will be
      referred to as \texttt{curr\_node}. The reason a copy is needed is
      that you will eventually need to reset whatever trie node you've
      stepped to back to the top of the trie, so using a copy lets you
      use the root node as a base to return to.
    \item You will need a monotonic counter to keep track of the next
      available code. This counter should start at \texttt{START\_CODE},
      as defined in the supplied \texttt{code.h} file. The counter
      should be a \texttt{Code} and will be referred as \texttt{next\_code}.
    \item You will also need two variables to keep track of the previous
      trie node and previously read symbol. We will refer to these as
      \texttt{prev\_node} and \texttt{prev\_sym}, respectively.
    \item Use \texttt{read\_sym()} in a loop to read in all the symbols
      from \texttt{infile}. Your loop should break when
      \texttt{read\_sym()} returns false. For each symbol read in, call
      it \texttt{curr\_sym}, perform the following:
      \begin{enumerate}
          \item Set \texttt{next\_node} to be \texttt{trie\_step(curr\_node,
              curr\_sym)}, stepping down from the current node to the
              currently read symbol.
          \item If \texttt{next\_node} is not \texttt{NULL}, that means we
              have seen the current prefix. Set \texttt{prev\_node} to be
              \texttt{curr\_node} and then \texttt{curr\_node} to be
              \texttt{next\_node}.
          \item Else, since \texttt{next\_node} is \texttt{NULL}, we know we
              have not encountered the current prefix. We write the pair
              (\texttt{curr\_node->code}, \texttt{curr\_sym}), where the
              bit-length of the written code is the bit-length of
              \texttt{next\_code}. The current prefix is then added to the trie.
              Let \texttt{curr\_node->children[curr\_sym]} be a new trie node
              whose code is \texttt{next\_code}. Reset \texttt{curr\_node} to
              point at the root of the trie and increment the value of
              \texttt{next\_code}.
          \item Check if \texttt{next\_code} is equal to \texttt{MAX\_CODE}.
              If it is, use \texttt{trie\_reset()} to reset the trie to just
              having the root node. This reset is necessary since we have a
              finite number of codes.
          \item Update \texttt{prev\_sym} to be \texttt{curr\_sym}.
      \end{enumerate}
    \item After processing all the characters in \texttt{infile}, check
      if \texttt{curr\_node} points to the root trie node. If it does
      not, it means we were still matching a prefix. Write the pair
      (\texttt{prev\_node->code}, \texttt{prev\_sym}). The bit-length of
      the code written should be the bit-length of \texttt{next\_code}.
      Make sure to increment \texttt{next\_code} and that it stays
      within the limit of \texttt{MAX\_CODE}. Hint: use the modulo
      operator.
    \item Write the pair (\texttt{STOP\_CODE}, 0) to signal the end of
      compressed output. Again, the bit-length of code written should be
      the bit-length of \texttt{next\_code}.
    \item Make sure to use \texttt{flush\_pairs()} to flush any unwritten,
      buffered pairs. Remember, calls to \texttt{write\_pair()} end up
      buffering them under the hood. So, we have to remember to flush
      the contents of our buffer.
    \item Use \texttt{close()} to close \texttt{infile} and \texttt{outfile}.
\end{enumerate}
