\section{Deliverables}
\epigraphwidth=0.66\textwidth
\epigraph{\textbf{Sheriff Bullard}: \emph{Don't ever do nothin' like this again. Don't come back up here.} \\
\textbf{Bobby}: \emph{You don't have to worry about that, Sheriff.}}
{---Sheriff Bullard, \emph{Deliverance}}

\noindent You will need to turn in the following source code and header
files:

\begin{enumerate}
  \item \texttt{decrypt.c}: This contains the implementation and
    \texttt{main()} function for the \texttt{decrypt} program.

  \item \texttt{encrypt.c}: This contains the implementation and
    \texttt{main()} function for the \texttt{encrypt} program.

  \item \texttt{keygen.c}: This contains the implementation and
    \texttt{main()} function for the \texttt{keygen} program.

  \item \texttt{numtheory.c}: This contains the implementations of the
    number theory functions.

  \item \texttt{numtheory.h}: This specifies the interface for the
    number theory functions.

  \item \texttt{randstate.c}: This contains the implementation of the
    random state interface for the RSA library and number theory
    functions.

  \item \texttt{randstate.h}: This specifies the interface for
    initializing and clearing the random state.

  \item \texttt{rsa.c}: This contains the implementation of the RSA
    library.

  \item \texttt{rsa.h}: This specifies the interface for the RSA
    library.
\end{enumerate}

Your code must pass \texttt{scan-build} \emph{cleanly}. If there are any
bugs or errors that are false positives, document them and explain why
they are false positives in your \texttt{README.md}. Improper
explanations will \emph{not} be considered. You will also need to turn
in the following:

\begin{enumerate}
  \item \texttt{Makefile}:
    \begin{itemize}
      \item \texttt{CC = clang} must be specified.

      \item \texttt{CFLAGS = -Wall -Wextra -Werror -Wpedantic} must be specified.

      \item \texttt{pkg-config} to locate compilation and include flags
        for the GMP library must be used.

      \item \texttt{make} must build the \texttt{encrypt},
        \texttt{decrypt}, and \texttt{keygen}
        executables, as should \texttt{make all}.

      \item \texttt{make decrypt} should build only the \texttt{decrypt}
        program.

      \item \texttt{make encrypt} should build only the \texttt{encrypt}
        program.

      \item \texttt{make keygen} should build only the \texttt{keygen}
        program.

      \item \texttt{make clean} must remove all files that are compiler
        generated.

      \item \texttt{make format} should format all your source code,
        including the header files.
    \end{itemize}

  \item \texttt{README.md}: This must use proper Markdown syntax and
    describe how to use your program and \texttt{Makefile}. It should
    also list and explain any command-line options that your program
    accepts. Any false positives reported by \texttt{scan-build} should
    be documented and explained here as well. Note down any known bugs
    or errors in this file as well for the graders.

  \item \texttt{DESIGN.pdf}: This document \emph{must} be a proper
    PDF\@. This design document must describe your design and design
    process for your program with enough detail such that a sufficiently
    knowledgeable programmer would be able to replicate your
    implementation. \textcolor{red}{This does not mean copying your
    entire program in verbatim}. You should instead describe how your
    program works with supporting pseudocode.
\end{enumerate}
