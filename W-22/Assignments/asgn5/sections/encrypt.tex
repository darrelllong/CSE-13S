\section{Encryptor}
\epigraph{\emph{The laws of mathematics are very commendable, but the only law that
applies in Australia is the law of Australia.}}{---Malcolm Turnbull, Australian Prime Minister}

Your encryptor program should accept the following command-line options:
\begin{itemize}
  \item \texttt{-i}\,: specifies the input file to encrypt (default:
    \texttt{stdin}).
  \item \texttt{-o}\,: specifies the output file to encrypt (default:
    \texttt{stdout}).
  \item \texttt{-n}\,: specifies the file containing the public key
    (default: \texttt{rsa.pub}).
  \item \texttt{-v}\,: enables verbose output.
  \item \texttt{-h}\,: displays program synopsis and usage.
\end{itemize}
The program should follow these steps:
\begin{enumerate}
  \item Parse command-line options using \texttt{getopt()} and handle
    them accordingly.
  \item Open the public key file using \texttt{fopen()}. Print a helpful
    error and exit the program in the event of failure.
  \item Read the public key from the opened public key file.
  \item If verbose output is enabled print the following, each with a
    trailing newline, in order:
    \begin{enumerate}
      \item username
      \item the signature \texttt{s}
      \item the public modulus \texttt{n}
      \item the public exponent \texttt{e}
    \end{enumerate}
    All of the \texttt{mpz\_t} values should be printed with information
    about the number of bits that constitute them, along with their
    respective values in \emph{decimal}. See the reference encryptor
    program for an example.
  \item Convert the username that was read in to an \texttt{mpz\_t}.
    This will be the expected value of the verified signature. Verify
    the signature using \texttt{rsa\_verify()}, reporting an error and
    exiting the program if the signature couldn't be verified.
  \item Encrypt the file using \texttt{rsa\_encrypt\_file()}.
  \item Close the public key file and clear any \texttt{mpz\_t}
    variables you have used.
\end{enumerate}
