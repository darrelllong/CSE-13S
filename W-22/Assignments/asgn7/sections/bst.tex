\section{Binary Search Trees}
\epigraph{\emph{The tree of liberty must be refreshed from time to time with the blood of patriots and tyrants.}}{---Thomas Jefferson}

\noindent
The definition of a binary search tree is a recursive one, as presented
in lecture. It is either \texttt{NULL}, or a node that points to up to
two subtrees. For any non-\texttt{NULL} node, the left subtree contains
keys that are less than it in value, and the right subtree contains keys
that are greater than it in value. Since our nodes will contain
oldspeak, each node's left subtree contains oldspeak that is
lexicographically less than it, and each node's right subtree contains
oldspeak that is lexicographically greater than it. You will want to
make use of the \texttt{strcmp()} function. The diagram below showcases
an example binary search tree that might appear for this assignment.

\begin{center}
  \begin{forest} for tree={draw,inner sep=5pt,l=10pt,l sep=6pt,s sep=14pt}
    [\texttt{window} $\rightarrow$ \texttt{okno}
      [\texttt{arm} $\rightarrow$ \texttt{rook}
        [\texttt{annoy} $\rightarrow$ \texttt{razdraz}]
        [\texttt{bad} $\rightarrow$ \texttt{baddiwad}]
      ]
      [\texttt{wipe} $\rightarrow$ \texttt{osoosh}
        [,phantom]
        [\texttt{zounds} $\rightarrow$ \texttt{NULL}]
      ]
    ]
  \end{forest}
\end{center}

The interface for the binary search tree ADT is provided in
\texttt{bst.h} and explained in the following sections.
\textcolor{red}{You may not modify this header file.}

\begin{funcdoc}{Node *bst\_create(void)}
  Constructor for a binary search tree that constructs an \emph{empty}
  tree. That means that the tree is \texttt{NULL}.
\end{funcdoc}

\begin{funcdoc}{void bst\_delete(Node **root)}
  Destructor for a binary search tree rooted at \texttt{root}. This
  should walk the tree using a \emph{postorder} traversal and delete
  each node of the tree.
\end{funcdoc}

\begin{funcdoc}{uint32\_t bst\_height(Node *root)}\label{bstheight}
  Returns the height of the binary search tree rooted at \texttt{root}.
\end{funcdoc}

\begin{funcdoc}{uint32\_t bst\_size(Node *root)}\label{bstsize}
  Returns the size of the binary search tree rooted at \texttt{root}.
  The size of a tree is equivalent to the number of nodes in the tree.
\end{funcdoc}

\begin{funcdoc}{Node *bst\_find(Node *root, char *oldspeak)}
  Searches for a node containing \texttt{oldspeak} in the binary search
  tree rooted at \texttt{root}. If a node is found, the pointer to the
  node is returned. Else, a \texttt{NULL} pointer is returned.
\end{funcdoc}

\begin{funcdoc}{Node *bst\_insert(Node *root, char *oldspeak, char *newspeak)}
  Inserts a new node containing the specified \texttt{oldspeak} and
  \texttt{newspeak} into the binary search tree rooted at \texttt{root}.
  Duplicates \emph{should not} be inserted.
\end{funcdoc}

\begin{funcdoc}{void bst\_print(Node *root)}
  Prints out each node in the binary search tree through an
  \emph{inorder} traversal. This will require the use of
  \texttt{node\_print()}.
\end{funcdoc}
