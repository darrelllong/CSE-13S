\section{Nodes}\label{sec:node}

\noindent As mentioned in \S\ref{sec:hashtable}, each array slot will contain a
node. Each node should contain a word and its count. The node \texttt{struct} is
defined for you in \texttt{node.h} as follows:

\begin{clisting}{}
struct Node {
    char *word;
    uint32_t count;
};
\end{clisting}

The rest of the interface for the node ADT is provided in \texttt{node.h}. The
node ADT is made transparent in order to simplify the program implementation.

\begin{funcdoc}{\texttt{Node *node\_create(char *word)}}
  The constructor for a node. You will want to make a \emph{copy} of the
  \texttt{word} that is passed in. This will require \emph{allocating memory}
  and copying over the characters for the \texttt{word}. You may find
  \texttt{strdup()} useful.
\end{funcdoc}

\begin{funcdoc}{\texttt{void node\_delete(Node **n)}}
  The destructor for a node. Since you have allocated memory for \texttt{word},
  remember to free the memory allocated to that as well. The pointer to the node
  should be set to \texttt{NULL}.
\end{funcdoc}

\begin{funcdoc}{\texttt{void node\_print(Node *n)}}
  A debug function to print out the contents of a node.
\end{funcdoc}
