\section{Specifics}

\epigraphwidth=0.4\textwidth
\epigraph{\emph{Jehosaphat the mongrel cat \\ Jumped off the roof today \\ Some
would say he fell but I could tell \\ He did himself away}}{---John Prine,
\emph{Living in the Future}}

Here are the specifics for your assignment implementation.

\begin{enumerate}
  \item Parse the command-line options by looping calls to \texttt{getopt()}.
    This should be similar to what you did in previous assignments.
  \item Use an initial call to \texttt{fscanf()} to read the number of rows and columns of
    the universe you will be populating from the specified input.
  \item Create \emph{two} universes using the dimensions that were obtained using \texttt{fscanf()}. Mark
    the universes toroidal if the \texttt{-t} option was specified. We will
    refer to these universes as universe $A$ and universe $B$.
  \item Populate universe $A$ using \texttt{uv\_populate()} with the remainder of the
    input.
  \item Setup the \texttt{ncurses} screen, as show by the example in \S 4.
  \item For each generation up to the set number of generations:
    \begin{enumerate}
      \item If \texttt{ncurses} isn't silenced by the \texttt{-s} option, clear
        the screen, display universe $A$, refresh the screen, then sleep for 50000
        microseconds.
      \item Perform one generation. This means taking a census of each cell in
        universe $A$ and either setting or clearing the corresponding cell in
        universe $B$, based off the 3 rules discussed in \S 2.
      \item Swap the universes. Think of universe $A$ as the current state of the
        universe and universe $B$ as the next state of the universe. To update the
        universe then, we simply have to swap $A$ and $B$. \textcolor{red}{Hint:
        swapping \emph{pointers} is much like swapping integers.}
    \end{enumerate}
  \item Close the screen with \texttt{endwin()}.
  \item Output universe $A$ to the specified file using \texttt{uv\_print()}.
    This is what you will be graded on. We will know if you properly
    evolved your universe for the set number of generations by comparing your output
    to that of the supplied program.
\end{enumerate}
