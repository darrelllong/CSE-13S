\section{Introduction}

\epigraphwidth=0.65\textwidth
\epigraph{\emph{Any inaccuracies in this index may be explained by the
fact that it has been sorted with the help of a computer.}}{---Donald
Knuth, Vol.  III, \emph{Sorting and Searching}}

\noindent Putting items into a sorted order is one of the most common
tasks in Computer Science. As a result, there are a myriad of library
routines that will do this task for you, but that does not absolve you
of the obligation of understanding how it is done. In fact, it behooves
you to understand the various algorithms in order to make wise choices.

The best execution time that can be accomplished, also referred to as
the \emph{lower bound}, for sorting using \emph{comparisons} is
$\Omega(n \log n)$, where $n$ is the number is elements to be sorted. If
the universe of elements to be sorted is small, then we can do better
using a \emph{Count Sort} or a \emph{Radix Sort} both of which have a
time complexity of $\operatorname{O}(n)$. The idea of \emph{Count Sort}
is to count the number of occurrences of each element in an array. For
\emph{Radix Sort}, a digit by digit sort is done by starting from the
least significant digit to the most significant digit.

What is this $\operatorname{O}$ and $\Omega$ stuff? It's how we talk
about the execution time (or space used) by a program. We will discuss
it in lecture and in section, and you will see it again in your Data
Structures and Algorithms class, now named CSE 101.

The sorting algorithms that you are expected to implement are Insertion Sort,
Batcher Sort, Heap Sort, and recursive Quicksort. The purpose of this assignment
is to get you fully familiarized with each sorting algorithm and for you to get
a \emph{feel} for computational complexity. \textcolor{red}{They are well-known
sorts. You can use the Python pseudocode provided to you as guides. Do not get
the code for the sorts from the Internet or you will be referred to for
cheating. We will be running plagiarism checkers.}
