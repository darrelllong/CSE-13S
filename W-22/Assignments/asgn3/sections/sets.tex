\section{Sets}

For this assignment, you are required to use a \emph{set} to track which
command-line options are specified when your program is run. The code
for sets is given in the resources repository in \texttt{set.h}.
\textcolor{red}{You may not modify this file.} Make sure to take the
time to go through and understand the code.

For manipulating the bits in a set, we use bit-wise operators. These
operators, as the name suggests, will perform an operation on every bit
in a number. The following are the six bit-wise operators specified in
\textbf{C}:

\begin{center}
  \begin{tabular}{|c|l|l|}
    \hline
    \verb|&| & bit-wise AND & Performs the AND operation on every bit
    of two numbers. \\
    \hline
    \verb||| & bit-wise OR & Performs the OR operation on every bit of
    two numbers. \\
    \hline
    \verb|~| & bit-wise NOT & Inverts all bits in the given number. \\
    \hline
    \verb|^| & bit-wise XOR & Performs the exclusive-OR operation on
    every bit of two numbers. \\
    \hline
    \verb|<<| & left shift & Shifts bits in a number to the left by a
    specified number of bits. \\
    \hline
    \verb|>>| & right shift & Shifts bits in a number to the right by a
    specified number of bits. \\
    \hline
  \end{tabular}
\end{center}

\noindent Recall that the basic set operations are: \emph{membership},
\emph{union}, \emph{intersection} and \emph{negation}. The functions
implementing these set operations are implemented for you. Using these
functions, you will set (make the bit 1) or clear (make the bit 0) bits
in the \texttt{Set} depending on the command-line options read by
\texttt{getopt()}. You can then check the states of all the bits (the
members) of the \texttt{Set} using a single \texttt{for} loop and
execute the corresponding sort. Note: you most likely won't use all the
functions, but you \emph{must} use sets to track which command-line
options are specified when running your program.

\begin{funcdoc}{\texttt{Set empty\_set(void)}}
  This function is used to return an empty set. In this context, an empty
  set would be a set in which all bits are equal to 0.
\end{funcdoc}

\begin{funcdoc}{\texttt{bool member\_set(uint32\_t x, Set s)}}

\begin{align*}
  x \in s \iff x\, \text{is a member of set}\, s
\end{align*}

  \noindent This function returns a \texttt{bool} indicating the
  presence of the given value \texttt{x} in the set \texttt{s}. The
  bit-wise AND operator is used to determine set membership. The first
  operand for the AND operation is the set \texttt{s}. The second
  operand is the value obtained by left shifting 1 \texttt{x} number of
  times. If the result of the AND operation is a non-zero value, then
  \texttt{x} is a member of \texttt{s} and \texttt{true} is returned to
  indicate this. \texttt{false} is returned if the result of the AND
  operation is 0.
\end{funcdoc}

\begin{funcdoc}{\texttt{Set insert\_set(uint32\_t x, Set s)}}
  This function inserts \texttt{x} into \texttt{s}. That is, it returns
  set \texttt{s} with the bit corresponding to \texttt{x} set to 1.
  Here, the bit is set using the bit-wise OR operator. The first operand
  for the OR operation is the set \texttt{s}. The second operand is
  value obtained by left shifting 1 by \texttt{x} number of bits.
\end{funcdoc}

\begin{funcdoc}{\texttt{Set delete\_set(uint32\_t x, Set s)}}

\begin{align*}
  s - x = \{y | y \in s \land y \ne x\}
\end{align*}

  This function deletes (removes) \texttt{x} from \texttt{s}. That is,
  it returns set \texttt{s} with the bit corresponding to \texttt{x}
  cleared to 0. Here, the bit is cleared using the bit-wise AND
  operator. The first operand for the AND operation is the set
  \texttt{s}. The second operand is a negation of the number 1 left
  shifted to the same position that \texttt{x} would occupy in the set.
  This means that the bits of the second operand are all 1s except for
  the bit at \texttt{x}'s position. The function returns set \texttt{s}
  after removing \texttt{x}.
\end{funcdoc}

\begin{funcdoc}{\texttt{Set union\_set(Set s, Set t)}}

\begin{align*}
  s \cup t = \{x | x \in s \lor x \in  t\}
\end{align*}

  \noindent The union of two sets is a collection of all elements in
  both sets. Here, to calculate the union of the two sets \texttt{s} and
  \texttt{t}, we need to use the OR operator. Only the bits
  corresponding to members that are equal to 1 in either \texttt{s} or
  \texttt{t} are in the new set returned by the function.
\end{funcdoc}

\begin{funcdoc}{\texttt{Set intersect\_set(Set s, Set t)}}

\begin{align*}
  s \cap t = \{x | x \in s \land x \in  t\}
\end{align*}

  \noindent The intersection of two sets is a collection of elements that
  are common to both sets. Here, to calculate the intersection of the
  two sets \texttt{s} and \texttt{t}, we need to use the AND operator.
  Only the bits corresponding to members that are equal to 1 in both
  \texttt{s} and \texttt{t} are in the new set returned by the function.
\end{funcdoc}

\begin{funcdoc}{\texttt{Set difference\_set(Set s, Set t)}}
  The difference of two sets refers to the elements of set \texttt{s}
  which are not in set \texttt{t}. In other words, it refers to the
  members of set \texttt{s} that are unique to set \texttt{s}. The
  difference is calculated using the AND operator where the two operands
  are set \texttt{s} and the negation of set \texttt{t}. The function
  then returns the set of elements in \texttt{s} that are not in
  \texttt{t}.

  This function can be used to find the complement of a given set as well,
  in which case the first operand would be the universal set $\mathbb{U}$
  and the second operand would be the set you want to complement as shown
  below.
\end{funcdoc}

\begin{align*}
  \overline{s} = \{ x | x \notin s\} = \mathbb{U} -s
\end{align*}

\begin{funcdoc}{\texttt{Set complement\_set(Set s)}}
  This function is used to return the complement of a given set. By
  complement we mean that all bits in the set are flipped using the NOT
  operator. Thus, the set that is returned contains all the elements of
  the universal set $\mathbb{U}$ that are not in \texttt{s} and contains
  none of the elements that are present in \texttt{s}.
\end{funcdoc}
