\section{Batcher's Odd-Even Merge Sort}

\epigraph{\emph{There are two ways of constructing a software design.
    One way is to make it so simple that there are obviously no
    deficiencies, and the other way is to make it so complicated that
    there are no obvious deficiencies. The first method is far more
difficult.}}{---C.A.R. Hoare}

Batcher's odd-even mergesort, or Batcher's method, is unlike the other presented
sorts in that it is actually a \emph{sorting network}. What is a sorting
network? Sorting networks, or \emph{comparator networks}, are circuits built for
the express purpose of sorting a set number of inputs. Sorting networks are
built with a fixed number of wires, one for each input to sort, and are
connected using \emph{comparators}. Comparators compare the values traveling
along the two wires they connect and swap the values if they're out of order.
Since there are a fixed number of comparators, a sorting network must
necessarily sort any input using a fixed number of comparisons. Refer to
Figure~\ref{fig:batcher} for a diagram of a sorting network using Batcher's
method.

Sorting networks are typically limited to inputs that are powers of 2. Batcher's
method is no exception to this. To remedy this, we apply Knuth's modification to
Batcher's method to allow it sort arbitrary-size inputs. This modification on
Batcher's method is also referred to as the \emph{Merge Exchange Sort}. You will
be implementing the merge exchange sort, or Batcher's method, using the
provided Python pseudocode.

\begin{pylisting}{Merge Exchange Sort (Batcher's Method) in Python}
def comparator(A: list, x: int, y: int):
    if A[x] > A[y]:
        A[x], A[y] = A[y], A[x]

def batcher_sort(A: list):
    if len(A) == 0:
        return

    n = len(A)
    t = n.bit_length()
    p = 1 << (t - 1)

    while p > 0:
        q = 1 << (t - 1)
        r = 0
        d = p

        while d > 0:
            for i in range(0, n - d):
                if (i & p) == r:
                    comparator(A, i, i + d)
            d = q - p
            q >>= 1
            r = p

        p >>= 1
\end{pylisting}

The pseudocode for Batcher's method can be a little mysterious, but it
effectively acts as a parallel \emph{Shell Sort}. Shell Sort acts a variation of
Insert Sort and first sorts pairs of elements which are far apart from each
other. The distance between these pairs of elements is called a \emph{gap}. Each
iteration of Shell Sort decreases the gap until a gap of $1$ is used.

Batcher's method is similar, but instead of sorting pairs of elements that are a
set gap apart, it \emph{$k$-sorts} the even and odd subsequences of the array,
where $k$ is some power of $2$. Given an array $A$ where $A_i$ denotes the
$i$-th index of $A$, an even subsequence refers to the sequence of values
$\{A_0, A_2, A_4, \dots \}$. Similarly, an odd subsequence refers to the sequence
of values $\{A_1, A_3, A_5, \dots \}$. The topic of $k$-sorting is beyond the
scope of this course, but all that you are required to understand is that if an
array is $k$-sorted, then all its elements are at most $k$ indices away from its
sorted position.

Consider an array of $16$ elements. Batcher's method first $8$-sorts the even
and odd subsequences, then $4$-sorts them, then $2$-sorts them, and finally
$1$-sorts them, \emph{merging} the sorted even and odd sequences. We will call
each level of $k$-sorting a \emph{round}. As stated previously, Batcher's method
works in \emph{parallel}. By virtue of the algorithm, any indices that appear in
a pairwise comparison when $k$-sorting a subsequence do not appear as indices in
another comparison during the same round. A clever use of the bitwise AND
operator, \texttt{\&}, guarantees this property.

When computing the bitwise AND operator of two integers $x$ and $y$, the
resulting integer is composed of $0$-bits except in the positions where both $x$
and $y$ have a $1$-bit. As an example, let $x = 10 = 1010_2$ and $y = 8 =
1000_2$. Bitwise AND-ing $x$ and $y$ yields $z = 8 = 1000_2$. In the provided
pseudocode, the variable \texttt{p} tracks the current round of $k$-sorting. It
is always a power of 2 since it starts off as a power of 2, and is only halved
in value using the right-shift operator (\verb|>>|). The condition on line 20 of
the pseudocode first computes the bitwise AND of \texttt{i} and \texttt{p}. This
effectively partitions values of \texttt{i} into partitions of size \texttt{p}.
The variable \texttt{r} effectively represents which partitions can be
considered for comparison. Thus, \texttt{(i \& p) == r} checks if the partition
that \texttt{i} falls into is eligible for comparison, and if it is, to execute
the comparison.

Indices are only ever compared with indices that are \texttt{d} indices away.
Since \texttt{p} is a power of $2$ and \texttt{d} is an odd multiple of
\texttt{p}, it follows that \texttt{i \& p != (i + d) \& p} for any \texttt{i}.
This means there is no overlap with any of the pairs of indices, which therefore
means that these comparisons can be run simultaneously, or in \emph{parallel},
with no ill effect. These parallel comparisons are shown clearly in
Figure~\ref{fig:batcher}.

Although Batcher's method can be run in parallel, your implementation of the
sort will run \emph{sequentially}, sorting the input over several rounds. For an
array size of $n$, the initial value to $k$-sort with is $k = \lceil \log_2(n)
\rceil$. The even and odd subsequences are first $k$-sorted, then
$\frac{k}{2}$-sorted, then $\frac{k}{4}$-sorted, and so on until they are
$1$-sorted. You will want to edit the provided pseudocode to print out the pairs
of indices that are being compared to see what is happening.

\begin{figure}[htb]
  \centering
  \begin{tikzpicture}[>=latex',scale=1.25]
    % Wire 0 line.
    \node at (0,-1.0) (wire0) {};
    \draw[-] (wire0.east) -- +(4.6,0);
    \node at (0.5,-1.0) (c01) [circle,fill,inner sep=1.5pt]{};
    \node at (1.0,-1.0) (c02) [circle,fill,inner sep=1.5pt]{};
    \node at (2.25,-1.0) (c03) [circle,fill,inner sep=1.5pt]{};

    % Wire 1 line.
    \node at (0,-1.5) (wire1) {};
    \draw[-] (wire1.east) -- +(4.6,0);
    \node at (0.5,-1.5) (c11) [circle,fill,inner sep=1.5pt]{};
    \node at (1.25,-1.5) (c12) [circle,fill,inner sep=1.5pt]{};
    \node at (1.75,-1.5) (c13) [circle,fill,inner sep=1.5pt]{};
    \node at (2.5,-1.5) (c14) [circle,fill,inner sep=1.5pt]{};
    \node at (4.25,-1.5) (c15) [circle,fill,inner sep=1.5pt]{};

    % Wire 2 line.
    \node at (0,-2.0) (wire2) {};
    \draw[-] (wire2.east) -- +(4.6,0);
    \node at (0.5,-2.0) (c21) [circle,fill,inner sep=1.5pt]{};
    \node at (1.0,-2.0) (c22) [circle,fill,inner sep=1.5pt]{};
    \node at (1.75,-2.0) (c23) [circle,fill,inner sep=1.5pt]{};
    \node at (2.75,-2.0) (c24) [circle,fill,inner sep=1.5pt]{};
    \node at (3.5,-2.0) (c25) [circle,fill,inner sep=1.5pt]{};
    \node at (4.25,-2.0) (c26) [circle,fill,inner sep=1.5pt]{};

    % Wire 3 line.
    \node at (0,-2.5) (wire3) {};
    \draw[-] (wire3.east) -- +(4.6,0);
    \node at (0.5,-2.5) (c31) [circle,fill,inner sep=1.5pt]{};
    \node at (1.25,-2.5) (c32) [circle,fill,inner sep=1.5pt]{};
    \node at (3.0,-2.5) (c33) [circle,fill,inner sep=1.5pt]{};
    \node at (3.75,-2.5) (c34) [circle,fill,inner sep=1.5pt]{};
    \node at (4.25,-2.5) (c35) [circle,fill,inner sep=1.5pt]{};

    % Wire 4 line.
    \node at (0,-3.0) (wire4) {};
    \draw[-] (wire4.east) -- +(4.6,0);
    \node at (0.5,-3.0) (c41) [circle,fill,inner sep=1.5pt]{};
    \node at (1.0,-3.0) (c42) [circle,fill,inner sep=1.5pt]{};
    \node at (2.25,-3.0) (c43) [circle,fill,inner sep=1.5pt]{};
    \node at (3.5,-3.0) (c44) [circle,fill,inner sep=1.5pt]{};
    \node at (4.25,-3.0) (c45) [circle,fill,inner sep=1.5pt]{};

    % Wire 5 line.
    \node at (0,-3.5) (wire5) {};
    \draw[-] (wire5.east) -- +(4.6,0);
    \node at (0.5,-3.5) (c51) [circle,fill,inner sep=1.5pt]{};
    \node at (1.25,-3.5) (c52) [circle,fill,inner sep=1.5pt]{};
    \node at (1.75,-3.5) (c53) [circle,fill,inner sep=1.5pt]{};
    \node at (2.5,-3.5) (c54) [circle,fill,inner sep=1.5pt]{};
    \node at (3.75,-3.5) (c55) [circle,fill,inner sep=1.5pt]{};
    \node at (4.25,-3.5) (c56) [circle,fill,inner sep=1.5pt]{};

    % Wire 6 line.
    \node at (0,-4.0) (wire6) {};
    \draw[-] (wire6.east) -- +(4.6,0);
    \node at (0.5,-4.0) (c61) [circle,fill,inner sep=1.5pt]{};
    \node at (1.0,-4.0) (c62) [circle,fill,inner sep=1.5pt]{};
    \node at (1.75,-4.0) (c63) [circle,fill,inner sep=1.5pt]{};
    \node at (2.75,-4.0) (c64) [circle,fill,inner sep=1.5pt]{};
    \node at (4.25,-4.0) (c65) [circle,fill,inner sep=1.5pt]{};

    % Wire 7 line.
    \node at (0,-4.5) (wire7) {};
    \draw[-] (wire7.east) -- +(4.6,0);
    \node at (0.5,-4.5) (c71) [circle,fill,inner sep=1.5pt]{};
    \node at (1.25,-4.5) (c72) [circle,fill,inner sep=1.5pt]{};
    \node at (3.0,-4.5) (c73) [circle,fill,inner sep=1.5pt]{};

    % Comparators.
    \draw[-] (c01) -- (c11);
    \draw[-] (c21) -- (c31);
    \draw[-] (c41) -- (c51);
    \draw[-] (c61) -- (c71);
    \draw[-] (c02) -- (c22);
    \draw[-] (c42) -- (c62);
    \draw[-] (c12) -- (c32);
    \draw[-] (c52) -- (c72);
    \draw[-] (c13) -- (c23);
    \draw[-] (c53) -- (c63);
    \draw[-] (c03) -- (c43);
    \draw[-] (c14) -- (c54);
    \draw[-] (c24) -- (c64);
    \draw[-] (c33) -- (c73);
    \draw[-] (c25) -- (c44);
    \draw[-] (c34) -- (c55);
    \draw[-] (c15) -- (c26);
    \draw[-] (c35) -- (c45);
    \draw[-] (c56) -- (c65);
  \end{tikzpicture}
  \caption{
    Batcher's odd-even mergesort sorting network with eight inputs. Inputs
    traveling along the wires are sorted as they move from left to right.
}
  \label{fig:batcher}
\end{figure}
