\section{Useful Commands}

This section will cover some commands that you'll find useful for this
assignment and for the course in general. This is not meant to be an exhaustive
list. We will cover all the commands that were used to create Figures
\ref{figure:collatz-length}, \ref{figure:collatz-maxval}, and
\ref{figure:collatz-hist}, but you may end up using commands that aren't listed.

\begin{itemize}
  \item \texttt{head} --- Prints out the first 10 lines of a file by default.
    Can be instructed to print out a specific number of lines.
  \item \texttt{ls} --- Lists the contents of the current working directory.
  \item \texttt{sort} --- Prints the sorted contents of a file. The contents are
    sorted lexicographically by default, but can be specified to be sorted
    numerically if desired.
  \item \texttt{uniq} --- Filters out repeated lines in a file. Can be used to
    both filter out repeated lines and count the number of times a line was
    repeated.
  \item \texttt{wc} --- Prints out the word, line, and character count of a
    file.
\end{itemize}

We have provided a brief description for each command, but should you find
yourself wanting more information, refer to the command's \texttt{man} page. You
can do so using \texttt{man <command>}. A \texttt{man} page (short for manual
page) is software documentation for tools and programs found on \Unix{} systems.
To view a \texttt{man} page:

\begin{shlisting}{}
$ man <function, program, tool>
\end{shlisting}

These manual pages are typically divided into sections, depending on
their respective purposes. General commands are found in section 1,
system calls in section 2, and library functions, such as the
\texttt{printf()} function used in this assignment, are found in section
3. So, to view the \texttt{man} page for \texttt{printf()}:

\begin{shlisting}{}
$ man 3 printf
\end{shlisting}

You will want to get into the habit of checking \texttt{man} pages. They are
written by programmers for programmers and will be invaluable when you get to
writing \C{} code.
