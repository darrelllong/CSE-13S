\section{Submission}

\epigraph{\emph{The cost of freedom is always high, but Americans have always
paid it. And one path we shall never choose, and that is the path of
surrender, or submission.}}{---John F.\ Kennedy}\noindent

\noindent Now that you have learned about some useful \texttt{git} commands,
it's time to put them to use. The steps to submitting assignments will not
change throughout the course. If you ever forget the steps, refer back to this
PDF. Remember: \emph{add, commit,} and \emph{push}! In the case you do mess
something up, \emph{don't panic.} Take a step back and think things throughly.
The Internet, TAs and tutors are here as resources.

\begin{enumerate}
  \item Add it!

\begin{shlisting}{}
$ git add CHEATING.pdf hello.c
\end{shlisting}

    As mentioned before, you will need to first add the files to your
    repository using the \texttt{git add <filenames>} command. You will
    be submitting these files into the \texttt{asgn0} directory.

  \item Commit it!

\begin{shlisting}{}
$ git commit -m "Your commit message here"
\end{shlisting}

  Changes to these files will be committed to the repository with \texttt{git
  commit}. The command should also include a commit message describing what
  changes are included in the commit.

  \item Push it!

\begin{shlisting}{}
$ git push
\end{shlisting}

  The committed changes are then sync'd up with the remote server
  using the \texttt{git push} command. You must be sure to push your
  changes to the remote server or else they will not be received by
  the graders.

\item Submit the commit ID on Canvas! You can find the most recent
  commit ID by using \texttt{git log} or searching for it through the
  GitLab web interface. \textcolor{red}{Your assignment is turned in
    \emph{only} after you have pushed and submitted the commit ID you
    want graded on Canvas. ``I forgot to push'' and ``I forgot to submit
    my commit ID'' are not valid excuses. It is \emph{highly}
  recommended to commit and push your changes \emph{often}.}
\end{enumerate}
