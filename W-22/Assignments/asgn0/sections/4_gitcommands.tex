\section{Useful Git Commands}

The following commands are used through \texttt{git} for version control. For
this assignment, you will have used the \texttt{clone}, \texttt{add}, and
\texttt{push} commands. This section will serve as a brief description and use
of frequently used \texttt{git} commands that you will most likely use
throughout the quarter, if not your entire career as a computing professional.

\subsection{\texttt{git config}}

This command lets you set configuration variables that tune \texttt{git}
to operate the way you want it to. The main things you will likely want
to do when you get started with \texttt{git} are establishing your
identity, as well as the default text editor when typing up longer
commits.

Perform the following command to set your name and email address. Notice
the \texttt{---global} in the command. That is a \emph{command-line
option} and indicates that the following configuration should be used
globally in every \texttt{git} repository you have. Unless otherwise
specified, configurations by default are applied only to the local
repository.

\begin{shlisting}{}
$ git config --global user.name "<your name>"
$ git config --global user.email "<your email>"
\end{shlisting}

To set your default editor as \texttt{vim}:
\begin{shlisting}{}
$ git config --global core.editor vim
\end{shlisting}

To check all the configurations simply run:
\begin{shlisting}{}
$ git config --list
\end{shlisting}

To check the value of a specific key, or setting, just supply it as the
sole argument after \texttt{git config}. For instance, to check the
configured email:
\begin{shlisting}{}
$ git config user.email
\end{shlisting}

\subsection{\texttt{git help}}

When starting out with \texttt{git}, you may find yourself frequently
needing to refresh your memory on certain commands. The command
\texttt{git help} will prove invaluable in this regard. There are three
ways to display the \texttt{man} page for any \texttt{git} command:
\begin{shlisting}{}
$ git help <command>
$ git <command> --help
$ man git-<command>
\end{shlisting}

For example, to view the \texttt{man} page for \texttt{git clone}, the
subject of the next section, any of the following can be run:

\begin{shlisting}{}
$ git help clone
$ git clone --help
$ man git-clone
\end{shlisting}

A \texttt{man} page (short for manual page) is software documentation
for tools and programs found on \textsc{Unix} systems. To view a
\texttt{man} page:

\begin{shlisting}{}
$ man <function, program, tool>
\end{shlisting}

These manual pages are typically divided into sections, depending on
their respective purposes. General commands are found in section 1,
system calls in section 2, and library functions, such as the
\texttt{printf()} function used in this assignment, are found in section
3. So, to view the \texttt{man} page for \texttt{printf()}:

\begin{shlisting}{}
$ man 3 printf
\end{shlisting}

\subsection{\texttt{git clone}}

This command clones a repository from a server onto your local machine. This
downloads a copy of the repository which is stored on a server for local
editing. Meaning, any changes that need to be sent back to the server will need
to be \emph{added}, \emph{committed} and \emph{pushed}. Here is an example of
cloning over \texttt{ssh}:

\begin{shlisting}{}
$ git clone user@somemachine:path/to/repo
\end{shlisting}

\subsection{\texttt{git add}}

This command allows you to add files into your repository and stages them to
the \texttt{git} source tree. Any file that has been changed since the time it was last
added needs to be added again.

\begin{shlisting}{}
$ git add file1 file2
\end{shlisting}

Keep in mind, adding files with this command does \emph{not} commit them. You
still need to commit the changes with the \texttt{git commit} command.

\subsection{\texttt{git commit}}

This command creates a checkpoint for each file which was added using the
previous command, \texttt{git add}. You can think of it like capturing a
snapshot of the current staged changes. These snapshots are then safely
committed. Each commit has an unique commit ID along with a message about the
commit.

\begin{shlisting}{}
$ git commit -m "A short informative message about any changes"
\end{shlisting}

To commit all the changed files, you can use the command \texttt{git commit -a}
which can also be combined with the \texttt{-m} option. This will only commit
files that have been added and committed at least once before. Without the
\texttt{-m} flag, you will be taken into the default \texttt{git} editor to enter your commit
message. A forewarning: don't commit rude comments --- the TAs and
graders will see them.

You should commit working versions of your code frequently so in the case you
mess something up, like accidentally deleting your code, you can use \texttt{git
checkout HEAD} to revert to the most recent commit.

\subsection{\texttt{git checkout}}

This command allows you to set the state of your repository to the state
of your repository at the time of a different commit. The reverting of
state can be performed per file, meaning that you can use \texttt{git
checkout} to restore a specific file to its state in a different commit.
To checkout a commit:

\begin{shlisting}{}
$ git checkout <commit>
\end{shlisting}

To checkout restore a file to its state at a different commit:

\begin{shlisting}{}
$ git checkout <commit> -- <file>
\end{shlisting}

This last command also works to retrieve files that you may have
accidentally deleted locally. This alone should provide good incentive
to add, commit, and push changes to your files often.

\subsection{\texttt{git log}}

This command provides a list of the commits that have been made on the
repository. It provides access to look up commit times, messages, and IDs.

\begin{shlisting}{}
$ git log
\end{shlisting}


\subsection{\texttt{git push}}

This command pushes all of your local commits to the upstream repository. It
pushes all of your changes to the directory which is stored on-line. You
\emph{must} do this to turn in your work for this class. If you do not run this
command after committing, \emph{none} of your work will be turned in.

\begin{shlisting}{}
$ git push
\end{shlisting}

\subsection{\texttt{git pull}}

This command fetches and downloads content from a remote repository. Your local
repository is immediately updated to match the fetched content. \texttt{git
pull} is actually a combination of \texttt{git fetch} followed by \texttt{git
merge}. The first half of \texttt{git pull} will execute \texttt{git fetch} on
the local branch that HEAD is pointed at. After the contents are fetched, the
second half of \texttt{git pull} will merge the work-flow creating a new merge
commit ID and HEAD is updated to point to the new commit.

\begin{shlisting}{}
$ git pull
\end{shlisting}

\subsection{\texttt{git ls-files}}

This command lists all files in the current directory that have been checked
into the repository. This will be useful for making sure you have submitted all
required deliverables for each assignment.

\begin{shlisting}{}
$ git ls-files
\end{shlisting}

\subsection{\texttt{git status}}

This command provides a status of which files have been added and staged for the
next commit, as well as unpushed changes.

\begin{shlisting}{}
$ git status
\end{shlisting}
