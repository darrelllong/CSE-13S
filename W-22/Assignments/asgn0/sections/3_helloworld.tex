\section{Hello World!}

You will be creating a simple \textbf{C} program which will simply print
``\texttt{Hello World!}'' \textcolor{red}{You can find also find a
tutorial of this program in Chapter 1 \S 1.1} in your textbook,
\textit{The C Programming Language} by Kernighan \& Ritchie.

\begin{enumerate}
  \item Make sure you are in the correct directory: \texttt{asgn0}. You can
    check your \emph{current working directory} using this command:

\begin{shlisting}{}
$ pwd
\end{shlisting}

  \item Create the program source \texttt{hello.c} with your text editor of
    choice. This means text editors such as \texttt{vi} and \texttt{emacs}.
    Notepad and Word are \emph{not} text editors. To open up \texttt{hello.c}
    for editing with \texttt{vi}:

\begin{shlisting}{}
$ vi hello.c
\end{shlisting}

    It should be noted that the \texttt{vim} text editor has largely
    succeeded \texttt{vi}. In fact, many systems simply have
    \texttt{vi} aliased to \texttt{vim}. You can check if your version
    of \texttt{vi} is an alias by running:

\begin{shlisting}{}
$ vi --version
\end{shlisting}

    Using either \texttt{vi} or \texttt{vim} is perfectly acceptable for
    your assignments, but you may find \texttt{vim} easier to get
    accustomed to with all the quality-of-life improvements that has
    been over the years.

  \item Include the header for the \texttt{<stdio.h>} library. This is needed by
    the \texttt{printf()} function that prints formatted strings to
    \texttt{stdout}, what you think of as the console.

\begin{clisting}{\texttt{hello.c}}
#include <stdio.h>
\end{clisting}

  \item Type your \texttt{main()} function. Every \textbf{C} program \emph{must}
    have a \texttt{main()} function which returns an \texttt{int}. A
    return value of 0 indicates program success, and a non-zero return
    indicates the occurrence of some error.

\begin{clisting}{\texttt{hello.c}}
#include <stdio.h>

int main(void) {
    return 0;
}
\end{clisting}

  \item In \texttt{main()} (between the curly braces) is where you will type the
    print statement. It is \emph{crucial} that your print statement matches the
    one given here. \textcolor{red}{You will be docked points otherwise.}

\begin{clisting}{\texttt{hello.c}}
#include <stdio.h>

int main(void) {
    printf("Hello World!\n");
    return 0;
}
\end{clisting}

  \item Save your work and exit your text editor to return to the
    command line. With \texttt{vi} this means entering normal mode by
    hitting \texttt{esc} and entering the command (indicated with a
    prefixed colon) ``\texttt{:wq}'' to save and quit. \texttt{ZZ} works
    as well. You are allowed to use \texttt{vi}, \texttt{vim},
    \texttt{nvim}, \texttt{emacs}, \texttt{atom}, \texttt{Sublime} ---
    but \emph{not} a Windows editor. \texttt{red}{Editing the file using
    Windows --- or Notepad, or even Word --- is grounds for receiving a
    zero.}

  \item You should now be back on the command line. You should now compile and
    run your code to verify its correctness. To compile your code, run:

\begin{shlisting}{}
$ clang -Wall -Wextra -Werror -Wpedantic -o hello hello.c
\end{shlisting}

    This will compile your code with the compiler flags required by the class.
    \texttt{clang} is the \textbf{C} compiler that we will be using --- not
    \texttt{gcc}, not \texttt{cc}. You \emph{must} use \texttt{clang}. The
    \texttt{-Wall -Wextra -Werror -Wpedantic} arguments are the set of compiler
    flags you must use when compiling your code. This specific set of compiler
    flags is commonly referred to as the ``take no prisoners'' compiler flags.
    Simply put, together they catch pretty much everything that a compiler can
    catch (there are a few more esoteric warnings that can be enabled). Here are
    some links for you to investigate what each flag does:

    \centerline{\url{https://releases.llvm.org/10.0.0/tools/clang/docs/UsersManual.html}}
    \centerline{\url{https://releases.llvm.org/10.0.0/tools/clang/docs/DiagnosticsReference.html}}

    The provided links are for version 10 of \texttt{clang}, but it is
    fine if you have version 13.

  \item If you've done everything correctly up to this point, the compilation
    process should run silently and return no errors. However, if you do run
    into any errors, lab sections, and Piazza will be your best friends. Resist
    the urge to immediately use Google.

  \item After successfully compiling your program, there should now be an
    executable file named \texttt{hello} in the current working directory. To
    list out all the files in the current working directory use \texttt{ls}:

\begin{shlisting}{}
$ ls
\end{shlisting}

  \item To display the current working directory, or what you think of
    as the directory you're currently in, use \texttt{pwd}:

\begin{shlisting}{}
$ pwd
\end{shlisting}

  \item To run the \texttt{hello} program, enter:

\begin{shlisting}{}
$ ./hello
\end{shlisting}

  \item The \texttt{.} (usually called ``dot'') refers to the
    \emph{current working directory.} Your shell has a \texttt{PATH}
    environment variable, a colon-delimited list of directories that it
    looks through when you enter a command. Since your current working
    directory should never be in your \texttt{PATH}, you must specify
    the directory that your program can be found in order to run it. If
    the output of running your program is correct, you should then
    submit your working \emph{source code} to \texttt{git}. You should
    submit source code \emph{only}: no executables. What is a
    \texttt{PATH}? It is an environment variable whose value is a
    colon-delimited list of directory names. Each directory in this list
    indicates a location where executable programs can be found.
    Directories that are usually included in the \texttt{PATH} include
    \texttt{/usr} and \texttt{/usr/bin}. The current working directory
    should \emph{never} be added to the \texttt{PATH} since doing so
    would be a serious vulnerability. Imagine some adversary managed to
    sneak in a compromised \texttt{ls} binary into your current working
    directory. If your current working directory was first in the
    \texttt{PATH}, then the compromised \texttt{ls} would be found and
    run first, most likely leaving your machine in an undesirable state.

\begin{shlisting}{}
$ git add hello.c
$ git commit -m "Adding finished hello.c"
$ git push
\end{shlisting}

    The above three commands will add, commit, and push \texttt{hello.c} to
    \texttt{git}. In-depth description of each of these commands will be
    provided in the following section. To verify that \texttt{hello.c} was
    added, check your repository:

    \centerline{\url{https://git.ucsc.edu/gitlab/cse13s/winter2022/<CruzID>/asgn0}}

    Only in this case do you perform one commit at the end. In general, you
    should commit after every significant change.
    \textcolor{red}{Warning: you should \emph{never} push binary files.
    This includes executable programs and object files, as well as any
    files generated during the compilation process of a program.}

  \item The only other file to be submitted for assignment 0 is your signed
    \texttt{CHEATING.pdf}. This file can be found on Canvas and/or under Piazza
    resources. \textcolor{red}{Note: You do not create a
    PDF file by simply appending \texttt{.pdf} to its name.} You will submit
    \texttt{CHEATING.pdf} the same way you did \texttt{hello.c}: adding,
    committing, then pushing.
\end{enumerate}
