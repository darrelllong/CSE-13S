\section{Your Task}\label{section:task}

Your first task for this assignment is to implement a small library of
mathematical functions, mimicking those found in \texttt{<math.h>}. The
interface for your math library is given in \texttt{mathlib.h}. The
implementation of the library should go in \texttt{mathlib.c}.
\textcolor{red}{You are \emph{strictly forbidden}} to use any functions from
\texttt{<math.h>} in your own math library. You are also forbidden to write a
factorial function. Each of the functions you will write must halt computation
using an $\epsilon = 10^{-14}$, which was discussed in \S\ref{section:exp}.

The functions you are expected to implement are as follows:

\begin{funcdoc}{double Exp(double x)}
  Returns the approximated value of $e^x$. Refer to \S\ref{section:exp}
  for specifics.
\end{funcdoc}

\begin{funcdoc}{double Sin(double x)}
  Returns the approximated value of $\sin(x)$. Refer to
  \S\ref{section:sincos} for specifics.
\end{funcdoc}

\begin{funcdoc}{double Cos(double x)}
  Returns the approximated value of $\cos(x)$. Refer to
  \S\ref{section:sincos} for specifics.
\end{funcdoc}

\begin{funcdoc}{double Sqrt(double x)}
  Returns the approximated value of $\sqrt{x}$. Refer to
  \S\ref{section:sqrtlog} for specifics.
\end{funcdoc}

\begin{funcdoc}{double Log(double x)}
  Returns the approximated value of $\log(x)$. Refer to
  \S\ref{section:sqrtlog} for specifics.
\end{funcdoc}

\vspace{0.5em}
Your second task for this assignment is to write a dedicated program,
\texttt{integrate}, that links with your implemented math library and computes
numerical integrations of various functions using the \emph{composite Simpson's
1/3 rule}.

\begin{funcdoc}{double integrate(double (*f)(double), double a, double
  b, uint32\_t n)}
  Computes the numerical integration of some function \texttt{f} over the
  interval [\texttt{a}, \texttt{b}]. This should be done with composite
  Simpson's 1/3 rule using \texttt{n} partitions. Note the special syntax for
  the parameter \texttt{f}. This syntax in \textbf{C} denotes that \texttt{f} is
  a pointer to a function -- a \emph{function pointer} -- that takes a single
  \texttt{double} as its sole argument and returns a \texttt{double}. Using a
  function pointer allows us to use this \texttt{integrate()} function with
  \emph{any} function that matches the function signature specified by
  \texttt{f}. Refer to \S\ref{section:compositesimpson} for the formula for the
  composite Simpson's 1/3 rule.
\end{funcdoc}

\subsection{Testing}
You will want to test each of your functions to make sure that they
are producing correct values. In this case, you \emph{can} and
\emph{should} compare the values that you compute with the ones
from \texttt{<math.h>}.

You are strongly encouraged to do this by writing small test programs for each
function. For some mysterious reason, students are reluctant to write such short
programs and prefer to add \texttt{printf()} to their code, or sit in
deep befuddlement when their final numbers are incorrect.

It is important to \emph{independently verify} each important
component. In the case of this program, those components are the
replacements for functions from \texttt{<math.h>}.
We include an example below.

\begin{clisting}{Comparing our sine against the C library}
#include <math.h>
#include <stdio.h>

#define EPSILON 1e-14

static inline double Abs(double x) { return x < 0 ? -x : x; }

double Sin(double x) {
    double sgn = 1, val = x, trm = x;
    for (int k = 3; abs(trm) > epsilon; k += 2) {
        trm = trm * (x * x) / ((k - 1) * k);
        sgn = -sgn;
        val += sgn * trm;
    }
    return val;
}

int main(void) {
    for (double x = -2.0 * M_PI; x < 2 * M_PI; x += 0.1) {
        printf("sin(%3.2lf) = %+lf (%+20.19lf)\n", x, Sin(x), sin(x) - Sin(x));
    }
    return 0;
}
\end{clisting}
