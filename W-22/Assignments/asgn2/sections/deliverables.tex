\section{Deliverables}

You will need to turn in the following source code and header files:

\begin{enumerate}
  \item \texttt{functions.c}: This file is provided and contains the
    implementation of the functions that your main program should
    integrate.
  \item \texttt{functions.h}: This file is provided and contains the
    function prototypes of the functions that your main program should
    integrate.
  \item \texttt{integrate.c}: This contains the \texttt{integrate()} and  the
\texttt{main()} function to perform the integration specified by the command-line over the specified
    interval.
  \item \texttt{mathlib.c}: This contains the implementation of each of
    your math library functions.
  \item \texttt{mathlib.h}: This file is provided and contains the
    interface for your math library.
\end{enumerate}

You may have other source and header files, but \emph{do not make things
over complicated}. \textcolor{red}{Any additional source code and header
files that you may use must not use global variables.}
You will also need to turn in the following:

\begin{enumerate}
  \item \texttt{Makefile}:
    \begin{itemize}
      \item \texttt{CC = clang} must be specified.
      \item \texttt{CFLAGS = -Wall -Wextra -Werror -Wpedantic} must be specified.
      \item \texttt{make} must build the \texttt{integrate}
        executable, as should \texttt{make all} and \texttt{make
        integrate}.
      \item \texttt{make clean} must remove all files that are compiler
        generated.
      \item \texttt{make format} should format all your source code,
        including the header files.
    \end{itemize}
  \item \texttt{README.md}: This must use proper Markdown syntax. It
    must describe how to use your program and \texttt{Makefile}. It
    should also list and explain any command-line options that your
    program accepts. Any false positives reported by \texttt{scan-build}
    should be documented and explained here as well. Note down any known
    bugs or errors in this file as well for the graders.
  \item \texttt{DESIGN.pdf}: This document \emph{must} be a proper
    PDF\@. This design document must describe your design and design
    process for your program with enough detail such that a sufficiently
    knowledgeable programmer would be able to replicate your
    implementation. \textcolor{red}{This does not mean copying your
    entire program in verbatim}. You should instead describe how your
    program works with supporting pseudocode.
  \item \texttt{WRITEUP.pdf}: This document \emph{must} be a proper
    PDF\@. This writeup must include, at least, the following:
      \begin{itemize}
        \item Graphs displaying the integrated values as you vary the
          number of partitions. You should be using \texttt{gnuplot} to
          produce these graphs. Attend section for examples of using
          \texttt{gnuplot} and other \textsc{Unix} tools. An example
          script for using \texttt{gnuplot} to help plot your graphs
          will be supplied in the resources repository.
        \item Analysis of the produced graphs and any lessons that you learned
          about floating-point numbers.
      \end{itemize}
\end{enumerate}
