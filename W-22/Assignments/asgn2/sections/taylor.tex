\section{Taylor Series}

\epigraphwidth=0.75\textwidth
\epigraph{\emph{Let us change our traditional attitude to the construction of
programs. Instead of imagining that our main task is to instruct a computer what
to do, let us concentrate rather on explaining to human beings what we want a
computer to do.}} {---Donald Knuth}

\noindent As we know, computers are simple machines that carry out a sequence of
very simple steps, albeit very quickly. Unless you have a special-purpose
processor, a computer can only compute \emph{addition}, \emph{subtraction},
\emph{multiplication}, and \emph{division}. If you think about it, you will see
that the functions that might interest you when dealing with real or complex
numbers can be built up from those four operations.  We use many of these
functions in nearly every program that we write, so we ought to understand how
they are created.

If you recall from your calculus class, with some conditions a function $f(x)$
can be represented by its Taylor series (Brook Taylor, 18 August 1685--29 December 1731)
expansion near some point $a$:
\[
  f(x) = f(a) + \sum_{k=1}^\infty \frac{f^{(k)}(a)}{k!}{(x-a)}^k.
\]
\textcolor{red}{Note: when you see $\Sigma$ with definite limits, you should generally think of a
\texttt{for} loop.}

Since we cannot compute an infinite series, we must be content to calculate a
finite number of terms. In general, the more terms that we compute, the more
accurate our approximation.

For example, if we expand to $10$ terms we get:
\begin{align*}
  f(x) = f(a) &+ \taylorterm{1} + \taylorterm{2} + \taylorterm{3} + \taylorterm{4} \\
              &+ \taylorterm{5} + \taylorterm{6} + \taylorterm{7} + \taylorterm{8} \\
              &+ \taylorterm{9} + \operatorname{O}({(x-a)}^{10}). \\
\end{align*}
In the case $a =0$, then it is called a \emph{Maclaurin series}.  Often we
choose $0$ since it is simpler, but the closer to the value of $x$ the better we
will approximate the function. \textcolor{red}{Note: $k! =
k(k-1)(k-2)\times...\times1$, and by definition, $0! = 1$. }

What is the $\operatorname{O}\left((x-a)^{10}\right)$ term? That is the \emph{error term} that
is ``on the order of'' the value in parentheses. This is different from the
\emph{big-O} that we will discuss with regard to algorithm analysis.

The number $e$, also known as \emph{Euler's number} (Leonhard Euler,
1707--1783), is an irrational mathematical constant approximately equal to
$2.71828$, that appears pervasively in the natural and mathematical worlds. It
is the base of the natural logarithm, it is the limit of
$\lim_{n\rightarrow\infty} (1 + \frac{1}{n})^n$ which was discovered by Jacob
Bernoulli (6 January 1655--16 August 1705) in his work on the calculation of
compound interest.

The function $f(x)=e^x$ is a very attractive function, since $f^{(k)}(x) = f^{(k-1)}(x) = \cdots = f^\prime(x) = f(x) = e^x$.
This is one of the simplest series when centered at $0$, since $e^0 = 1$.
\begin{align*}
  e^x & = \sum_{k=0}^{\infty} \frac{x^k}{k!} = 1 + \mlterm{1} + \mlterm{2} +
  \mlterm{3} + \mlterm{4} + \mlterm{5} + \mlterm{6} + \mlterm{7} + \mlterm{8} +
  \mlterm{9} + \ldots
\end{align*}
