\section{Shell Sort}

\epigraph{\emph{There are two ways of constructing a software design.
    One way is to make it so simple that there are obviously no
    deficiencies, and the other way is to make it so complicated that
    there are no obvious deficiencies. The first method is far more
difficult.}}{---C.A.R. Hoare}

\noindent
Donald L. Shell (March 1, 1924--November 2, 2015) was an American
computer scientist who designed the Shell sort sorting algorithm.
He earned his Ph.D. in Mathematics from the University of Cincinnati
in 1959, and published the Shell Sort algorithm in the Communications
of the ACM in July that same year.

Shell Sort is a variation of insertion sort, which sorts pairs of
elements which are far apart from each other. The \emph{gap} between the
compared items being sorted is continuously reduced. Shell Sort starts
with distant elements and moves out-of-place elements into position
faster than a simple nearest neighbor exchange. What is the expected
time complexity of Shell Sort? It depends entirely upon the gap
sequence.

The following is the pseudocode for Shell Sort. The gap sequence is
represented by the array \texttt{gaps}. You will be given a gap
sequence, the Pratt sequence ($2^p 3^q$ also called $3$-smooth), in the
header file \texttt{gaps.h}.  For each \texttt{gap} in the gap sequence,
the function compares all the pairs in \texttt{arr} that are
\texttt{gap} indices away from each other. Pairs are swapped if they are
in the wrong order.

\begin{pylisting}{Shell Sort in Python}
def shell_sort(arr):
    for gap in gaps:
        for i in range(gap, len(arr)):
            j = i
            temp = arr[i]
            while j >= gap and temp < arr[j - gap]:
                arr[j] = arr[j - gap]
                j -= gap
            arr[j] = temp
\end{pylisting}
