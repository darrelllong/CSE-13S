\section{Your Task}\label{task}

\epigraph{\emph{Find out the reason that commands you to write; see whether it has spread its roots into the very depth of your heart; confess to yourself you would have to die if you were forbidden to write.}}{---Rainer Maria Rilke}

\noindent Your task for this assignment is as follows:

\begin{enumerate}
  \item Implement Shell Sort, Batcher Sort (Batcher's method), Heapsort, and
    recursive Quicksort based on the provided Python pseudocode in \textbf{C}.
    The interface for these sorts will be given as the header files
    \texttt{shell.h}, \texttt{batcher.h}, \texttt{heap.h}, and \texttt{quick.h}.
    \textcolor{red}{You are not allowed to modify these files for any reason.}
  \item Implement a test harness for your implemented sorting
    algorithms. In your test harness, you will creating an array of
    pseudorandom elements and testing each of the sorts. Your test harness
    \emph{must} be in the file \texttt{sorting.c}.
  \item Gather statistics about each sort and its performance. The
    statistics you will gather are the \emph{size} of the array, the
    number of \emph{moves} required, and the number of
    \emph{comparisons} required. \textcolor{red}{Note: a comparison is
    counted only when two array elements are compared.}
\end{enumerate}

Your test harness must support any combination of the following
command-line options:

\begin{itemize}
  \item \texttt{-a}\ : Employs \emph{all} sorting algorithms.
  \item \texttt{-h}\ : Enables Heap Sort.
  \item \texttt{-b}\ : Enables Batcher Sort.
  \item \texttt{-s}\ : Enables Shell Sort.
  \item \texttt{-q}\ : Enables Quicksort.
  \item \texttt{-r seed}\ : Set the random seed to \texttt{seed}.
    The \emph{default} seed should be 13371453.
  \item \texttt{-n size}\ : Set the array size to \texttt{size}. The
    \emph{default} size should be 100.
  \item \texttt{-p elements}\ : Print out \texttt{elements} number of
    elements from the array. The \emph{default} number of elements to
    print out should be 100. \textcolor{red}{If the size of the array is
      less than the specified number of elements to print, print out the
    entire array and nothing more.}
  \item \texttt{-H}\ : Prints out program usage. See reference program
    for example of what to print.
\end{itemize}

It is important to read this \emph{carefully}. None of these options are
\emph{exclusive} of any other (you may specify any number of them,
including \emph{zero}). The most natural data structure for this
problem is a \emph{set}.
