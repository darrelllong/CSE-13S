\section{Testing}

\begin{itemize}
  \item You will test each of the sorts specified by command-line option
    by sorting an array of pseudorandom numbers generated with
    \texttt{random()}. Each of your sorts should sort the \emph{same}
    pseudorandom array. \textcolor{red}{Hint: make use of
    \texttt{srandom()}.}
  \item The pseudorandom numbers generated by \texttt{random()} should
    be \emph{bit-masked} to fit in \emph{30} bits. \textcolor{red}{Hint: use
    bit-wise AND.}
  \item Your test harness \emph{must} be able to test your sorts with
    array sizes \emph{up to the memory limit of the computer}. That
    means that you will need to dynamically allocate the array.
  \item Your program should have no \emph{memory leaks}. Make sure you
    \texttt{free()} before exiting. \texttt{valgrind} should pass
    cleanly with any combination of the specified command-line options.
  \item Your algorithms \emph{must} correctly sort. Any algorithm that
    does not sort correctly will receive a \emph{zero}.
\end{itemize}

A large part of this assignment is understanding and comparing the
performance of various sorting algorithms. You essentially conducting an
experiment. As stated in \S\ref{task}, you \emph{must} collect the following
statistics on each algorithm:

\begin{itemize}
  \item The \emph{size} of the array,
  \item The number of \emph{moves} required (each time you transfer an
    element in the array, that counts), and
  \item The number of \emph{comparisons} required (comparisons
    \emph{only} count for \emph{elements}, not for logic).
\end{itemize}
