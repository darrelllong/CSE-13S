\section{Output}
\epigraph{\emph{Books are not made to be believed, but to be subjected to inquiry. When we consider a book, we mustn’t ask ourselves what it says but what it means.}}{---Umberto Eco}

\noindent
The output your test harness produces \emph{must} be formatted like in
the following examples:

\begin{shlisting}{}
$ ./sorting -q -n 1000 -p 0
Quick Sort, 1000 elements, 18642 moves, 10531 compares
$ ./sorting -h -n 15 -p 0
Heap Sort, 15 elements, 144 moves, 70 compares
$ ./sorting -a -n 15
Shell Sort, 15 elements, 90 moves, 59 compares
     34732749     42067670     54998264    102476060    104268822
    134750049    182960600    538219612    629948093    783585680
    954916333    966879077    989854347    994582085   1072766566
Batcher Sort, 15 elements, 90 moves, 59 compares
     34732749     42067670     54998264    102476060    104268822
    134750049    182960600    538219612    629948093    783585680
    954916333    966879077    989854347    994582085   1072766566
Heap Sort, 15 elements, 144 moves, 70 compares
     34732749     42067670     54998264    102476060    104268822
    134750049    182960600    538219612    629948093    783585680
    954916333    966879077    989854347    994582085   1072766566
Quick Sort, 15 elements, 135 moves, 51 compares
     34732749     42067670     54998264    102476060    104268822
    134750049    182960600    538219612    629948093    783585680
    954916333    966879077    989854347    994582085   1072766566
\end{shlisting}

For each sort that was specified, print its name, the statistics for the
run, then the specified number of array elements to print. The array
elements should be printed out in a table with 5 columns. Each array
element should be printed with a width of 13. You should make use of the
following \texttt{printf()} statement:

\begin{clisting}{}
printf("%13" PRIu32); // Include <inttypes.h> for PRIu32.
\end{clisting}
