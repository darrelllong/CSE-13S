\section{Heapsort}

\epigraph{\emph{Increasingly, people seem to misinterpret complexity as
sophistication, which is baffling -- the incomprehensible should cause
suspicion rather than admiration.}}{---Niklaus Wirth}

\noindent Heapsort, along with the heap data structure, was invented in
1964 by J. W. J. Williams. The heap data structure is typically
implemented as a specialized binary tree. There are two kinds of heaps:
\emph{max} heaps and \emph{min} heaps. In a max heap, any parent node
\emph{must} have a value that is greater than or equal to the values of
its children. For a min heap, any parent node \emph{must} have a value
that is less than or equal to the values of its children. The heap is
typically represented as an \emph{array}, in which for any index $k$,
the index of its left child is $2k$ and the index of its right child is
$2k + 1$. It's easy to see then that the parent index of any index $k$
should be $\lfloor \frac{k}{2} \rfloor$.

\begin{figure}[b]
  \begin{center}
    \begin{forest} for tree={circle,draw,inner sep=5pt,l=10pt,l sep=6pt,s sep=14pt}
      [13
        [5
          [0]
          [1]
        ]
        [8
          [2]
          [3]
        ]
      ]
    \end{forest}
  \end{center}
  \begin{center}
    \begin{tikzpicture}
      \matrix (A) [matrix of nodes, nodes={draw, minimum size=8mm},
          column sep=-\pgflinewidth]{
          13 & 5 & 8 & 0 & 1 & 2 & 3\\};
    \end{tikzpicture}
  \end{center}
  \caption{A max heap and its array representation.}
\end{figure}

Heapsort, as you may imagine, utilizes a heap to sort elements. Heapsort
sorts its elements using two routines that 1) build a heap, 2) fix a
heap.

\begin{enumerate}
  \item \textbf{Building a heap.} The first routine is taking the array
    to sort and building a heap from it. This means ordering the array
    elements such that they obey the constraints of a max or min heap.
    For our purposes, the constructed heap will be a \emph{max heap}.
    This means that the largest element, the root of the heap, is the
    first element of the array from which the heap is built.
  \item \textbf{Fixing a heap.} The second routine is needed as we sort
    the array. The gist of Heapsort is that the largest array elements
    are repeatedly removed from the top of the heap and placed at the
    end of the sorted array, if the array is to be sorted in increasing
    order. After removing the largest element from the heap, the heap
    needs to be \emph{fixed} so that it once again obeys the constraints
    of a heap.
\end{enumerate}

In the following Python code for Heapsort, you will notice a lot of
indices are shifted down by 1. Why is this? Recall how indices of
children are computed. The formula of the left child of $k$ being $2k$
only works assuming 1-based indexing. We, in Computer Science,
especially in \textbf{C}, use 0-based indexing. So, we will run the
algorithm assuming 1-based indexing for the Heapsort algorithm itself,
subtracting 1 on each array index access to account for 0-based
indexing.

\begin{pylisting}{Heap maintenance in Python}
def max_child(A: list, first: int, last: int):
    left = 2 * first
    right = left + 1
    if right <= last and A[right - 1] > A[left - 1]:
        return right
    return left

def fix_heap(A: list, first: int, last: int):
    found = False
    mother = first
    great = max_child(A, mother, last)

    while mother <= last // 2 and not found:
        if A[mother - 1] < A[great - 1]:
            A[mother - 1], A[great - 1] = A[great - 1], A[mother - 1]
            mother = great
            great = max_child(A, mother, last)
        else:
            found = True
\end{pylisting}

\begin{pylisting}{Heapsort in Python}
def build_heap(A: list, first: int, last: int):
    for father in range(last // 2, first - 1, -1):
        fix_heap(A, father, last)

def heap_sort(A: list):
    first = 1
    last = len(A)
    build_heap(A, first, last)
    for leaf in range(last, first, -1):
        A[first - 1], A[leaf - 1] = A[leaf - 1], A[first - 1]
        fix_heap(A, first, leaf - 1)
\end{pylisting}
