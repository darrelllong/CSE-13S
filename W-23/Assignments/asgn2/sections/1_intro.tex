\section{Introduction}

\epigraphwidth=0.6\textwidth \epigraph{\emph{Programming is one of the
most difficult branches of applied mathematics; the poorer
mathematicians had better remain pure mathematicians.}}{---Edsger
Dijkstra}

\noindent
As we know, computers are simple machines that carry out a sequence of
elementary steps, albeit very quickly. Unless you have a special-purpose
processor, a computer can only compute \emph{addition},
\emph{subtraction}, \emph{multiplication}, and \emph{division}. If you
think about it, you will see that the functions that might interest you
when dealing with real or complex numbers can be built up from those
four operations. We use many of these functions in nearly every program
that we write, so we ought to understand how they are created.

We cannot expect the computer to solve integrals such as
$$
\left [ \int_{-\infty}^\infty e^{-x^2} \, dx \right ] ^2
= \pi $$
in order to calculate $\pi$, and computing \emph{numerical
integrals} is more complicated than we want to attempt in this class.
But, suppose that
you wanted to calculate that integral, how would you do it? You might use a formula like this:
$$
\frac{h}{3}
   \bigg[f(x_0) + 2\sum_{j=1}^{\lfloor n/2-1 \rfloor} f(x_{2j}) + 4\sum_{j=1}^{\lfloor n/2 \rfloor} f(x_{2j-1}) + f(x_n)\bigg]
$$
which is called the \emph{composite Simpson's $\tfrac{1}{3}$ rule}.
If you were to compute $\int_{-5}^{+5}$ the error would be just $9.69935\times10^{-12}$ or about $11$ digits of precision.

Fortunately, you may recall from your Calculus class, with some
conditions a function $f(x)$ can be represented by its Taylor series
expansion near some point $f(a)$: $$ f(x) = f(a) + \sum_{k=1}^\infty
\frac{f^{(k)}(a)}{k!}{(x-a)}^k. $$ We will make extensive use of
infinite series, suitably truncated, to solve many problems of interest.
Since we cannot compute an infinite series, we must be content to
calculate a finite number of terms. In general, the more terms that we
calculate, the more accurate our approximation. \textcolor{red}{Note:
when you see $\Sigma$ or $\Pi$, you should generally think of a \texttt{for}
loop.}

If you have forgotten (or have never taken) Calculus, do not despair.
Attend a laboratory section for review: the concepts required for this
assignment are do not extend beyond derivatives.
