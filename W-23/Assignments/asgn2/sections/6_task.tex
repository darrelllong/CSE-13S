\section{Your Task}

Your task for this assignment is to implement a small number of
mathematical functions ($e^x$ and $\sqrt{x}$), mimicking
\texttt{<math.h>}, and using them to compute the fundamental constants
$e$ and $\pi$. You will also need to write a dedicated test harness
comparing your implemented functions with that of the math library's,
then analyzing and presenting your findings in a writeup. The test
harness should be a program named \texttt{mathlib-test}. The interface
for your math library will be given in \texttt{mathlib.h}.
\textcolor{red}{You may not modify this file.} The following sections
will describe the functions that you need to write, and the files that
should contain the functions.

\textcolor{red}{You are \emph{strictly forbidden} to use any functions
from \texttt{<math.h>} in your own math library. You are also forbidden
to write a \texttt{factorial()} function.}

Each of the functions you will write must halt computation using an
$\epsilon = 10^{-14}$, which will be defined in \texttt{mathlib.h}. For
example, consider approximating the value of $e$. For sufficiently large
$k$, $|x^k| < k!$. As seen in Figure \ref{growth}, $x^k$ dominates
briefly, but is quickly overwhelmed by $k!$, making the ratio rapidly
approach zero.

\subsection{\texttt{e.c}}

This file should contain two functions: \texttt{e()} and
\texttt{e\_terms()}. The former function will approximate the value of
$e$ using the Taylor series presented in \S3 and track the number of
computed terms by means of a \texttt{static} variable local to the file.
The latter function will simply return the number of computed terms.

\subsection{\texttt{madhava.c}}

This file should contain two functions: \texttt{pi\_madhava()} and
\texttt{pi\_madhava\_terms()}. The former function will approximate the
value of $\pi$ using the Madhava series presented in \S4.2 and track the
number of computed terms with a \texttt{static} variable, exactly like
in \texttt{e.c}. The latter function will simply return the number of
computed terms.

\subsection{\texttt{euler.c}}

This file should contain two functions: \texttt{pi\_euler()} and
\texttt{pi\_euler\_terms()}. The former function will approximate the
value of $\pi$ using the formula derived from Euler's solution to the
Basel problem, as described in \S4.4. It should also track the number of
computed terms. The latter function will simply return the number of
computed terms.

\subsection{\texttt{bbp.c}}

This file should contain two functions: \texttt{pi\_bbp()} and
\texttt{pi\_bbp\_terms()}. The former function will approximate the value
of $\pi$ using the Bailey-Borwein-Plouffe formula presented in \S4.5 and
track the number of computed terms. The latter function will simply
return the number of computed terms.

\subsection{\texttt{viete.c}}

This file should contain two functions: \texttt{pi\_viete()} and
\texttt{pi\_viete\_factors()}. The former function will approximate the
value of $\pi$ using Vi\`{e}te's formula as presented in \S4.6 and track
the number of computed factors. The latter function will simply return
the number of computed factors.

\subsection{\texttt{newton.c}}

This file should contain two functions: \texttt{sqrt\_newton()} and
\texttt{sqrt\_newton\_iters()}. The former function will approximate the
the square root of the argument passed to it using the Newton-Raphson
method presented in \S5. This function should also track the number of
iterations taken, which the latter function will return.

\subsection{\texttt{mathlib-test.c}}

This file will contain the main test harness for your implemented math
library. It should support the following
command-line options:

\begin{itemize}
  \item \textbf{\texttt{-a}\,:} Runs all tests.
  \item \textbf{\texttt{-e}\,:} Runs $e$ approximation test.
  \item \textbf{\texttt{-b}\,:} Runs Bailey-Borwein-Plouffe $\pi$ approximation test.
  \item \textbf{\texttt{-m}\,:} Runs Madhava $\pi$ approximation test.
  \item \textbf{\texttt{-r}\,:} Runs Euler sequence $\pi$ approximation test.
  \item \textbf{\texttt{-v}\,:} Runs Vi\`{e}te $\pi$ approximation test.
  \item \textbf{\texttt{-n}\,:} Runs Newton-Raphson square root approximation tests.
  \item \textbf{\texttt{-s}\,:} Enable printing of statistics to see
    computed terms and factors for each tested function.
  \item \textbf{\texttt{-h}\,:} Display a help message detailing program
    usage.
\end{itemize}

The expected output for each of the $e$ or $\pi$ tests should resemble
the following:

\begin{shlisting}{}
$ ./mathlib-test -e -b -v
e() = 2.718281828459046, M_E = 2.718281828459045, diff = 0.000000000000000
pi_bbp() = 3.141592653589793, M_PI = 3.141592653589793, diff = 0.000000000000000
pi_viete() = 3.141592653589775, M_PI = 3.141592653589793, diff = 0.000000000000018
\end{shlisting}

Note that the newline should occur \emph{after} the printed difference.
You can refer to the reference program in the resources repostiroy for
the example output. With the statistics option enabled, the output
should resemble the following:

\begin{shlisting}{}
$ ./mathlib-test -e -b -v -s
e() = 2.718281828459046, M_E = 2.718281828459045, diff = 0.000000000000000
e terms = 18
pi_bbp() = 3.141592653589793, M_PI = 3.141592653589793, diff = 0.000000000000000
pi_bbp() terms = 11
pi_viete() = 3.141592653589775, M_PI = 3.141592653589793, diff = 0.000000000000018
pi_viete() terms = 23
\end{shlisting}

You will specially test your \texttt{sqrt\_newton()} function in the
range $[0,\,10)$ in steps of $0.1$. Again, you can refer to the
reference program in the resources repository for the example output.
Any \texttt{double} that is printed should use the following
\texttt{printf()} format specifier:

\begin{clisting}{}
double pi = M_PI;
printf("%16.15lf\n", pi); // Newline is included as an example.
\end{clisting}

