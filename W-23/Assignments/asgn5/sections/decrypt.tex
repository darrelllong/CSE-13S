\section{Decryptor}
\epigraphwidth=0.33\textwidth
\epigraph{\emph{So much technology, so little talent.}}{---Vernor Vinge, \emph{Rainbow's End}}

\noindent
Your decryptor program should accept the following command-line options:
\begin{itemize}
  \item \texttt{-i}\,: specifies the input file to decrypt (default:
    \texttt{stdin}).
  \item \texttt{-o}\,: specifies the output file to decrypt (default:
    \texttt{stdout}).
  \item \texttt{-n}\,: specifies the file containing the private key
    (default: \texttt{ss.priv}).
  \item \texttt{-v}\,: enables verbose output.
  \item \texttt{-h}\,: displays program synopsis and usage.
\end{itemize}
The program should follow these steps:
\begin{enumerate}
  \item Parse command-line options using \texttt{getopt()} and handle
    them accordingly.
  \item Open the private key file using \texttt{fopen()}. Print a helpful
    error and exit the program in the event of failure.
  \item Read the private key from the opened private key file.
  \item If verbose output is enabled print the following, each with a
    trailing newline, in order:
    \begin{enumerate}
      \item the private modulus \texttt{pq}
      \item the private key \texttt{d}
    \end{enumerate}
    Both these values should be printed with information about the
    number of bits that constitute them, along with their respective
    values in \emph{decimal}. See the reference decryptor program for an
    example.
  \item Decrypt the file using \texttt{ss\_decrypt\_file()}.
  \item Close the private key file and clear any \texttt{mpz\_t}
    variables you have used.
\end{enumerate}
