\section{Mathematics of SS}\label{proof}

\epigraphwidth=0.7\textwidth
\epigraph{\emph{If $\,{\cal P} = \NNP$, then all of modern cryptography collapses. On this happy thought\ldots}}{Michael O.\xspace Rabin, November 1998}

\noindent
The mathematics of SS are based on arithmetic in a set of integers
modulo $n$, denoted $\mathbb{Z}/n$. This is the set $\{ 0, \ldots, n - 1
\}$ and all sets that are equivalent to it. For example, $\{ n, \ldots,
2n -1 \} \pmod{n} = \{ 0, \ldots, n - 1 \} \pmod{n}$, and there are an
infinite number of such sets. Since they are \emph{all the same} we will
only concern ourselves with the one with the smallest numbers. What do
we mean when we say that are the same? We mean that if $x \equiv k
\pmod{n}$ then $a n + x \equiv k \pmod{n}, \forall a \ge 0, a \in
\mathbb{Z}$. In other words, additional integer products of $n$ do not
matter.

The Euler-Fermat theorem says that if $a\in\mathbb{N}$ and
$n\in\mathbb{N}$ are \emph{coprime}, that is $\gcd(a, n) = 1$, then
$$a^{\varphi(n)} \equiv 1 \pmod{n}.$$ This will allow us, for example,
to take a message $M$ and have $M^{\varphi(n)} \equiv 1 \pmod{n}$.

What is $\varphi(n)$? It is the Euler totient function, and gives the
number of positive integers than that or equal to $n$ that are
relatively prime to $n$. For any prime number $p$,
$$
\varphi(p) = p - 1.
$$

For the SS algorithm, we choose two large primes $p$ and $q$, where
$p \nmid q - 1$ and $q \nmid p - 1$, and we make $n = p^2 q$.
Then we calculate $d$ such that $d \cdot n \equiv 1 \pmod{\lcm(p - 1, q - 1)}$.
What does this mean for us with respect to $\varphi(pq)$?
\begin{align*}
        \varphi(pq) & = \varphi(p) \times \varphi(q) \\
        & = (p - 1) (q - 1) \\
        & = pq - (p + q) + 1 .
\end{align*}

Now because we chose $p$ and $q$ such that $p \nmid q - 1$ and $q \nmid p - 1$,
$n = p^2 q$ is relatively prime to $\varphi(pq)$,
that is, $\gcd(n, \varphi(pq)) = 1$. Because of this we know that a decryption key $d$
exists, such that $n \times d \equiv 1 \pmod{\varphi(pq)}$.
Our encryption algorithm is simply $E(M) = M^n \pmod{n} = C$, and our
decryption algorithm is $D(C) = C^d \pmod{pq} = M$.

Observe that,
\[
  M^{n d} \equiv M^{k \varphi(pq) + 1} \pmod{pq}
\]
for some integer $k \ge 1$. This is true because, prior to the
modular reduction, $nd$ must be one greater than some multiple of
$\varphi(pq)$; applying the modulus is what makes
$n d \equiv 1 \pmod{\varphi(pq)}$. We can rewrite this as
\[
  M^{n d} \equiv (M^{\varphi(pq)})^k \times M \pmod{pq}.
\]
Here we apply Euler's theorem, which states that if $a$ and $n$ are
coprime integers, then $a^{\varphi(n)} \equiv 1 \pmod{n}$. So, assuming
that $M$ is coprime with $pq$, we can simplify the above equation to
\begin{align*}
  M^{n d} &\equiv (1)^k \times M  &&\pmod{pq} \\
          &\equiv (1) \times M  &&\pmod{pq} \\
          &\equiv M  &&\pmod{pq}
\end{align*}
which proves that decryption really does work on encrypted messages.

In practice, Carmichael's function can be used in place of Euler's totient in
the computation of the SS key pair. Carmichael's function is denoted with
$\lambda(n)$, where $\lambda(n) = \lcm(\lambda(p), \lambda(q))$.
We indicate the least common multiple of some $a$ and $b$ with
$\lcm(a, b)$. From the definition of the least common multiple, we
see that
\begin{align*}
  \lambda(n) &= \lambda(pq) \\
             &= \lcm(\lambda(p), \lambda(q)) \\
             &= \frac{|\lambda(p) \times \lambda(q)|}{\gcd(\lambda(p), \lambda(q))}.
\end{align*}
If $p$ and $q$ are prime, then we know
$\lambda(p) = \varphi(p) = p - 1$, and $\lambda(q) = \varphi(q) = q - 1$. Thus,
\begin{align*}
  \lambda(n) &= \frac{|\lambda(p) \times \lambda(q)|}{\gcd(\lambda(p), \lambda(q))} \\
             &= \frac{|\varphi(p) \times \varphi(q)|}{\gcd(p - 1, q - 1)} \\
             &= \frac{|\varphi(n)|}{\gcd(p - 1, q - 1)}.
\end{align*}
It should be clear from this that $\varphi(pq)$ is a multiple of $\lambda(pq)$.
Therefore,
\begin{align*}
  M^{n d} &\equiv (M^{\varphi(pq)})^k \times M &&\pmod{pq} \\
          &\equiv (M^{j \times \lambda(pq)})^k \times M &&\pmod{pq} \\
          &\equiv (M^{\lambda(pq)})^{jk} \times M &&\pmod{pq}
\end{align*}
for some multiplier $j \ge 1$.

From Carmichael's generalization of Euler's theorem, we know that
$a^{\lambda(pq)} \equiv 1 \pmod{pq}$. Thus,
\begin{align*}
  M^{n d} &\equiv (M^{\lambda(pq)})^{jk} \times M &&\pmod{pq} \\
          &\equiv (1)^{jk} \times M  &&\pmod{pq} \\
          &\equiv (1) \times M  &&\pmod{pq} \\
          &\equiv M  &&\pmod{pq},
\end{align*}
which demonstrates how $\lambda(pq)$ can be used in place of $\varphi(pq)$ for the
computation of the SS key pair. For your assignment, you will be using
Carmichael's function when computing your SS key pair.
