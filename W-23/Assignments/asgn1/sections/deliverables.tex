\section{Deliverables}

You will need to turn in the following source code and header files:

\begin{enumerate}
  \item \texttt{plot.sh}: This \Bash{} script should produce the Monte Carlo
    method plots used in your report. This script should produce plots similar to Figures  \ref{figure:monte_carlo_grid}, \ref{figure:monte_carlo}. You can create more plots than these if you
    choose to.
  \item \texttt{monte\_carlo.c}: This file is provided and contains the
    implementation of the Monte Carlo program.
  \item \texttt{Makefile}: This file is provided and directs the compilation
    process of the Monte Carlo program.
\end{enumerate}

You may have other source and header files, but \emph{do not make things
over complicated}. \textcolor{red}{Any additional source code and header
files that you may use must not use global variables.}
You will also need to turn in the following:

\begin{enumerate}
        \item \texttt{README.md}: This must use proper \emph{Markdown} syntax. It
    must describe how to use your script and \texttt{Makefile}.
    should also list and explain any command-line options that your
    program accepts.
  \item \texttt{DESIGN.pdf}: This document \emph{must} be a proper PDF\@. This
    design document must describe your design and design process for your
    program with enough detail such that a sufficiently knowledgeable programmer
    would be able to replicate your implementation. \textcolor{red}{This does
    not mean copying your entire program in verbatim}. You should instead
    describe how your program works with supporting pseudocode.
  \item \texttt{WRITEUP.pdf}: This document \emph{must} be a proper
    PDF\@. This writeup must include the plots that you produced using your
    \Bash{} script, as well as discussion on which \Unix{} commands you
    used to produce each plot and why you chose to use them.
                \textcolor{red}{You should use \LaTeX to produce this document.}
\end{enumerate}

