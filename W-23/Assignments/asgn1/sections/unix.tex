\section{\Unix{} and the Shell}\label{section:unix}

\noindent To understand why a \emph{shell} is used to interact with
\Unix{}, we need to first delve into why \Unix{} was developed the
way it was. The development of \Unix{} started by
Ken Thompson and Dennis Ritchie, with contributions by many others, for use at Bell
Laboratories (Bell Labs). The development continued throughout most of the 70s, eventually leading
to the licensing of \Unix{} to other parties, especially universities in the late 70s.

Along with \Unix{} came the \Unix{} philosophy. This philosophy is
based on the belief that a tool should have a limited, yet specific
function. All the tools on the original \Unix{} system shipped with this
philosophy. Examples include \texttt{head}, which prints out the first $n$ lines
of a file, \texttt{ls}, which lists out files in a directory, and \texttt{diff},
which reports differences between two files. The idea is that the output of one
tool, or \emph{command}, can become the input of another command.

Commands are the main way that users are intended to interact with
\Unix{}. Each command follows the pattern of \texttt{command
[arguments]} (here the square brackets mean the arguments are
optional). Each command can be executed an optional number of
arguments. Consider the following command:

\begin{shlisting}{}
$ wc -l roster.csv
\end{shlisting}

This particular command uses the \texttt{wc} command, which is a program that
counts the number of words, lines, and characters in a file. The argument
\texttt{-l} tells \texttt{wc} to only consider the line count. The final
argument, \texttt{roster.csv}, is the file to get the line count from. Note the
\texttt{\$} prefixing the command. The \texttt{\$} it not itself part of the
command, but is commonly used to indicate that the command is run in a
\emph{shell} and acts as a \emph{shell prompt}.

A shell is a program (just like any other) that runs in an infinite loop, parsing and executing
commands. In other words, it acts as a \emph{command line interpreter}. Examples
of popular shells include \Bash{} and \Zsh{}. Both these shells have two modes of
operation: interactive and batch. Using a shell in interactive mode means
issuing commands one at a time, with the result of the command appearing
immediately after the command is executed. Using a shell in batch mode means
giving the shell a \emph{series} of commands to run in sequence. Typically this
means placing all the commands in a file. Files that contain a series of
commands to run are typically called \emph{shell scripts}. You will be writing a
shell script for this assignment.
