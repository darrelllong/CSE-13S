\section{Your Task}
Use the \texttt{Makefile} that is provided for you.

Create a program called \texttt{classify.c} that takes two arguments:

\texttt{-p} that determines whether to print only perfect numbers

\texttt{-n \#} that determines the largest number to classify

It will classify each number into one of \emph{four} categories: prime, perfect, deficient or abundant.

You will do this using the `getopt()` library

\subsection{\em bonus pro opere superior}

Some of you may be thinking that this assignment is too simple, it
does not test your abilities. Consequently, we are offering you the
opportunity to earn  an extra 20\% by finding the so-called
\emph{weird numbers}.

A \emph{semiperfect number} is
a natural number that is equal to the sum of \emph{some or all} of its
proper divisors. A semiperfect number that is equal to the sum of
all its proper divisors is a perfect number.  Most abundant numbers
are also semiperfect; abundant numbers that are not semiperfect are
called \emph{weird numbers}.

Determining whether a number is semiperfect requires that the sum
of \emph{every subset} of its proper divisors be checked. The set
of all subsets of a set $S$ is called its \emph{power set}.
Let $|S|$ be the number of elements in $S$, then the power set, written
${\cal P} (S)$ and has cardinality $2^k, k = |S|$.

An algorithm for cycling through all subsets naturally comes from
looking at the binary representation of a number. Assume that we
have an array $d_0, \ldots, d_{k-1}$ that hold the $k$ proper
divisors of our number $n$. Let $i = 1, \ldots, 2^{k-1}$. For each
such $k$ select the divisors that correspond to $1$ bits in the
binary representation of $i$.

Let's see how we might accomplish this in Python:

\begin{pylisting}{Semiperfection}
def combination(d):
    for c in range(1 << len(d)):
        yield [ _ for _ in range(len(d)) if c & (1 << _) ]

def semiperfect(d, n):
    for c in combination(d):
        if sum([ d[_] for _ in c ]) == n:
            return True
    return False
\end{pylisting}

How to we accomplish a similar operation in \textbf{C}? Consider this simple program:

\begin{clisting}{Checking for set bits}
#include <stdio.h>

int main(void) {
    int x = 1962;
    for (int i = 0; i < 32; i += 1) {
        if (x & 1 << i) { printf("bit %d is 1\n", i); }
    }
    return 0;
}
\end{clisting}

If you decide to try for extra credit, what should your output look
like? Here's a simple example. Be warned, do not expect finding
weird numbers to be fast---they are few in number, and you have to
evaluate the power set for each number.

\begin{shlisting}{Example}
eco $ ./classify -w -n 100 | grep weird
70: [1, 2, 5, 7, 10, 14, 35], abundant and weird
836 {11}: [1, 2, 4, 11, 19, 22, 38, 44, 76, 209, 418], abundant and weird
\end{shlisting}
