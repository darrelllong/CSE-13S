\section{Specifics}\label{section:specifics}

You will be writing a \C program, and the first thing is to include
the appropriate header files.

\begin{clisting}{Header Files}
#include <stdbool.h> // Gives us the bool data type
#include <stdio.h>   // ... standard I/O
#include <stdlib.h>  // ... strtoul() and other useful things
#include <string.h>  // ... string utilities
#include <unistd.h>  // ... getopt() and other useful things
\end{clisting}

You will need to process the command line arguments.
The \texttt{-p} option determines whether you will print just
\emph{perfect numbers} or all numbers (which is the default).

\begin{clisting}{Command line arguments}
#define DEFAULT_MAXIMUM 1000
#define OPTIONS "pn:"

int main(int argc, char **argv) {
    int n = DEFAULT_MAXIMUM;
    bool imperfect = true;
    int opt;

    while ((opt = getopt(argc, argv, OPTIONS)) != -1) {
        switch (opt) {
        case 'p':
            imperfect = !imperfect;
            break;
        case 'n':
            n = strtoul(optarg, NULL, 10);
            break;
        }
    }
    // Your other code goes here ...
}
\end{clisting}

For each positive integer from the ordered set $n\in\left\{ 2, \ldots, n\right\}$ you will classify it
as \emph{prime}, \emph{perfect}, \emph{deficient} or \emph{abundant}.

You will print the number, followed by a list of all of its
\emph{proper divisors}, followed by its classification.
Classifying a number requires you to determine \emph{all} of its proper divisors.
\begin{itemize}
\item How do you know if a number $n$ is \emph{prime}? It has only
$1$ as a \emph{proper divisor}.

\item How do you know if a number $n$ is \emph{perfect}? The \emph{sum}
of the proper divisors $s(n) = n$.

\item How do you know if a number $n$ is \emph{deficient}? The
\emph{sum} of the proper divisors $s(n) < n$.

\item How do you know if a number $n$ is \emph{abundant}? The
\emph{sum} of the proper divisors $s(n) > n$.

\end{itemize}
You should see a pattern here that you can (and should) exploit.

You will want to keep an array of the proper divisors, alternatively,
you can keep a string of their decimal representation, but that seems inelegant, especially if you are trying for \emph{extra credit}. You will need to
do this because the \texttt{-p} option will prevent printing of all
\emph{imperfect} numbers. You will not know which class a number
falls into until you have examined all divisors.

\begin{shlisting}{Example}
eco :: ./classify -n 20
2 : [1] is prime
3 : [1] is prime
4 : [1, 2] is deficient
5 : [1] is prime
6 : [1, 2, 3] is perfect
7 : [1] is prime
8 : [1, 2, 4] is deficient
9 : [1, 3] is deficient
10 : [1, 2, 5] is deficient
11 : [1] is prime
12 : [1, 2, 3, 4, 6] is abundant
13 : [1] is prime
14 : [1, 2, 7] is deficient
15 : [1, 3, 5] is deficient
16 : [1, 2, 4, 8] is deficient
17 : [1] is prime
18 : [1, 2, 3, 6, 9] is abundant
19 : [1] is prime
20 : [1, 2, 4, 5, 10] is abundant
\end{shlisting}

\begin{shlisting}{Example}
eco :: ./classify -n 10000 -p
6 : [1, 2, 3] is perfect
28 : [1, 2, 4, 7, 14] is perfect
496 : [1, 2, 4, 8, 16, 31, 62, 124, 248] is perfect
8128 : [1, 2, 4, 8, 16, 32, 64, 127, 254, 508, 1016, 2032, 4064] is perfect
\end{shlisting}

\subsection{Executables}

An executable on \Unix{} is just another name for a program. The terms
\emph{program}, \emph{executable}, \emph{executable binary}, and sometimes just
\emph{binary}, all refer to a program that can be run, or \emph{executed}. The
distinction between an plain old executable and an executable binary is in its
representation. An executable binary is platform specific and is simply a file
full of machine code: pure binary. An executable could be a shell script, which
contains readable text. The only thing these two share is that they must have
the executable bit set in order to be run. Refer to the discussion of \Unix{}
file permissions in assignment 0 if you have forgotten what the executable bit
refers to.
