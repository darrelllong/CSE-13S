\section{Deliverables}

\noindent You will need to turn in the following source code and header files:

\begin{enumerate}
  \item \texttt{bf.h}: Defines the interface for the Bloom filter ADT.
    \textcolor{red}{Do not modify this.}

  \item \texttt{bf.c}: Contains the implementation of the Bloom filter
    ADT.

  \item \texttt{bv.h}: Defines the interface for the bit vector ADT.
    \textcolor{red}{Do not modify this.}

  \item \texttt{bv.c}: Contains the implementation of the bit vector
    ADT.

  \item \texttt{ht.h}: Defines the interface for the hash table ADT and the hash
    table iterator ADT. \textcolor{red}{Do not modify this.}

  \item \texttt{ht.c}: Contains the implementation of the hash table
    ADT and the hash table iterator ADT.

  \item \texttt{identify.c}: Contains \texttt{main()} and the implementation of
    the author identification program.

  \item \texttt{metric.h}: Defines the enumeration for the distance metrics and
    their respective names stored in an array of strings. \textcolor{red}{Do not
    modify this.}

  \item \texttt{node.h}: Defines the interface for the node ADT.
    \textcolor{red}{Do not modify this.}

  \item \texttt{node.c}: Contains the implementation of the node ADT.

  \item \texttt{parser.h}: Defines the interface for the regex parsing
    module. \textcolor{red}{Do not modify this.}

  \item \texttt{parser.c}: Contains the implementation of the regex
    parsing module.

  \item \texttt{pq.h}: Defines the interface for the priority queue ADT.
    \textcolor{red}{Do not modify this.}

  \item \texttt{pq.c}: Contains the implementation for the priority queue ADT.

  \item \texttt{salts.h}: Defines the primary, secondary, and tertiary
    salts to be used in your Bloom filter implementation. Also defines
    the salt used by the hash table in your hash table implementation.

  \item \texttt{speck.h}: Defines the interface for the hash function
    using the SPECK cipher. \textcolor{red}{Do not modify this.}

  \item \texttt{speck.c}: Contains the implementation of the hash
    function using the SPECK cipher. \textcolor{red}{Do not modify this.}

  \item \texttt{text.h}: Defines the interface for the text ADT.
    \textcolor{red}{Do not modify this.}

  \item \texttt{text.c}: Contains the implementation for the text ADT.
\end{enumerate}

You may have other source and header files, but \emph{do not try to be
overly clever}. You will also need to turn in the following:

\begin{enumerate}
  \item \texttt{Makefile}: This is a file that will allow the grader to
    type \texttt{make} to compile your program.

    \begin{itemize}
      \item \texttt{CC = clang} must be specified.
      \item \texttt{CFLAGS = -Wall -Wextra -Werror -Wpedantic}
        must be included.
      \item \texttt{make} should build the \texttt{identify}
        executable, as should \texttt{make all} and \texttt{make
        identify}.
      \item \texttt{make clean} must remove all files that are compiler
        generated.
      \item \texttt{make format} should format all your source code,
        including the header files.
    \end{itemize}

  \item Your code must pass \texttt{scan-build} \emph{cleanly}. If there
    are any bugs or errors that are false positives, document them and
    explain why they are false positives in your \texttt{README.md}.

  \item \texttt{README.md}: This must be in \emph{Markdown}. This must
    describe how to use your program and \texttt{Makefile}. This
    includes listing and explaining the command-line options that your
    program accepts. Any false positives reported by \texttt{scan-build}
    should go here as well.

  \item \texttt{DESIGN.pdf}: This \emph{must} be a PDF. The design
    document should describe your design for your program with enough
    detail that a sufficiently knowledgeable programmer would be able to
    replicate your implementation. This does not mean copying your
    entire program in verbatim. You should instead describe how your
    program works with supporting pseudocode. For this program, pay
    extra attention to how you build each necessary component.

  \item \texttt{WRITEUP.pdf}: This document \emph{must} be a PDF. The
    writeup must discuss what you observe about your program's behavior as you
    tune the number of noise words that are filtered out and the amount of text
    that you feed your program. Does your program accurately identify the author
    for a small passage of text? What about a large passage of text? How do the
    different metrics (Euclidean, Manhattan, Cosine) compare against each other?
\end{enumerate}
