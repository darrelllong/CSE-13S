\section{Hashing with the SPECK Cipher}\label{sec:speck}

\noindent You will need a good hash function to use in your Bloom filter
(discussed in \S\ref{sec:bloom}) and hash table (discussed in
\S\ref{sec:hashtable}). We will have discussed hash functions in lecture, and
rather than risk having a poor one implemented, we will simply provide you one.
The SPECK\footnote{ Ray Beaulieu, Stefan Treatman-Clark, Douglas Shors, Bryan
Weeks, Jason Smith, and Louis Wingers, ``The {SIMON} and {SPECK} lightweight
block ciphers.'' In Proceedings of the 52nd ACM/EDAC/IEEE Design Automation
Conference (DAC), pp.  1--6. IEEE, 2015.} block cipher is provided for use as a
hash function.

SPECK is a family of lightweight block ciphers publicly released by the
National Security Agency (NSA) in June 2013.  SPECK has been optimized
for performance in software implementations, while its sister algorithm,
SIMON, has been optimized for hardware implementations. SPECK is an
add-rotate-xor (ARX) cipher. The reason a cipher is used for this is
because encryption generates random output given some input; exactly
what we want for a hash.

Encryption is the process of taking some file you wish to protect,
usually called plaintext, and transforming its data such that only
authorized parties can access it. This transformed data is referred to
as ciphertext. Decryption is the inverse operation of encryption, taking
the ciphertext and transforming the encrypted data back to its original
state as found in the original plaintext. Encryption algorithms that
utilize the same key for both encryption and decryption, like SPECK, are
symmetric-key algorithms, and algorithms that don't, such as RSA, are
asymmetric-key algorithms.

You will be given two files, \texttt{speck.h} and \texttt{speck.c}. The
former will provide the interface to using the SPECK hash function which
has been named \texttt{hash()}, and the latter contains the
implementation. The hash function \texttt{hash()} takes two parameters:
a 128-bit salt passed in the form of an array of two
\texttt{uint64\_t}s, and a key to hash. The function will return a
\texttt{uint32\_t} which is exactly the index the key is mapped to.

\begin{clisting}{}
uint32_t hash(uint64_t salt[], char *key);
\end{clisting}
