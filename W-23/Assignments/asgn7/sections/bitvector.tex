\section{Bit Vectors}\label{bitvector}

\noindent A bit vector is an ADT that represents a one dimensional array
of bits, the bits in which are used to denote if something is true or
false (1 or 0). This is an efficient ADT since, in order to represent
the truth or falsity of a bit vector of n items, we can use $\lceil
\frac{n}{8} \rceil$ \texttt{uint8\_t}s instead of $n$, and being able to
access 8 indices with a single integer access is extremely cost
efficient. Since we cannot directly access a bit, we must use bitwise
operations to get, set, and clear a bit within a byte. Much of the bit
vector implementation can be derived from your implementation of the
\texttt{Code} ADT used in assignment 6. If there were any issues with
getting, setting, or clearing bits in that assignment, make sure you
address them here.

\begin{clisting}{}
struct BitVector {
  uint32_t length;
  uint8_t *vector;
};
\end{clisting}

This \texttt{struct} definition \emph{must} go in \texttt{bv.c}.

\begin{funcdoc}{\texttt{BitVector *bv\_create(uint32\_t length)}}
  The constructor for a bit vector that holds \texttt{length} bits. In
  the even that sufficient memory cannot be allocated, the function must
  return \texttt{NULL}. Else, it must return a \texttt{BitVector *)}, or
  a pointer to an allocated \texttt{BitVector}. Each bit of the bit
  vector should be initialized to 0.
\end{funcdoc}

\begin{funcdoc}{\texttt{void bv\_delete(BitVector **bv)}}
  The destructor for a bit vector. Remember to set the pointer to
  \texttt{NULL} after the memory associated with the bit vector is
  freed.
\end{funcdoc}

\begin{funcdoc}{\texttt{uint32\_t bv\_length(BitVector *bv)}}
  Returns the length of a bit vector.
\end{funcdoc}

\begin{funcdoc}{\texttt{bool bv\_set\_bit(BitVector *bv, uint32\_t i)}}
  Sets the $i^\text{th}$ bit in a bit vector. If \texttt{i} is out of
  range, return \texttt{false}. Otherwise, return \texttt{true} to
  indicate success.
\end{funcdoc}

\begin{funcdoc}{\texttt{bool bv\_clr\_bit(BitVector *bv, uint32\_t i)}}
  Clears the $i^\text{th}$ bit in the bit vector. If \texttt{i} is out
  of range, return \texttt{false}. Otherwise, return \texttt{true} to
  indicate success.
\end{funcdoc}

\begin{funcdoc}{\texttt{bool bv\_get\_bit(BitVector *bv, uint32\_t i)}}
  Returns the $i^\text{th}$ bit in the bit vector. If \texttt{i} is out
  of range, return \texttt{false}. Otherwise, return \texttt{false} if
  the value of bit \texttt{i} is \texttt{0} and return \texttt{true} if
  the value of bit \texttt{i} is \texttt{1}.
\end{funcdoc}

\begin{funcdoc}{\texttt{void bv\_print(BitVector *bv)}}
  A debug function to print the bits of a bit vector. That is, iterate
  over each of the bits of the bit vector. Print out either 0 or 1
  depending on whether each bit is set. \textcolor{red}{You should write
  this immediately after the constructor}.
\end{funcdoc}
