\section{Bloom Filters}\label{sec:bloom}

\noindent With open addressing, it's possible that you need to iterate over the
entire hash table (if it's full), before you can come to the conclusion that
some key isn't present. This is very unsatisfactory with very large and full
hash tables. Is there a faster way to check if a key has never been seen? That
is where the \emph{Bloom filter} comes into play.

A Bloom filter is a space-efficient probabilistic data structure, conceived by
Burton H. Bloom in 1970, and is used to test whether an element is a member of a
set. False-positive matches are possible, but false negatives are not---in other
words, a query for set membership returns either ``possibly in the set'' or
``definitely not in the set.'' Elements can be added to the set but not removed
from it; the more elements added, the higher the probability of false positives.

A Bloom filter can be represented as an array of $m$ bits, or a \textbf{bit
vector}. A Bloom filter should utilize $k$ different hash functions. Using these
hash functions, a set element added to the Bloom filter is mapped to at most $k$
of the $m$ bit indices, generating a uniform pseudo-random distribution.
Typically, $k$ is a small constant which depends on the desired false error rate
$\epsilon$, while $m$ is proportional to $k$ and the number of elements to be
added.

Assume you are adding a word $w$ to your Bloom filter and are using $k=3$ hash
functions, $f(x)$, $g(x)$, and $h(x)$. To add $w$ to the Bloom filter, you
simply set the bits at indices $f(w)$, $g(w)$, and $h(w)$. To check if some word
$w'$ has been added to the same Bloom filter, you check if the bits at indices
$f(w')$, $g(w')$, and $h(w')$ are set. If they are all set, then $w'$ has
\emph{most likely} been added to the Bloom filter. If \emph{any one} of those
bits was cleared, then $w'$ has definitely \emph{not} been added to the Bloom
filter. The fact that the Bloom filter can only tell if some word has \emph{most
likely} been added to the Bloom filter means that \emph{false positives} can
occur. The larger the Bloom filter, the lower the chances of getting false
positives.

So what do Bloom filters mean for you? It means you can add each word of some
sample of text into your Bloom filter. When comparing the words of this sample
with the words of another sample, it is much more efficient to simply query the
Bloom filter of the other sample of text for whether not a word has been seen,
rather than look it up in the hash table. You decide to implement a Bloom filter
with \emph{three} salts for \emph{three} different hash functions. Why? To
reduce the chance of a \emph{false positive}.

You can think of a ``salt'' as an initialization vector or a key. Using
different salts with the same hash function results in a different, unique hash.
Since you are equipping your Bloom filter with three different salts, you are
effectively getting three different hash functions: $f(x)$, $g(x)$, and $h(x)$.
Hashing a word $w$, with extremely high probability, should result in $f(w) \ne
g(w) \ne h(w)$. These salts are to be used for the SPECK cipher, which requires
a 128-bit key, so we have used MD5 \footnote{Rivest, R.. “The MD5 Message-Digest
Algorithm.” RFC 1321 (1992): 1-21.} ``message-digest'' to reduce three books
down to 128 bits each. These salts are provided for you in \texttt{salts.h}.
\textcolor{red}{Do not change them.} You will use the SPECK cipher as a hash
function, as discussed in \S\ref{sec:speck}.

The following \texttt{struct} defines the \texttt{BloomFilter} ADT. The three
salts will be stored in the \texttt{primary}, \texttt{secondary}, and
\texttt{tertiary} fields. Each salt is 128 bits in size. To hold these 128 bits,
we use an array of two \texttt{uint64\_t}s.

\begin{clisting}{}
struct BloomFilter {
  uint64_t primary[2];   // Primary hash function salt.
  uint64_t secondary[2]; // Secondary hash function salt.
  uint64_t tertiary[2];  // Tertiary hash function salt.
  BitVector *filter;
};
\end{clisting}

This \texttt{struct} definition \emph{must} go in \texttt{bf.c}.

\begin{funcdoc}{\texttt{BloomFilter *bf\_create(uint32\_t size)}}
  The constructor for a Bloom filter. The primary, secondary, and
  tertiary salts that should be used are provided in \texttt{salts.h}.
  Note that you will also have to implement the bit vector ADT for your
  Bloom filter, as it will serve as the array of bits necessary for a
  proper Bloom filter. Bit vectors will be discussed in
  \S\ref{bitvector}.
\end{funcdoc}

\begin{funcdoc}{\texttt{void bf\_delete(BloomFilter **bf)}}
  The destructor for a Bloom filter. As with all other destructors, it
  should free any memory allocated by the constructor and null out the
  pointer that was passed in.
\end{funcdoc}

\begin{funcdoc}{\texttt{uint32\_t bf\_size(BloomFilter *bf)}}
  Returns the size of the Bloom filter. In other words, the number of
  bits that the Bloom filter can access. Hint: this is the length of the
  underlying bit vector.
\end{funcdoc}

\begin{funcdoc}{\texttt{void bf\_insert(BloomFilter *bf, char *word)}}
  Takes \texttt{word} and inserts it into the Bloom filter. This entails hashing
  \texttt{word} with each of the three salts for three indices, and setting the
  bits at those indices in the underlying bit vector.
\end{funcdoc}

\begin{funcdoc}{\texttt{bool bf\_probe(BloomFilter *bf, char *word)}}
  Probes the Bloom filter for \texttt{word}. Like with \texttt{bf\_insert()},
  \texttt{word} is hashed with each of the three salts for three indices. If all
  the bits at those indices are set, return \texttt{true} to signify that
  \texttt{word} was most likely added to the Bloom filter. Else, return
  \texttt{false}.
\end{funcdoc}

\begin{funcdoc}{\texttt{void bf\_print(BloomFilter *bf)}}
  A debug function to print out the bits of a Bloom filter. This will ideally
  utilize the debug print function you implement for your bit vector.
\end{funcdoc}
