\section{Hash Tables}\label{sec:hashtable}

You will require some method to quickly store and retrieve unique words in a
sample of text, as well as the count of each unique word. A hash table would be
the best data structure for this purpose. A hash table maps keys to values and
provides, on average, fast \emph{constant time} look-ups. It does so typically
by taking a key $k$, hashing it with some hash function $h(x)$, and placing the
key's corresponding value in an underlying array at index $h(k)$. This is the
perfect way to store all unique words in a sample of text along with their
respective counts. So what happens when two words have the same hash value? This
is called a \emph{hash collision}, and must be resolved. To resolve these
collisions, you will perform \emph{open addressing} using \emph{linear probing}.

The open addressing method of resolving hash collisions requires searching
through the hash table if index $h(k)$ for some key $k$ doesn't contain $k$.
This act of searching, or \emph{probing}, can be done in multiple ways, the
simplest of which is \emph{linear probing}. With linear probing, successive
indices of the hash table starting from $h(k)$ are probed until either $k$ is
found, or an empty slot shows up. If an empty slot shows up before $k$ is found,
it means $k$ was never inserted. Using open addressing, it's possible that a key
cannot be inserted into a hash table because it is completely filled up.

A hash function maps data of arbitrary size to fixed-size values. Its values are
called hash values, hash codes, digests, or simply hashes, which index a
fixed-size table called a hash table.  Hash functions and their associated hash
tables are used in data storage and retrieval applications to access data (on
average) at a nearly constant time per retrieval. They require a storage space
that is only fractionally greater than the total space needed for the data.
Hashing avoids the non-linear access time of ordered and unordered lists and
some trees and the often exponential storage requirements of direct access of
large state spaces.

Hash functions rely on statistical properties of key and function interaction:
worst-case behavior is an exhaustive search but with a vanishingly small
probability, and average-case behavior can be nearly constant (with minimal
collisions).

Below is the \texttt{struct} definition for a hash table. The hash table
contains a salt that will be passed to the given hash function, \texttt{hash()}
--- more on this in \S\ref{sec:speck}. Each hash table entry is a \texttt{Node}
that contains a unique word and its count: the number of times is appears in a
sample of text. The node ADT will be discussed in \S\ref{sec:node}. Since we are
resolving hash collisions with open addressing, it only makes sense to store
these nodes in an array.

\begin{clisting}{}
struct HashTable {
  uint64_t salt[2]; // The salt to use for the hash function.
  uint32_t size;    // The number of slots in the hash table.
  Node **slots;     // The array of hash table items.
};
\end{clisting}

This \texttt{struct} definition \emph{must} go in \texttt{ht.c}.

\begin{funcdoc}{\texttt{HashTable *ht\_create(uint32\_t size)}}
  The constructor for a hash table. The \texttt{size} parameter denotes the
  number of slots that the hash table can index up to. The salt for the hash
  table is provided in \texttt{salts.h}.
\end{funcdoc}

\begin{funcdoc}{\texttt{void ht\_delete(HashTable **ht)}}
  The destructor for a hash table. This should free any remaining nodes left in
  the hash table. Remember to set the pointer to the freed hash table to
  \texttt{NULL}.
\end{funcdoc}

\begin{funcdoc}{\texttt{uint32\_t ht\_size(HashTable *ht)}}
  Returns the hash table's size, the number of slots it can index up to.
\end{funcdoc}

\begin{funcdoc}{\texttt{Node *ht\_lookup(HashTable *ht, char *word)}}
  Searches for an entry, a node, in the hash table that contains \texttt{word}.
  If the node is found, the pointer to the node is returned. Otherwise, a
  \texttt{NULL} pointer is returned.
\end{funcdoc}

\begin{funcdoc}{\texttt{Node *ht\_insert(HashTable *ht, char *word)}}
  Inserts the specified \texttt{word} into the hash table. If the word already
  exists in the hash table, increment its count by 1. Otherwise, insert a new
  node containing \texttt{word} with its count set to 1. Again, since we're
  using open addressing, it's possible that an insertion fails if the hash table
  is filled. To indicate this, return a pointer to the inserted node if the
  insertion was successful, and return \texttt{NULL} if unsuccessful.
\end{funcdoc}

\begin{funcdoc}{\texttt{void ht\_print(HashTable *ht)}}
  A debug function to print out the contents of a hash table.
  \textcolor{red}{Write this immediately after the constructor}.
\end{funcdoc}
