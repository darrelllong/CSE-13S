\section{Iterating Over Hash Tables}\label{sec:iterator}

You will need some mechanism to iterate over all the entries in a hash table.
High-level languages such as \textbf{Python} and \textbf{Rust} natively provide
iterators to do exactly this. \textbf{C}, on the other hand, provides no such
utility. This means that you will need to implement your own hash table iterator
type. This will be called \texttt{HashTableIterator} and should be defined in
\texttt{ht.c}. The hash table and the hash table iterator should be implemented
in the same file. \textcolor{red}{You must use the following \texttt{struct}
definition for your hash table iterator.}

\begin{clisting}{}
struct HashTableIterator {
    HashTable *table; // The hash table to iterate over.
    uint32_t slot;    // The current slot the iterator is on.
};
\end{clisting}

Storing each hash table entry in its own array index makes the hash table
iterator implementation almost trivial. All an iterator needs to do is keep
track of which slot it has iterated up to in the hash table. The following
sections are the functions that must be implemented for your hash table iterator
interface.

\begin{funcdoc}{HashTableIterator *hti\_create(HashTable *ht)}
  Creates a new hash table iterator. This iterator should iterate over the
  \texttt{ht}. The \texttt{slot} field of the iterator should be initialized to
  0.
\end{funcdoc}

\begin{funcdoc}{void hti\_delete(HashTableIterator **hti)}
  Deletes a hash table iterator. You \emph{should not} delete the \texttt{table}
  field of the iterator, as you may need to iterate over that hash table at a
  later time.
\end{funcdoc}

\begin{funcdoc}{Node *ht\_iter(HashTableIterator *hti)}
  Returns the pointer to the next valid entry in the hash table. This may
  require incrementing the \texttt{slot} field of the iterator multiple times to
  get to the next valid entry. Return \texttt{NULL} if the iterator has iterated
  over the entire hash table.
  \begin{clisting}{Example hash table iterator usage.}
HashTable *ht = ht_create(4);
ht_insert(ht, "hello");
ht_insert(ht, "world");
HashTableIterator *hti = hti_create(ht);
Node *n = NULL;
while ((n = ht_iter(hti)) != NULL) {
  print("%s\n", n->word);
}
hti_delete(&hti);
ht_delete(&ht);
  \end{clisting}
  The above code should print either "hello" then "world" or "world" then
  "hello" depending on how the hash function works.
\end{funcdoc}
