\section{Texts}\label{sec:text}

We will need a new ADT to encapsulate the parsing of a text and serve as the
in-memory representation of the distribution of words in the file. This is the
\texttt{Text} ADT. It will contain a hash table, Bloom filter, and the count of
how many words are in the text. This is \emph{not} the number of unique words in
the text, but the \emph{total} number of words in the text. The following
\texttt{struct} definition \emph{must} go in \texttt{text.c}.

\begin{clisting}{}
struct Text {
  HashTable *ht;
  BloomFilter *bf;
  uint32_t word_count;
};
\end{clisting}

\begin{funcdoc}{Text *text\_create(FILE *infile, Text *noise)}
  The constructor for a text. Using the regex-parsing module, get each word of
  \texttt{infile} and convert it to \emph{lowercase}. The \texttt{noise}
  parameter is a \texttt{Text} that contains noise words to filter out. That is,
  each parsed, lowercase word is only added to the created \texttt{Text} if and
  only if the word doesn't appear in the noise \texttt{Text}.

  Why are we ignoring certain words? As you can imagine, certain
  words of the English language occur quite frequently in writing, words like
  ``a'', ``the'', and ``of''. These words aren't great indicators of an author's
  unique diction and thus add additional noise to the computed distance. Hence,
  they should be ignored. If \texttt{noise} is \texttt{NULL}, then the
  \texttt{Text} that is being created is the noise text itself.

  If sufficient memory cannot be allocated, the function must return
  \texttt{NULL}. Else, it must return a \texttt{Text *}, or a pointer to an
  allocated \texttt{Text}. The hash table should be created with a size of
  $2^{19}$ and the Bloom filter should be created with a size of $2^{21}$.
\end{funcdoc}

\begin{funcdoc}{void text\_delete(Text **text)}
  Deletes a text. Remember to free both the hash table and the Bloom filter in
  the text before freeing the text itself. Remember to set the pointer to
  \texttt{NULL} after the memory associated with the text is freed.
\end{funcdoc}

\begin{funcdoc}{double text\_dist(Text *text1, Text *text2, Metric metric)}
  This function returns the distance between the two texts depending on the
  metric being used. This can be either the Euclidean distance, the Manhattan
  distance, or the cosine distance. The \texttt{Metric} enumeration is provided
  to you in \texttt{metric.h} and will be mentioned as well in \S\ref{sec:task}.

  Remember that the nodes contain the counts for their respective words and
  still need to be normalized with the total word count from the text.
\end{funcdoc}

\begin{funcdoc}{double text\_frequency(Text *text, char* word)}
  Returns the frequency of the word in the text. If the word is not in the text,
  then this must return 0. Otherwise, this must return the normalized frequency
  of the word.
\end{funcdoc}

\begin{funcdoc}{bool text\_contains(Text *text, char* word)}
  Returns whether or not a word is in the text. This should return \texttt{true}
  if \texttt{word} is in the text and \texttt{false} otherwise.
\end{funcdoc}

\begin{funcdoc}{void text\_print(Text *text)}
  A debug function to print the contents of a text. You may want to just call
  the respective functions of the component parts of the text.
\end{funcdoc}
