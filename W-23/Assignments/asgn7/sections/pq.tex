\section{Priority Queue}\label{sec:pq}

You will use a priority queue to store the names of author along with the
distance calculated between text authored by the author the anonymous sample of
text. This interface for this priority queue is designed specifically for this
task. It should enqueue and dequeue pairs of the author and the corresponding
distance as separate parameters instead of as a single struct. This means that
you can implement your priority queue as you wish, so long as you follow the
defined API.

\begin{funcdoc}{PriorityQueue *pq\_create(uint32\_t capacity)}
  The constructor for a priority queue that holds up to \texttt{capacity}
  elements. In the event that sufficient Memory cannot be allocated, the
  function must return \texttt{NULL}. Else, it must return a
  \texttt{PriorityQueue *}, or a pointer to an allocated \texttt{PriorityQueue}.
  The priority queue should initially contain no elements.
\end{funcdoc}

\begin{funcdoc}{void pq\_delete(PriorityQueue **q)}
  The destructor for a priority queue. Remember to set the pointer to
  \texttt{NULL} after the memory associated with the priority queue is freed.
  Anything left in the priority queue that hasn't been dequeued should be freed
  as well.
\end{funcdoc}

\begin{funcdoc}{bool pq\_empty(PriorityQueue *q)}
  Returns \texttt{true} if the priority queue is empty and \texttt{false}
  otherwise.
\end{funcdoc}

\begin{funcdoc}{bool pq\_full(PriorityQueue *q)}
  Returns \texttt{true} if the priority queue is full and \texttt{false}
  otherwise.
\end{funcdoc}

\begin{funcdoc}{uint32\_t pq\_size(PriorityQueue *q)}
  Returns the number of elements in the priority queue.
\end{funcdoc}

\begin{funcdoc}{bool enqueue(PriorityQueue *q, char *author, double dist)}
  Enqueue the \texttt{author}, \texttt{dist} pair into the priority queue. If
  the priority queue is full, return \texttt{false}. Otherwise, return
  \texttt{true} to indicate success.
\end{funcdoc}

\begin{funcdoc}{bool dequeue(PriorityQueue *q, char **author, double *dist)}
  Dequeue the \texttt{author}, \texttt{dist} pair from the priority queue. The
  pointer to the author string is passed back with the \texttt{author} double
  pointer. The distance metric value is passed back with the \texttt{dist}
  pointer. If the priority queue is empty, return \texttt{false}. Otherwise,
  return \texttt{true} to indicate success.
\end{funcdoc}

\begin{funcdoc}{void pq\_print(PriorityQueue *q)}
  A debug function to print the priority queue.
\end{funcdoc}
