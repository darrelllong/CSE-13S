\section{Deliverables}

\epigraphwidth=0.4\textwidth
\epigraph{\emph{It is such a sweet temptation \\ It gives such grief relief \\
It is such a false sensation \\ How come that's so hard to believe?}}{---Ray
Wylie Hubbard, \emph{Loco Gringo's Lament}}

You will need to turn in the following source code and header files:

\begin{enumerate}
  \item Your program \emph{must} have the following source and header
    files:
  \begin{itemize}
    \item \texttt{universe.c} implements the \texttt{Universe} ADT.
    \item \texttt{universe.h} specifies the interface to the \texttt{Universe}
      ADT. This file is provided and \emph{may not} be modified.
    \item \texttt{life.c} contains \texttt{main()} and \emph{may}
      contain any other functions necessary to complete your implementation of
      the Game of Life.
  \end{itemize}
\end{enumerate}

You can have other source and header files, but \emph{do not try to be overly
clever}. You will also need to turn in the following:

\begin{enumerate}
  \item \texttt{Makefile}:
    \begin{itemize}
      \item \texttt{CC = clang} must be specified.
      \item \texttt{CFLAGS = -Wall -Wextra -Werror -Wpedantic} must be specified.
      \item \texttt{make} must build the \texttt{life} executable, as should
        \texttt{make all} and \texttt{make life}.
      \item \texttt{make clean} must remove all files that are compiler
        generated.
      \item \texttt{make format} should format all your source code,
        including the header files.
    \end{itemize}
  \item \texttt{README.md}: This must use proper Markdown syntax. It
    must describe how to use your program and \texttt{Makefile}. It
    should also list and explain any command-line options that your
    program accepts. Any false positives reported by \texttt{scan-build}
    should be documented and explained here as well. Note down any known
    bugs or errors in this file as well for the graders.
  \item \texttt{DESIGN.pdf}: This document \emph{must} be a proper
    PDF\@. This design document must describe your design and design
    process for your program with enough detail such that a sufficiently
    knowledgeable programmer would be able to replicate your
    implementation. \textcolor{red}{This does not mean copying your
    entire program in verbatim}. You should instead describe how your
    program works with supporting pseudocode.
  \item \texttt{WRITEUP.pdf}: This document \emph{must} be a proper
    PDF\@. This writeup document must include everything you learned from
    this assignment. Make sure to mention everything in detail while being as precise as possible. How well you explain all the lessons you have learned in this assignment will be really important here. 
\end{enumerate}
