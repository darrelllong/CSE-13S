\section{Controlling the Cursor}

\epigraphwidth=0.6\textwidth
\epigraph{\emph{An orphan's curse would drag to hell \\ A spirit from on high;
\\ But oh! more horrible than that \\ Is the curse in a dead man's eye! \\ Seven
days, seven nights, I saw that curse, \\ And yet I could not die.}}{---Samuel
Taylor Coleridge, \emph{The Rime of the Ancient Mariner}}

\texttt{ncurses} is a programming library used to develop text-based user
interfaces. It is used in vi (vim, neovim), emacs and many other programs. You
will use this library to display the state of the universe after each evolution.
Here is some code that showcases moving an `\texttt{o}' horizontally across the
screen:

\begin{clisting}{Short \texttt{ncurses} example.}
#include <ncurses.h>
#include <unistd.h> // For usleep().

#define ROW 0
#define DELAY 50000

int main(void) {
    initscr();                    // Initialize the screen.
    curs_set(FALSE);              // Hide the cursor.
    for (int col = 0; col < 40; col += 1) {
        clear();                  // Clear the window.
        mvprintw(ROW, col, "o");  // Displays "o".
        refresh();                // Refresh the window.
        usleep(DELAY);            // Sleep for 50000 microseconds.
    }
    endwin();                     // Close the screen.
    return 0;
}
\end{clisting}

To test this code snippet out, place it in \texttt{example.c} and compile it
with the command:

\begin{shlisting}{}
$ clang -o example example.c -lncurses
\end{shlisting}

The \texttt{-lncurses} at the end serves to \emph{link} the \texttt{ncurses}
library. Linked libraries should always be linked at the end. Why? \textsc{Unix}
links binaries from left to right. When an \emph{undefined} reference is
encountered, it is expected to be defined \emph{later}. An \emph{unreferenced}
item is ignored, so if you list the library first, it will be unreferenced and
is thus ignored.
