\documentclass[12pt]{article}
\usepackage{fullpage,fourier}
\usepackage [english]{babel}
\usepackage [autostyle, english = american]{csquotes}
\MakeOuterQuote{"}
\title{Essay on \emph{Just Scoring Points}}
\author{ChatGPT}
\date{\today}
\begin{document}\maketitle
In "Just Scoring Points," the author, Walter R. Tschinkel, argues
that professors and students have different perceptions of education,
and that this disconnect is causing students to miss out on a true
education. He explains that students often view education as a way
to score points and get good grades, while professors view it as a
way to build understanding and knowledge. This difference in
perception leads to students not retaining information or connecting
it to other subjects, which ultimately diminishes their overall
understanding of the material.

The author uses the metaphor of building a structure to illustrate
the professor's perspective on education. He notes that for the
structure to be strong, each brick, or piece of knowledge, must be
connected and built upon. However, he observes that students do not
retain information from previous courses, and are unable to connect
it to new information. Instead, they focus on getting good grades,
which is similar to winning a match in sports. The author then
points out that this "sports approach" is not conducive to a true
education.

The author also suggests that the problem lies not only with the
students, but with the professors as well. Professors often assume
that students have a strong foundation of knowledge and fail to
demand or encourage long-term memory or cross-disciplinary thinking.
They also focus on the latest and most interesting discoveries,
rather than ensuring that students understand the fundamentals.

In conclusion, the author argues that the disconnect between the
professors' and students' perceptions of education is causing
students to miss out on a true education. He suggests that students
should focus on building understanding and knowledge, rather than
just getting good grades, and that professors should focus on
ensuring that students retain information and can connect it to
other subjects. This will lead to a more well-rounded education for
students.

\end{document}
