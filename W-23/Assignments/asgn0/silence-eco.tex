\documentclass[11pt,twocolumn]{article}
\usepackage[T1]{fontenc}
\usepackage[colorlinks=true,urlcolor=blue]{hyperref}
\usepackage[utf8]{inputenc}
\usepackage{Zallman,lettrine}
\usepackage{amsmath,amssymb}
\usepackage{booktabs}
\usepackage{color}
\usepackage{ebgaramond}
\usepackage{epigraph}
\usepackage{float}
\usepackage{fourier}
\usepackage{fullpage}
\usepackage{graphicx}
\usepackage{listings}
\usepackage{multicol}
\usepackage{subcaption}
\usepackage{url}
\usepackage{wrapfig}
\usepackage{xspace}

\title{Censorship and Silence}
\author{Umberto Eco \\ \small \textsf{translated from the Italian by Richard Dixon}}
\date{}

\begin{document}\maketitle

\lettrine[lraise=0.15,nindent=0.4em,lines=2]{\textin{T}}{hose of you}
who are younger may think that \emph{veline}
are pretty girls who dance about on television shows, and that a
casino is a chaotic mess\footnote{\emph{Translator's note}: The
word casino in Italian is in effect two words, with two pronunciations---a
casin\`o,with the accent on the final syllable, is the same as the
English casino, or gambling house; but here we are concerned with
the other word casino, pronounced, confusingly, with the stress on
the penultimate syllable exactly the same way as the English word.}
Anyone of my generation knows that the word casino used to mean
``brothel'' and only later, by connotation, did it come to mean
``somewhere chaotic,'' so that it lost its initial meaning, and
today anyone, perhaps even a bishop, uses it to indicate disorder.
Likewise, once upon a time a bordello was a brothel, but my
grandmother, a woman of the most upright morals, used to say, ``Don't
make a bordello,'' meaning ``Don't make too much racket''; the word
had completely lost its original meaning. The younger ones among
you may not know that, during the Fascist regime, \emph{veline}
were sheets of paper that the government department responsible for
controlling culture (called the Ministry of Popular Culture, shortened
to MinCulPop---they didn't have sufficient sense of humor to avoid
such an ambiguous-sounding name) sent to the newspapers. These
sheets of thin copy paper told the newspapers what they had to keep
quiet about and what they had to print. The \emph{velina}, in
journalistic jargon, therefore came to symbolize censorship, the
inducement to conceal, to make information disappear.\footnote{
Now that we have established what \emph{veline} originally were, I
can explain how the word came to take on its present meaning.  When
Antonio Ricci started the television entertainment show \emph{Striscia
la notizia} in the 1990s, he wanted some girls, usually appearing
on roller-skates, to bring messages for the two presenters, and he
called them \emph{veline}.  But the choice is very significant; it
means that when Ricci created \emph{Striscia la notizia}, the fact
that he could make a joke out of the word \emph{veline} indicated
there was still an audience that remembered and knew what the
\emph{veline} sent out by the MinCulPop were. If no one knows this
today, it is another reflection that can be made on ``noise,'' on
the superimposition of information; in the space of two decades one
notion is canceled out because it has been taken over by the obsessive
use of another.}

The \emph{veline} that we know today---the television showgirls---are,
however, the exact opposite: they are, as we all know, the celebration
of outward appearance, visibility, indeed of fame achieved through
pure visibility, where appearance signifies excellence---even that
kind of appearance that would once have been considered unseemly.

We find ourselves with two forms of \emph{velina}, which I would like to
compare with two forms of censorship. The first is censorship through
silence; the second is censorship through noise; I use the word
\emph{velina}, therefore, as a symbol of the television event, the show,
entertainment, news coverage, and so on.

Fascism had understood (as dictators generally do) that deviant
behavior is encouraged by the fact that the media give it coverage.
For example, the \emph{veline} used to say ``Don't write about suicide''
because the mere mention of suicide might inspire someone to commit
suicide a few days later. This is absolutely correct---we shouldn't
assume all that went through the minds of the Fascist hierarchy was
wrong---and it is quite true that we know about events of national
significance that have occurred only because the media have talked
about them. For example, the student protests of 1977 and 1989:
they were short-lived events that sought to repeat the protests of
1968 only because the newspapers had begun saying ``1968 is about
to return.'' Anyone involved in those events knows perfectly well
that they were created by the press, in the same way that the press
generates revenge attacks, suicides, classroom shootings---news
about one school shooting provokes other school shootings, and a
great many Romanians have probably been encouraged to rape old
ladies because the newspapers told them it is the exclusive speciality
of immigrants and is extremely easy to commit: all you have to do
is loiter in any pedestrian passage, near a railway station, and
so forth.

If the old-style \emph{velina} used to say, ``To avoid causing behavior
considered to be deviant, don't talk about it,'' the \emph{velina} culture
of today says, ``To avoid talking about deviant behavior, talk a
great deal about other things.'' I have always taken the view that
if, by some chance, I discovered that tomorrow's newspapers were
going to take up some wrong I had committed that would cause me
serious harm, the first thing I'd do would be plant a bomb outside
the local police headquarters or railway station. The next day the
newspaper front pages would he full of it and my personal misdemeanor
would end up as a small inside story. And who knows how many real
bombs have been planted to make other front-page stories disappear.
The example of the bomb is sonically appropriate, as it is an example
of a great noise that silences everything else.

Noise becomes a cover. I would say that the ideology of this
censorship through noise can be expressed, with apologies to
Wittgenstein, by saying, ``Whereof one cannot speak, thereof one
must talk a great deal.'' The flagship \emph{TG1} news program on
Italian state television, for example, is a master of this technique,
full of news items about calves born with two heads and bags snatched
by petty thieves---in other words, the sort of minor stories papers
used to put low on an inside page---which now serve to fill up
three-quarters of an hour of information, to ensure we don't notice
other news stories they ought to have covered have not been covered.
Several months ago, the press controlled by Berlusconi, in order
to undermine the authority of a magistrate who criticized the
premier, followed him for days, reporting that he sat smoking on a
bench, went to the barber, and wore turquoise socks. To make a
noise, you don't have to invent stories. All you have to do is
report a story that is real but irrelevant, yet creates a hint of
suspicion by the simple fact that it has been reported. It is true
and irrelevant that the magistrate wears turquoise socks, but the
fact it has been reported creates a suggestion of something not
quite confessed, leaving a mark, an impression. Nothing is more
difficult to dispose of than an irrelevant but true story.

The error made by \emph{La Repubblica} in its campaign against Berlusconi
was to give too much coverage to a relevant story (the party at
Noemi's house):\footnote{
\emph{Translator's note}: Silvio Berlusconi appeared as
guest at a girl's eighteenth-birthday party in April 2009)
prompting his wife to file for divorce.}
If, instead, it had reported something like this---``Berlusconi went
into Piazza Navona yesterday morning, met his cousin, and they had
a beer together \ldots\xspace how curious''---it would have triggered such a
series of insinuations, suspicions, and embarrassments that the
premier would have resigned long ago. In short, a fact that is too
relevant can be challenged, whereas an accusation that is not an
accusation cannot be challenged.

At the age of ten I was stopped in the doorway of a bar by a lady
who said, ``I'll give you one lira if you write a letter for me---I've
hurt my hand.'' Being a decent child I replied that I didn't want
any money and would do it simply as a favor, but the lady insisted
on buying me an ice cream. I wrote the letter for her and explained
what had happened when I got home. ``Good Lord,'' said my mother,
``they've made you write an anonymous letter. Heaven knows what will
happen to us when they find out!'' ``Look,'' I explained, ``there's
nothing terrible in that letter.'' In fact, it was addressed to a
wealthy businessman, whom I also knew (he had a shop in the city
center) and it said, ``It has come to our attention that you intend
to ask for the hand of Signorina {\em X} in marriage. We wish to inform
you that Signorina {\em X} is from a respectable and prosperous family
and is highly regarded throughout the city.'' Now, you don't usually
see an anonymous letter that praises the subject of the letter
rather than damning her. But what was the purpose of that anonymous
letter? Since the lady who recruited me clearly had no grounds for
saying anything else, she wanted at least to create unease. The
recipient would have wondered, ``Why should they send me such a
letter? What does `highly regarded throughout the city' actually
mean?'' I believe the wealthy businessman would have decided in the
end to postpone the idea of marriage for fear of setting up home
with someone so gossiped about.

This form of noise doesn't even require that the transmitted messages
be of any particular interest, since one message adds to another,
and together they create noise. Noise can sometimes take the form
of superfluous excess. A few months ago there was a fine article
by Berselli in \emph{L'Espresso} magazine, saying, Do you realize that
advertising no longer has any effect on us? No one can prove that
our soap powder is better than another (in fact they are all the
same), so for the past fifty years the only method anyone has come
up with shows us housewives who refuse the offer of two packets in
exchange for their own brand, or grandmothers who tell us that this
recalcitrant stain will disappear if we use the right powder. Soap
companies therefore carry out an intensive and relentless campaign,
consisting of the same message, which everyone knows by heart, so
that it becomes proverbial: ``Omo washes whiter than white,'' and so
on. Its purpose is twofold: partly to repeat the brand name (in
certain cases it becomes a successful strategy: if I have to go
into a supermarket and ask for soap powder, I will ask for Tide or
Omo because I have known these names for the past fifty years), and
partly to prevent anyone from realizing that no epideictic discussion
can be made about soap powder---either for or against. And the
same happens with other forms of advertising: Berselli observes
that in every mobile phone advert, none of us actually understand
what the characters are saying. But there's no need to understand
what they say~ it is the great noise that sells cell phones. I think
it is most probable that companies have jointly agreed to stop
promoting their own particular brands and to carry out general
publicity, to spread the mobile telephone culture. If you buy Nokia
instead of Samsung, you will be persuaded by other factors, but not
by advertising. In fact the main function of the publicity noise
is to remind you of the advertising sketch, not the product. Try
to think of the most pleasant, the most enjoyable piece of advertising---
some are even quite funny and to remember which product it relates
to. It is very rare that you manage to remember the name of the
product to which that advertisement refers: the child who mispronounces
``Simmenthal,'' or perhaps ``No Martini, no party'' or ``Ramazzotti is
always good for you.'' In all other cases the noise compensates for
the fact that there is no way to demonstrate the excellence of the
product.

The Internet, of course, generates, with no intention
to censor, the greatest noise that yields no information. Or rather:
first, you receive information, but you don't know whether it is
reliable; second, you try searching for information on the Internet:
only we academics and researchers, after ten minutes' work, can
begin to select the information we want. Most other users are stuck
on blogs, or on a porn site, and so forth, without surfing too far,
because surfing isn't going to help them find reliable information.

Looking further at cases of noise that do not presuppose any intention
to censure, but nevertheless tend toward censorship, we should also
mention the newspaper with sixty-four pages. Sixty-four pages are
too many to give real prominence to the most essential information.
Here again, some of you will say, ``But I buy a newspaper to find
the news that interests me.'' Certainly, but those who do that are
an elite who know how to deal with information---and there must be
some good explanation for the frightening drop in the number of
newspapers being sold and read. Young people no longer read newspapers.
It is easier to find the \emph{La Repubblica} or \emph{Corriere della Sera} sites
on the Internet---there, at least, it is all on one screen---or
to read the free sheets at the train station, where the news is set
out on two pages.

Therefore, as a result of noise, we have a deliberate censorship---this
is what is happening in the world of television, in creating political
scandals, and so forth-and we have an involuntary but fatal censorship
whereby, for reasons that are entirely legitimate in themselves
(such as advertising revenue, product sales, and so forth), excess
of information is transformed into noise.  This (and here I am
moving from communications to ethics) has also created a psychology
and morality of noise. Look at that idiot walking along the street,
wearing his iPod headphones; be cannot spend an hour on the train
reading a newspaper or looking at the countryside, but has to go
straight to his mobile phone during the first part of the journey
to say ``I've just left'' and on the second part of the journey to
say ``I'm just arriving.'' There are people now who cannot live
away from noise.  And it is for this reason that restaurants, already
noisy places, offer extra noise from a television screen---sometimes
two---and music; and if you ask for them to be switched off;
people stare at you as if you're mad. This great need for noise is
like a drug; it is a way to avoid focusing on what is really
important. \emph{Redi in interiorem hominem}: yes, in the end, the example
of Saint Augustine could still provide a good ideal for the world
of politics and television.

It is in silence alone that the only truly powerful means of
information becomes effective---word of mouth. All people, even when
they are oppressed by the most censorious tyrants, have been able
to find out all that is going on in the world through popular word
of mouth. Publishers know that books do not become bestsellers
through publicity or reviews but by what the French call \emph{bouche \`a oreille}
and
the Italians call \emph{passaparola}---books achieve success through word
of mouth. In losing the condition of silence, we lose the possibility
of hearing what other people are saying, which is the only basic
and reliable means of communication.

And that is why, in conclusion, I would say that one of the ethical
problems we face today is how to return to silence. And one of the
semiotic problems we might consider is the closer study of the
function of silence in various aspects of communication, to examine
a semiotics of silence: it maybe a semiotics of reticence, a
semiotics of silence in theater, a semiotics of silence in politics,
a semiotics of silence in political debate---in other words, the long
pause, silence as creation of suspense, silence as threat, silence
as agreement, silence as denial, silence in music. Look how many
subjects there are to study concerning the semiotics of silence.
I invite you to consider, therefore, not words but silence.

\vspace{1.25cm}
\centerline{\includegraphics[height=3cm]{../../images/Jonny-Plank.png}}
\vspace{0.5cm}
\noindent Lecture given during the conference of the \emph{Associazione ltaliana di Semiotica}, 2009.

\end{document}
